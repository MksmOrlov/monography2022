\begin{partbacktext}
\part*{Авторское предисловие}
\markboth{АВТОРСКОЕ ПРЕДИСЛОВИЕ}{АВТОРСКОЕ ПРЕДИСЛОВИЕ}
\label{chap_preface_auth}
\addcontentsline{toc}{part}{Авторское предисловие}

Ключевой особенностью \textbf{\textit{компьютерных систем нового поколения}} является их \textbf{\textit{интероперабельность}}, т.е. способность к эффективному осознанному взаимодействию, необходимым условием которой является способность к взаимопониманию, т.е. \textbf{\textit{семантическая совместимость}}. Таким образом, каждая \textit{компьютерная система нового поколения} должна:

\begin{textitemize}
	\item знать свои обязанности и возможности;
	\item уметь координировать свои действия с другими компьютерными системами нового поколения
	\begin{textitemize}
	\item в ходе коллективного решения сложных задач;
	\item в заранее не предусмотренных (нештатных) обстоятельствах.
	\end{textitemize}	
\end{textitemize}

Переход от современных компьютерных систем (в том числе, и интеллектуальных компьютерных систем) к компьютерным системам нового поколения, которые, очевидно, должны обладать достаточно высоким уровнем интеллекта, означает переход к принципиально новому технологическому укладу в области автоматизации различных видов человеческой деятельности и предполагает переосмысление и использование всего опыта, накопленного при разработке и эксплуатации самого различного вида \textbf{\textit{компьютерных систем}}. Поэтому участие в проекте создания \textbf{\textit{Технологии комплексной поддержки жизненного цикла компьютерных систем нового поколения}} не требует от специалистов радикального изменения области их научных интересов. Требуется просто учет дополнительных требований к формализации своих результатов. Основным из этих требований является \textbf{\textit{семантическая совместимость}} с другими (смежными) результатами.

К настоящему моменту завершен первый этап разработки \textit{Технологии поддержки жизненного цикла компьютерных систем} нового поколения, которая названа нами \textbf{\textit{Технологией OSTIS}} (Open Semantic Technology for Intelligent Systems). Началом указанного первого этапа является первая конференция OSTIS (Минск, 10 -- 12 февраля 2011 года).

К текущему моменту сформировался работоспособный стартовый авторский коллектив в рамках созданного Учебно-научного объединения по Искусственному интеллекту, в состав которого входят ведущие университеты РБ (в четырех из которых открыта специальность ``Искусственный интеллект''), и ряд других организаций.

Результаты первого этапа разработки \textit{Технологии OSTIS} отражены в материалах проведенных \textbf{\textit{конференций OSTIS}}, в первой версии формализованного \textbf{\textit{Стандарта Технологии OSTIS}} и в первой версии \textbf{\textit{Метасистемы OSTIS}}, которая непосредственно и осуществляет автоматизацию проектирования, реализации, реинжиниринга и эксплуатации \textbf{\textit{интеллектуальных компьютерных систем нового поколения}} и гарантирует поддержку их \textbf{\textit{семантической совместимости}} и интероперабельности не только на этапе проектирования, но и в ходе эксплуатации.

На следующем этапе разработки \textit{Технологии поддержки жизненного цикла интеллектуальных компьютерных систем нового поколения} требуется существенное расширение фронта работ и соответствующего авторского коллектива. Этому, в частности, была посвящена конференция OSTIS-2022 (24-26 ноября 2022 года), которая внесла важный вклад в развитие Открытого проекта развития технологии комплексной поддержки жизненного цикла интеллектуальных компьютерных систем нового поколения (Проекта OSTIS).
\end{partbacktext}