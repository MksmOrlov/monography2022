\section{Общие принципы интеграции Экосистемы OSTIS с современными сервисами и информационными ресурсами}
{\label{sec_integration_common_principles}}

\begin{SCn}
    
    \begin{scnrelfromlist}{подраздел}
        \scnitem{\ref{sec_integration_services}~\nameref{sec_integration_services}}
        \scnitem{\ref{sec_integration_resources}~\nameref{sec_integration_resources}}
    \end{scnrelfromlist}
    
\end{SCn}

\subsection*{Введение в \ref{sec_integration_common_principles}}

Процесс \textit{интеграции} \textit{Экосистемы OSTIS} с \textit{сервисами} и \textit{информационными ресурсами} требует глубокого понимания базовых принципов \textit{Технологии OSTIS}. 
В этом параграфе рассмотрены общие принципы \textit{интеграции} \textit{Экосистемы OSTIS} с современными \textit{сервисами} и \textit{информационными ресурсами}, как она может быть адаптирована для интеграции с различными \textit{сервисами} и \textit{ресурсами}, включая \textit{базы данных}, \textit{веб-сервисы} и \textit{облачные системы}.

Общие принципы \textit{интеграции} \textit{цифровой экосистемы} с современными \textit{сервисами} и \textit{информационными ресурсами} выглядят следующим образом (см. \scncite{Valdez2019}, \scncite{Li2012a}):
\begin{textitemize}
    \item \textit{стандартизация} и \textit{совместимость}, что достигается путем использования стандартизованных протоколов и форматов обмена данными.
    \item \textit{открытость} и доступность \textit{цифровой экосистемы} для различных участников, что обеспечениватся открытыми и удобными \textit{интерфейсами}.
    \item \textit{безопасность} и конфиденциальность, что достигается путем использования криптографических методов защиты данных и контроля доступа к ресурсам.
    \item \textit{автоматизация} и \textit{масштабируемость}, что позволит обеспечить эффективность и производительность \textit{цифровой экосистемы} при работе с большим количеством \textit{сервисов} и \textit{ресурсов}.
    \item анализ и управление данными, что поможет определить эффективность \textit{интеграции} и улучшить ее в дальнейшем.
\end{textitemize}

Каждый из этих принципов играет важную роль в \textit{интеграции} \textit{цифровой экосистемы} с современными \textit{сервисами} и \textit{информационными ресурсами} и должен быть учтен при разработке и реализации интеграционных решений. Важно отметить, что для успешной \textit{интеграции} необходимо учитывать специфику каждой конкретной системы и ее потребностей. Это требует анализа и понимания всех особенностей и требований, которые могут повлиять на процесс \textit{интеграции}.

В целом, для успешной \textit{интеграции} \textit{цифровой экосистемы} с современными \textit{сервисами} и \textit{информационными ресурсами} необходимо учитывать все указанные принципы и использовать современные технологии, такие как \textit{Технология OSTIS}, которые позволяют реализовать эти принципы в практической работе.

\textit{Технология OSTIS} представляет собой мощный инструмент для разработки и реализации \textit{цифровых экосистем}, которые могут интегрироваться с различными \textit{сервисами} и \textit{информационными ресурсами}. Она обладает рядом преимуществ, таких как использование открытых стандартов и моделей, возможность масштабирования и автоматизации процессов, высокий уровень безопасности и конфиденциальности, а также удобные \textit{интерфейсы} для работы с данными.

\textit{ostis-системы} могут выполнять роль системных интеграторов различных \textit{ресурсов} и \textit{сервисов}, реализованных современными \textit{компьютерными системами}, поскольку \textit{уровень интеллекта} \textit{ostis-систем} позволяет им с любой степенью детализации специфицировать интегрируемые \textit{компьютерные системы} и, следовательно, достаточно адекватно понимать, что знает и/или умеет каждая из них. Также \textit{ostis-системы} способны достаточно качественно координировать деятельность стороннего \textit{ресурса} и \textit{сервиса} и обеспечивать релевантный поиск нужного \textit{ресурса} или \textit{сервиса}.

Кроме того \textit{ostis-системы} могут выполнять роль интеллектуальных help-систем --- помощников и консультантов по вопросам эффективной эксплуатации различных \textit{компьютерных систем} со сложными функциональными возможностями, имеющими \textit{пользовательский интерфейс} с нетривиальной семантикой и использующимися в сложных предметных областях. 
Такие интеллектуальные help-системы можно сделать интеллектуальными посредниками между соответствующими \textit{компьютерными системами} их пользователями.

На первых этапах перехода к Обществу 5.0 нет необходимости преобразовывать в \textit{ostis-системы} все современные системы автоматизации некоторых видов и областей человеческой деятельности. 
Однако, \textit{ostis-системы} должны взять на себя координационно-связующую роль благодаря высокому уровню своей \textit{интероперабельности}. 
\textit{ostis-системы} должны научиться либо выполнять миссию активной интероперабельной надстройки над различными современными средствами автоматизации, либо ставить перед современными средствами автоматизации выполнимые для них задачи, обеспечивая их непосредственное участие в решении сложных комплексных задач и организуя управление взаимодействием различных средств автоматизации в процессе коллективного решения сложных комплексных задач.

\subsection{Принципы интеграции Экосистемы OSTIS с разнородными сервисами}
{\label{sec_integration_services}} 

В контексте интеграции \textit{Экосистемы OSTIS} с разнородными \textit{сервисами}, под \textit{сервисами} понимаются приложения, программы, \textit{веб-сервисы} и другие \textit{информационные системы}, которые предоставляют определенный функционал, механизм преобразования информации в соответствии с заданной функцией. Зачастую такое приложение может предоставить \textit{программный интерфейс}, который можно использовать с определённым форматом входов, которым будут соответствовать определённые форматы выходов.

Под \textit{интеграцией} \textit{Экосистемы OSTIS} с \textit{сервисом} следует понимать возможность использовать функционал \textit{сервиса} для изменения внутреннего состояния \textit{базы знаний} \textit{Экосистемы OSTIS}. 

При \textit{интеграции} \textit{сервисов} в \textit{цифровых экосистемах} возникают ряд проблем, которые могут затруднять процесс \textit{интеграции} и уменьшать эффективность экосистемы (см. \scncite{Valdez2019}, \scncite{Li2012a}). Некоторые из этих проблем могут включать в себя:

\begin{textitemize}
    \item различные форматы данных и протоколы обмена, которые могут привести к ошибкам при \textit{обмене информацией}, что затрудняет взаимодействие между \textit{сервисами};
    \item \textit{несовместимость} версий приложений, что может привести к конфликтам при \textit{обмене информацией};
    \item разные уровни безопасности, что может стать причиной утечки конфиденциальной информации;
    \item отсутствие единой точки управления, что затрудняет мониторинг и управление процессами \textit{интеграции};
    \item отсутствие механизмов для анализа и управления информацией, что затрудняет контроль над процессами \textit{обмена информации}.
\end{textitemize}

Перечисленные проблемы значительно усложняют разработку самих \textit{сервисов} и ведёт к значительному увеличению временных и материальных затрат. Для решения этих проблем при интеграции \textit{цифровых экосистем} с различными \textit{сервисами} и \textit{ресурсами} используются различные подходы и технологии (см \scncite{Caldarola2015}, \scncite{Bork2019}). Некоторые из них могут включать в себя:

\begin{textitemize}
\item использование стандартных протоколов и форматов обмена данных, таких как XML, JSON и другие, что позволяет сделать \textit{обмен информацией} более надежным и универсальным;
\item разработка единой схемы данных и правил доступа, что позволяет сделать \textit{интеграцию} более простой и управляемой;
\item реализация механизмов для автоматической обработки ошибок и конфликтов, что позволяет снизить количество ошибок и улучшить надежность \textit{цифровой экосистемы};
\item использование инструментов и технологий для анализа и управления информацией, таких как системы бизнес-аналитики и управления информацией, что позволяет контролировать процессы \textit{обмена информации} и оптимизировать их работу.
\end{textitemize}


В рамках \textit{Экосистемы OSTIS} выделено несколько уровней \textit{интеграции}, которые позволяют взаимодействовать с различными \textit{информационными ресурсами} и \textit{сервисами}. 

Полная \textit{интеграция} предполагает исполнение функции \textit{сервиса} на платформонезависимом уровне, где вся программа исполняется в самой \textit{базе знаний} \textit{Экосистемы OSTIS}. То есть, задача интеграции такого \textit{сервиса} сводится к выделению алогиртма обработки графовой конструкции и его реализации в рамках \textit{базы знаний}. В результате такой полной интеграции надобность использовать сторонний \textit{сервис} отпадает. 

Частичная \textit{интеграция} предполагает реализацию взаимодействия и изменения состояния \textit{базы знаний} \textit{Экосистемы OSTIS} на этапах исполнения функции \textit{сервиса}. Степень глубины \textit{интеграции} может различаться, в некоторых случаях \textit{сервис} может обращаться к \textit{базе знаний} для получения дополнительной информации или для записи промежуточных результатов. 

В простейшем случае изменение \textit{базы знаний} \textit{Экосистемы OSTIS} может происходить единожды, после получения результата отработки функции \textit{сервиса}. На основе такого принципа можно выделить специальные \textit{sc-агенты}, использующие сторонний \textit{сервис}. 

Для обеспечения \textit{интеграции} функционального \textit{сервиса} необходимо выполнение следующих минимальных требований:
\begin{textitemize}
    \item спецификация входной конструкции в \textit{базе знаний} системы: определение структуры в \textit{базе знаний} системы, которая будет преобразовываться в формат данных, совместимый с \textit{сервисом};
    \item спецификация выходной конструкции в \textit{базе знаний} системы: определение структуры в \textit{базе знаний} системы, которая будет формироваться из исходной структуры впоследствии преобразования данных \textit{сервиса} в знания;
    \item реализация \textit{sc-агента}, который преобразует конструкцию \textit{базы знаний} в формат, который может быть использован в \textit{сервисе}, а также погружать результаты работы \textit{сервиса} обратно в \textit{базу знаний} системы в соответствии со спецификацией.
\end{textitemize}

Удовлетворение данных требований позволит обеспечить эффективную \textit{интеграцию} функционального \textit{сервиса}, что, в свою очередь, позволит использовать данные и функциональность \textit{сервиса} в различных \textit{ostis-системах}. Задача формирования спецификации рассмотрена в \ref{sec_kb_design_methods}~\nameref{sec_kb_design_methods}.

Обобщенный алгоритм \textit{sc-агента}, использующего сторонний \textit{сервис} для \textit{интеграции} функциональных возможностей, может быть описан следующим образом:

\begin{textitemize}
    \item извлечение из \textit{базы знаний} необходимых структур знаний, соответствующих требованиям функционального \textit{сервиса};
    \item преобразование извлечённых знаний в формат, необходимый для подачи на вход функциональному \textit{сервису};
    \item отправка запроса на функциональный \textit{сервис} и ожидание его ответа;
    \item формирование структур знаний на основе полученных данных от функционального \textit{сервиса};
    \item погружение новых структур знаний в \textit{базу знаний} \textit{интеллектуальной системы}, с целью обеспечения их дальнейшей использования.
\end{textitemize}

Следует учесть, что при \textit{интеграции} функциональных \textit{сервисов}, возможно потребуется проводить дополнительную обработку и преобразование данных, например, для обеспечения их \textit{совместимости} с форматами данных, или для обеспечения безопасности передачи и хранения данных.

Таким образом, внедрение возможности использования стороннего \textit{сервиса} в \textit{Экосистеме OSTIS} предполагает выполнение следующих шагов:
\begin{textitemize}
    \item анализ требований к интегрируемому \textit{сервису}, определение необходимого функционала, форматов входных и выходных данных, и других характеристик сервиса.
    \item разработка спецификации \textit{интеграции}, которая будет определять форматы данных и правила взаимодействия между \textit{базой знаний} \textit{Экосистемы OSTIS} и сторонним \textit{сервисом};
    \item разработка \textit{sc-агента}, который будет обеспечивать взаимодействие между \textit{базой знаний} и сторонним \textit{сервисом} в соответствии со спецификацией \textit{интеграции};
    \item тестирование и отладка;
    \item внедрение в \textit{Экосистему OSTIS}, что позволит использовать возможности интегрированного стороннего \textit{сервиса} в различных \textit{ostis-системах}.
\end{textitemize}

Использование описанного подхода описано в работе \scncite{Kroshchanka2022} на примере интеграции сервисов компьютерного зрения с ostis-системой. 

Для подключения \textit{sc-агента} используются различные подходы, один из которых заключается в подключении \textit{sc-агента} в рамках уже существующей, основной \textit{ostis-системы}. С точки зрения масштабируемости реализации такого подхода следует отметить \textit{монолитную архитектуру} получаемой \textit{ostis-системы}, что позволяет упростить процесс внедрения новых \textit{сервисов} и \textit{sc-агентов} в \textit{ostis-систему}.

Преимуществом такого подхода является более простое и удобное внедрение новых \textit{сервисов} и \textit{sc-агентов} в \textit{ostis-систему}, а также упрощение процесса управления зависимостями. Использование реализации ostis-системы с \textit{монолитной архитектурой} может быть применено в случаях, когда функциональный \textit{сервис} не нуждается в частой модификации и обладает достаточно простой структурой. Кроме того, \textit{монолитная архитектура} может быть более удобна в случаях, когда обращение к \textit{сервису} происходит через внутреннюю сеть и требует низкой задержки и высокой производительности. Так могут быть интегрированы и \textit{сервисы} получения знаний из внешних источников (получение прогноза погоды, обработка статистической информации, и так далее), и функциональные \textit{сервисы} (обработка аудиоинформации и погружение результатов обработки в \textit{базу знаний}, получение синтаксического анализа предложения). 

Альтернативным вариантом внедрения является реализация отдельной \textit{ostis-системы}, в рамках которой будет интегрирована функция \textit{сервиса}. Это позволяет перейти к использованию \textit{микросервисной архитектуры}, что характеризуется распределенным взаимодействием \textit{ostis-систем}.

Преимуществами такого подхода является большая \textit{гибкость} и возможность \textit{масштабирования}. Подход характеризуется высокой степенью распределённости, децентрализованности и доступности нового функционала. Система выходит за рамки технических ограничений, функционал может быть распределён на различном аппаратном обеспечении. Полученный функционал может быть использован различными \textit{ostis-системами} в рамках \textit{Экосистемы OSTIS} для достижения своих целей. Минусами такой системы является сложность разработки, а также увеличение временных затрат на общение систем друг с другом по сетевым протоколам. 

\textit{микросервисную архитектуру} ostis-систем предпочтительно использовать в случаях, когда функциональный \textit{сервис} обладает сложной структурой, а также в случаях, когда требуется масштабирование и гибкость всей системы. В качестве примера можно привести \textit{сервис}, который взаимодействует с внешними источниками данных и может быть подвержен частым изменениям.  Примерами \textit{интеграции} на основе \textit{микросервисной архитектуры} могут служить \textit{сервисы} считывания эмоционального состояния пользователя, естественно-языковая обработка, задачи классификации и идентификации, и так далее.

Таким образом, выбор подхода к \textit{интеграции} функциональных \textit{сервисов} зависит от конкретных требований и условий проекта. Использование \textit{Технологии OSTIS} позволяет создавать гибкие и эффективные системы, которые могут быть адаптированы под различные потребности и требования пользователей.

\subsection{Принципы интеграции Экосистемы OSTIS со структурированными информационными ресурсами}
{\label{sec_integration_resources}} 

\begin{SCn}
\bigskip

\begin{scnrelfromlist}{библиографическая ссылка}
    \scnitem{\scncite{RDF}}
    \scnitem{\scncite{R2RML}}
    \scnitem{\scncite{R2RMLIO}}
    \scnitem{\scncite{INFOLAKE}}
\end{scnrelfromlist}
\end{SCn}

Существует несколько причин, по которым следует интегрировать интеллектуальную систему с информационными источниками:
\begin{textitemize}
    \item Обеспечение полноты и точности данных: Интеллектуальная система создается для обработки больших объемов данных и принятия решений на их основе. Информационные источники являются основой для этих данных, и интеграция системы с ними гарантирует полноту и точность данных.
    \item Уменьшение времени и усиление эффективности работы: Интеграция информационных источников в интеллектуальную систему обеспечивает быстрый и удобный доступ к нужной информации. Это уменьшает время на поиск и обработку данных, что повышает эффективность работы системы и уменьшает количество ошибок.
    \item Расширение возможностей системы: Информационные источники обладают большим количеством данных, которые могут быть полезны для работы интеллектуальной системы. Интеграция системы с различными источниками расширяет возможности системы и позволяет ей повысить качество работы.
    \item Повышение надежности: Разнообразные источники данных обеспечивают резервирование и возможность сравнения, что позволяет интеллектуальной системе работать более надежно и безопасно в случае сбоя одного или нескольких источников.
    \item Улучшение качества прогнозирования: Интеграция информационных источников с интеллектуальной системой способствует улучшению качества прогнозирования, так как позволяет объединять данные из различных источников и анализировать их вместе для получения более точных результатов.
\end{textitemize}

Существует большое количество информационных источников, из которых можно получать информацию. Основными можно назвать следующие:
\begin{textitemize}
    \item Интернет --- сайты, блоги, форумы, социальные сети, новостные порталы и другие ресурсы в сети.
    \item Книги и учебники --- доступные в библиотеках, книжных магазинах или в электронном формате.
    \item СМИ --- телевизионные программы, радио, газеты, журналы и другие источники новостей.
    \item Официальные документы и отчеты --- включая законы, правительственные статистические данные, отчеты об исследованиях и другие официальные документы.
\end{textitemize}

Принципы интеграции \textit{Экосистемы OSTIS} со структурированными информационными ресурсами основаны на пополнении базы знаний системы новыми знаниями. Один из востребованных подходов в этом направлении --- интеграция с ресурсами на основе RDF (Resource Description Framework, см. \scncite{RDF}), который является моделью данных, предложенной консорциумом W3C.

Для успешной интеграции структурированных информационных ресурсов в Экосистему OSTIS, важно уделить должное внимание пониманию и применению принципов RDF-модели, поскольку они играют ключевую роль в организации связей между различными ресурсами. RDF используется для описания ресурсов в сети Интернет, и является основой для построения семантических веб-приложений, таких как Linked Open Data.

Основная структура абстрактного синтаксиса RDF --- это тройка, состоящая из субъекта, предиката и объекта. Набор таких троек называется графом RDF. Граф RDF может быть визуализирован как диаграмма узла и направленной дуги, в которой каждая тройка представлена как связь ``узел --- дуга --- узел''.

Графы RDF атемпоральны, т.е. представляют собой статические снимки информации. Однако графы RDF могут выражать информацию о событиях и временных аспектах других сущностей, учитывая соответствующие термины из словаря. Поскольку графы RDF определены как математические наборы, добавление или удаление троек из графа RDF дает другой граф RDF.

Узел может иметь следующий тип:
\begin{textitemize}
    \item IRI. Представляет собой короткую последовательность символов, идентифицирующую абстрактный или физический ресурс на любом языке мира. IRI представляе собой обобщение URI;
    \item Литерал. Представляя собой структуру, состоящую из лексической формы (UNICODE-строка) и типа данных;
    \item Пустой узел. Представляет собой локальный идентификатор, который используются в некоторых конкретных синтаксисах RDF или реализациях хранилища RDF.
\end{textitemize}

RDF поддерживает основные типы данных, такие как строковый (string), логический (boolean), числовые (integer, double, float и др.), временные и некоторые другие.

В RDF существует такое понятие, как словарь RDF. Он представляет собой совокупность IRI, ссылающихся на другие графы с классами, литералами и др. Часто группа IRI может начинаться с одинакового префикса.

RDF нашел широкое применение. Так, например, RDF используется в оформлении \textit{баз знаний} в рамках различных проектов во множестве институтов, университетов и иных организаций. Поисковые системы предлагают веб-мастерам использовать RDF и аналогичные языки разметки страниц для повышения информативности ссылок на их сайты в результатах поиска. Социальные сети, с подачи Facebook, предлагают веб-мастерам использовать RDF для описания свойств страниц, так же позволяющих красиво оформить ссылку на неё в записи пользователя социальной сети.

В ходе анализа были выявлены следующие подходы к интеграции информационных ресурсов на основе RDF с другими системами:
\begin{textitemize}
    \item R2RML (см. \scncite{R2RML}) --- это стандарт W3C для выражения настраиваемых отображений из реляционных БД в RDF. Такие отображения предоставляют возможность просматривать существующие реляционные данные в модели данных RDF, выраженные в структуре и целевом словаре по выбору автора сопоставления;
    \item R2RML.io (см. \scncite{R2RMLIO}) --- это open-source проект, разрабатываемый с 2013 года. Данная технология предназначена для генерации базы знаний на основе данных из полуструктурированных источников;
    \item ``Озеро данных'' (см. \scncite{INFOLAKE}) --- это централизованное хранилище, которое позволяет хранить все структурированные и неструктурированные данные в любом масштабе. ``Семантическое озеро данных'' --- это особая форма озер данных, в которых верхний семантический слой обогащает и связывает данные семантически. Семантический уровень преодолевает разрозненность данных и обеспечивает семантический поиск по всем данным.
\end{textitemize}

Интеграция ostis-систем с внешними информационными ресурсами удобна по многим причинам. Технология OSTIS изначально предлагает инструменты для описания синтаксиса и семантики внешних языков (см. \textit{Главу \ref{chapter_ext_lang} \nameref{chapter_ext_lang}}). Данный инструментарий позволяет сократить время разработки в несколько раз. Также из достоинств можно выделить:
\begin{textitemize}
    \item способность осуществлять интеграцию знаний в своей памяти на высоком уровне;
    \item возможность интегрировать различные виды знаний;
    \item возможность интегрировать различные модели решения задач.
\end{textitemize}

Для интеграции информационного ресурса на основе RDF в Экосистему OSTIS был реализован соответствующий \textit{абстрактный sc-агент}. Его работу можно разбить на следующие этапы:
\begin{textitemize}
    \item интеграция с использованием готовых правил;
    \item интеграция с сохранением исходной схемы;
    \item дополнительные преобразования.
\end{textitemize}

\textbf{Интеграция с использованием готовых правил}

На этом этапе ко всем сгенерированным тройкам применяются готовые правила интеграции, хранящиеся в \textit{базе знаний}. Создание и применение подобных правил необходимо в ситуациях, когда способ представления конкретного знания во внешнем информационном ресурсе по какой-то причине не соответствует представлению аналогичного знания в ostis-системе.

\textbf{Интеграция с сохранением исходной схемы}

На данном этапе оставшиеся тройки будут преобразованы с сохранением той структуры отношения, в которой находились участвовавшие в нем сущности. Это значит, что порядок элементов в итоговой конструкции будет аналогичен порядку сущностей в исходной.

\textbf{Дополнительные преобразования}

На данном этапе проходят оставшиеся интеграционные преобразования, которым не нашлось места в предыдущих пунктах, но которые необходимы для завершения процесса интеграции.

Для выгрузки информации из \textit{базы знаний} в какой-либо внешний формат можно использовать те же правила, что и для загрузки, так как в основном они представляют собой утверждения об \textit{эквиваленции}. То есть изначально производится поиск необходимых конструкций, затем они к ним применяется соответствующее правило, и, в результате получается множество троек. В дальнейшем данные тройки преобразуются в необходимый формат.