\section{Общие принципы интеграции Экосистемы OSTIS с современными сервисами и информационными ресурсами}
{\label{sec_integration_common_principles}}

\begin{SCn}

    \bigskip
    \begin{scnrelfromlist}{ключевое понятие}
        \scnitem{цифровая экосистема}
        \scnitem{ostis-система}
        \scnitem{интеграция}
    \end{scnrelfromlist}

    \bigskip
    \begin{scnrelfromlist}{ключевой знак}
        \scnitem{Технологии OSTIS}
        \scnitem{Экосистемы OSTIS}
    \end{scnrelfromlist}
    
\end{SCn}

Процесс \textit{интеграции} \textit{Экосистемы OSTIS} с \textit{сервисами} и \textit{информационными ресурсами} требует глубокого понимания базовых принципов \textit{Технологии OSTIS}. 
В этом параграфе рассмотрены общие принципы \textit{интеграции} \textit{Экосистемы OSTIS} с современными \textit{сервисами} и \textit{информационными ресурсами}, как она может быть адаптирована для интеграции с различными \textit{сервисами} и \textit{ресурсами}, включая \textit{базы данных}, \textit{веб-сервисы} и \textit{облачные системы}.

Общие принципы \textit{интеграции} \textit{цифровой экосистемы} с современными \textit{сервисами} и \textit{информационными ресурсами} выглядят следующим образом (см. \scncite{Valdez2019}, \scncite{Li2012a}):
\begin{textitemize}
    \item \textit{стандартизация} и \textit{совместимость}, что достигается путем использования стандартизованных протоколов и форматов обмена данными.
    \item \textit{открытость} и доступность \textit{цифровой экосистемы} для различных участников, что обеспечениватся открытыми и удобными \textit{интерфейсами}.
    \item \textit{безопасность} и конфиденциальность, что достигается путем использования криптографических методов защиты данных и контроля доступа к ресурсам.
    \item \textit{автоматизация} и \textit{масштабируемость}, что позволит обеспечить эффективность и производительность \textit{цифровой экосистемы} при работе с большим количеством \textit{сервисов} и \textit{ресурсов}.
    \item анализ и управление данными, что поможет определить эффективность \textit{интеграции} и улучшить ее в дальнейшем.
\end{textitemize}

Каждый из этих принципов играет важную роль в \textit{интеграции} \textit{цифровой экосистемы} с современными \textit{сервисами} и \textit{информационными ресурсами} и должен быть учтен при разработке и реализации интеграционных решений. Важно отметить, что для успешной \textit{интеграции} необходимо учитывать специфику каждой конкретной системы и ее потребностей. Это требует анализа и понимания всех особенностей и требований, которые могут повлиять на процесс \textit{интеграции}.

В целом, для успешной \textit{интеграции} \textit{цифровой экосистемы} с современными \textit{сервисами} и \textit{информационными ресурсами} необходимо учитывать все указанные принципы и использовать современные технологии, такие как \textit{Технология OSTIS}, которые позволяют реализовать эти принципы в практической работе.

\textit{Технология OSTIS} представляет собой мощный инструмент для разработки и реализации \textit{цифровых экосистем}, которые могут интегрироваться с различными \textit{сервисами} и \textit{информационными ресурсами}. Она обладает рядом преимуществ, таких как использование открытых стандартов и моделей, возможность масштабирования и автоматизации процессов, высокий уровень безопасности и конфиденциальности, а также удобные \textit{интерфейсы} для работы с данными.

\textit{ostis-системы} могут выполнять роль системных интеграторов различных \textit{ресурсов} и \textit{сервисов}, реализованных современными \textit{компьютерными системами}, поскольку \textit{уровень интеллекта} \textit{ostis-систем} позволяет им с любой степенью детализации специфицировать интегрируемые \textit{компьютерные системы} и, следовательно, достаточно адекватно понимать, что знает и/или умеет каждая из них. Также \textit{ostis-системы} способны достаточно качественно координировать деятельность стороннего \textit{ресурса} и \textit{сервиса} и обеспечивать релевантный поиск нужного \textit{ресурса} или \textit{сервиса}.

Кроме того \textit{ostis-системы} могут выполнять роль интеллектуальных help-систем --- помощников и консультантов по вопросам эффективной эксплуатации различных \textit{компьютерных систем} со сложными функциональными возможностями, имеющими \textit{пользовательский интерфейс} с нетривиальной семантикой и использующимися в сложных предметных областях. 
Такие интеллектуальные help-системы можно сделать интеллектуальными посредниками между соответствующими \textit{компьютерными системами} их пользователями.

На первых этапах перехода к Обществу 5.0 нет необходимости преобразовывать в \textit{ostis-системы} все современные системы автоматизации некоторых видов и областей человеческой деятельности. 
Однако, \textit{ostis-системы} должны взять на себя координационно-связующую роль благодаря высокому уровню своей \textit{интероперабельности}. 
\textit{ostis-системы} должны научиться либо выполнять миссию активной интероперабельной надстройки над различными современными средствами автоматизации, либо ставить перед современными средствами автоматизации выполнимые для них задачи, обеспечивая их непосредственное участие в решении сложных комплексных задач и организуя управление взаимодействием различных средств автоматизации в процессе коллективного решения сложных комплексных задач.
