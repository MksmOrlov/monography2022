\section{Принципы интеграции Экосистемы OSTIS с разнородными сервисами}
{\label{sec_integration_services}} 

\begin{SCn}

    \bigskip
    
    \begin{scnrelfromlist}{ключевое понятие}
        \scnitem{цифровая экосистема}
        \scnitem{сервис}
        \scnitem{монолитная архитектура}
        \scnitem{микросервисная архитектура}
    \end{scnrelfromlist}
    
    \bigskip
    
    \begin{scnrelfromlist}{ключевой знак}
        \scnitem{Экосистема OSTIS}
        \scnitem{Технология OSTIS}
    \end{scnrelfromlist}
    
\end{SCn}

В контексте интеграции \textit{Экосистемы OSTIS} с разнородными \textit{сервисами}, под \textit{сервисами} понимаются приложения, программы, \textit{веб-сервисы} и другие \textit{информационные системы}, которые предоставляют определенный функционал, механизм преобразования информации в соответствии с заданной функцией. Зачастую такое приложение может предоставить \textit{программный интерфейс}, который можно использовать с определённым форматом входов, которым будут соответствовать определённые форматы выходов.

Под \textit{интеграцией} \textit{Экосистемы OSTIS} с \textit{сервисом} следует понимать возможность использовать функционал \textit{сервиса} для изменения внутреннего состояния \textit{базы знаний} \textit{Экосистемы OSTIS}. 

При \textit{интеграции} \textit{сервисов} в \textit{цифровых экосистемах} возникают ряд проблем, которые могут затруднять процесс \textit{интеграции} и уменьшать эффективность экосистемы (см. \scncite{Valdez2019}, \scncite{Li2012a}). Некоторые из этих проблем могут включать в себя:

\begin{textitemize}
    \item различные форматы данных и протоколы обмена, которые могут привести к ошибкам при \textit{обмене информацией}, что затрудняет взаимодействие между \textit{сервисами};
    \item \textit{несовместимость} версий приложений, что может привести к конфликтам при \textit{обмене информацией};
    \item разные уровни безопасности, что может стать причиной утечки конфиденциальной информации;
    \item отсутствие единой точки управления, что затрудняет мониторинг и управление процессами \textit{интеграции};
    \item отсутствие механизмов для анализа и управления информацией, что затрудняет контроль над процессами \textit{обмена информации}.
\end{textitemize}

Перечисленные проблемы значительно усложняют разработку самих \textit{сервисов} и ведёт к значительному увеличению временных и материальных затрат. Для решения этих проблем при интеграции \textit{цифровых экосистем} с различными \textit{сервисами} и \textit{ресурсами} используются различные подходы и технологии (см \scncite{Caldarola2015}, \scncite{Bork2019}). Некоторые из них могут включать в себя:

\begin{textitemize}
\item использование стандартных протоколов и форматов обмена данных, таких как XML, JSON и другие, что позволяет сделать \textit{обмен информацией} более надежным и универсальным;
\item разработка единой схемы данных и правил доступа, что позволяет сделать \textit{интеграцию} более простой и управляемой;
\item реализация механизмов для автоматической обработки ошибок и конфликтов, что позволяет снизить количество ошибок и улучшить надежность \textit{цифровой экосистемы};
\item использование инструментов и технологий для анализа и управления информацией, таких как системы бизнес-аналитики и управления информацией, что позволяет контролировать процессы \textit{обмена информации} и оптимизировать их работу.
\end{textitemize}


В рамках \textit{Экосистемы OSTIS} выделено несколько уровней \textit{интеграции}, которые позволяют взаимодействовать с различными \textit{информационными ресурсами} и \textit{сервисами}. 

Полная \textit{интеграция} предполагает исполнение функции \textit{сервиса} на платформонезависимом уровне, где вся программа исполняется в самой \textit{базе знаний} \textit{Экосистемы OSTIS}. То есть, задача интеграции такого \textit{сервиса} сводится к выделению алогиртма обработки графовой конструкции и его реализации в рамках \textit{базы знаний}. В результате такой полной интеграции надобность использовать сторонний \textit{сервис} отпадает. 

Частичная \textit{интеграция} предполагает реализацию взаимодействия и изменения состояния \textit{базы знаний} \textit{Экосистемы OSTIS} на этапах исполнения функции \textit{сервиса}. Степень глубины \textit{интеграции} может различаться, в некоторых случаях \textit{сервис} может обращаться к \textit{базе знаний} для получения дополнительной информации или для записи промежуточных результатов. 

В простейшем случае изменение \textit{базы знаний} \textit{Экосистемы OSTIS} может происходить единожды, после получения результата отработки функции \textit{сервиса}. На основе такого принципа можно выделить специальные \textit{sc-агенты}, использующие сторонний \textit{сервис}. 

Для обеспечения \textit{интеграции} функционального \textit{сервиса} необходимо выполнение следующих минимальных требований:
\begin{textitemize}
    \item спецификация входной конструкции в \textit{базе знаний} системы: определение структуры в \textit{базе знаний} системы, которая будет преобразовываться в формат данных, совместимый с \textit{сервисом};
    \item спецификация выходной конструкции в \textit{базе знаний} системы: определение структуры в \textit{базе знаний} системы, которая будет формироваться из исходной структуры впоследствии преобразования данных \textit{сервиса} в знания;
    \item реализация \textit{sc-агента}, который преобразует конструкцию \textit{базы знаний} в формат, который может быть использован в \textit{сервисе}, а также погружать результаты работы \textit{сервиса} обратно в \textit{базу знаний} системы в соответствии со спецификацией.
\end{textitemize}

Удовлетворение данных требований позволит обеспечить эффективную \textit{интеграцию} функционального \textit{сервиса}, что, в свою очередь, позволит использовать данные и функциональность \textit{сервиса} в различных \textit{ostis-системах}. Задача формирования спецификации рассмотрена в \ref{sec_kb_design_methods}~\nameref{sec_kb_design_methods}.

Обобщенный алгоритм \textit{sc-агента}, использующего сторонний \textit{сервис} для \textit{интеграции} функциональных возможностей, может быть описан следующим образом:

\begin{textitemize}
    \item извлечение из \textit{базы знаний} необходимых структур знаний, соответствующих требованиям функционального \textit{сервиса};
    \item преобразование извлечённых знаний в формат, необходимый для подачи на вход функциональному \textit{сервису};
    \item отправка запроса на функциональный \textit{сервис} и ожидание его ответа;
    \item формирование структур знаний на основе полученных данных от функционального \textit{сервиса};
    \item погружение новых структур знаний в \textit{базу знаний} \textit{интеллектуальной системы}, с целью обеспечения их дальнейшей использования.
\end{textitemize}

Следует учесть, что при \textit{интеграции} функциональных \textit{сервисов}, возможно потребуется проводить дополнительную обработку и преобразование данных, например, для обеспечения их \textit{совместимости} с форматами данных, или для обеспечения безопасности передачи и хранения данных.

Таким образом, внедрение возможности использования стороннего \textit{сервиса} в \textit{Экосистеме OSTIS} предполагает выполнение следующих шагов:
\begin{textitemize}
    \item анализ требований к интегрируемому \textit{сервису}, определение необходимого функционала, форматов входных и выходных данных, и других характеристик сервиса.
    \item разработка спецификации \textit{интеграции}, которая будет определять форматы данных и правила взаимодействия между \textit{базой знаний} \textit{Экосистемы OSTIS} и сторонним \textit{сервисом};
    \item разработка \textit{sc-агента}, который будет обеспечивать взаимодействие между \textit{базой знаний} и сторонним \textit{сервисом} в соответствии со спецификацией \textit{интеграции};
    \item тестирование и отладка;
    \item внедрение в \textit{Экосистему OSTIS}, что позволит использовать возможности интегрированного стороннего \textit{сервиса} в различных \textit{ostis-системах}.
\end{textitemize}

Для подключения \textit{sc-агента} используются различные подходы, один из которых заключается в подключении \textit{sc-агента} в рамках уже существующей, основной \textit{ostis-системы}. С точки зрения масштабируемости реализации такого подхода следует отметить \textit{монолитную архитектуру} получаемой \textit{ostis-системы}, что позволяет упростить процесс внедрения новых \textit{сервисов} и \textit{sc-агентов} в \textit{ostis-систему}.

Преимуществом такого подхода является более простое и удобное внедрение новых \textit{сервисов} и \textit{sc-агентов} в \textit{ostis-систему}, а также упрощение процесса управления зависимостями. Использование реализации ostis-системы с \textit{монолитной архитектурой} может быть применено в случаях, когда функциональный \textit{сервис} не нуждается в частой модификации и обладает достаточно простой структурой. Кроме того, \textit{монолитная архитектура} может быть более удобна в случаях, когда обращение к \textit{сервису} происходит через внутреннюю сеть и требует низкой задержки и высокой производительности. Так могут быть интегрированы и \textit{сервисы} получения знаний из внешних источников (получение прогноза погоды, обработка статистической информации, и так далее), и функциональные \textit{сервисы} (обработка аудиоинформации и погружение результатов обработки в \textit{базу знаний}, получение синтаксического анализа предложения). 

Альтернативным вариантом внедрения является реализация отдельной \textit{ostis-системы}, в рамках которой будет интегрирована функция \textit{сервиса}. Это позволяет перейти к использованию \textit{микросервисной архитектуры}, что характеризуется распределенным взаимодействием \textit{ostis-систем}.

Преимуществами такого подхода является большая \textit{гибкость} и возможность \textit{масштабирования}. Подход характеризуется высокой степенью распределённости, децентрализованности и доступности нового функционала. Система выходит за рамки технических ограничений, функционал может быть распределён на различном аппаратном обеспечении. Полученный функционал может быть использован различными \textit{ostis-системами} в рамках \textit{Экосистемы OSTIS} для достижения своих целей. Минусами такой системы является сложность разработки, а также увеличение временных затрат на общение систем друг с другом по сетевым протоколам. 

\textit{микросервисную архитектуру} ostis-систем предпочтительно использовать в случаях, когда функциональный \textit{сервис} обладает сложной структурой, а также в случаях, когда требуется масштабирование и гибкость всей системы. В качестве примера можно привести \textit{сервис}, который взаимодействует с внешними источниками данных и может быть подвержен частым изменениям.  Примерами \textit{интеграции} на основе \textit{микросервисной архитектуры} могут служить \textit{сервисы} считывания эмоционального состояния пользователя, естественно-языковая обработка, задачи классификации и идентификации, и так далее.

Таким образом, выбор подхода к \textit{интеграции} функциональных \textit{сервисов} зависит от конкретных требований и условий проекта. Использование \textit{Технологии OSTIS} позволяет создавать гибкие и эффективные системы, которые могут быть адаптированы под различные потребности и требования пользователей.
