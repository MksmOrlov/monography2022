\section{Принципы интеграции Экосистемы OSTIS с разнородными сервисами}
{\label{sec_integration_services}} 

\begin{SCn}

    \bigskip
    
    \begin{scnrelfromlist}{ключевое понятие}
        \scnitem{цифровая экосистема}
        \scnitem{база знаний}
        \scnitem{sc-агент}
        \scnitem{ostis-система}
        \scnitem{интеллектуальная система}
    \end{scnrelfromlist}
    
    \bigskip
    
    \begin{scnrelfromlist}{ключевой знак}
        \scnitem{Экосистема OSTIS}
        \scnitem{Технология OSTIS}
    \end{scnrelfromlist}
    
    \bigskip
    
    \begin{scnrelfromlist}{библиографическая ссылка}
        \scnitem{\scncite{...}}
    \end{scnrelfromlist}
    
\end{SCn}

В контексте интеграции \textit{Экосистемы OSTIS} с разнородными сервисами, под "сервисами" понимаются приложения, программы, веб-сервисы и другие информационные системы, которые предоставляют определенный функционал, механизм преобразования данных в соответствии с заданной функцией. Зачастую такое приложение может предоставить программный интерфейс (API --- Application Programming Interface), который можно использовать с определённым форматом входов, которым будут соответствовать определённые форматы выходов.

Под интеграцией \textit{Экосистемы OSTIS} с сервисом следует понимать возможность использовать функционал сервиса для изменения внутреннего состояния базы знаний системы. 

При интеграции сервисов в \textit{цифровых экосистемах} возникают ряд проблем, которые могут затруднять процесс интеграции и уменьшать эффективность экосистемы. Некоторые из этих проблем могут включать в себя:

\begin{textitemize}
    \item различные форматы данных и протоколы обмена, которые могут привести к ошибкам при обмене данными между сервисами, что затрудняет взаимодействие между сервисами;
    \item несовместимость версий приложений, что может привести к конфликтам при обмене данными;
    \item разные уровни безопасности, что может стать причиной утечки конфиденциальной информации;
    \item отсутствие единой точки управления, что затрудняет мониторинг и управление процессами интеграции;
    \item отсутствие механизмов для анализа и управления данными, что затрудняет контроль над процессами обмена данными.
\end{textitemize}

Перечисленные проблемы значительно усложняют разработку самих сервисов и ведёт к значительному увеличению временных и материальных затрат. Для решения этих проблем при интеграции \textit{цифровых экосистем} с различными сервисами и ресурсами используются различные подходы и технологии. Некоторые из них могут включать в себя:

\begin{textitemize}
\item использование стандартных протоколов и форматов обмена данных, таких как XML, JSON и другие, что позволяет сделать обмен данными более надежным и универсальным;
\item разработка единой схемы данных и правил доступа, что позволяет сделать интеграцию более простой и управляемой;
\item реализация механизмов для автоматической обработки ошибок и конфликтов, что позволяет снизить количество ошибок и улучшить надежность экосистемы;
\item использование инструментов и технологий для анализа и управления данными, таких как системы бизнес-аналитики и управления данными, что позволяет контролировать процессы обмена данными и оптимизировать их работу.
\end{textitemize}


В рамках \textit{Экосистемы OSTIS} выделено несколько уровней интеграции, которые позволяют взаимодействовать с различными информационными ресурсами и сервисами. 

Полная интеграция предполагает исполнение функции сервиса на платформонезависимом уровне, где вся программа исполняется в самой базе знаний. То есть, задача интеграции такого сервиса сводится к выделению алогиртма обработки графовой конструкции и его реализации в рамках \textit{базы знаний} системы. В результате такой полной интеграции надобность использовать сторонний сервис отпадает. 

Частичная интеграция предполагает реализацию взаимодействия и изменения состояния \textit{базы знаний} системы на этапах исполнения функции сервиса. Степень глубины интеграции может различаться. В некоторых случаях сервис может обращаться к базе знаний для получения дополнительной информации или для записи промежуточных результатов. 

В простейшем случае изменение \textit{базы знаний} может происходить единожды, после получения результата отработки функции сервиса. На основе такого принципа можно выделить специальные \textit{sc-агенты}, использующего сторонний сервис. 

Для обеспечения интеграции функционального сервиса необходимо выполнение следующих минимальных требований:
\begin{textitemize}
    \item спецификация входной конструкции в \textit{базе знаний} системы, определение структуры в \textit{базе знаний} системы, которая будет преобразовываться в формат данных, совместимый с сервисом;
    \item спецификация выходной конструкции в \textit{базе знаний} системы, определение структуры в \textit{базе знаний} системы, которая будет формироваться из исходной структуры впоследствии преобразования данных сервиса в знания;
    \item реализация \textit{sc-агента}, который преобразует конструкцию базы знаний в формат, который может быть использован в сервисе, а также погружать результаты работы сервиса обратно в \textit{базу знаний} системы в соответствии со спецификацией.
\end{textitemize}

Удовлетворение данных требований позволит обеспечить эффективную интеграцию функционального сервиса, что, в свою очередь, позволит использовать данные и функциональность сервиса в различных \textit{ostis-системах}. Задача формирования спецификации рассмотрена в \ref{sec_kb_design_methods}~\nameref{sec_kb_design_methods}.

Обобщенный алгоритм \textit{sc-агента}, использующего сторонний сервис для интеграции функциональных возможностей, может быть описан следующим образом:

\begin{textitemize}
    \item извлечение из \textit{базы знаний} необходимых структур знаний, соответствующих требованиям функционального сервиса;
    \item преобразование извлечённых знаний в формат, необходимый для подачи на вход функциональному сервису;
    \item отправка запроса на функциональный сервис и ожидание ответа;
    \item формирование структур знаний на основе полученных данных от функционального сервиса;
    \item погружение новых структур знаний в \textit{базу знаний} \textit{интеллектуальной системы}, с целью обеспечения их дальнейшей использования.
\end{textitemize}

Следует учесть, что при интеграции функциональных сервисов, возможно потребуется проводить дополнительную обработку и преобразование данных, например, для обеспечения их совместимости с форматами данных, или для обеспечения безопасности передачи и хранения данных.

Таким образом, внедрение возможности использования стороннего сервиса в \textit{Экосистеме OSTIS} предполагает выполнение следующих шагов:
\begin{textitemize}
    \item анализ требований к интегрируемому сервису, определение необходимого функционала, форматов входных и выходных данных, и других характеристик сервиса.
    \item разработка спецификации интеграции, которая будет определять форматы данных и правила взаимодействия между базой знаний системы и сторонним сервисом;
    \item разработка \textit{sc-агента}, который будет обеспечивать взаимодействие между базой знаний системы и сторонним сервисом в соответствии со спецификацией интеграции;
    \item тестирование и отладка;
    \item внедрение в \textit{Экосистему OSTIS}, что позволит использовать возможности интегрированного стороннего сервиса в различных \textit{ostis-системах}.
\end{textitemize}

Для подключения \textit{sc-агента} можно использовать различные подходы. Один из таких подходов заключается в подключении \textit{sc-агента} в рамках уже существующей, основной \textit{ostis-системы}. При этом используется монолитная архитектура, что позволяет упростить процесс внедрения новых сервисов и \textit{sc-агентов} в \textit{ostis-систему}.

Преимуществом такого подхода является более простое и удобное внедрение новых сервисов и агентов в систему, а также упрощение процесса управления зависимостями. Монолитная архитектура может быть использована в случаях, когда функциональный сервис не нуждается в частой модификации и обладает достаточно простой структурой. Кроме того, монолитная архитектура может быть более удобна в случаях, когда обращение к сервису происходит через внутреннюю сеть и требует низкой задержки и высокой производительности. Так могут быть интегрированы и сервисы получения знаний из внешних источников (получение прогноза погоды, обработка статистической информации, и так далее), и функциональные сервисы (обработка аудиоинформации и погружение результатов обработки в \textit{базу знаний}, получение синтаксического анализа предложения). 

Альтернативным вариантом внедрения является реализация отдельной \textit{ostis-системы}, в рамках которой будет интегрирована функция сервиса. Это позволяет перейти к использованию микросервисной архитектуры, что характеризуется распределенным взаимодействием \textit{ostis-систем}.

Преимуществами такого подхода является большая гибкость и возможность масштабирования, а также повышение производительности и сокращение времени отклика в системе. Подход характеризуется высокой степенью распределённости, децентрализованности и доступности нового функционала. Система выходит за рамки технических ограничений, функционал может быть распределён на различном аппаратном обеспечении. Полученный функционал может быть использован различными \textit{ostis-системами} в рамках \textit{Экосистемы OSTIS} для достижения своих целей. Минусами такой системы является сложность разработки, а также увеличение временных затрат на общение систем друг с другом по сетевым протоколам. 

Микросервисная архитектура более удобна в случаях, когда функциональный сервис обладает сложной структурой, а также в случаях, когда требуется масштабирование и гибкость системы. Например, если сервис взаимодействует с внешними источниками данных и может быть подвержен изменениям, то микросервисная архитектура может быть более предпочтительной.  Примерами интеграции могут служить сервисы считывания эмоционального состояния пользователя, или же помощи принятия решения на основе актуальных знаний.

Таким образом, выбор подхода к интеграции функциональных сервисов зависит от конкретных требований и условий проекта. При этом использование \textit{Технологии OSTIS} позволяет создавать гибкие и эффективные системы, которые могут быть адаптированы под различные потребности и требования пользователей.
