\section{Семантически совместимые интеллектуальные ostis-порталы научных знаний}
{\label{sec_ostis_scientific_portal}} 

\begin{SCn}

\bigskip

\begin{scnrelfromlist}{ключевое понятие}
    \scnitem{...}
\end{scnrelfromlist}

\bigskip

\begin{scnrelfromlist}{ключевое знание}
    \scnitem{...}
\end{scnrelfromlist}

\bigskip

\begin{scnrelfromlist}{библиографическая ссылка}
    \scnitem{\scncite{...}}
\end{scnrelfromlist}

\end{SCn}


Без Общей формальной теории интеллектуальных систем невозможно построить набор методов и средств, обеспечивающий комплексную поддержку разработки интеллектуальных компьютерных систем различного назначения и с различным набором навыков, которыми могут обладать интеллектуальные компьютерные системы, но необязательно каждая из них. 
При этом важно не просто построить Общую теорию интеллектуальных систем и довести ее до строгого формального уровня, но также довести представление такой формальной теории до уровня базы знаний соответствующего портала научных знаний.

Целями интеллектуального портала научных знаний являются:
\begin{textitemize}
    \item ускорение погружения каждого человека в новые для него научные области при постоянном сохранении общей целостной картины Мира (образовательная цель);
    \item фиксация в систематизированном виде новых научных результатов так, чтобы все основные связи новых результатов с известными были четко обозначены;
    \item автоматизация координации работ по рецензированию новых результатов;
    \item автоматизация анализа текущего состояния базы знаний.
\end{textitemize}

Создание интеллектуальных порталов научных знаний, обеспечивающих повышение темпов интеграции и согласования различных точек зрения, – это способ существенного повышения темпов эволюции научно-технической деятельности.
Совместимые порталы научных знаний, реализованные в виде ostis-систем, входящих в Экосистему OSTIS, являются основой новых принципов организации научной деятельности, в которой результатами этой деятельности являются не статьи, монографии, отчеты и другие научно-технические документы, а фрагменты глобальной базы знаний, разработчиками которых являются свободно формируемые научные коллективы, состоящие из специалистов в соответствующих научных дисциплинах. 
С помощью порталов научных знаний осуществляется как координация процесса рецензирования новой научно-технической информации, поступающей от научных работников в базы знаний этих порталов, так и процесс согласования различных точек зрения ученых (в частности, введению и семантической корректировке понятий, а также введению и корректировке терминов, соответствующих различным сущностям).

Реализация семейства семантически совместимых порталов научных знаний в виде совместимых ostis-систем, входящих в состав Экосистемы OSTIS, предполагает разработку иерархической системы семантически согласованных формальных онтологий, соответствующих различным научно-техническим дисциплинам, с четко заданным наследованием свойств описываемых сущностей и с четко заданными междисциплинарными связями, которые описываются связями между соответствующими формальными онтологиями и специфицируемыми ими предметными областями.

Примером портала научных знаний, построенного в виде ostis-системы является Метасистема OSTIS, содержащая все известные на текущий момент знания и навыки, входящие в состав Технологии OSTIS.
