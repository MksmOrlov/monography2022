\section{Семантически совместимые интеллектуальные корпоративные ostis-системы различного назначения}
{\label{sec_corporate_ostis_system}} 

\begin{SCn}

\bigskip

\begin{scnrelfromlist}{ключевое понятие}
    \scnitem{...}
\end{scnrelfromlist}

\bigskip

\begin{scnrelfromlist}{ключевое знание}
    \scnitem{...}
\end{scnrelfromlist}

\bigskip

\begin{scnrelfromlist}{библиографическая ссылка}
    \scnitem{\scncite{...}}
\end{scnrelfromlist}

\end{SCn}


Корпоративная ostis-система позволяет отслеживать, анализировать и постепенно автоматизировать все процессы обработки данных в рамках ostis-сообщества. 
Такая система действует по следующим принципам:
\begin{textitemize}
    \item интеллектуальные подсистемы (агенты) упорядочивают структуру данных таким образом, что актуальная информация всегда доступна, а устаревшая информация автоматически архивируется или удаляется в соответствии с законами о хранении и защите данных;
    \item запросы к системе выполняются в виде простых инструкций, система помогает менеджерам вводить необходимую информацию, осуществляет частичную или полную автоматизацию обновления информации из баз данных, доступных через Интернет;
    \item интеллектуальные подсистемы выполняют структуризацию и классификацию документов и информации для принятия быстрых и правильных решений, автоматически обрабатывает документы и доступные базы данных для отбора ключевой информации, необходимой в данный момент и в будущем;
    \item существующее системное окружение на предприятии может быть легко подключено к системе через открытые интерфейсы, вся информация остается доступной;
    \item все ключевые системы данных синхронизируются с основной системой, данные постоянно сравниваются друг с другом, чтобы избежать потерь;
    \item вся информация доступна в базе знаний, которая является источником данных для рабочих процессов, отчетности и комплексных проверок;
\end{textitemize}

Таким образом, предлагаемая платформа позволяет представить всю ифнормации об ostis-сообществе единым целостным образом. 
Достоинствами внедрения предложенной системы являются:
\begin{textitemize}
    \item помощь сбора и оценки информации без преднамеренных искажений или ошибок, связанных с человеческим фактором;
    \item предоставление возможности полного контроля своих данных;
    \item система предоставляет только высококачественные, достоверные и актуальные данные;
    \item цифровое представление всех процессов сообщества обеспечивает интегрированную обработку информации внутри сообщества, что дает полную прозрачность управления, облегчает доступ ко всей информации и ее анализ;
    \item Благодаря поддержке интеллектуальных подсистем все необходимые данные из документов, процессов и внешних источников могут быть извлечены, структурированы и грамотно оценены.
\end{textitemize}

Корпоративные ostis-системы могут быть применены в различных областях: медицина и здравоохранение, образовательная деятельность широкого профиля, страховой бизнес, промышленная деятельность, административная деятельность, недвижимость, транспорт и т.д.
