\section{Семантически совместимые интеллектуальные корпоративные ostis-системы различного назначения}
{\label{sec_corporate_ostis_system}} 

\begin{SCn}

\begin{scnrelfromlist}{ключевой знак}
    \scnitem{Экосистема OSTIS}
\end{scnrelfromlist}

\begin{scnrelfromlist}{ключевое понятие}
    \scnitem{корпоративная система}
    \scnitem{корпоративная ostis-систем}
\end{scnrelfromlist}

\bigskip

\begin{scnrelfromlist}{библиографическая ссылка}
    \scnitem{\scncite{...}}
\end{scnrelfromlist}

\end{SCn}

\textit{корпоративные системы} представляют собой программные решения, предназначенные для автоматизации бизнес-процессов и управления ресурсами и данными внутри организации. Они могут включать в себя различные подсистемы, такие как управление отношениями с клиентами, управление контентом, управление проектами, управление ресурсами предприятия, управление документами и многое другое.

Роль \textit{корпоративных систем} в современных организациях заключается в обеспечении эффективного управления бизнес-процессами и ресурсами, повышении производительности и качества работы, а также обеспечении прозрачности и оперативности принятия решений на основе актуальных данных.

\textit{корпоративные системы} могут использоваться для следующих целей:

\begin{textitemize}
    \item автоматизация многих рутинных задач, таких как обработка заказов, управление складом, учет финансовых операций и так далее. Это позволяет сократить время на выполнение задач и уменьшить количество ошибок.
    \item сбор, хранение и обработка данных о бизнес-процессах и ресурсах организации. Это позволяет увеличить точность и оперативность принятия решений, а также обеспечить прозрачность в управлении организацией.
    \item эффективное управление ресурсами организации, такими как финансы, трудовые ресурсы, материальные и технические ресурсы и так далее. Это позволяет сократить затраты на управление ресурсами и повысить эффективность их использования.
    \item управление отношениями с клиентами, автоматизация процессов продаж и обслуживания, а также анализ данных о клиентах. Это позволяет повысить удовлетворенность клиентов и увеличить объемы продаж.
    \item управление проектами, планирование и отслеживание выполнения работ, управление ресурсами и расписание проектов. Это позволяет повысить эффективность выполнения проектов, уменьшить сроки выполнения работ и снизить затраты на проекты.
    \item управление документами, контроль версиями, автоматизация процессов редактирования и утверждения документов. Это позволяет повысить эффективность работы с документами и обеспечить безопасность их хранения и передачи.
\end{textitemize}

Хотя \textit{корпоративные системы} могут принести значительные выгоды для организаций, они также могут столкнуться с рядом проблем, связанных с их внедрением и эксплуатацией. Ниже перечислены некоторые из этих проблем:

\begin{textitemize}
    \item Внедрение корпоративных систем может быть дорогостоящим и трудоемким процессом, который требует значительных ресурсов и экспертизы. Кроме того, многие системы могут потребовать изменения бизнес-процессов и требовать адаптации культуры организации.
    \item Корпоративные системы могут столкнуться с проблемами совместимости с другими системами, используемыми в организации. Это может привести к проблемам с обменом данными и снижению эффективности работы.
    \item Корпоративные системы могут стать мишенью для кибератак, поэтому важно обеспечить безопасность хранения и передачи данных, используемых в системах.
    \item Корпоративные системы могут потребовать значительных затрат на обслуживание и поддержку, включая установку обновлений, устранение ошибок и техническую поддержку.
    \item Внедрение новых корпоративных систем может потребовать обучения персонала, что может быть трудоемким и затратным процессом.
    \item Внедрение корпоративных систем может потребовать изменения бизнес-процессов, что может быть сложным и вызвать сопротивление со стороны сотрудников.
\end{textitemize}

Для создания семантически совместимых интеллектуальных \textit{корпоративных систем} необходимо обеспечить высокую степень гибкости, масштабируемости, автоматизации и интеграции. Это позволит организациям более эффективно управлять ресурсами и данными и повысить их конкурентоспособность на рынке. Для достижения этих целей необходимо использовать современные технологии, такие как аналитика данных, машинное обучение, искусственный интеллект и технологии распределенных вычислений. Кроме того, необходимо учитывать особенности организации и ее бизнес-процессов, чтобы обеспечить максимальную эффективность использования системы. 

Для решения задачи формирования корпоративной системы целесообразно применение \textit{Технологии OSTIS}. \textit{корпоративная ostis-система} позволяет отслеживать, анализировать и постепенно автоматизировать все процессы обработки данных в рамках ostis-сообщества. 
Такая система действует по следующим принципам:
\begin{textitemize}
    \item интеллектуальные подсистемы (агенты) упорядочивают структуру данных таким образом, что актуальная информация всегда доступна, а устаревшая информация автоматически архивируется или удаляется в соответствии с законами о хранении и защите данных в режиме реального времени;
    \item запросы к системе выполняются в виде простых инструкций, система помогает менеджерам вводить необходимую информацию, осуществляет частичную или полную автоматизацию обновления информации из баз данных, доступных через Интернет;
    \item интеллектуальные подсистемы выполняют структуризацию и классификацию документов и информации для принятия быстрых и правильных решений, автоматически обрабатывает документы и доступные базы данных для отбора ключевой информации, необходимой в данный момент и в будущем;
    \item существующее системное окружение на предприятии может быть легко подключено к системе через открытые интерфейсы, вся информация остается доступной;
    \item все ключевые системы данных синхронизируются с основной системой, данные постоянно сравниваются друг с другом, чтобы избежать потерь;
    \item вся информация доступна в базе знаний, которая является источником данных для рабочих процессов, отчетности и комплексных проверок;
\end{textitemize}

Достоинствами внедрения предложенной системы являются:
\begin{textitemize}
    \item помощь сбора и оценки информации без преднамеренных искажений или ошибок, связанных с человеческим фактором;
    \item предоставление возможности полного контроля своих данных;
    \item система предоставляет только высококачественные, достоверные и актуальные данные;
    \item цифровое представление всех процессов сообщества обеспечивает интегрированную обработку информации внутри сообщества, что дает полную прозрачность управления, облегчает доступ ко всей информации и ее анализ;
    \item благодаря поддержке интеллектуальных подсистем все необходимые данные из документов, процессов и внешних источников могут быть извлечены, структурированы и грамотно оценены.
\end{textitemize}

С точки зрения структуры \textit{Экосистемы OSTIS}, \textit{корпоративная ostis-система} осуществляет координацию и эволюцию деятельности некоторых групп \textit{ostis-систем} и их пользователей. 
Основная цель \textit{корпоративных ostis-систем} – локализовать \textit{базы знаний} указанных групп \textit{ostis-систем}, перевести их из статуса виртуальных в статус реальных и автоматизировать их эволюцию.


\textit{корпоративные ostis-системы} могут быть применены в различных областях: медицина и здравоохранение, образовательная деятельность широкого профиля, страховой бизнес, промышленная деятельность, административная деятельность, недвижимость, транспорт и так далее.
