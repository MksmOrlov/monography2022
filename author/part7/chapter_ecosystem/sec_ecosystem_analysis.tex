\section{Анализ существующих подходов к созданию цифровых экосистем}
{\label{sec_ecosystem_automation_principles}} 

\begin{SCn}

\bigskip

\begin{scnrelfromlist}{ключевое понятие}
    \scnitem{...}
\end{scnrelfromlist}

\bigskip

\begin{scnrelfromlist}{ключевое знание}
    \scnitem{...}
\end{scnrelfromlist}

\bigskip

\begin{scnrelfromlist}{библиографическая ссылка}
    \scnitem{\scncite{...}}
\end{scnrelfromlist}

\end{SCn}    

% Добавить описание, про что вообще глава, сделать введение более плавным.  
Понятие архетипа сети сводится к объединению множества автономных объектов друг с другом. Эти объекты сильно связны друг с другом, и при этом не имеют какого-либо центра. Таким образом, они образуют децентрализованную сеть, в которой отсутствует единый центр управления.
% \cite{Briscoe_agents_evolving_2012}

Можно выделить следующие характеристики такой системы:
\begin{textitemize}
    \item отсутствие или слабовыраженность централизованного управления;
    \item автономная природа участников, объектов такой сети;
    \item сильная связность участников такой сети друг с другом;
    \item влияние участников такой сети друг на друга нелинейно и достаточно сложно.
\end{textitemize}

У таких распределённых искусственных систем выделяются как преимущества (высокий уровень адаптивности, устойчивости, связности), так и недостатки (неоптимальность, неуправляемость, непредсказуемость поведения). Наиболее подходящим примером реализованной технологии на основе концепции сети является интернет.

Архетип сети удобно использовать для отображения сложных процессов, взаимозависимость компонентов, экономических, социальных, экологических процессов, процессов коммуникации. В таких процессах нет начала или конца, всё является центром. Сеть является единственной топологией, способной к безграничному расширению или самостоятельному обучению, остальные топологии имеют собственные ограничения. "The Atom is the icon of 20th century science. The symbol of science for the next century is the dynamical Net"{}.
% \cite{Kelly_out_of_control_1995}

Концепция экосистемы стала популярным способом описания взаимодействия самостоятельно действующих систем во внешней среде. 
% \cite{Boley_ecosystem_principles_2007}
Экосистемы имеют две отличительные характеристики по сравнению с другими концепциями сотрудничества: взаимодополняемость и взаимозависимость присутствуют одновременно, и система не полностью иерархически контролируется.
% \cite{Masahary_ecosystem_concept_2019}

При традиционных подходах к решению проблемы формирования экосистемы возникают проблемы, связанные с низким уровнем интероперабельности таких систем.
% \cite{li2012problems}
Зачастую каждая из систем будет иметь свой специализированный программный интерфейс и формат данных для общения с ней, что ведёт к дополнительным расходам на устранение недостатков таких проблем. Поддержка жизненного цикла, модификация уже существующих систем может накладывать дополнительные временные и ресурсные затраты. 

Ключевым направлением повышения уровня интеллекта индивидуальных интеллектуальных кибернетических систем --- это переход от абсолютно независимых друг от друга индивидуальных интеллектуальных кибернетических систем к их универсальным сообществам, т.е. к многоагентным системам, самостоятельными агентнами которых являются указанные индивидуальные интеллектуальные кибернетические системы.
В рамках таких систем обеспечивается возможность коммуникации каждого агента с каждым, а также обеспечивается возможность формирования специализированных коллективов для коллективного решения сложных коллективных задач.
Реализация указанного универсального сообщества интероперабельных интеллектуальных кибернетических систем осуществляется в виде Глобальной Экосистемы OSTIS. 