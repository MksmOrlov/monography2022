\section{Анализ проблем при создании цифровых экосистем}
{\label{sec_ecosystem_analysis}} 

\begin{SCn}

\begin{scnrelfromlist}{ключевое понятие}
	\scnitem{сеть}
	\scnitem{распределенная система}
	\scnitem{цифровая экосистема}
\end{scnrelfromlist}


\begin{scnrelfromlist}{библиографическая ссылка}
    \scnitem{\scncite{...}}
\end{scnrelfromlist}

\end{SCn}    

\textit{сеть} может быть определена как система, состоящая из различных элементов, которые взаимодействуют друг с другом для передачи данных и ресурсов. Архетип \textit{сети} удобно использовать для отображения сложных процессов, взаимозависимость компонентов, экономических, социальных, экологических процессов, процессов коммуникации. В таких процессах нет начала или конца, всё является центром. \textit{сеть} является единственной топологией, способной к безграничному расширению или самостоятельному обучению, остальные топологии имеют собственные ограничения.

В случае, если элементы \textit{сети} сильно связны друг с другом, и при этом не имеют какого-либо центра, они образуют децентрализованную \textit{сеть}, в которой отсутствует единый центр управления (\scncite{Briscoe_agents_evolving_2012}). Децентрализация представляет собой распределение контроля и ресурсов между узлами \textit{сети}, что делает систему более гибкой и устойчивой к отказам. \textit{распределенная система} - это система, которая состоит из множества автономных узлов, которые взаимодействуют друг с другом и выполняют задачи вместе.
Можно выделить следующие характеристики \textit{распределённых систем}:
\begin{textitemize}
    \item отсутствие или слабовыраженность централизованного управления;
    \item автономная природа участников, элементов такой \textit{сети};
    \item сильная связность участников такой \textit{сети} друг с другом;
    \item влияние участников такой \textit{сети} друг на друга нелинейно и достаточно сложно.
\end{textitemize}

У таких \textit{распределённых систем} выделяются как преимущества (высокий уровень адаптивности, устойчивости, связности), так и недостатки (неоптимальность, неуправляемость, непредсказуемость поведения). Наиболее подходящим примером реализованной технологии на основе концепции \textit{сети} является интернет.

Концепция экосистемы получила все большее значение как подход к описанию взаимодействия различных самостоятельно действующих систем во внешней среде (\scncite{Boley_ecosystem_principles_2007}). Экосистемы отличаются от других концепций сотрудничества тем, что они объединяют различные самостоятельно действующие системы, обладающие взаимодополняющими и взаимозависимыми характеристиками, в рамках единой среды обитания. В отличие от полностью иерархически контролируемых систем, экосистемы представляют собой децентрализованные структуры, которые обеспечивают более гибкое и устойчивое управление (\scncite{Masahary_ecosystem_concept_2019}).

\textit{цифровая экосистема} может быть определена как совокупность цифровых продуктов и сервисов, которые взаимодействуют друг с другом и с внешней средой, образуя единую среду обитания. \textit{распределенные системы} часто используются как ключевой элемент в создании \textit{цифровых экосистем}, так как они обеспечивают гибкость, масштабируемость и отказоустойчивость.

При традиционных подходах к решению проблемы формирования \textit{цифровой экосистемы} возникают проблемы, связанные с низким уровнем \textit{интероперабельности} таких систем (\scncite{li2012problems}). Традиционные подходы к решению данной проблемы зачастую неэффективны, поскольку каждая из систем имеет свой специализированный программный интерфейс и формат данных для взаимодействия. Это приводит к дополнительным расходам на устранение недостатков таких проблем. Поддержка жизненного цикла и модификация уже существующих систем может также потребовать дополнительных временных и ресурсных затрат.

Использование современных подходов к формированию \textit{цифровой экосистемы}, таких как открытые стандарты и протоколы взаимодействия, может значительно упростить задачу обеспечения \textit{интероперабельности} между различными системами. Это позволяет повысить эффективность и экономическую целесообразность проектов цифровой трансформации, снизить временные и финансовые затраты на разработку и поддержку \textit{цифровой экосистемы}.
