\section{Персональные ostis-ассистенты пользователей}
{\label{sec_ostis_assistant}} 

\begin{SCn}

\begin{scnrelfromlist}{ключевое понятие}
    \scnitem{персональный ассистент}
    \scnitem{персональный ostis-ассистент}
\end{scnrelfromlist}

\bigskip

\begin{scnrelfromlist}{библиографическая ссылка}
    \scnitem{\scncite{Meurisch2017}}
    \scnitem{\scncite{Meurisch2020}}
    \scnitem{\scncite{Jeni2022}}
    \scnitem{\scncite{Akbar2022}}
\end{scnrelfromlist}

\end{SCn}

Общество должно обеспечивать персональную поддержку каждому человеку, учитывая его индивидуальные особенности, с целью достижения следующих целей:
\begin{textitemize}
    \item максимального уровня физического здоровья, активности и долголетия;
    \item максимального уровня физического комфорта, личного пространства и материального благосостояния;
    \item максимального уровня социального комфорта и защиты прав и свобод.
\end{textitemize}

Для этого должен осуществляться:
\begin{textitemize}
    \item персональный мониторинг каждой личности по всем направлениям;
    \item диагностика и устранение нежелательных отклонений;
    \item оказание своевременной персональной помощи в уточнении направлений дальнейшей эволюции каждой личности.
\end{textitemize}

Необходимо перейти от оказания услуг в решении различных проблем по инициативе самих лиц, столкнувшихся с этими проблемами, к своевременному обнаружению возможности возникновения этих проблем и к соответствующей профилактике. 
Это возможно только при наличии четкой системной организации персонального мониторинга. 

Цифровые \textit{персональные ассистенты} – это программы, основанные на технологиях искусственного интеллекта и машинного обучения, которые помогают пользователям в выполнении повседневных задач, таких как составление расписания, управление контактами, поиск информации, напоминание о важных событиях и так далее (см. \scncite{Meurisch2017}, \scncite{Meurisch2020}, \scncite{Jeni2022}, \scncite{Akbar2022}).

\textit{Персональный ассистент} должен учитывать, что роли пользователя в обществе могут меняться, расширяться, также как и его интересы и цели. 
При этом, все \textit{персональные ассистенты} должны быть семантически совместимыми с целью понимания друг друга, а также обладать способностью самостоятельно взаимодействовать в рамках различных \textit{корпоративных систем}, представляя интересы своих пользователей.

Одной из основных проблем, связанных с реализацией цифровых \textit{персональных ассистентов}, является необходимость точного понимания запросов и задач, поступающих от пользователя. Это может быть вызвано различными факторами, такими как нечеткость и неоднозначность формулировок, использование аббревиатур и сленга, а также многозначность некоторых слов.

Пользователь не обязан знать множество сервисов, из которых он должен выбирать подходящий ему функционал. Комплекс семантически совместимых сервисов должен располагаться "за кадром"{}. Следовательно, все используемые информационные ресурсы и сервисы должны быть семантически совместимы. Выбор подходящего для пользователя ресурса или сервиса должен производить его \textit{персональный ассистент}.

Таким образом, при реализации цифровых \textit{персональных ассистентов} необходимо обеспечить их масштабируемость и адаптивность к потребностям пользователей. Это означает, что система должна быть способна автоматически адаптироваться к изменениям в поведении пользователя, учитывая его предпочтения, особенности работы и другие факторы.

\textit{Технология OSTIS} позволяет создавать семантически совместимые системы, которые способны обрабатывать запросы и задачи пользователей, учитывая их контекст и смысл. Это достигается за счет использования семантических сетей, которые позволяют описывать знания и связи между ними. Кроме того, \textit{технология OSTIS} обеспечивает масштабируемость и гибкость системы, что позволяет ей адаптироваться к изменениям в поведении пользователей и изменениям в их потребностях.

\textit{Персональный ostis-ассистент} есть \textit{ostis-система}, являющаяся \textit{персональным ассистентом} пользователя в рамках \textit{Экосистемы OSTIS}.
Такая система предоставляет возможности:
\begin{textitemize}
    \item анализа деятельности пользователя и формирования рекомендаций по ее оптимизации;
    \item адаптации под настроение пользователя, его личностные качества, общую окружающую обстановку, задачи, которые чаще всего решает пользователь;
    \item перманентного обучения самого ассистента в процессе решения новых задач, при этом обучаемость потенциально не ограничена;
    \item вести диалог с пользователем на естественном языке, в том числе в речевой форме;
    \item отвечать на вопросы различных классов, при этом если системе что-то не понятно, то она сама может задавать встречные вопросы;
    \item автономного получения информации от всей окружающей среды, а не только от пользователя (в текстовой или речевой форме).
\end{textitemize}

При этом система может как анализировать доступные информационные источники (например, в интернете), так и анализировать окружающий ее физический мир, например, окружающие предметы или внешний вид пользователя.

Достоинства \textit{персонального ostis-ассистента}:
\begin{textitemize}
    \item пользователю нет необходимости хранить разную информацию в разной форме в разных местах, вся информация хранится в единой базе знаний компактно и без дублирований;
    \item благодаря неограниченной обучаемости ассистенты могут потенциально автоматизировать практически любую деятельность, а не только самую рутинную;
    \item благодаря базе знаний, ее структуризации и средствам поиска информации в базе знаний пользователь может получить более точную информацию более быстро.
\end{textitemize}

\textit{Персональные ассистенты} имеют самое различное назначение и могут быть использованы для самых различных категорий пользователей (пациент, юридическое обслуживание, административное обслуживание, покупатель, потребитель различных услуг). \textit{Персональный ostis-ассистент} может использовать знания и данные, хранящиеся в других \textit{ostis-системах}, таких как \textit{корпоративные ostis-системы}, чтобы предоставлять пользователю более полную и актуальную информацию. Это может быть особенно полезно для пользователей, которые работают с большим количеством данных и информации. \textit{Персональный ostis-ассистент} автоматически интегрируется с другими \textit{ostis-системами}, что позволяет ему более эффективно работать с данными и информацией. Он может использовать технологии машинного обучения и искусственного интеллекта для адаптации к поведению пользователя и улучшения его производительности и эффективности. \textit{Персональный ostis-ассистент} может быть создан и настроен с учетом конкретных потребностей организации и ее процессов, что может привести к значительным экономическим и производственным преимуществам.

Таким образом, \textit{персональные ostis-ассистенты} обладают рядом преимуществ по сравнению с другими реализациями цифровых \textit{персональных ассистентов}, таких как более точное понимание запросов и задач пользователей, доступ к актуальным данным и информации, автоматическая интеграция с другими \textit{ostis-системами} в рамках \textit{Экосистемы OSTIS} и адаптация к потребностям организации и ее процессов.
