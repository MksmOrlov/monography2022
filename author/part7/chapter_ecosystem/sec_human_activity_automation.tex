\section{Автоматизация человеческой деятельности в области Искусственного интеллекта, осуществляемая в рамках Экосистемы OSTIS}
{\label{sec_human_activity_automation}} 

\begin{SCn}

\bigskip

\begin{scnrelfromlist}{ключевое понятие}
    \scnitem{...}
\end{scnrelfromlist}

\bigskip

\begin{scnrelfromlist}{ключевое знание}
    \scnitem{...}
\end{scnrelfromlist}

\bigskip

\begin{scnrelfromlist}{библиографическая ссылка}
    \scnitem{\scncite{...}}
\end{scnrelfromlist}

\end{SCn}

Экосистема OSTIS представляет собой саморазвивающуюся сеть ostis-систем, которая обеспечивает комплексную автоматизацию всевозможных видов и областей человеческой деятельности. 
Особое место среди ostis-систем, входящих в состав Экосистемы OSTIS, занимают корпоративные ostis-системы, через которое осуществляется координация и эволюция деятельности некоторых групп ostis-систем и их пользователей. 
Основная цель корпоративных ostis-систем – локализовать базы знаний указанных групп ostis-систем, перевести их из статуса виртуальных в статус реальных и автоматизировать их эволюцию.

Экосистема OSTIS является следующим этапом развития человеческого общества, обеспечивающий существенное повышение уровня общественного (коллективного) интеллекта путем преобразования человеческого общества в экосистему, состоящую из людей и семантически совместимых интеллектуальных систем. 
Экосистема OSTIS --- предлагаемый подход к реализации smart-общества или Общества 5.0, построенного на основе Технологии OSTIS.

Сверхзадачей Экосистемы OSTIS является не просто комплексная автоматизация всех видов человеческой деятельности (разумеется, только тех видов деятельности, автоматизация которых целесообразна), но и существенное повышение уровня интеллекта различных человеческих (точнее человеко-машинных) сообществ и всего человеческого общества в целом.