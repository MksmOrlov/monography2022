\chapauthor{Шункевич Д.В.}
\chapter{Универсальная модель интерпретатора внутренних агентов решателя задач интеллектуальной компьютерной системы нового поколения}
\chapauthortoc{Шункевич Д.В.}
\label{chapter_interpreter}

\abstract{Аннотация к главе.}

\section{Методы и средства реализации ostis-систем}

В общем случае разработка любой искусственной системы, в частности, \textit{интеллектуальной компьютерной системы}, предполагает выполнение двух этапов:
\begin{itemize}
\item этапа проектирования, то есть построения формальной модели системы, достаточной для понимания принципов ее устройства и выполнения последующего этапа ее реализации. Результатом данного этапа является некоторая формальная модель системы, иначе называемая \textit{проектом};
\item этапа реализации, то есть непосредственно воплощения разработанной модели с использованием конкретных средств (инструментов, материалов, комплектующих и т.д.). В случае компьютерных систем выполнение данного этапа обычно предполагает выбор конкретных языков программирования, библиотек, сторонних средств, таких как СУБД и различные сервисы и т.д., а также собственно программирование и отладку системы с использованием выбранных средств.
\end{itemize}

Для каждого из указанных этапов могут существовать свои методики, а также средства автоматизации соответствующих процессов.

Рассмотрим предлагаемый подход к организации реализации ostis-систем. Одним из ключевых принципов предлагаемой нами Технологии OSTIS является обеспечение платформенной независимости ostis-систем, то есть строгое разделение логико-семантическая модели кибернетической системы (\textit{sc-модели кибернетической системы}) и платформы интерпретации sc-моделей кибернетических системы (\textit{ostis-платформы}). Преимущества такого строгого разделения достаточно очевидны:
\begin{itemize}
	\item Перенос ostis-системы с одной платформы на другую (например более новую и эффективную или ориентированную на определенный класс устройств) выполняется с минимальными накладными расходами (в идеальном случае -- вообще сводится просто к загрузке \textit{sc-модели кибернетической системы} на платформу);
	\item Компоненты ostis-систем становятся универсальными, то есть могут использоваться в любых ostis-системах, где их использование является целесообразным;
	\item Развитие платформы и развитие sc-моделей систем может осуществляться параллельно и независимо друг от друга, в общем случае отдельными независимыми коллективами разработчиков по своим собственным правилам и методикам.
\end{itemize}

Рассмотрим более детально понятие \textit{логико-семантической модели кибернетической системы}.

\begin{SCn}
\scnheader{логико-семантическая модель кибернетической системы}
\scnidtf{формальная модель (формальное описание) функционирования кибернетической системы, состоящая из (1) формальной модели информации, хранимой в памяти кибернетической системы и (2) формальной модели коллектива агентов, осуществляющих обработку указанной информации.}
\scnsuperset{sc-модель кибернетической системы}
\begin{scnindent}
	\scnidtf{логико-семантическая модель кибернетической системы, представленная в SC-коде}
	\scnidtf{логико-семантическая модель ostis-системы, которая, в частности, может быть функционально эквивалентной моделью какой-либо кибернетической системы, не являющейся ostis-системой}
\end{scnindent}

\scnheader{кибернетическая система}
\scnsuperset{компьютерная система}
\begin{scnindent}
	\scnidtf{искусственная кибернетическая система}
	\scnsuperset{ostis-система}
	\begin{scnindent}
		\scnidtf{компьютерная система, построенная по Технологии OSTIS на основе интерпретации спроектированной логико-семантической sc-модели этой системы}
	\end{scnindent}
\end{scnindent}

\scnheader{ostis-система}
\begin{scnrelfromset}{базовая декомпозиция}
	\scnitem{sc-модель кибернетической системы}
	\scnitem{ostis-платформа}
\end{scnrelfromset}

\scnheader{sc-модель кибернетической системы}
\begin{scnrelfromset}{базовая декомпозиция}
	\scnitem{sc-память}
	\scnitem{sc-модель базы знаний}
	\scnitem{sc-модель решателя задач}
	\scnitem{sc-модель интерфейса кибернетической системы}
\end{scnrelfromset}

\scnheader{sc-память}
\scnidtf{абстрактная sc-память}
\scnidtf{семантическая память, хранящая конструкции SC-кода}

\end{SCn}

Sc-память представляет собой с одной стороны общую среду для хранения знаний из базы знаний, а с другой стороны -- среду для взаимодействия sc-агентов. При этом каждый sc-агент опирается при работе на некоторые известные ему sc-элементы, хранящиеся в этой памяти (\textit{ключевые sc-элементы} данного sc-агента).

Явное выделение в рамках \textit{sc-модели кибернетической системы} \textit{sc-модели базы знаний}, \textit{sc-модели решателя задач} и \textit{sc-модели интерфейса кибернетической системы} является в известной мере условным, поскольку для обеспечения платформенной независимости \textit{sc-модели кибернетической системы} и решатель задач и интерфейс системы описываются средствами SC-кода и, таким образом, тоже являются частью базы знаний. Такое явное выделение указанных компонентов обусловлено исключительно удобством проектирования и сопровождения системы.

Таким образом, при условии строгого разделения \textit{sc-модели кибернетической системы} и \textit{ostis-платформы}, а также обеспечении универсальности \textit{ostis-платформы}, то есть возможности интерпретировать любую \textit{sc-модель кибернетической системы} на любом варианте \textit{ostis-платформы}, этап \uline{реализации} ostis-системы фактически сводится к загрузке \textit{sc-модели кибернетической системы} на выбранный вариант \textit{ostis-платформы}.

Важно отметить, что универсальность конкретного варианта реализации \textit{ostis-платформы} очевидно ограничивается физической (аппаратной) частью этой реализации. Например, если аппаратная часть выбранного варианта платформы представляет собой обычный персональный компьютер, то без добавления дополнительных аппаратных компонентов система не сможет решать задачи, связанные с физическим перемещением себя и других объектов в пространстве, даже если программная часть системы способна выполнить необходимые расчеты. Говоря другими словами, любая \textit{ostis-платформа} всегда будет ограничена в решении \textit{поведенческих задач} каких-либо классов, какими бы мощными физическими ресурсами она не обладала. Таким образом, корректнее говорить об \uline{универсальности \textit{ostis-платформы} в контексте решения \textit{информационных задач}}, то есть возможности интерпретировать любые \textit{sc-модели кибернетических систем} независимо от того, какого рода \textit{информационные задачи} рещают эти системы. 

Исходя из этого можно сформулировать ключевое требование, предъявляемое к \textit{sc-модели кибернетической системы} -- ни на одном из этапов решения любой \textit{информационной задачи} в данной системе не должны учитываться особенности той платформы, на которой в дальнейшем будет интерпретироваться указанная sc-модель. Аналогично ключевым требованием к \textit{ostis-платформе} является обеспечение интерфейса доступа (поиска и преобразования) к хранимой в sc-памяти информации некоторым универсальным способом, не зависящим от особенностей реализации конкретной платформы. Таким образом, важнейшей задачей для обеспечения платформенной независимости ostis-систем является четкая спецификация требований, предъявляемых к каждой реализации \textit{ostis-платформы}, то есть \uline{стандартизация} \textit{ostis-платформ}.

Важно подчеркнуть, что несмотря на преимущества платформенно-независимой реализации ostis-систем часто оказывается целесообразным реализовывать некоторые компоненты ostis-систем (например, конкретные sc-агенты или компоненты пользовательского интерфейса) на уровне платформы. В случае подобной реализации программ sc-агентов можно провести аналогию с реализацией каких-либо подпрограмм на уровне языков микропрограммирования для современных компьютеров. Чаще всего целесообразность такого решения обусловлена повышением производительности таких компонентов и системы в целом, поскольку реализация компонента с учетом особенностей платформы в общем случае является более эффективной. В то же время заметим, что последнее утверждение не всегда верно, поскольку при реализации компонента на уровне логико-семантической модели может быть реализованы, например, модели параллельной обработки информации, не всегда легко и понятно реализуемые на уровне платформы.

Таким образом, при проектировании каждой конкретной ostis-системы разработчику необходимо принимать решение о реализации тех или иных компонентов на платформенно-независимом уровне или уровне платформы. При этом очевидно, что с точки зрения развития технологии и накопления проектного опыта более приоритетной является реализация компонентов ostis-систем на платформенно-независимом уровне.

Исходя из сказанного, можно предположить существование ostis-систем, в которых все sc-агенты реализованы на уровне платформы, которая в таком случае по сути "заточена"{} под конкретную ostis-систему и может рассматриваться как аналог специализированного компьютера, ориентированного на решение задач только определенного ограниченного класса. Назовем такой вариант реализации ostis-систем \textit{минимальной конфигурацией ostis-системы}. Для того, чтобы \textit{минимальная конфигурация ostis-системы} вообще могла считаться \textit{ostis-системой}, то есть системой, построенной в соответствии с принципами \textit{Технологии OSTIS}, она должна удовлетворять хотя бы следующему минимальному набору требований:
\begin{SCn}
	\item использование SC-кода как базового языка кодирования информации в базе знаний, и, соответственно, наличие памяти, хранящей конструкции SC-кода;
	\item наличие базы знаний, определяющей денотационную семантику понятий, используемых системой;
	\item наличие хотя бы одного \textit{внутреннего sc-агента}, осуществляющего обработку знаний в памяти ostis-системы. Этот sc-агент может быть реализован на уровне платформы, соответственно база знаний такой системы может не содержать процедурных знаний (методов);
\end{SCn}

Такой вариант \textit{минимальной конфигурации ostis-системы} обладает только внутренним sc-агентом и, соответственно, не имеет возможности общаться с внешним миром (можно сказать, что такая ostis-система не обладает "органами чувств"). Для того, чтобы система имела возможность общаться с внешним миром, необходимо добавить к \textit{минимальной конфигурации ostis-системы} хотя бы один рецепторный sc-агент и хотя бы один эффекторный sc-агент.

Важно отметить, что, как видно из представленного описания \textit{минимальной конфигурации ostis-системы}, \textit{ostis-система} не обязана по умолчанию быть \textit{интеллектуальной системой}. Применение \textit{Технологии OSTIS} для разработки компьютерных систем не делает их автоматически интеллектуальными, оно позволяет обеспечить возможность последующей неограниченной интеллектуализации таких систем с минимальными накладными расходами при условии соблюдения при их разработке всех принципов \textit{Технологии OSTIS}.

\section{Базовые интерпретаторы логико-семантических моделей ostis-систем}

\begin{SCn}
\scnheader{ostis-платформа}
\scnidtf{платформа интерпретации sc-моделей компьютерных систем}
\scnidtf{интерпретатор унифицированных логико-семантических моделей компьютерных систем}
\scnidtf{Семейство платформ интерпретации sc-моделей компьютерных систем}
\scnidtf{платформа реализации sc-моделей компьютерных систем}
\scnsubset{встроенная ostis-система}
\scnidtf{встроенная пустая ostis-система}
\end{SCn}

Реализация \textit{ostis-платформы} (\textit{интерпретатора sc-моделей кибернетических систем}) может иметь большое число вариантов -- как программно, так и аппаратно реализованных. Логическая архитектура \textit{ostis-платформы} обеспечивает независимость проектируемых систем от многообразия вариантов реализации интерпретатора их моделей и в общем случае включает в себя:

\begin{itemize}
	\item хранилище \textit{sc-текстов} (\textit{sc-хранилище}, хранилище знаковых конструкций, представленных SC-коде);
	\item файловую память, то есть память для хранения информационных конструкций, представленных не в SC-коде (в виде файлов \textit{ostis-системы});
	\item средства, обеспечивающие взаимодействие \textit{sc-агентов} над общей памятью (sc-памятью);
	\item базовые средства интерфейса для взаимодействия системы с внешним миром (пользователем или другими системами). Указанные средства включают в себя, как минимум, редактор, транслятор (в sc-память и из нее) и визуализатор для одного из базовых универсальных вариантов представления \textit{SC-кода} (\textit{SCg-код}, \textit{SCs-код}, \textit{SCn-код}), средства, позволяющие задавать системе вопросы из некоторого универсального класса (например, запрос семантической окрестности некоторого объекта);
	\item реализацию \textit{Абстрактной scp-машины}, то есть интерпретатор \textit{scp-программ} (программ Языка SCP).
\end{itemize}

При необходимости, в \textbf{\textit{ostis-платформу}} могут быть заранее на платформенно-зависимом уровне включены какие-либо компоненты машин обработки знаний или баз знаний, например, с целью упрощения создания первой версии \textit{прикладной ostis-системы}.

Реализация платформы может осуществляться на основе произвольного набора существующих технологий, включая аппаратную реализацию каких-либо ее частей. С точки зрения компонентного подхода любая \textbf{\textit{ostis-платформа}} является \textbf{\textit{платформенно-зависимым многократно используемым компонентом}}.

\begin{SCn}
\scnheader{ostis-платформа}
\begin{scnrelfromset}{разбиение}
	\scnitem{базовая ostis-платформа}
	\begin{scnindent}
		\scnidtf{базовый интерпретатор логико-семантических моделей ostis-систем}
		\scnidtf{минимальная универсальная ostis-платформа, обеспечивающая интерпретацию sc-модели любой ostis-системы и включающая интерпретатор базового языка программирования ostis-систем (Языка SCP)}
		\scnidtf{универсальный интерпретатор sc-моделей ostis-систем}
		\scnidtf{универсальная базовая ostis-система, обеспечивающая имитацию любой ostis-системы путем интерпретации sc-модели имитируемой ostis-системы}
	\end{scnindent}
	\scnitem{расширенная ostis-платформа}
	\begin{scnindent}
		\scnidtf{ostis-платформа, содержащая дополнительные компоненты, реализованные на уровне платформы}
		\scnidtf{базовая ostis-платформа и множество компонентов, реализованных на уровне платформы}
	\end{scnindent}
	\scnitem{специализированная ostis-платформа}
	\begin{scnindent}
		\scnidtf{ostis-платформа, не содержащая реализацию интерпретатора языка SCP}
		\scnidtf{неуниверсальная ostis-платформа}
	\end{scnindent}
\end{scnrelfromset}
\end{SCn}

Понятие \textit{базовой ostis-платформы} является ключевым с точки зрения обеспечения платформенной независимости ostis-систем. Универсальность \textit{базовой ostis-платформы} подразумевает возможность интерпретации на ее основе любой sc-модели кибернетической системы. Это достигается за счет наличия в рамках \textit{Технологии OSTIS} средств, позволяющих описывать на уровне sc-модели базу знаний, решатель задач и интерфейс кибернетической системы, а также наличия Базового универсального языка программирования для ostis-систем (\textit{Языка SCP}). Язык SCP в таком случае выступает в роли базового низкоуровневого стандарта (ассемблера) обработки конструкций SC-кода, гарантирующего полноту с точки зрения обработки, то есть, обеспечивающего возможность осуществить любое преобразование любого фрагмента SC-кода при условии сохранения синтаксической корректности этого фрагмента. Следует отметить, что в общем случае таких функционально эквивалентных ассемблеров может быть несколько, но для обеспечения совместимости в рамках \textit{Технологии OSTIS} один из таких вариантов выбирается в качестве стандарта и описывается в главе \nameref{chapter_situation_management}. 

Таким образом, основным и \uline{единственным требованием}, предъявляемым ко всем \textit{базовым ostis-платформам} для обеспечения их универсальности, является необходимость обеспечения интерпретации \textit{Языка SCP}, стандартизированного в рамках \textit{Технологии OSTIS}. При этом важно отметить, что все \textit{базовые ostis-платформы} обязаны быть \uline{функционально эквивалентными}, поскольку интерпретируют один и тот же стандарт \textit{Языка SCP}.

Каждая \textit{базовая ostis-платформа} включает в себя:
\begin{itemize}
	\item реализацию средств хранения конструкций SC-кода (sc-памяти);
	\item реализацию файловой памяти;
	\item реализацию средств обработки конструкций SC-кода -- \textit{scp-интерпретатора};
	\item реализацию базового набора рецепторных и эффекторных sc-агентов, обеспечивающих минимально необходимый обмен информацией между ostis-системой и внешней средой. Конкретный перечень таких агентов требует уточнения, однако можно сказать, что в общем случае они могут быть реализованы как в составе scp-интерпретатора (в этом случае им будут соответствать определенные классы scp-операторов), так и отдельно от него в составе платформы.
	\item реализацию базового набора sc-агентов, обеспечивающих базовые функции ostis-системы, связанные с обеспечением ее жизнедеятельности, которые принципиально не могут быть реализованы на платформенно-независимом уровне. К таким функциям относятся, например, запуск системы, загрузка базы знаний в память системы, запуск \textit{scp-интерпретатора} и т.д.
\end{itemize}

\textit{Расширенная ostis-платформа} представляет собой \textit{базовую ostis-платформу}, дополненную каким-либо множеством компонентов (хотя бы одним), реализованных на уровне платформы, при условии сохранения при этом всех возможностей \textit{базовой ostis-платформы}. Таким образом, \textit{расширенная ostis-платформа} по сути представляет собой \textit{базовую ostis-платформу}, адаптированную для более эффективного решения задач определенных классов в рамках конкретного класса ostis-систем. Компонент, реализуемый на уровне платформы, становится частью этой платформы и, таким образом,  преобразует \textit{базовую ostis-платформу} в \textit{расширенную ostis-платформу}. 

Введением понятия \textit{расширенной ostis-платформы} позволяет сформулировать ряд дополнительных принципов реализации \textit{ostis-систем}:
\begin{itemize}
	\item Может существовать произвольное количество ostis-систем, каждая из которых будет иметь свою уникальную \textit{расширенную ostis-платформу}, но при этом все они будут основаны на одном и том же варианте \textit{базовой ostis-платформы}.
	\item Для каждого варианта \textit{базовой ostis-платформы} может существовать своя \textit{библиотека многократно используемых компонентов ostis-платформ}, совместимых с данным вариантом \textit{базовой ostis-платформы}, и позволяющая компоновать различные варианты \textit{расширенной ostis-платформы} на основе \textit{базовой ostis-платформы}.
\end{itemize} 

\textit{Специализированная ostis-платформа} представляет собой ограниченный вариант реализации \textit{ostis-платформы}, не содержащий реализацию \textit{scp-интерпретатора}. Таким образом, \uline{все} sc-агенты, в рамках \textit{ostis-системы}, основанной на \textit{специализированной ostis-платформе} должны быть реализованы на платформенно-зависимом уровне. Такая  \textit{специализированная ostis-платформа} является аналогом специализированного компьютера, реализованного для конкретной компьютерной системы. Таким образом, в общем случае каждая \textit{ostis-система}, ориентированная на работу на  \textit{специализированной ostis-платформе} будет иметь свою \uline{уникальную} \textit{специализированную ostis-платформу}.

\textit{Специализированная ostis-платформа} может быть получена из \textit{базовой ostis-платформы} путем исключения из нее реализации  \textit{scp-интерпретатора} и реализации всех необходимых sc-агентов на уровне платформы (или заимствования всех или части агентов из соответствующей данному варианту \textit{базовой ostis-платформы} \textit{библиотеки многократно используемых компонентов ostis-платформ}).

Применение \textit{специализированных ostis-платформ} может быть целесообразным на стартовом этапе развития \textit{Технологии OSTIS}, а также с целью повышения производительности конкретных наиболее высоконагруженных ostis-систем, однако активное развитие таких \textit{специализированных ostis-платформ} и их компонентов с точки зрения \textit{Технологии OSTIS} является нецелесообразным, поскольку:
\begin{itemize}
	\item если какой-либо компонент разработан с ориентацией на конкретную платформу, то нет гарантий возможности его повторного использования в других вариантах реализации ostis-платформы (как минимум, компоненты, разработанные для \textit{программного варианта реализации ostis-платформы} не смогут быть использованы в рамках \textit{семантического ассоциативного компьютера});
	\item наличие большого числа платформенно-зависимых компонентов требует развития и сопровождения отдельной инфраструктуры библиотек для хранения и повторного использования таких компонентов. Чем больше будет вариантов ostis-платформ и чем больше будет число платформенно-зависимых компонентов, тем более сложной и громоздкой будет такая инфраструктура. Как минимум, необходимо будет отслеживать совместимость компонентов с разными версиями разных вариантов реализации ostis-платформ;
	\item изменения в \textit{специализированной ostis-платформе}, например, связанные с переходом на более новую и эффективную версию \textit{базовой ostis-платформы}, на основе которой построена данная \textit{специализированная ostis-платформа} в общем случае могут привести к необходимости внесения изменений в компоненты, зависящие от данного варианта реализации ostis-платформы. Чем больше таких платформенно-зависимых компонентов, тем больше потенциальных изменений может потребоваться и, соответственно, тем сложнее будет осуществляться эволюция платформы при условии сохранения работоспособности ostis-систем, в которых она используется.
\end{itemize} 

Перечисленные тезисы справедливы и для \textit{расширенных ostis-платформ}, однако в случае \textit{расширенной ostis-платформы} проблемы, связанные с переходом на более новую версию платформы и изменениями в соответствующих компонентах всегда могут быть решены путем временной замены платформенно-зависимых компонентов на их платформенно-независимые версии с соответствующим снижением производительности, но зато сохранением функциональной целостности системы.

\begin{SCn}
\scnheader{ostis-платформа}
\begin{scnrelfromset}{разбиение}
	\scnitem{однопользовательский вариант реализации ostis-платформы}
	\begin{scnindent}
		\scnidtf{вариант реализации платформы интерпретации sc-моделей компьютерных систем, рассчитанный на то, что с конкретной ostis-системой взаимодействует только один пользователь (владелец)}
	\end{scnindent}
	\scnitem{многопользовательский вариант реализации ostis-платформы}
	\begin{scnindent}
		\scnidtf{вариант реализации платформы интерпретации sc-моделей компьютерных систем, рассчитанный на то, что с конкретной ostis-системой одновременно или в разное время могут взаимодействовать разные пользователи, в общем случае обладающие разными правами, сферами ответственности, уровнем опыта, и имеющие свою конфиденциальную часть хранимой в базе знаний информации}
	\end{scnindent}
\end{scnrelfromset}
\end{SCn}

При однопользовательском варианте реализации платформы оказывается невозможным реализовать некоторые важные принципы \textit{Технологии OSTIS}, например, коллективную согласованную разработку базы знаний системы в процессе ее эксплуатации. При этом могут использоваться различные сторонние средства, например для разработки базы знаний на уровне исходных текстов.

\begin{SCn}
\scnheader{ostis-платформа}
\begin{scnrelfromset}{разбиение}
	\scnitem{программный вариант реализации ostis-платформы}
	\begin{scnindent}
		\scnidtf{программная платформа интерпретации sc-моделей ostis-систем}
		\scnidtf{программный базовый интерпретатор sc-моделей ostis-систем}
	\end{scnindent}
	\scnitem{семантический ассоциативный компьютер}
	\begin{scnindent}
		\scnidtf{аппаратная платформа интерпретации sc-моделей ostis-систем}
		\scnidtf{аппаратно реализованный базовый интерпретатор sc-моделей ostis-систем}
	\end{scnindent}
\end{scnrelfromset}
\end{SCn}

Важно отметить, что в любом варианте реализации \textit{ostis-платформы} всегда присутствует как программная, так и аппаратная часть. Так, любой \textit{программный вариант реализации ostis-платформы} предполагает его последующую интерпретацию на какой-либо аппаратной основе, например, на персональном компьютере с традиционной архитектурой. В то же время, разработка \textit{ostis-платформы} в виде \textit{семантического ассоциативного компьютера} предполагает разработку набора микропрограмм, реализующих базовые операции поиска и преобразования sc-конструкций, хранящихся в sc-памяти. 

Таким образом, разделение множества возможных реализаций \textit{ostis-платформы} на программный и аппаратный варианты скорее отражает вариант аппаратной архитектуры, на которую в конечном итоге ориентирован тот или иной вариант реализации платформы -- либо на традиционную фон-неймановскую архитектуру, либо на специализированную архитектуру \textit{семантического ассоциативного компьютера} со структурно-перестраиваемой (графодинамической) памятью. \textit{Программный вариант реализации ostis-платформы} по сути является моделью (виртуальной машиной) \textit{семантического ассоциативного компьютера}, построенной на базе традиционной фон-неймановской архитектуры, а \textit{Язык SCP} выступает в роли ассемблера для \textit{семантического ассоциативного компьютера} и также может интерпретироваться как в рамках аппаратной реализации такого компьютера, так и в рамках его программной модели. 

Целесообразность разработки \textit{программных вариантов реализации ostis-платформы} на настоящий момент обусловлена очевидной распространенностью фон-неймановской архитектуры и, соответственно, необходимостью реализации ostis-систем на современных компьютерах различного вида. В то же время очевидно, что разработка специализированных \textit{семантических ассоциативных компьютеров} позволит существенно повысить эффективность работы ostis-систем, а четкое разделение \textit{sc-модели кибернетической системы} и платформы ее интерпретации позволит осуществить перевод уже работающих ostis-систем с традиционных архитектур на \textit{семантические ассоциативные компьютеры} с минимальными накладными расходами.

Каждой конкретной \textit{ostis-системе} однозначно соответствует конкретная \textit{ostis-платформа}, которая может относиться к разному набору классов \textit{ostis-платформ}. В тоже время очевидно, что на этапе разработки платформы проектируется и реализуется некоторый вариант \textit{ostis-платформы}, который затем тиражируется в разные \textit{ostis-системы}. Впоследствии в каждой \textit{ostis-системе} в этот вариант \textit{ostis-платформы} могут быть внесены изменения, но в общем случае в большом количестве \textit{ostis-систем} могут использоваться полностью эквивалентные \textit{ostis-платформы}. Таким образом, по аналогии с \textit{sc-агентами} и \textit{абстрактными sc-агентами}, целесообразно говорить о классах синтаксически и функционально эквивалентных \textit{ostis-платформ}, которые мы будем называть \textit{абстрактными ostis-платформами}.

Классификация \textit{абстрактных ostis-платформ} аналогична классификации \textit{ostis-платформ}.

\begin{SCn}
\scnheader{ostis-платформа}
\begin{scnrelfromset}{разбиение}
	\scnitem{абстрактная базовая ostis-платформа}
	\scnitem{абстрактная расширенная ostis-платформа}
	\scnitem{абстрактная специализированная ostis-платформа}
\end{scnrelfromset}
\begin{scnrelfromset}{разбиение}
	\scnitem{абстрактный однопользовательский вариант реализации ostis-платформы}
	\scnitem{абстрактный многопользовательский вариант реализации ostis-платформы}
\end{scnrelfromset}
\begin{scnrelfromset}{разбиение}
	\scnitem{абстрактный программный вариант реализации ostis-платформы}
	\scnitem{абстрактный семантический ассоциативный компьютер}
\end{scnrelfromset}
\end{SCn}

\section{Ассоциативные семантические компьютеры для ostis-систем}

Применение для разработки ostis-систем современных программно-аппаратных платформ, ориентированных на адресный доступ к хранящимся в памяти данным, не всегда оказывается эффективным, поскольку при разработке интеллектуальных систем фактически приходится моделировать нелинейную память на базе линейной. Повышение эффективности решения задач интеллектуальными системами требует разработки специализированных платформ, в том числе аппаратных, ориентированных на унифицированные семантические модели представления и обработки информации. Таким образом, основной целью создания \textit{семантических ассоциативных компьютеров} является повышение производительности ostis-систем.

Рассмотрим более детально особенности логической организации традиционной архитектуры компьютерных систем, существенно затрудняющие эффективную реализацию ostis-систем на ее основе:
\begin{itemize}
	\item низкий уровень доступа к памяти, т.е. сложность и громоздкость выполнения процедуры ассоциативного поиска нужного фрагмента знаний; 
	\item линейная организация памяти и чрезвычайно простой вид конструктивных объектов, непосредственно хранимых в памяти. Это приводит к тому, что в интеллектуальных системах, построенных на базе современных компьютеров, манипулирование знаниями осуществляется с большим трудом. Во-первых, приходится оперировать не самими структурами, а их громоздкими линейными представлениями (списками, матрицами смежности, матрицами инцидентности); во-вторых, линеаризация сложных структур разрушает локальность их преобразований;
	\item представление информации в памяти современных компьютеров имеет уровень весьма далекий от семантического, что делает переработку знаний довольно громоздкой, требующей учета большого количества деталей, касающейся не смысла перерабатываемой информации, а способа ее представления в памяти;
	\item в современных компьютерах имеет место весьма низкий уровень аппаратно реализуемых операций над нечисловыми данными и полностью отсутствует аппаратная поддержка логических операций над фрагментами знаний, имеющих сложную структуру, что делает манипулирование такими фрагментами весьма сложным.
\end{itemize}

Перечисленные особенности, по существу, не устраняются также и в развиваемых в настоящее время подходах к построению нетрадиционных высокопроизводительных компьютеров (например, компьютеров, предназначенных для обучения и интерпретации искусственных нейронных сетей), ибо, в основном, все эти подходы (даже если они достаточно далеко отходят от предложенных фон Нейманом базовых принципов организации вычислительных машин) неявно сохраняют точку зрения на компьютер как на большой арифмометр и тем самым сохраняют ее ориентацию на задачи числового характера.

\textit{SC-код}, являющийся формальной основой \textit{Технологии OSTIS} изначально разрабатывался как язык кодирования информации в памяти \textit{семантических ассоциативных компьютеров}, таким образом в нем изначально заложены такие принципы, как универсальность (возможность представить знания любого рода) и унификация (единообразие) представления, а также минимизация \textit{Алфавита SC-кода}, которая, в свою очередь, позволяет облегчить создание аппаратной платформы, позволяющей хранить и обрабатывать тексты \textit{SC-кода}.

\begin{SCn}
\scnheader{семантический ассоциативный компьютер}
\scnidtf{аппаратно реализованный интерпретатор семантических моделей (sc-моделей) компьютерных систем}
\scnidtf{семантический ассоциативный компьютер, управляемый знаниями}
\scnidtf{компьютер с нелинейной структурно перестраиваемой (графодинамической) ассоциативной памятью, переработка информации в которой сводится не к изменению состояния элементов памяти, а к изменению конфигурации связей между ними}
\scnidtf{sc-компьютер}
\scnidtf{scp-компьютер}
\scnidtf{универсальный компьютер нового поколения, специально предназначенный для реализации семантически совместимых гибридных интеллектуальных компьютерных систем}
\scnidtf{универсальный компьютер нового поколения, ориентированный на аппаратную интерпретацию логико-семантических моделей интеллектуальных компьютерных систем}
\scnidtf{универсальный компьютер нового поколения, ориентированный на аппаратную интерпретацию ostis-систем}
\scnidtf{ostis-компьютер}
\scnidtf{компьютер для реализации ostis-систем}
\scnidtf{компьютер, управляемый знаниями, представленными в SC-коде}
\scnidtf{компьютер, ориентированный на обработку текстов SC-кода}
\end{SCn}





Рассмотрим принципы, лежащие в основе реализации \textit{семантических ассоциативных компьютеров}:
\begin{itemize}
	\item нелинейная память -- каждый элементарный фрагмент хранимого в памяти текста может быть инцидентен неограниченному числу других элементарных фрагментов этого текста. Таким образом, ячейки памяти, в отличие от обычной памяти, связываются не фиксированными условными связями, задающими фиксированную последовательность (порядок) ячеек в памяти, a реально (физически) проводимыми связями произвольной конфигурации. Эти связи соответствуют дугам, ребрам, гиперребрам записанного в памяти графа (sc-текста);
	\item структурно-перестраиваемая (реконфигурируемая) память -- процесс отработки хранимой в памяти информации сводится не только к изменению состояния элементов, но и к реконфигурации связей между ними. То есть, в ходе переработки информации в структурно-перестраиваемой памяти меняются на только и даже не столько состояния ячеек памяти, как это имеет место в обычной памяти, сколько конфигурация связей между этими ячейками. Т.е. в структурно-перестраиваемой памяти в ходе переработки информации не только перераспределяются метки на вершинах записанного в памяти графа, но и меняется структура самого этого графа;	
	\item в качестве внутреннего способа кодирования знаний, хранимых в памяти семантического ассоциативного компьютера, используется универсальный (!) способ нелинейного (графоподобного) смыслового представления знаний -- SC-код;
	\item обработка информации осуществляется коллективом агентов, работающих над общей памятью. Каждый из них реагирует на соответствующую ему ситуацию или событие в памяти (компьютер, управляемый хранимыми знаниями);
	\item есть программно реализуемые агенты, поведение которых описывается хранимыми в памяти агентно-ориентированными программами, которые интерпретируются соответствующими коллективами агентов;
	\item есть базовые агенты, которые не могут быть реализованы программно (в частности, это агенты интерпретации агентных программ, базовые рецепторные агенты-датчики, базовые эффекторные агенты);
	\item все агенты работают над общей памятью одновременно. Более того, если для какого-либо агента в некоторый момент времени в различных частях памяти возникает сразу несколько условий его применения, разные информационные процессы, соответствующие указанному агенту в разных частях памяти могут выполняться одновременно;
	\item для того, чтобы информационные процессы агентов, параллельно выполняемые в общей памяти не "мешали"{} друг другу, для каждого информационного процесса в памяти фиксируется и постоянно актуализируется его текущее состояние. То есть каждый информационный процесс сообщает всем остальным о своих намерениях и пожеланиях, которым остальные информационные процессы не должны препятствовать. Реализация такого подхода может выполняться, например, на основе механизма блокировок элементов семантической памяти, рассмотренного в главе \nameref{chapter_situation_management};
	\item процессор и память семантического ассоциативного компьютера глубоко интегрированы и составляют единую процессоро-память. Процессор семантического ассоциативного компьютера равномерно "распределен"{} по его памяти так, что процессорные элементы одновременно являются и элементами памяти компьютера. То есть каждая ячейка дополняется функциональным (процессорным) элементом, a перестраиваемые связи между ячейками становятся коммутируемыми каналами связи между функциональными элементами. Каждый функциональный элемент при этом имеет свою специальную внутреннюю регистровую память, отражающую важные для данного функционального элемента аспекты текущего состояния процесса выполнения элементарных операций внутреннего языка.

	Обработка информации в \textit{семантическом ассоциативном компьютере} сводится к реконфигурации каналов связи между процессорными элементами,  следовательно память такого компьютера есть не что иное, как \uline{коммутатор} (!) указанных каналов связи. Таким образом, текущее состояние конфигурации этих каналов связи и есть текущее состояние обрабатываемой информации. Этот принцип обеспечивает значительное ускорение переработки информации благодаря исключению этапов передачи информации из памяти в процессор и обратно, но оплачивается ценой большой избыточности функциональных (процессорных) средств, равномерно распределяемых по памяти.
	
\end{itemize}
	
%TODO Что даст разработка семантических компьютеров
	
%%%%%%%%%%%%%%%%%%%%%%%%% referenc.tex %%%%%%%%%%%%%%%%%%%%%%%%%%%%%%
% sample references
% %
% Use this file as a template for your own input.
%
%%%%%%%%%%%%%%%%%%%%%%%% Springer-Verlag %%%%%%%%%%%%%%%%%%%%%%%%%%
%
% BibTeX users please use
% \bibliographystyle{}
% \bibliography{}
%
\biblstarthook{In view of the parallel print and (chapter-wise) online publication of your book at \url{www.springerlink.com} it has been decided that -- as a genreral rule --  references should be sorted chapter-wise and placed at the end of the individual chapters. However, upon agreement with your contact at Springer you may list your references in a single seperate chapter at the end of your book. Deactivate the class option \texttt{sectrefs} and the \texttt{thebibliography} environment will be put out as a chapter of its own.\\\indent
References may be \textit{cited} in the text either by number (preferred) or by author/year.\footnote{Make sure that all references from the list are cited in the text. Those not cited should be moved to a separate \textit{Further Reading} section or chapter.} If the citatiion in the text is numbered, the reference list should be arranged in ascending order. If the citation in the text is author/year, the reference list should be \textit{sorted} alphabetically and if there are several works by the same author, the following order should be used:
\begin{enumerate}
\item all works by the author alone, ordered chronologically by year of publication
\item all works by the author with a coauthor, ordered alphabetically by coauthor
\item all works by the author with several coauthors, ordered chronologically by year of publication.
\end{enumerate}
The \textit{styling} of references\footnote{Always use the standard abbreviation of a journal's name according to the ISSN \textit{List of Title Word Abbreviations}, see \url{http://www.issn.org/en/node/344}} depends on the subject of your book:
\begin{itemize}
\item The \textit{two} recommended styles for references in books on \textit{mathematical, physical, statistical and computer sciences} are depicted in ~\cite{science-contrib, science-online, science-mono, science-journal, science-DOI} and ~\cite{phys-online, phys-mono, phys-journal, phys-DOI, phys-contrib}.
\item Examples of the most commonly used reference style in books on \textit{Psychology, Social Sciences} are~\cite{psysoc-mono, psysoc-online,psysoc-journal, psysoc-contrib, psysoc-DOI}.
\item Examples for references in books on \textit{Humanities, Linguistics, Philosophy} are~\cite{humlinphil-journal, humlinphil-contrib, humlinphil-mono, humlinphil-online, humlinphil-DOI}.
\item Examples of the basic Springer style used in publications on a wide range of subjects such as \textit{Computer Science, Economics, Engineering, Geosciences, Life Sciences, Medicine, Biomedicine} are ~\cite{basic-contrib, basic-online, basic-journal, basic-DOI, basic-mono}. 
\end{itemize}
}

\begin{thebibliography}{99.}%
% and use \bibitem to create references.
%
% Use the following syntax and markup for your references if 
% the subject of your book is from the field 
% "Mathematics, Physics, Statistics, Computer Science"
%
% Contribution 
\bibitem{science-contrib} Broy, M.: Software engineering --- from auxiliary to key technologies. In: Broy, M., Dener, E. (eds.) Software Pioneers, pp. 10-13. Springer, Heidelberg (2002)
%
% Online Document
\bibitem{science-online} Dod, J.: Effective substances. In: The Dictionary of Substances and Their Effects. Royal Society of Chemistry (1999) Available via DIALOG. \\
\url{http://www.rsc.org/dose/title of subordinate document. Cited 15 Jan 1999}
%
% Monograph
\bibitem{science-mono} Geddes, K.O., Czapor, S.R., Labahn, G.: Algorithms for Computer Algebra. Kluwer, Boston (1992) 
%
% Journal article
\bibitem{science-journal} Hamburger, C.: Quasimonotonicity, regularity and duality for nonlinear systems of partial differential equations. Ann. Mat. Pura. Appl. \textbf{169}, 321--354 (1995)
%
% Journal article by DOI
\bibitem{science-DOI} Slifka, M.K., Whitton, J.L.: Clinical implications of dysregulated cytokine production. J. Mol. Med. (2000) doi: 10.1007/s001090000086 
%
\bigskip

% Use the following (APS) syntax and markup for your references if 
% the subject of your book is from the field 
% "Mathematics, Physics, Statistics, Computer Science"
%
% Online Document
\bibitem{phys-online} J. Dod, in \textit{The Dictionary of Substances and Their Effects}, Royal Society of Chemistry. (Available via DIALOG, 1999), 
\url{http://www.rsc.org/dose/title of subordinate document. Cited 15 Jan 1999}
%
% Monograph
\bibitem{phys-mono} H. Ibach, H. L\"uth, \textit{Solid-State Physics}, 2nd edn. (Springer, New York, 1996), pp. 45-56 
%
% Journal article
\bibitem{phys-journal} S. Preuss, A. Demchuk Jr., M. Stuke, Appl. Phys. A \textbf{61}
%
% Journal article by DOI
\bibitem{phys-DOI} M.K. Slifka, J.L. Whitton, J. Mol. Med., doi: 10.1007/s001090000086
%
% Contribution 
\bibitem{phys-contrib} S.E. Smith, in \textit{Neuromuscular Junction}, ed. by E. Zaimis. Handbook of Experimental Pharmacology, vol 42 (Springer, Heidelberg, 1976), p. 593
%
\bigskip
%
% Use the following syntax and markup for your references if 
% the subject of your book is from the field 
% "Psychology, Social Sciences"
%
%
% Monograph
\bibitem{psysoc-mono} Calfee, R.~C., \& Valencia, R.~R. (1991). \textit{APA guide to preparing manuscripts for journal publication.} Washington, DC: American Psychological Association.
%
% Online Document
\bibitem{psysoc-online} Dod, J. (1999). Effective substances. In: The dictionary of substances and their effects. Royal Society of Chemistry. Available via DIALOG. \\
\url{http://www.rsc.org/dose/Effective substances.} Cited 15 Jan 1999.
%
% Journal article
\bibitem{psysoc-journal} Harris, M., Karper, E., Stacks, G., Hoffman, D., DeNiro, R., Cruz, P., et al. (2001). Writing labs and the Hollywood connection. \textit{J Film} Writing, 44(3), 213--245.
%
% Contribution 
\bibitem{psysoc-contrib} O'Neil, J.~M., \& Egan, J. (1992). Men's and women's gender role journeys: Metaphor for healing, transition, and transformation. In B.~R. Wainrig (Ed.), \textit{Gender issues across the life cycle} (pp. 107--123). New York: Springer.
%
% Journal article by DOI
\bibitem{psysoc-DOI}Kreger, M., Brindis, C.D., Manuel, D.M., Sassoubre, L. (2007). Lessons learned in systems change initiatives: benchmarks and indicators. \textit{American Journal of Community Psychology}, doi: 10.1007/s10464-007-9108-14.
%
%
% Use the following syntax and markup for your references if 
% the subject of your book is from the field 
% "Humanities, Linguistics, Philosophy"
%
\bigskip
%
% Journal article
\bibitem{humlinphil-journal} Alber John, Daniel C. O'Connell, and Sabine Kowal. 2002. Personal perspective in TV interviews. \textit{Pragmatics} 12:257--271
%
% Contribution 
\bibitem{humlinphil-contrib} Cameron, Deborah. 1997. Theoretical debates in feminist linguistics: Questions of sex and gender. In \textit{Gender and discourse}, ed. Ruth Wodak, 99--119. London: Sage Publications.
%
% Monograph
\bibitem{humlinphil-mono} Cameron, Deborah. 1985. \textit{Feminism and linguistic theory.} New York: St. Martin's Press.
%
% Online Document
\bibitem{humlinphil-online} Dod, Jake. 1999. Effective substances. In: The dictionary of substances and their effects. Royal Society of Chemistry. Available via DIALOG. \\
http://www.rsc.org/dose/title of subordinate document. Cited 15 Jan 1999
%
% Journal article by DOI
\bibitem{humlinphil-DOI} Suleiman, Camelia, Daniel C. O'Connell, and Sabine Kowal. 2002. `If you and I, if we, in this later day, lose that sacred fire...': Perspective in political interviews. \textit{Journal of Psycholinguistic Research}. doi: 10.1023/A:1015592129296.
%
%
%
\bigskip
%
%
% Use the following syntax and markup for your references if 
% the subject of your book is from the field 
% "Computer Science, Economics, Engineering, Geosciences, Life Sciences"
%
%
% Contribution 
\bibitem{basic-contrib} Brown B, Aaron M (2001) The politics of nature. In: Smith J (ed) The rise of modern genomics, 3rd edn. Wiley, New York 
%
% Online Document
\bibitem{basic-online} Dod J (1999) Effective Substances. In: The dictionary of substances and their effects. Royal Society of Chemistry. Available via DIALOG. \\
\url{http://www.rsc.org/dose/title of subordinate document. Cited 15 Jan 1999}
%
% Journal article by DOI
\bibitem{basic-DOI} Slifka MK, Whitton JL (2000) Clinical implications of dysregulated cytokine production. J Mol Med, doi: 10.1007/s001090000086
%
% Journal article
\bibitem{basic-journal} Smith J, Jones M Jr, Houghton L et al (1999) Future of health insurance. N Engl J Med 965:325--329
%
% Monograph
\bibitem{basic-mono} South J, Blass B (2001) The future of modern genomics. Blackwell, London 
%
\end{thebibliography}
