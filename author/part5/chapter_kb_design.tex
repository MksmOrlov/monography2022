\chapauthor{Бутрин С.В.\\Шункевич Д.В.\\Банцевич К.А.}
\chapter{Методика и средства проектирования и анализа качества баз знаний ostis-систем}
\chapauthortoc{Бутрин С.В.\\Шункевич Д.В.\\Банцевич К.А.}
\label{chapter_kb_design}

\abstract{Аннотация к главе.}

Правки:
\begin{itemize}
	\item Написать аннотацию
	\item Перенести материал по коллективной разработке от Давыденко
	\item Материал по sc-web и sc-builder 
	\item Описать проводившиеся работы по верификации и что есть сейчас (ничего ;) )
\end{itemize}

Вопросы:
\begin{itemize}
	\item Нужна помощь в переносе материала. Кто может? (Ксюша?)
	\item Объеденить разделы про противоречия и информационные дыры?
	\item Где должно быть сравнительный анализ? Насколько большой?
	\item Нужно ли рассказывать про преимущества технологии в рамках задачи верификации?
	\item Должен ли быть описан решатель? В статье был пример, но лучше абстрогироваться и описать алгоритм? Есть ли примеры формального описания алгоритмов? (блок-схемы?)
\end{itemize}

\section{Действия и методики проектирования баз знаний ostis-систем}
\section{Логико-семантическая модель комплекса встраиваемых ostis-систем автоматизации проектирования баз знаний ostis-систем}
\section{Логико-семантическая модель ostis-системы редактирования, сборки и ввода исходных текстов различных компонентов проектируемой базы знаний в память ostis-системы}
\section{Логико-семантическая модель ostis-системы редактирования проектируемой базы знаний ostis-системы на уровне её внутреннего представления}
\section{Логико-семантическая модель ostis-системы обнаружения и анализа ошибок и противоречий в базе знаний ostis-системы}

Важным этапом в разработке любой системы является контроль качества, так как именно на этом этапе определяется степень жизнеспособности и эффективности системы.

Верификация является видом анализа качества и часть процесса разработки. Она заключается в проверке информации на правильность и точность.
Ее целью является выявление ошибок, различных дефектов и недоработок для своевременного их устранения.

Cуществующие на данный момент методы верификации хорошо развиты и разработано большое количество различных моделей верификации, использующих расширенные таблицы решения, сети Петри\cite{Petri}, различные логики, например, логики с векторной семантикой\cite{VTF1}\cite{VTF2} и другие модели. Более того формируются специализированные онтологии для описания многообразия средств и моделей верификации баз знаний\cite{RybinaAlgo}. Однако механизма взаимодействия средств, использующих данные методы, нет.

Поэтому средства верификации баз знаний, существующие на данный момент, обладают рядом таких проблем как \cite{ZhangProblems}:
\begin{itemize}
    \item зависимость от формата представления информации, из-за чего приходится тратить дополнительно время на конвертирование информацию
    \item проблема невозможности быть переиспользованными, так как средства обычно делаются с учетом особенностей конкретной системы
    \item проблема отсутствия механизма взаимодействия средств верификации и анализа знаний
    \item высокая роль человека в процессе верификации, так как самым распространенным способом верификации баз знаний является ручная проверка базы экспертом, человек выступает как администратор, принимающий единогласное решение, навязывая свое мнение системе
    \item современные средства не учитывают и не рассматривают процесс верификации в рамках взаимодействия систем друг с другом.
\end{itemize}

Эти проблемы могли бы быть решены, если:
\begin{itemize}
    \item использовать унифицированный и удобный формат представления знаний
    \item системы создавались бы по общей методологии и были бы совместимы друг с другом
    \item продумать и реализовать механизм позволяющий системе стремиться самой принимать решение относительно своего состояния и наличия в нем проблемных моментов и ошибок, система может допускать ошибки и не всегда принимать верные решения, но это должны быть ее ошибки, а не экспертов и разработчиков
\end{itemize}

Преимуществами Технологии OSTIS в рамках задачи верификации являются:
\begin{itemize}
    \item наличие общей методологии проектирования интеллектуальных систем, позволяющая решить проблему совместимости систем при их коллективном взаимодействии;
    \item все знания представлены в унифицированном виде, что позволяет эффективно их обрабатывать, сводя затраты на конвертирование к минимуму;
   \item средства, с помощью которых производится выявление, анализ и устранение противоречий описаны в самой базе знаний, а так же их спецификация представлены в самой базе знаний системы, тем самым обеспечивая легкость их расширения и позволяя системе знать, каким инструментарием она обладает;
   \item отсутствие семантических эквивалентных фрагментов, что обеспечивает локальность вносимых исправлений и исключает необходимость вносить исправления многократно в разных местах;
   \item многоагентный подход, который позволяет рассматривать средства анализа и верификации баз знаний как коллектив агентов, способных взаимодействовать друг с другом и дальнейшем принимать общее решение касательно состояния базы знаний. 
\end{itemize}

Предлагаемый подход сводится к разработке:
\begin{itemize}
    \item специализированной \textit{предметной области и онтологии}, которая бы содержала бы в себе все необходимые знания о возможных видах проблемных фрагментов базы знаний и способах их исправления
    \item алгоритма, позволяющего системе выявить в себе проблемные фрагменты и устранить их, при этом обеспечив согласованность работы средств самой системы
    \item специализированного решателя задач, содержащего необходимые агенты для выявления и устранения проблемных фрагментов
\end{itemize}

%% ССылки и объяснение что есть что
Качество базы знаний во многом определяется уровнем наличия/отсутствия не-факторов\cite{Narinjani2004} в базе знаний.
\begin{SCn}
\scnheader{не-фактор}
\scnidtf{группа семантических свойств, определяющих качество информации, хранимой в памяти кибернетической системы}
\begin{scneqtoset}
 \scnitem{корректность/некорректность информации, хранимой в памяти кибернетической системы}
 \scnitem{однозначность/неоднозначность информации, хранимой в памяти кибернетической системы}
 \scnitem{целостность/нецелостность информации, хранимой в памяти кибернетической системы}
 \scnitem{чистота/загрязненность информации, хранимой в памяти кибернетической системы}
 \scnitem{достоверность/недостоверность информации, хранимой в памяти кибернетической системы}
 \scnitem{точность/неточность информации, хранимой в памяти кибернетической системы}
 \scnitem{четкость/нечеткость информации, хранимой в памяти кибернетической системы}
 \scnitem{определенность/недоопределенность информации, хранимой в памяти кибернетической системы}
\end{scneqtoset}
\end{SCn}

%\begin{SCn}
%\scnheader{непротиворечивость/противоречивость информации, хранимой в памяти кибернетической системы}
%\scnidtf{уровень присутствия в хранимой информации различного вида противоречий и, в частности, ошибок}

%\scnheader{противоречие*}
%\scnidtf{пара противоречащих друг другу фрагментов информации, хранимой в памяти кибернетической системы*}
%\scntext{примечание}{Чаще всего противоречащими друг другу информационными фрагментами являются:
	%\begin{scnitemize}
	%\item явно представленная в памяти некоторая закономерность (некоторое правило)
	%\item информационный фрагмент, не соответствующий (противоречащий) указанной закономерности
	%\end{scnitemize}}
	
	%\scnheader{полнота/неполнота информации, хранимой в памяти кибернетической системы}
	%\scnidtf{уровень того, насколько информация, хранимая в памяти кибернетической системы, описывает среду существования этой системы и используемые ею методы решения задач достаточно полно (достаточно детально) для того, чтобы кибернетическая система могла действительно решать все множество соответствующих ей задач}
	
	%\scnheader{чистота/загрязненность информации, хранимой в памяти кибернетической системы}
	%\scnidtf{многообразие форм и общее количество информационного мусора, входящего в состав информации, хранимой в памяти кибернетической системы}
	%\end{SCn}


%% Описать что это такое и зачем выделяется
\begin{SCn}
\scnheader{проблемная структура}
\scnidtf{структура, описывающая проблемный фрагмент базы знаний}
\scnidtf{структура, описывающая некачественный фрагмент базы знаний}
\begin{scnreltoset}{объединение}
\scnitem{некорректная структура}
\begin{scnindent}
	\scnidtf{структура, содержащая фрагменты, противоречащие каким либо правилам или закономерностям описанным в базе знаний}
\end{scnindent}
\scnitem{структура, описывающая неполноту в базе знаний}
\begin{scnindent}
	\scnidtf{структура, в которой имеется неполнота (то есть имеется некоторое количество информационных дыр)}
	\scntext{примечание}{Под структурой, описывающей неполноту в базе знаний, понимается структура, содержащая фрагмент базы знаний, в котором отсутствует какая-либо информация, которая необходима (или, по крайней мере, желательна) для однозначного и полного понимания смысла данного фрагмента.}
	%% Переформулировать
\end{scnindent}
\scnitem{информационный мусор}
\begin{scnindent}
	\scnidtf{структура, удаление которой существенно не усложнит деятельность системы}
	\scnidtf{структура, содержащая фрагмент базы знаний, который по каким-либо причинам стал ненужным и требует удаления}
\end{scnindent}
\end{scnreltoset}
\end{SCn}
% Больше про определение противоречия, уточнить что чему противоречит 
\begin{SCn}
\scnheader{противоречие*}
\scnidtf{пара противоречащих друг другу фрагментов информации, хранимой в памяти кибернетической системы*}
\scntext{примечание}{Чаще всего противоречащими друг другу информационными фрагментами являются:
	\begin{scnitemize}
		\item явно представленная в памяти некоторая закономерность (некоторое правило)
		\item информационный фрагмент, не соответствующий (противоречащий) указанной закономерности
	\end{scnitemize}
	В этом случае некорректность может присутствовать:
	\begin{scnitemize}
		\item либо в информационном фрагменте, который противоречит указанной закономерности;
		\item либо в самой этой закономерности;
		\item либо и там и там.
	\end{scnitemize}
		%% Переформулировать
}
\end{SCn}
	
\begin{SCn}
\scnheader{некорректная структура}
\begin{scnreltoset}{включение}
	\scnitem{дублирование системных идентификаторов}
	\scnitem{несоответствие элементов связки доменам отношения}
	\scnitem{цикл по отношению порядка}
	\scnitem{структура, противоречащая свойству единственности}
\end{scnreltoset}
	
\scnheader{структура, описывающая неполноту в базе знаний}
\begin{scnreltoset}{включение}
	\scnitem{не указан максимальный класс объектов исследования предметной области}
	\scnitem{для сущности указан системный, но не указаны основные идентификаторы для всех внешних языков}
	\scnitem{не указаны домены отношения}
	\scnitem{понятие не соотнесено ни с одной предметной областью}
\end{scnreltoset}
\end{SCn}

%% Больше описать, привести примеры, вообщем пересмотреть

\begin{SCn}
\scnheader{требующее внимание разработчика}
\scnidtf{проблемная структура, для исправления которой требуется участие разработчика}
\scnheader{множество элементов, которые должны быть удалены для исправления структуры*}
\scnidtf{множество элементов, удаление которых из структуры позволяет устранить в ней противоречие}
\scnheader{множество элементов, которые должны быть добавлены для исправления структуры*}
\scnidtf{множество элементов, добавление которых в структуру позволяет устранить в ней противоречия}
\scnheader{структура, которую система не способна исправить сама}
\scnidtf{структура, в которой система не способна автоматически устранить противоречия}
%% Возможно переформулировать на что-нибудь покороче


\scnheader{следует отличать*}
\begin{scnhaselementset}
\scnitem{структура, которая система не может решить сама}
\begin{scnindent}
	\scntext{примечание}{Здесь структура, которую система не может решить сама, не может быть исправлена при взаимодействии с разработчиком и требует полного исправления от самого разработчика}
\end{scnindent}
\scnitem{требующее внимание разработчика}
\begin{scnindent}
	\scntext{примечание}{Здесь структура, требующая внимания разработчика, может быть решена в процессе верификации, но потребуется участие разработчика}
\end{scnindent}
\end{scnhaselementset}

%% Описать идею мезанизма согласования средств верификации

\scnheader{Решатель задач средств выявления и устранения противоречий}
\begin{scnreltoset}{декомпозиция абстрактного sc-агента}
\scnitem{Неатомарный агент выявления противоречий}
	\begin{scnindent}
		\scnidtf{Множество агентов, обеспечивающих поиск и фиксирование противоречий в структуре}
	\end{scnindent}
\scnitem{Неатомарный агент устранения противоречий}
	\begin{scnindent}
	\scnidtf{Множество агентов, создающих предложения по исправлению противоречий}
	\scntext{примечание}{Результатом работы таких агентов будут множества предлагаемых к удалению из структуры или добавлению в структуру элементов}
	\end{scnindent}
\scnitem{Агент слияния структур}
	\begin{scnindent}
	\scnidtf{Агент, создающий структуру содержащую все элементы сливаемых структур}
	\end{scnindent}
\scnitem{Агент применения предложений по устранению противоречий}
\scnitem{Агент внесения исправлений в базу знаний}
	\begin{scnindent}
	\scntext{примечание}{Внесение изменений подразумевает не только исправление в базе знаний изначальной проблемной структуры, но и фиксацию самого факта изменения состояния базы знаний}.
	\end{scnindent}
\scnitem{Неатомарный агент верификации структуры}
	\begin{scnindent}
	\scntext{примечание}{Агент обеспечивающий полный цикл верификации структуры координирую другие агенты}
	\end{scnindent}
\end{scnreltoset}
\end{SCn}
%% Добавить про алгоритм

%% Не ограничиваться средствами верификации, затронуть механизмы предотвращения проблемных ситуаций 

%% Рассмотреть верификацию в рамках взаимодействия систем


\section{Логико-семантическая модель ostis-системы обнаружения и анализа информационных дыр в базе знаний ostis-системы} 
\section{Логико-семантическая модель ostis-системы автоматизации управления взаимодействием разработчиков различных категорий в процессе проектирования базы знаний ostis-системы}


Процесс создания и редактирования базы знаний ostis-системы сводится к формированию разработчиками предложений по редактированию того или иного раздела базы знаний.
В дальнейшем данные предложения рассматриваются администраторами базы знаний.
Схема базы знаний с вносимыми предложениями представлена на рисунке.

%% Описать механизм редактирования, сравнение с аналогами, акцент на описание стадий с учетом каждой роли 
\begin{SCn}
\scnheader{пользователь, обладающий правом просмотра sc-структуры*}
\scnidtf{бинарное отношение, связывающее sc-элемент, обозначающий sc-структуру (например, фрагмент sc-модели базы знаний), и sc-элемент, обозначающий пользователя этой ostis-системы, который обладает правом просмотра этой sc-структуры.}
\scniselement{бинарное отношение}
\scniselement{ориентированное отношение}
\scntext{примечание*}{Пользователь, обладающий правом просмотра sc-структуры базы знаний ostis-системы может быть зарегистрирован или не зарегистрирован в sc-модели базы знаний.}

\scnheader{пользователь, обладающий правом редактирования sc-структуры*}
\scnidtf{бинарное отношение, связывающее sc-элемент, обозначающий sc-структуру (например, фрагмент sc-модели базы знаний), и sc-элемент, обозначающий зарегистрированного пользователя ostis-системы, который обладает правом редактирования этой sc-структуры.}
\scniselement{бинарное отношение}
\scniselement{ориентированное отношение}
\scnsuperset{пользователь, обладающий правом просмотра sc-структуры*}
\scntext{пояснение}{Связки отношения пользователя, обладающий правом редактирования sc-структуры ostis-системы* связывают sc-структуру (не обязательно всю sc-модель базы знаний) и пользователя, зарегистрированного в этой sc-модели базы знаний.}
\begin{scnrelfromset}{покрытие}
	\scnitem{пользователь, обладающий правом редактирования sc-структуры посредством формирования предложений по внесению изменений в согласованную часть базы знаний этой ostis-системы*}
	\scnitem{пользователь, обладающий правом редактирования sc-структуры с автоматическим формированием и принятием предложений по внесению изменений в согласованную часть базы знаний этой ostis-системы*}
\end{scnrelfromset}

\scnheader{разработчик*}
\scnsubset{пользователь, обладающий правом редактирования sc-структуры*}
\scnidtf{бинарное отношение, связывающее sc-элемент, обозначающий некоторый раздел базы знаний (в пределе – всю базы знаний), и sc-элемент, обозначающий пользователя ostis-системы, который может быть разработчиком данного раздела базы знаний, т. е. выполнять проектные задачи в рамках данного раздела}
\end{SCn}
%% Добавить больше про классификацию пользователей

\section{Многократно используемые компоненты баз знаний ostis-систем}
\label{ostis_library_knowledge_base}

На сегодняшний день разработано большое число \textit{баз знаний} по самым различным предметным областям. Однако в большинстве случаев каждая база знаний разрабатывается отдельно и независимо от других, в отсутствие единой унифицированной формальной основы для представления знаний, а также единых принципов формирования систем понятий для описываемой предметной области. В связи с этим разработанные базы оказываются, как правило, несовместимы между собой и не пригодны для повторного использования. Компонентный подход к разработке интеллектуальных компьютерных систем, реализуемый в виде \textbf{\textit{библиотеки многократно используемых компонентов ostis-систем}}, позволяет решить описанные проблемы.

%%%%%%%%%%%%%%%%%%%%%%%%% referenc.tex %%%%%%%%%%%%%%%%%%%%%%%%%%%%%%
% sample references
% %
% Use this file as a template for your own input.
%
%%%%%%%%%%%%%%%%%%%%%%%% Springer-Verlag %%%%%%%%%%%%%%%%%%%%%%%%%%
%
% BibTeX users please use
% \bibliographystyle{}
% \bibliography{}
%
\biblstarthook{In view of the parallel print and (chapter-wise) online publication of your book at \url{www.springerlink.com} it has been decided that -- as a genreral rule --  references should be sorted chapter-wise and placed at the end of the individual chapters. However, upon agreement with your contact at Springer you may list your references in a single seperate chapter at the end of your book. Deactivate the class option \texttt{sectrefs} and the \texttt{thebibliography} environment will be put out as a chapter of its own.\\\indent
References may be \textit{cited} in the text either by number (preferred) or by author/year.\footnote{Make sure that all references from the list are cited in the text. Those not cited should be moved to a separate \textit{Further Reading} section or chapter.} If the citatiion in the text is numbered, the reference list should be arranged in ascending order. If the citation in the text is author/year, the reference list should be \textit{sorted} alphabetically and if there are several works by the same author, the following order should be used:
\begin{enumerate}
\item all works by the author alone, ordered chronologically by year of publication
\item all works by the author with a coauthor, ordered alphabetically by coauthor
\item all works by the author with several coauthors, ordered chronologically by year of publication.
\end{enumerate}
The \textit{styling} of references\footnote{Always use the standard abbreviation of a journal's name according to the ISSN \textit{List of Title Word Abbreviations}, see \url{http://www.issn.org/en/node/344}} depends on the subject of your book:
\begin{itemize}
\item The \textit{two} recommended styles for references in books on \textit{mathematical, physical, statistical and computer sciences} are depicted in ~\cite{science-contrib, science-online, science-mono, science-journal, science-DOI} and ~\cite{phys-online, phys-mono, phys-journal, phys-DOI, phys-contrib}.
\item Examples of the most commonly used reference style in books on \textit{Psychology, Social Sciences} are~\cite{psysoc-mono, psysoc-online,psysoc-journal, psysoc-contrib, psysoc-DOI}.
\item Examples for references in books on \textit{Humanities, Linguistics, Philosophy} are~\cite{humlinphil-journal, humlinphil-contrib, humlinphil-mono, humlinphil-online, humlinphil-DOI}.
\item Examples of the basic Springer style used in publications on a wide range of subjects such as \textit{Computer Science, Economics, Engineering, Geosciences, Life Sciences, Medicine, Biomedicine} are ~\cite{basic-contrib, basic-online, basic-journal, basic-DOI, basic-mono}. 
\end{itemize}
}

\begin{thebibliography}{99.}%
% and use \bibitem to create references.
%
% Use the following syntax and markup for your references if 
% the subject of your book is from the field 
% "Mathematics, Physics, Statistics, Computer Science"
%
% Contribution 
\bibitem{science-contrib} Broy, M.: Software engineering --- from auxiliary to key technologies. In: Broy, M., Dener, E. (eds.) Software Pioneers, pp. 10-13. Springer, Heidelberg (2002)
%
% Online Document
\bibitem{science-online} Dod, J.: Effective substances. In: The Dictionary of Substances and Their Effects. Royal Society of Chemistry (1999) Available via DIALOG. \\
\url{http://www.rsc.org/dose/title of subordinate document. Cited 15 Jan 1999}
%
% Monograph
\bibitem{science-mono} Geddes, K.O., Czapor, S.R., Labahn, G.: Algorithms for Computer Algebra. Kluwer, Boston (1992) 
%
% Journal article
\bibitem{science-journal} Hamburger, C.: Quasimonotonicity, regularity and duality for nonlinear systems of partial differential equations. Ann. Mat. Pura. Appl. \textbf{169}, 321--354 (1995)
%
% Journal article by DOI
\bibitem{science-DOI} Slifka, M.K., Whitton, J.L.: Clinical implications of dysregulated cytokine production. J. Mol. Med. (2000) doi: 10.1007/s001090000086 
%
\bigskip

% Use the following (APS) syntax and markup for your references if 
% the subject of your book is from the field 
% "Mathematics, Physics, Statistics, Computer Science"
%
% Online Document
\bibitem{phys-online} J. Dod, in \textit{The Dictionary of Substances and Their Effects}, Royal Society of Chemistry. (Available via DIALOG, 1999), 
\url{http://www.rsc.org/dose/title of subordinate document. Cited 15 Jan 1999}
%
% Monograph
\bibitem{phys-mono} H. Ibach, H. L\"uth, \textit{Solid-State Physics}, 2nd edn. (Springer, New York, 1996), pp. 45-56 
%
% Journal article
\bibitem{phys-journal} S. Preuss, A. Demchuk Jr., M. Stuke, Appl. Phys. A \textbf{61}
%
% Journal article by DOI
\bibitem{phys-DOI} M.K. Slifka, J.L. Whitton, J. Mol. Med., doi: 10.1007/s001090000086
%
% Contribution 
\bibitem{phys-contrib} S.E. Smith, in \textit{Neuromuscular Junction}, ed. by E. Zaimis. Handbook of Experimental Pharmacology, vol 42 (Springer, Heidelberg, 1976), p. 593
%
\bigskip
%
% Use the following syntax and markup for your references if 
% the subject of your book is from the field 
% "Psychology, Social Sciences"
%
%
% Monograph
\bibitem{psysoc-mono} Calfee, R.~C., \& Valencia, R.~R. (1991). \textit{APA guide to preparing manuscripts for journal publication.} Washington, DC: American Psychological Association.
%
% Online Document
\bibitem{psysoc-online} Dod, J. (1999). Effective substances. In: The dictionary of substances and their effects. Royal Society of Chemistry. Available via DIALOG. \\
\url{http://www.rsc.org/dose/Effective substances.} Cited 15 Jan 1999.
%
% Journal article
\bibitem{psysoc-journal} Harris, M., Karper, E., Stacks, G., Hoffman, D., DeNiro, R., Cruz, P., et al. (2001). Writing labs and the Hollywood connection. \textit{J Film} Writing, 44(3), 213--245.
%
% Contribution 
\bibitem{psysoc-contrib} O'Neil, J.~M., \& Egan, J. (1992). Men's and women's gender role journeys: Metaphor for healing, transition, and transformation. In B.~R. Wainrig (Ed.), \textit{Gender issues across the life cycle} (pp. 107--123). New York: Springer.
%
% Journal article by DOI
\bibitem{psysoc-DOI}Kreger, M., Brindis, C.D., Manuel, D.M., Sassoubre, L. (2007). Lessons learned in systems change initiatives: benchmarks and indicators. \textit{American Journal of Community Psychology}, doi: 10.1007/s10464-007-9108-14.
%
%
% Use the following syntax and markup for your references if 
% the subject of your book is from the field 
% "Humanities, Linguistics, Philosophy"
%
\bigskip
%
% Journal article
\bibitem{humlinphil-journal} Alber John, Daniel C. O'Connell, and Sabine Kowal. 2002. Personal perspective in TV interviews. \textit{Pragmatics} 12:257--271
%
% Contribution 
\bibitem{humlinphil-contrib} Cameron, Deborah. 1997. Theoretical debates in feminist linguistics: Questions of sex and gender. In \textit{Gender and discourse}, ed. Ruth Wodak, 99--119. London: Sage Publications.
%
% Monograph
\bibitem{humlinphil-mono} Cameron, Deborah. 1985. \textit{Feminism and linguistic theory.} New York: St. Martin's Press.
%
% Online Document
\bibitem{humlinphil-online} Dod, Jake. 1999. Effective substances. In: The dictionary of substances and their effects. Royal Society of Chemistry. Available via DIALOG. \\
http://www.rsc.org/dose/title of subordinate document. Cited 15 Jan 1999
%
% Journal article by DOI
\bibitem{humlinphil-DOI} Suleiman, Camelia, Daniel C. O'Connell, and Sabine Kowal. 2002. `If you and I, if we, in this later day, lose that sacred fire...': Perspective in political interviews. \textit{Journal of Psycholinguistic Research}. doi: 10.1023/A:1015592129296.
%
%
%
\bigskip
%
%
% Use the following syntax and markup for your references if 
% the subject of your book is from the field 
% "Computer Science, Economics, Engineering, Geosciences, Life Sciences"
%
%
% Contribution 
\bibitem{basic-contrib} Brown B, Aaron M (2001) The politics of nature. In: Smith J (ed) The rise of modern genomics, 3rd edn. Wiley, New York 
%
% Online Document
\bibitem{basic-online} Dod J (1999) Effective Substances. In: The dictionary of substances and their effects. Royal Society of Chemistry. Available via DIALOG. \\
\url{http://www.rsc.org/dose/title of subordinate document. Cited 15 Jan 1999}
%
% Journal article by DOI
\bibitem{basic-DOI} Slifka MK, Whitton JL (2000) Clinical implications of dysregulated cytokine production. J Mol Med, doi: 10.1007/s001090000086
%
% Journal article
\bibitem{basic-journal} Smith J, Jones M Jr, Houghton L et al (1999) Future of health insurance. N Engl J Med 965:325--329
%
% Monograph
\bibitem{basic-mono} South J, Blass B (2001) The future of modern genomics. Blackwell, London 
%
\end{thebibliography}
