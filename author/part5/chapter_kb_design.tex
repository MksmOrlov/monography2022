\chapter{Методика и средства проектирования и анализа качества баз знаний ostis-систем}
\chapauthortoc{Бутрин С.В.\\Шункевич Д.В.\\Банцевич К.А.}
\label{chapter_kb_design}

\vspace{-7\baselineskip}

\scntext{эпиграф}{Дырявая и запутанная сеть хорошего улова не принесет.}

\bigskip

\begin{SCn}
	\begin{scnrelfromlist}{автор}
		\scnitem{Бутрин С.В.}
		\scnitem{Шункевич Д.В.}
		\scnitem{Банцевич К.А.}
	\end{scnrelfromlist}

\bigskip

\scntext{аннотация}{В главе рассматриваются актуальные проблемы текущего состояния средств проектирования и анализа качества \textit{баз знаний}, предложен подход к их решению на основе \textit{Технологии OSTIS}. Сформулированы принципы коллективного проектирования и разработки \textit{баз знаний}. Сформулированы принципы верификации \textit{баз знаний}.}

\bigskip

\begin{scnrelfromlist}{подраздел}
	\scnitem{\ref{sec_kb_design_methods}~\nameref{sec_kb_design_methods}}
	\scnitem{\ref{sec_kb_design_individual}~\nameref{sec_kb_design_individual}}	\scnitem{\ref{sec_kb_design_collective}~\nameref{sec_kb_design_collective}}
	\scnitem{\ref{sec_kb_design_contradiction}~\nameref{sec_kb_design_contradiction}}
	\scnitem{\ref{sec_kb_design_developers}~\nameref{sec_kb_design_developers}}
	\scnitem{\ref{sc_kb_design_components}~\nameref{sc_kb_design_components}}
\end{scnrelfromlist}

\bigskip

\begin{scnrelfromlist}{ключевое понятие}
	\scnitem{база знаний}
	\scnitem{редактор баз знаний}
	\scnitem{разработка баз знаний}
	\scnitem{разработчик}
	\scnitem{противоречие}
	\scnitem{информационная дыра}
	\scnitem{многократно используемые компоненты}
\end{scnrelfromlist}

\bigskip

\begin{scnrelfromlist}{библиографическая ссылка}
	\scnitem{\scncite{Davydenko2016}}
	\scnitem{\scncite{Davydenko2017}}
	\scnitem{\scncite{Davydenko2018}}
	\scnitem{\scncite{Knowledge-base-editor2022}}
	\scnitem{\scncite{Arshinskiy2020}}
	\scnitem{\scncite{Rybina2007}}
	\scnitem{\scncite{Zhang2023}}
	\scnitem{\scncite{Narinjani2004}}
	\scnitem{\scncite{Ivashenko2011}}
	\scnitem{\scncite{Davydenko2013}}
\end{scnrelfromlist}

\end{SCn}

\section*{Введение в Главу \ref{chapter_kb_design}}

Разработка \textit{базы знаний} является трудоемким и продолжительным процессом, требующим высокого уровня квалификации \textit{разработчиков баз знаний}. Данный факт приводит к высокой себестоимости как самих \textit{баз знаний}, так и соответствующих им \textit{интеллектуальных систем}, а также к дефициту специалистов в области \textit{инженерии знаний}.

Расширение областей применения \textit{интеллектуальных систем} требует поддержки решения комплексных задач.
Решение каждой такой задачи предполагает совместное использование различных видов знаний и моделей их представления, что приводит к компенсации недостатков одних моделей возможностями и достоинствами других.

Существующие \textit{средства создания баз знаний} предполагают, что процессы разработки и модификации базы знаний осуществляются отдельно от процесса ее использования, что приводит к дополнительному усложнению решения задачи обеспечения совместимости различного вида знаний.
Отсутствие удовлетворительного решения этой задачи приводит к несовместимости \textit{компонентов баз знаний}, разрабатываемых для разных систем, и невозможности их повторного использования в других системах.
Данный факт приводит к многократной повторной разработке содержательно одних и тех же компонентов для разных баз знаний.

Таким образом, актуальной является задача разработки модели \textit{баз знаний}, которая, с одной стороны, обеспечит общий унифицированный формальный фундамент для представления различных видов знаний в рамках одной \textit{базы знаний} и их совместного использования при решении комплексных задач, а с другой стороны, обеспечит возможность расширения числа видов знаний, используемых \textit{интеллектуальной системой} (см. \scncite{Davydenko2016}).


\section{Действия и методики проектирования баз знаний ostis-систем}
\label{sec_kb_design_methods}

\textit{База знаний} является ключевым компонентом \textit{интеллектуальной системы}, которая в систематизированном виде включает в себя все знания, необходимые \textit{интеллектуальной системе} для ее функционирования.

Для задач, решаемых \textit{интеллектуальными системами}, в общем случае неизвестно, какие данные и знания должны быть использованы для решения этих задач, какие должны быть использованы методы их решения.
При решении таких задач необходима локализация \textit{фрагмента базы знаний}, содержащего данные и знания, достаточные для решения задачи, и исключающего те данные и знания, которые заведомо для этого не нужны, а также выделение из имеющегося многообразия методов решения задач тех \textit{методов}, которых достаточно для решения данной задачи.

Однако для обеспечения совместного использования различных видов знаний в единой \textit{базе знаний} необходимо обеспечить совместимость этих видов знаний и, как следствие, совместимость \textit{компонентов баз знаний}, которая включает два аспекта: обеспечение синтаксической совместимости, что подразумевает унификацию формы представления знаний, и обеспечение семантической совместимости, что подразумевает однозначную и единую для всех компонентов трактовку используемых понятий.

Существующие подходы к разработке \textit{баз знаний}, как правило, предполагают решение задачи обеспечения синтаксической совместимости знаний путем соединения разнородных моделей представления знаний, а также разработки новых интегрированных моделей и новых языков представления знаний.
Разработка \textit{базы знаний} таким методом приводит к дополнительным накладным расходам при интеграции и обработке разнородных знаний, и, как следствие, к резкому увеличению трудозатрат при модификации таких \textit{баз знаний} и добавлении новых видов знаний.
Таким образом, \textit{интеллектуальная система}, способная решать комплексные задачи, должна обладать способностью приобретать новые знания и навыки в процессе ее эксплуатации, сохраняя при этом \textit{корректность} и \textit{целостность} \textit{базы знаний}.

В свою очередь, это обуславливает требование модицифируемости к такой \textit{базе знаний}, то есть снижения трудоемкости внесения изменений в базу знаний.
Таким образом, можно сформулировать требования к \textit{базе знаний} \textit{систем, способных решать комплексные задачи}:

\begin{textitemize}
\item возможность согласованного использования различных видов знаний, хранимых в одной \textit{базе знаний}, при решении каждой комплексной задачи;

\item возможность реализации различных аспектов спецификации \textit{сущностей}, описываемых в \textit{базе знаний};

\item \textit{модифицируемость базы знаний}, позволяющая непосредственно в процессе эксплуатации \textit{интеллектуальной системы} добавлять в базу знаний новые фрагменты, в том числе новые виды знаний без внесения изменений в существующую структуру \textit{базы знаний};

\item возможность неограниченного перехода в рамках каждой \textit{базы знаний} от знаний к \textit{метазнаниям}, от метазнаний к \textit{метаметазнаниям} и так далее, что, в частности, предоставляет неограниченные возможности типологии и систематизации знаний, хранимых в составе \textit{базы знаний}, и неограниченные возможности \textit{декомпозиции} и \textit{структуризации} самой базы знаний;

\item наличие языковых средств, позволяющих в рамках базы знаний представлять метазнания, описывающие \textit{качество базы знаний} (противоречия, неполноту, избыточность);

\item наличие языковых средств, позволяющих в рамках базы знаний представлять метазнания, описывающие \textit{историю эволюции} и \textit{планы дальнейшей эволюции} базы знаний;

\item возможность для каждой решаемой задачи явно задавать и уточнять в ходе решения задачи \textit{область решения}, то есть такой фрагмент базы знаний, использование которого является достаточным для решения этой задачи.
\end{textitemize}

А также следующие дополнительные требования к функциональности \textit{системы поддержки коллективной разработки гибридных баз знаний}, учитывающие недостатки рассмотренных аналогов:

\begin{textitemize}
\item обеспечение возможности как ручного, так и автоматического \textit{редактирования баз знаний};

\item обеспечение возможности \textit{автоматической верификации базы знаний}, в том числе анализ \textit{корректности} и \textit{полноты} \textit{базы знаний};

\item обеспечение возможности создания \textit{базы знаний} распределенным \textit{коллективом разработчиков}, включая механизм согласования вносимых в \textit{базу знаний} изменений, а также механизм хранения истории вносимых изменений с указанием авторства.
\end{textitemize}

Реализация перечисленных возможностей подразумевает отказ от работы с \textit{файлами исходных текстов базы знаний}. В данном случае предполагается, что все изменения осуществляются непосредственно в памяти системы, где хранится вся база знаний, что позволяет осуществлять разработку базы знаний компьютерной системы в процессе ее эксплуатации (см. \scncite{Davydenko2017}).

%%%%%%%%%%%%%%%%%%%%

Проектирование базы знаний включает в себя два аспекта: индивидуальный и коллективный.
Индивидуальный представляет собой наполнение изолировнной части \textit{базы знаний} конкретным пользователем.
Коллективный представляет собой согласование всей \textit{базы знаний} и включает в себя процесс внесения предложений и внедрения изменений.

Таким образом индивидуальный аспект включает в себя средства и методы позволяющие пользователю непосредственно наполнять \textit{базу знаний} посредством \textit{редакторов}, \textit{трансляторов} и так далее. При этом \textit{процесс редактирования базы знаний} должен быть максимально удобным и понятным для пользователя.

Коллективный же аспект включает в себя средства и методы для автоматического или полуавтоматического \textit{согласования общей базы знаний} с индивидуальными \textit{базами знаний} пользователей. То есть он включает в себя процессы создания предложений по изменению базы знаний, их рассмотрение и внесение в общую базу знаний.

Так как в основе любой современной базы знаний лежат \textit{онтологии}, то методы и средства разработки онтологий являются важнейшей частью технологий разработки баз знаний.
Методология разработки онтологий представляет собой набор инструкций и руководств, описывающих процесс выполнения сложных процедур разработки онтологий.
Она детализирует различные задачи, как они должны быть выполнены, в каком порядке и каким образом осуществлять документирование работы по созданию онтологий.
Существующие методологии можно условно разделить на представленные ниже группы.

\begin{SCn}
\scnheader{методология разработки онтологий}
\scnrelfrom{разбиение}{\scnkeyword{Типология методологий по поддержке коллективной разработки\scnsupergroupsign}}
\begin{scnindent}
	\begin{scneqtoset}
		\scnitem{методология, поддерживающая совместную коллективную разработку онтологии}
		\scnitem{методология, не поддерживающая совместную коллективную разработку онтологии}
	\end{scneqtoset}
\end{scnindent}

\scnrelfrom{разбиение}{\scnkeyword{Типология методологий по степени зависимости от инструментария\scnsupergroupsign}}
\begin{scnindent}
	\begin{scneqtoset}
		\scnitem{методология, зависимая от инструментария}
		\scnitem{методология, полузависимая от инструментария}
		\scnitem{методология, независимая от инструментария}
	\end{scneqtoset}
\end{scnindent}

\scnrelfrom{разбиение}{\scnkeyword{Типология методологий по типу используемой модели жизненного цикла онтологии\scnsupergroupsign}}
\begin{scnindent}
	\begin{scneqtoset}
		\scnitem{методология без указания модели жизненного цикла онтологии}
		\scnitem{методология с итеративной моделью жизненного цикла онтологии}
		\scnitem{методология с моделью жизненного цикла онтологии на основе эволюционного прототипирования}
		\scnitem{методология с моделью жизненного цикла приложения}
	\end{scneqtoset}
\end{scnindent}

\scnrelfrom{разбиение}{\scnkeyword{Типология методологий по возможности формализации\scnsupergroupsign}}
\begin{scnindent}
	\begin{scneqtoset}
		\scnitem{методология, предусматривающая методы формализации}
		\scnitem{методология, не предусматривающая формализации}
	\end{scneqtoset}
\end{scnindent}

\scnrelfrom{разбиение}{\scnkeyword{Типология методологий по возможности повторного использования разрабатываемых онтологий\scnsupergroupsign}}
\begin{scnindent}
	\begin{scneqtoset}
		\scnitem{методология, поддерживающая повторное использование}
		\scnitem{методология, не поддерживающая повторного использования}
	\end{scneqtoset}
\end{scnindent}

\scnrelfrom{разбиение}{\scnkeyword{Типология методологий по стратегии выделения концептов предметной области\scnsupergroupsign}}
\begin{scnindent}
	\begin{scneqtoset}
		\scnitem{методология снизу вверх (bottom-up)}
		\scnitem{методология сверху вниз (top-down)}
		\scnitem{методология от середины (middle-out)}
		\scnitem{методология, сочетающая различные стратегии}
	\end{scneqtoset}
\end{scnindent}

\scnrelfrom{разбиение}{\scnkeyword{Типология методологий по возможности поддержки совместимости разрабатываемых онтологий\scnsupergroupsign}}
\begin{scnindent}
	\begin{scneqtoset}
		\scnitem{методология, поддерживающая совместимость}
		\scnitem{методология, не поддерживающая совместимость}
	\end{scneqtoset}
\end{scnindent}

\end{SCn}

Большинство методологий не поддерживают совместную разработку \textit{баз знаний}, поддержку совместимости разрабатываемых \textit{баз знаний} и, как следствие, поддержку повторного использования уже разработанных \textit{баз знаний} и их \textit{компонентов}.

Кроме того, подавляющее большинство \textit{методологий разработки баз знаний} описывают процесс разработки в общих чертах, не регламентируя действия участников на каждом этапе разработки \textit{онтологии}, не уточняя принципы согласования новых \textit{понятий} с уже существующими, высоким оказывается субъективное влияние разработчиков.

%%%%%%%%%%%%%%%%%%%%%%%

Отличительной особенностью предлагаемой методики от существующих \textit{методологий разработки баз знаний} является совершенствование \textit{базы знаний} \textit{коллективом разработчиков} непосредственно в процессе ее использования, а также создание новых и использование уже имеющихся \textit{компонентов баз знаний} в процессе разработки каждой \textit{базы знаний}.

Данная методика предполагает два основных этапа --- этап создания стартовой версии разрабатываемой \textit{ostis-системы}, \textit{база знаний} которой синтезируется из \textit{компонентов}, входящих в \textit{библиотеку многократно используемых компонентов баз знаний ostis-систем}(см. \textit{\ref{ostis_library_section}~\nameref{ostis_library_section}}), и этап расширения и совершенствования \textit{базы знаний} разрабатываемой \textit{ostis-системы}, осуществляемый в рамках этой системы.

Стартовая версия \textit{ostis-системы} содержит набор знаний и средств решения задач, достаточный для дальнейшего развития системы.

Процесс создания стартовой версии ostis-системы можно разделить на пять основных этапов:
\begin{textitemize}
\item выбор и установка \textit{ostis-платформы} для интерпретации \textit{sc-модели ostis-системы}(см. \textit{Главу \ref{chapter_soft_platform}~\nameref{chapter_soft_platform}});

\item установка \textit{Ядра sc-модели базы знаний ostis-системы} из \textit{библиотеки многократно используемых компонентов баз знаний}(см. \textit{\ref{sc_kb_design_components}~\nameref{sc_kb_design_components}});

\item установка \textit{Ядра решателей задач из библиотеки многократно используемых компонентов решателей задач}, то есть набора базовых \textit{многократно используемых компонентов решателей задач}, необходимых для работы стартовой версии \textit{ostis-системы}(см. \textit{\ref{sec_ps_components}~\nameref{sec_ps_components}});

\item установка \textit{Ядра sc-моделей интерфейсов}, то есть набора базовых \textit{многократно используемых компонентов пользовательского интерфейса ostis-систем}, необходимых для работы стартовой версии \textit{ostis-системы}(см. \textit{\ref{sec_reusable_UI_components}~\nameref{sec_reusable_UI_components}});

\item установка \textit{системы поддержки коллективной разработки гибридных баз знаний}.
\end{textitemize}

\textit{Процесс разработки базы знаний} включает в себя следующие стадии:
\begin{textitemize}
	\item Формирование начальной структуры \textit{гибридной базы знаний}, которая предполагает:
		\begin{textitemize}
			\item  формирование структуры разделов базы знаний, соответствующей варианту структуризации \textit{базы знаний} с точки зрения разработчиков;
			\item выявление описываемых \textit{предметных областей};
			\item построение иерархической системы описываемых \textit{предметных областей};
			\item построение иерархии разделов \textit{базы знаний} в рамках предметной	части базы знаний, учитывающей построенную на предыдущем этапе иерархию \textit{предметных областей}.
		\end{textitemize}
	\item Выявление \textit{компонентов базы знаний}, которые могут быть заимствованы из \textit{библиотеки многократно используемых компонентов баз знаний}, и включение их в состав разрабатываемой \textit{базы знаний};
	\item Формирование проектных заданий на разработку недостающих \textit{фрагментов базы знаний} и распределение заданий между разработчиками;
	\item Разработка и согласование \textit{фрагментов базы знаний}, которые, в свою	очередь, могут в дальнейшем быть включены в состав \textit{библиотеки многократно используемых компонентов баз знаний};
	\item Верификация и отладка базы знаний.
\end{textitemize}

Следует отметить, что в процессе совершенствования базы знаний этапы 3 - 5 выполняются циклически.

Для обеспечения свойства рефлексивности \textit{интеллектуальной системы}, в частности, возможности автоматизации анализа истории эволюции базы знаний и генерации планов по ее развитию, вся деятельность, связанная с разработкой базы знаний, должна специфицироваться в самой этой базе знаний теми же средствами, что и предметная часть (см. \scncite{Davydenko2018}).

\section{Индивидуальный аспект проектирования и разработки баз знаний ostis-систем}
\label{sec_kb_design_individual}

Для решения задачи индивудуального наполения базы знаний предлагается использовать специализированный инструментарий, который включает в себя различного рода редакторы и трансляторы.

Текущая реализация \textit{ostis-платформы} и решателя задач поддерживает работу с файлами исходных текстов базы знаний. Для создания таких файлов исходных текстов на \textit{SCs-коде} можно воспользоваться любым текстовым редактором.

Для создания файлов исходных текстов в \textit{SCg-коде} может быть использован редактор \textbf{\textit{KBE}} (Knowledge Base source Editor, см. \scncite{Knowledge-base-editor2022}). \textit{KBE} является приложением, которое направлено на помощь в создании и редактировании фрагментов баз знаний интеллектуальных систем, проектирование которых основано на \textit{Технологии OSTIS}. В основу данного редактора положен принцип визуализации данных, хранящихся в базе знаний, что намного упрощает процесс их редактирования и ускоряет процесс проектирования баз знаний.

\textit{пользовательский интерфейс} инструмента представляет собой главное окно, в котором пользователь может создавать вкладки. В каждой вкладке может происходить редактирование различных файлов исходных текстов баз знаний, представленных с помощью \textit{SCg-кода}.

В рамках главного окна имеется панель инструментов и меню приложения. 

Меню приложения представляет собой некоторый набор команд. Команды, которые отображаются в меню делятся на два типа:
\begin{textitemize}
\item команды, которые являются общими для всех вкладок. В частности к ним относятся команды сохранения, загрузки, помощи и так далее;
\item команды, которые специфичны для активной вкладки. Зависят от типа активной вкладки.
\end{textitemize}

На панель инструментов, как и в пользовательских интерфейсах большинства приложений, вынесены наиболее часто используемые команды:
\begin{textitemize}
\item Создать новый файл;
\item Открыть файл;
\item Сохранить;
\item Сохранить как;
\item Закрыть.
\end{textitemize}

Основная идея, которая преследуется в данном редакторе SCg-кода --- это упрощение и ускорение процесса редактирования sc.g-текстов.

В процессе редактирования пользователю доступны различные режимы редактирования.

Всего выделено 4 режима:
\begin{textitemize}
\item Режим выделения и создания узлов.
В данном режиме пользователь может работать со всеми объектами выделяя и перемещая их, вызывая контекстное меню с командами.
Отличительной особенностью данного режима является то, что в нем можно создавать sc.g-узлы;

\item Режим создания sc.g-дуг.
Создание sc.g-дуги начинается с того, что пользователь указывает объект из которого она будет выходить, далее он может указать точки излома дуги, завершается создание указанием конечного объекта.
В процессе создания пользователь может отменять последнее действие (указание начального объекта, точки излома);

\item \textbf{\textit{Режим создания sc.g-шин}}.
sc.g-шины используются для увеличения контактной площади узла, поэтому они могут создаваться лишь для sc.g-узлов. 
Создание шины начинается с указания sc.g-узла, далее как и при создании sc.g-дуг указываются точки излома. 
Как и при создании дуг пользователь может отменять последнее действие нажатием правой клавиши мыши;

\item \textbf{\textit{Режим создания sc.g-контуров}}.
Создание sc.g-контура начинается с указаний первой его точки. Далее, как и в случае с sc.g-дугами и sc.g-шинами, указываются точки.
Стоит отметить, что все объекты, которые попадут внутрь созданного
контура, будут добавлены в него автоматически.
Как и при создании дуг и шин пользователь может отменять последнее действие.
\end{textitemize}

Кроме перечисленных выше команд существует еще целый ряд команд редактирования:
\begin{textitemize}
\item Команда изменения основного текстового идентификатора элемента;
\item Команда изменения типа элемента;
\item Команда установки содержимого.
\end{textitemize}

Полученные файлы исходных текстов в дальнейшем могут быть погружены в \textit{базу знаний }ostis-системы с помощью \textit{Реализации транслятора файлов исходных текстов \textit{базы знаний} в sc-память ostis-платформы}.

\begin{SCn}
\scnheader{Реализация транслятора файлов исходных текстов базы знаний в sc-память ostis-платформы}
\scnidtf{sc-builder}
\scniselement{многократно используемый компонент ostis-систем, хранящийся в виде файлов исходных текстов}
\scnrelfrom{используемый язык}{SCs-код}
\begin{scnrelfromset}{зависимости компонента}
	\scnitem{Библиотека методов и структура данных C++ Standard Library}
\end{scnrelfromset}
\scnrelto{программный компонент}{Программный вариант реализации ostis-платформы}
\end{SCn}

\textit{Реализация транслятора файлов исходных текстов базы знаний в sc-память ostis-платформы} позволяет осуществить сборку \textit{базы знаний} из набора файлов исходных текстов, записанных в SCs-коде с ограничениями в бинарный формат, воспринимаемый \textit{Программной моделью sc-памяти} (см. \textit{\ref{sec_soft_platform_scin_code_example}~\nameref{sec_soft_platform_scin_code_example}}).
При этом возможна как сборка \scnqq{с нуля} (с уничтожением ранее созданного слепка памяти), так и аддитивная сборка, когда информация, содержащаяся в заданном множестве файлов, добавляется к уже имеющемуся слепку состояния памяти.
В текущей реализации сборщик осуществляет \scnqq{склеивание} (\scnqq{слияние}) sc-элементов, имеющих на уровне файлов исходных текстов одинаковые \textit{системные sc-идентификаторы}.

Кроме \textit{KBE} существует редактор текстов базы знаний, являющийся частью \textit{Реализации интерпретатора sc-моделей пользовательских интерфейсов}, обладающий схожим с \textit{KBE} функционалом, но при этом позволяющий редактировать базу знаний в режиме реального времени и без создания файлов исходных текстов базы знаний, именно им рекомендуется пользоваться для редактирования базы знаний.

\section{Коллективный аспект проектирования и разработки баз знаний ostis-систем}
\label{sec_kb_design_collective}

Для решения задачи \textit{коллективной разработки баз знаний} разработана модель деятельности, направленной на создание \textit{гибридных баз знаний} коллективом разработчиков.

Данная модель базируется на модели деятельности различных субъектов и реализована в виде онтологии предметной области деятельности разработчиков, направленной на разработку и модификацию гибридных баз знаний.

Процесс создания и редактирования \textit{базы знаний} \textit{ostis-системы} сводится к формированию разработчиками предложений по редактированию того или иного раздела \textit{базы знаний} и последующему рассмотрению этих предложений администраторами \textit{базы знаний}.

Кроме того, предполагается, что в случае необходимости для верификации поступающих предложений по редактированию базы знаний могут привлекаться эксперты, а управление процессом разработки осуществляется менеджерами соответствующих проектов по разработке базы знаний.
При этом формирование проектных заданий и их спецификация осуществляются также при помощи механизма предложений по редактированию соответствующего раздела базы знаний.

Таким образом, вся информация, связанная с текущими процессами разработки базы знаний, историей и планами ее развития, хранится в той же базе знаний, что и ее предметная часть, то есть часть базы знаний, доступная конечному пользователю системы. Такой подход обеспечивает широкие возможности автоматизации процесса создания баз знаний, а также последующего анализа и совершенствования базы знаний.

Каждое предложение по редактированию базы знаний представляет собой структуру, содержащую sc-текст, который предлагается включить в состав согласованной части базы знаний. В состав таких предложений могут входить знаки действий по редактированию базы знаний, которые автоматически инициируются и выполняются соответствующими агентами после утверждения предложения.

\section{Логико-семантическая модель ostis-системы обнаружения и анализа ошибок и противоречий в базе знаний ostis-системы}
\label{sec_kb_design_contradiction}

Важным этапом в разработке любой системы является контроль ее качества, так как именно на этом этапе определяется степень жизнеспособности и эффективности системы.

Верификация является видом анализа качества и частью процесса разработки системы. Она заключается в проверке информации на правильность и полноту.
Ее целью является выявление ошибок, различных дефектов и недоработок для своевременного их устранения.

Существующие на данный момент методы верификации хорошо развиты и разработано большое количество различных моделей верификации, использующих расширенные таблицы решения, сети Петри, различные логики, например, логики с векторной семантикой (см. \scncite{Arshinskiy2020}) и другие модели. Более того формируются специализированные онтологии для описания многообразия средств и моделей верификации баз знаний (см. \scncite{Rybina2007}). Однако механизма взаимодействия средств, использующих данные методы, нет.

Поэтому средства верификации баз знаний, существующие на данный момент, обладают рядом таких проблем как (см. \scncite{Zhang2023}):
\begin{textitemize}
    \item зависимость от формата представления информации, из-за чего приходится тратить дополнительно время на конвертирование информации;
    \item проблема невозможности быть переиспользованными, так как средства обычно делаются с учетом особенностей конкретной системы;
    \item проблема отсутствия механизма взаимодействия средств верификации и анализа знаний;
    \item высокая роль человека в процессе верификации, так как самым распространенным методом верификации баз знаний является ручная проверка базы экспертом, человек выступает как администратор, принимающий единогласное решение, навязывая свое мнение системе;
    \item современные средства не учитывают и не рассматривают процесс верификации в рамках взаимодействия систем друг с другом.
\end{textitemize}

Эти проблемы могли бы быть решены, если:
\begin{textitemize}
    \item использовать унифицированный и удобный формат представления знаний;
    \item системы создавались бы по общей методологии и были бы совместимы друг с другом;
    \item продумать и реализовать механизм позволяющий системе стремиться самой принимать решение относительно своего состояния и наличия в нем проблемных моментов и ошибок, система может допускать ошибки и не всегда принимать верные решения, но это должны быть ее ошибки, а не экспертов и разработчиков.
\end{textitemize}

Преимуществами \textit{Технологии OSTIS} в рамках задачи верификации являются:
\begin{textitemize}
   \item наличие общей методологии проектирования интеллектуальных систем, позволяющая решить проблему совместимости систем при их коллективном взаимодействии;
   \item все знания представлены в унифицированном виде, что позволяет эффективно их обрабатывать, сводя затраты на конвертирование к минимуму;
   \item средства, с помощью которых производится выявление, анализ и устранение противоречий, описаны в самой базе знаний, а также их спецификация представлены в самой базе знаний системы, тем самым обеспечивая легкость их расширения и позволяя системе знать, каким инструментарием она обладает;
   \item отсутствие семантических эквивалентных фрагментов, что обеспечивает локальность вносимых исправлений и исключает необходимость вносить исправления многократно в разных местах;
   \item многоагентный подход, который позволяет рассматривать средства анализа и верификации баз знаний как коллектив агентов, способных взаимодействовать друг с другом и в дальнейшем принимать общее решение касательно состояния базы знаний. 
\end{textitemize}

Предлагаемый подход сводится к разработке:
\begin{textitemize}
    \item специализированной \textit{предметной области и онтологии}, которая бы содержала в себе все необходимые знания о возможных видах проблемных фрагментов базы знаний и методах их исправления;
    \item алгоритма, позволяющего системе выявить в себе проблемные фрагменты и устранить их, при этом обеспечив согласованность работы средств самой системы;
    \item \textit{специализированного решателя задач}, содержащего необходимые агенты для выявления и устранения проблемных фрагментов.
\end{textitemize}

%% ССылки и объяснение что есть что
Качество базы знаний во многом определяется уровнем наличия/отсутствия \textit{не-факторов} (см. \scncite{Narinjani2004}) в \textit{базе знаний}.
\begin{SCn}
\scnheader{не-фактор}
\scnidtf{группа семантических свойств, определяющих качество информации, хранимой в памяти кибернетической системы}
\begin{scneqtoset}
 \scnitem{корректность/некорректность информации, хранимой в памяти кибернетической системы}
 \scnitem{однозначность/неоднозначность информации, хранимой в памяти кибернетической системы}
 \scnitem{целостность/нецелостность информации, хранимой в памяти кибернетической системы}
 \scnitem{чистота/загрязненность информации, хранимой в памяти кибернетической системы}
 \scnitem{достоверность/недостоверность информации, хранимой в памяти кибернетической системы}
 \scnitem{точность/неточность информации, хранимой в памяти кибернетической системы}
 \scnitem{четкость/нечеткость информации, хранимой в памяти кибернетической системы}
 \scnitem{определенность/недоопределенность информации, хранимой в памяти кибернетической системы}
\end{scneqtoset}
\end{SCn}

\begin{SCn}
\scnheader{проблемная структура}
\scnidtf{структура, описывающая проблемный фрагмент базы знаний}
\scnidtf{структура, описывающая некачественный фрагмент базы знаний}
\begin{scnreltoset}{объединение}
\scnitem{некорректная структура}
\begin{scnindent}
	\scnidtf{структура, содержащая фрагменты, противоречащие каким-либо правилам или закономерностям описанным в базе знаний}
\end{scnindent}
\scnitem{структура, описывающая неполноту в базе знаний}
\begin{scnindent}
	\scnidtf{структура, в которой имеется неполнота (то есть имеется некоторое количество информационных дыр)}
	\scntext{примечание}{Под структурой, описывающей неполноту в базе знаний, понимается структура, содержащая фрагмент базы знаний, в котором отсутствует какая-либо информация, которая необходима (или, по крайней мере, желательна) для однозначного и полного понимания смысла данного фрагмента.}
	%% Переформулировать
\end{scnindent}
\scnitem{информационный мусор}
\begin{scnindent}
	\scnidtf{структура, удаление которой существенно не усложнит деятельность системы}
	\scnidtf{структура, содержащая фрагмент базы знаний, который по каким-либо причинам стал ненужным и требует удаления}
\end{scnindent}
\end{scnreltoset}
\end{SCn}
% Больше про определение противоречия, уточнить что чему противоречит 
\begin{SCn}
\scnheader{противоречие*}
\scnidtf{пара противоречащих друг другу фрагментов информации, хранимой в памяти кибернетической системы*}
\scntext{примечание}{Чаще всего противоречащими друг другу информационными фрагментами являются:
	\begin{scnitemize}
		\item явно представленная в памяти некоторая закономерность (некоторое правило);
		\item информационный фрагмент, не соответствующий (противоречащий) указанной закономерности.
	\end{scnitemize}
	В этом случае некорректность может присутствовать:
	\begin{scnitemize}
		\item либо в информационном фрагменте, который противоречит указанной закономерности;
		\item либо в самой этой закономерности;
		\item либо и там и там.
	\end{scnitemize}
		%% Переформулировать
}
\end{SCn}
	
\begin{SCn}
\scnheader{некорректная структура}
\begin{scnreltoset}{включение}
	\scnitem{дублирование системных идентификаторов}
	\scnitem{несоответствие элементов связки доменам отношения}
	\scnitem{цикл по отношению порядка}
	\scnitem{структура, противоречащая свойству единственности}
\end{scnreltoset}
	
\scnheader{структура, описывающая неполноту в базе знаний}
\begin{scnreltoset}{включение}
	\scnitem{не указан максимальный класс объектов исследования предметной области}
	\scnitem{для сущности указан системный, но не указаны основные идентификаторы для всех внешних языков}
	\scnitem{не указаны домены отношения}
	\scnitem{понятие не соотнесено ни с одной предметной областью}
\end{scnreltoset}
\end{SCn}

%% Больше описать, привести примеры, вообщем пересмотреть

\begin{SCn}
\scnheader{структура, требующее внимание разработчика}
\scnidtf{проблемная структура, для исправления которой требуется участие разработчика}
\scnheader{множество элементов, которые должны быть удалены для исправления структуры*}
\scnidtf{множество элементов, удаление которых из структуры позволяет устранить в ней противоречия}
\scnheader{множество элементов, которые должны быть добавлены для исправления структуры*}
\scnidtf{множество элементов, добавление которых в структуру позволяет устранить в ней противоречия}
\scnheader{структура, которую система не способна исправить сама}
\scnidtf{структура, в которой система не способна автоматически устранить противоречия}
%% Возможно переформулировать на что-нибудь покороче


\scnheader{следует отличать*}
\begin{scnhaselementset}
\scnitem{структура, которую система не может решить сама}
\begin{scnindent}
	\scntext{примечание}{Здесь структура, которую система не может решить сама, не может быть исправлена при взаимодействии с разработчиком и требует полного исправления от самого разработчика}
\end{scnindent}
\scnitem{требующее внимание разработчика}
\begin{scnindent}
	\scntext{примечание}{Здесь структура, требующая внимания разработчика, может быть решена в процессе верификации, но потребуется участие разработчика}
\end{scnindent}
\end{scnhaselementset}

%% Описать идею мезанизма согласования средств верификации

\scnheader{Решатель задач средств выявления и устранения противоречий}
\begin{scnrelfromset}{декомпозиция абстрактного sc-агента}
\scnitem{Неатомарный абстрактный sc-агент выявления противоречий}
	\begin{scnindent}
		\scnidtf{Множество агентов, обеспечивающих поиск и фиксирование противоречий в структуре}
	\end{scnindent}
\scnitem{Неатомарный абстрактный sc-агент устранения противоречий}
	\begin{scnindent}
	\scnidtf{Множество агентов, создающих предложения по исправлению противоречий}
	\scntext{примечание}{Результатом работы таких агентов будут множества предлагаемых к удалению из структуры или добавлению в структуру элементов}
	\end{scnindent}
\scnitem{Абстрактный sc-агент объединения структур}
	\begin{scnindent}
	\scnidtf{Агент, создающий структуру содержащую все элементы сливаемых структур}
	\end{scnindent}
\scnitem{Абстрактный sc-агент применения предложений по устранению противоречий}
\scnitem{Абстрактный sc-агент внесения исправлений в базу знаний}
	\begin{scnindent}
	\scntext{примечание}{Внесение изменений подразумевает не только исправление в базе знаний изначальной проблемной структуры, но и фиксацию самого факта изменения состояния базы знаний}
	\end{scnindent}
\scnitem{Неатомарный абстрактный sc-агент верификации структуры}
	\begin{scnindent}
	\scntext{примечание}{Агент, обеспечивающий полный цикл верификации структуры и координирующий другие агенты}
	\end{scnindent}
\end{scnrelfromset}
\end{SCn}


%\section{Логико-семантическая модель ostis-системы обнаружения и анализа информационных дыр в базе знаний ostis-системы} 
%\label{sec_kb_design_hole}
%Нужно согласовать и возможно объеденить в раздел выше

\section{Логико-семантическая модель ostis-системы автоматизации управления взаимодействием разработчиков различных категорий в процессе проектирования базы знаний ostis-системы}
\label{sec_kb_design_developers}

В первую очередь все пользователи любой \textit{ostis-системы} делятся на \textit{зарегистрированных пользователей} и \textit{незарегистрированных пользователей}.

\begin{SCn}
\scnheader{пользователь базы знаний ostis-системы*}
\scnidtf{бинарное отношение, связывающее sc-модель базы знаний ostis-системы и sc-элемент, обозначающий персону, участвующую в разработке или эксплуатации этой базы знаний}
\scniselement{бинарное отношение}
\scniselement{ориентированное отношение}

\scnrelfrom{разбиение}{\scnkeyword{Типология отношений между базами знаний ostis-систем и их пользователями по наличию факта прохождения регистрации в этих ostis-системах\scnsupergroupsign}}
\begin{scnindent}
	\begin{scneqtoset}
		\scnitem{зарегистрированный пользователь*}
		\scnitem{незарегистрированный пользователь*}
	\end{scneqtoset}
\end{scnindent}
\end{SCn}

Данное отношение отражает связь \textit{пользователя} и \textit{базы знаний} в целом, при этом тот же сам пользователь может быть связан другими более частными отношениями с какими-либо фрагментами этой же \textit{базы знаний}.

\textit{зарегистрированный пользователь} имеет доступ на чтение всей базы знаний и внесение предложений ко всей базе знаний, может выполнять роль конечного пользователя ostis-системы, то есть работать в режиме эксплуатации, а также роль ее разработчика. 

При этом независимо от роли, которую выполняет тот или иной \textit{пользователь}, он может делать предложения по редактированию любой из частей базы знаний, которые в зависимости от его уровня будут либо приняты автоматически, либо будут отдельно рассматриваться.

\begin{SCn}
\scnheader{пользователь, обладающий правом просмотра sc-структуры базы знаний ostis-систем*}
\scnidtf{бинарное отношение, связывающее sc-элемент, обозначающий sc-структуру (например, фрагмент sc-модели базы знаний), и sc-элемент, обозначающий пользователя этой ostis-системы, который обладает правом просмотра этой sc-структуры.}
\scniselement{бинарное отношение}
\scniselement{ориентированное отношение}
\end{SCn}

\textbf{\textit{пользователь, обладающий правом просмотра sc-структуры базы знаний ostis-системы*}} может быть зарегистрирован или не зарегистрирован в \textit{sc-модели базы знаний}.

\begin{SCn}
\scnheader{пользователь, обладающий правом редактирования sc-структуры базы знаний ostis-систем*}
\scnidtf{бинарное отношение, связывающее sc-элемент, обозначающий sc-структуру (например, фрагмент sc-модели базы знаний), и sc-элемент, обозначающий зарегистрированного пользователя ostis-системы, который обладает правом редактирования этой sc-структуры.}
\scniselement{бинарное отношение}
\scniselement{ориентированное отношение}
\scnsuperset{пользователь, обладающий правом просмотра sc-структуры*}
\begin{scnrelfromset}{покрытие}
	\scnitem{пользователь, обладающий правом редактирования sc-структуры посредством формирования предложений по внесению изменений в согласованную часть базы знаний этой ostis-системы*}
	\scnitem{пользователь, обладающий правом редактирования sc-структуры с автоматическим формированием и принятием предложений по внесению изменений в согласованную часть базы знаний этой ostis-системы*}
\end{scnrelfromset}
\end{SCn}

Связки отношения \textit{пользователя, обладающего правом редактирования sc-структуры ostis-системы*} связывают sc-структуру (не обязательно всю sc-модель базы знаний) и пользователя, зарегистрированного в этой sc-модели базы знаний.

\begin{SCn}
\scnheader{разработчик*}
\scnsubset{пользователь, обладающий правом редактирования sc-структуры*}
\scnidtf{бинарное отношение, связывающее sc-элемент, обозначающий некоторый раздел базы знаний (в пределе --- всю базы знаний), и sc-элемент, обозначающий пользователя ostis-системы, который может быть разработчиком данного раздела базы знаний, то есть выполнять проектные задачи в рамках данного раздела}

\scnrelfrom{разбиение}{\scnkeyword{Типология разработчиков баз знаний ostis-систем\scnsupergroupsign}}
\begin{scnindent}
	\begin{scneqtoset}
		\scnitem{администратор*}
		\scnitem{менеджер*}
		\scnitem{эксперт*}
	\end{scneqtoset}
\end{scnindent}

\scnheader{администратор*}
\scnidtf{бинарное отношение, связывающее sc-элемент, обозначающий некоторый раздел базы знаний (в пределе --- всю базы знаний), и sc-элемент, обозначающий пользователя ostis-системы, который является администратором данного раздела базы знаний}
\begin{scnrelfromset}{функции}
	\scnfileitem{контроль целостности и непротиворечивости всей базы знаний}
	\scnfileitem{определение уровней доступа других пользователей}
	\scnfileitem{принятие решения относительно принятия или отклонения предложений в различные части базы знаний, в том числе при необходимости отправка их на экспертизу}
	\scnfileitem{самостоятельное внесение изменений в различные части базы знаний путем использования соответствующих команд редактирования (при этом изменения автоматически оформляются как предложения и заносятся в раздел истории развития ostis-системы)}
\end{scnrelfromset}


\scnheader{менеджер*}
\scnidtf{бинарное отношение, связывающее sc-элемент, обозначающий некоторый раздел базы знаний (в пределе --- всю базы знаний), и sc-элемент, обозначающий персону, которая является менеджером данного раздела базы знаний}
\begin{scnrelfromset}{функции}
	\scnfileitem{планирование объемов работ по разработке базы знаний}
	\scnfileitem{детализация проектных задач на подзадачи, непосредственно формулирование проектных задач, назначение исполнителей проектных задач}
	\scnfileitem{установка приоритетов и сроков выполнения задач}
	\scnfileitem{контроль сроков выполнения проектных задач}
\end{scnrelfromset}


\scnheader{эксперт*}
\scnidtf{бинарное отношение, связывающее sc-элемент, обозначающий какой-либо проект по разработке раздела базы знаний ostis-системы (в общем случае --- всей базы знаний), и sc-элемент, обозначающий персону, которая является экспертом данного раздела базы знаний}
\begin{scnrelfromset}{функции}
	\scnfileitem{верификация результатов выполнения проектных задач}
	\scnfileitem{при необходимости эксперт может оставлять комментарии к любому фрагменту базы знаний относительно его корректности. Все комментарии попадают в раздел, описывающий план развития компьютерной системы}
\end{scnrelfromset}

\end{SCn}

При необходимости разработки объемной \textit{базы знаний} может вводиться иерархия разработчиков, соответствующая иерархии разделов разрабатываемой \textit{базы знаний}.

В этом случае утверждение какого-либо предложения администратором раздела нижнего уровня не приводит к интеграции предложения в соответствующий раздел, а требует рассмотрения администраторами более высокого уровня. Окончательное решение принимается администратором всей базы знаний.

Кроме того, любой участник процесса разработки имеет возможность оставить естественно-языковой комментарий к любому фрагменту или элементу базы знаний, таким образом, может осуществляться обсуждение каких-либо вопросов, связанных с указанным фрагментом или элементом базы знаний. 

Такого рода комментарии попадают в раздел базы знаний текущие процессы развития компьютерной системы.

\section{Многократно используемые компоненты баз знаний ostis-систем}
\label{sc_kb_design_components}

\begin{SCn}
	\begin{scnrelfromlist}{ключевое понятие}
		\scnitem{компонентное проектирование баз знаний интеллектуальных систем}
	\end{scnrelfromlist}
\end{SCn}

\begin{SCn}
	\begin{scnrelfromlist}{библиографическая ссылка}
		\scnitem{\scncite{Ivashenko2011}}
		\scnitem{\scncite{Golenkov2013}}
		\scnitem{\scncite{Davydenko2013}}
	\end{scnrelfromlist}
\end{SCn}

Для широкого применения интеллектуальных систем, способных повысить качество решения
прикладных задач, разработано большое число \textit{баз знаний} по самым различным предметным областям. Однако в большинстве случаев каждая база знаний разрабатывается отдельно и независимо от других, в отсутствие единой унифицированной формальной основы для представления знаний, а также единых принципов формирования систем понятий для описываемой предметной области. В связи с этим разработанные базы оказываются, как правило, несовместимы между собой и не пригодны для повторного использования. Для быстрой разработки достаточного количества баз знаний, кроме наличия средств разработки интеллектуальных систем, обеспечивающих разработку и проектирование различных компонентов интеллектуальной системы, включая базу знаний, требуется наличие соответствующей отлаженной технологии проектирования баз знаний (\textit{см. \scncite{Ivashenko2011}}). 

Компонентный подход к разработке интеллектуальных компьютерных систем, реализуемый в виде \textbf{\textit{библиотеки многократно используемых компонентов ostis-систем}}, позволяет решить описанные проблемы (см. \textit{\ref{ostis_library_section}~\nameref{ostis_library_section}}). \textit{Библиотека многократно используемых компонентов баз знаний ostis-систем в составе Метасистемы OSTIS} является важнейшим фрагментом \textit{Метасистемы OSTIS}, который обеспечивает надежность и совместимость проектируемых фрагментов баз знаний, а также повышение скорости разработки \textit{баз знаний} \textit{интеллектуальных компьютерных систем}.

Описываемая библиотека включает в себя множество компонентов баз знаний и их спецификаций.

Рассмотрим классификацию многократно используемых компонентов баз знаний ostis-систем\\ (\textit{см. \scncite{Davydenko2013}}).

\begin{SCn}
\scnheader{многократно используемый компонент баз знаний ostis-систем}
\scnsuperset{предметная область и онтология}
\begin{scnindent}
\scnsubset{раздел базы знаний}
\end{scnindent}
\scnsuperset{семантическая окрестность}
\begin{scnindent}
\scnsuperset{семантическая окрестность по инцидентным коннекторам}
\scnsuperset{полная семантическая окрестность}
\scnsuperset{базовая семантическая окрестность}
\scnsuperset{специализированная семантическая окрестность}
\end{scnindent}
\scnsuperset{базовые фрагменты предметных областей и онтологий}
\begin{scnindent}
\scnnote{Базовый фрагмент предметной области и онтологии включает в себя теоретико-множественную, логическую онтологии, а также терминологические фрагменты.}
\scnnote{Данный вид многократно используемых компонентов позволяет использовать только те знания, которые непосредственно необходимы для функционирования интеллектуальных систем, исключив то, что никак не влияет на работу конечной системы (пояснения, примеры, дидактический материал и так далее).}
\scnhaselement{Базовый фрагмент теории логических формул, высказываний и логических sc-языков}
\scnhaselement{Базовый фрагмент теории множеств}
\scnhaselement{Базовый фрагмент теории связок и отношений}
\end{scnindent}
\scnsuperset{база знаний}
\begin{scnindent}
\scnnote{Целые базы знаний могут быть многократно используемыми компонентами в случае разработки интеллектуальных систем, назначение которых совпадает.}
\end{scnindent}
\end{SCn}

Важнейшими компонентами, которые входят в состав библиотеки, являются \textit{онтологии предметных областей}. \textit{онтологии предметных областей}, описывающие виды знаний, которые являются основой для построения \textit{базы знаний} любой \textit{интеллектуальной системы}, входят в \textit{Ядро базы знаний}, поскольку являются \textit{онтологиями верхнего уровня} (см. \textit{\ref{sec_top_level_ontologies}~\nameref{sec_top_level_ontologies}}). Следовательно, \textit{Ядро базы знаний} представляет собой компонент, входящий в состав каждой базы знаний, разрабатываемой по \textit{Технологии OSTIS}, и устанавливающийся в первую очередь.

\begin{SCn}
\scnheader{многократно используемый компонент баз знаний ostis-систем}
\scnhaselement{Ядро базы знаний}
\begin{scnindent}
\scnhaselement{Предметная область и онтология множеств}
\scnhaselement{Предметная область и онтология связок и отношений}
\scnhaselement{Предметная область и онтология структур}
\scnhaselement{Предметная область и онтология семантических окрестностей}   
\scnhaselement{Предметная область и онтология предметных областей}
\scnhaselement{Предметная область и онтология онтологий}
\end{scnindent}
\end{SCn}

\textit{онтологии предметных областей}, которые используются в большинстве интеллектуальных систем, являются частью \textit{Расширенного ядра базы знаний}.

\begin{SCn}
\scnheader{многократно используемый компонент баз знаний ostis-систем}
\scnhaselement{Расширенное ядро базы знаний}
\begin{scnindent}
\scnrelfrom{включение}{Ядро базы знаний}
\scnexplanation{В отличие от \textit{Ядра базы знаний} \textit{Расширенное ядро базы знаний} содержит в себе не только обязательные для установки \textit{онтологии предметных областей}, но и такие \textit{онтологии предметных областей}, которые используются в большинстве \textit{интеллектуальных компьютерных систем}. Следовательно, являются компонентами, которые наиболее часто устанавливаются пользователями \textit{Библиотеки многократно используемых компонентов баз знаний ostis-систем в составе Метасистемы OSTIS}.}
\scnhaselement{Предметная область и онтология параметров, величин и шкал}
\scnhaselement{Предметная область и онтология чисел и числовых структур}
\scnhaselement{Предметная область и онтология темпоральных сущностей}
\scnhaselement{Предметная область и онтология пространственных сущностей различных форм}
\scnhaselement{Предметная область и онтология материальных сущностей}
\end{scnindent}
\end{SCn}

Подробное описание представленных \textit{многократно используемых компонентов баз знаний} можно найти в \textit{Главе \ref{chapter_top_ontologies}~\nameref{chapter_top_ontologies}}, а также в \textit{Главе \ref{chapter_kb}~\nameref{chapter_kb}}.

Представленный список \textit{многократно используемых компонентов баз знаний} не является окончательным. В случае, когда разработчик \textit{базы знаний} интеллектуальной системы считает, что разработанный им компонент сможет стать неотъемлемой частью библиотеки, то компонент будет добавлен в библиотеку, как многократно используемый, если:
\begin{textitemize}
	\item компонент специфицирован;
	\item компонент прошел верификацию и соответствует требованиям разработчиков библиотеки	(см. \textit{\ref{reusable_component_section}~\nameref{reusable_component_section}}).
\end{textitemize}

Чтобы \textit{многократно используемый компонент баз знаний} мог быть принят в библиотеку, он должен быть корректно специфицирован. Для этого используются отношения класса \textit{необходимое для установки отношение, специфицирующее многократно используемый компонент ostis-систем}, а также \textit{необязательное для установки отношение, специфицирующее многократно используемый компонент ostis-систем}. В \textit{\ref{reusable_component_section}~\nameref{reusable_component_section}} описана спецификация, общая для любых типов компонентов. Однако в зависимости от типа компонента, спецификация может расширяться. Рассмотрим \textit{необязательное для установки отношение, специфицирующее многократно используемый компонент баз знаний ostis-систем}, его поиска и установки в \textit{дочернюю ostis-систему}, если таковым компонентом является \textit{предметная область и онтология}.

\begin{SCn}
\scnheader{необязательное для установки отношение, специфицирующее многократно используемый компонент баз знаний ostis-систем}
\scnsubset{необязательное для установки отношение, специфицирующее многократно используемый компонент ostis-систем}
\scnhaselement{максимальный класс объектов исследования'}
\begin{scnindent}
\scnidtf{класс объектов исследования, для которого в заданной предметной области отсутствует другой класс объектов исследования, который был бы его надмножеством'}
\scnrelfrom{первый домен}{предметная область и онтология}
\scnrelfrom{второй домен}{понятие}
\end{scnindent}
\scnhaselement{немаксимальный класс объектов исследования'}
\begin{scnindent}
\scnrelfrom{первый домен}{предметная область и онтология}
\scnrelfrom{второй домен}{понятие}
\end{scnindent}
\scnhaselement{исследуемое отношение'}
\begin{scnindent}
\scnrelfrom{первый домен}{предметная область и онтология}
\scnrelfrom{второй домен}{понятие}
\end{scnindent}
\end{SCn}

Для компонентов, которые являются частью \textit{Библиотеки многократно используемых компонентов баз знаний ostis-систем в составе Метасистемы OSTIS}, также существуют средства поиска, обновления (см. \textit{\ref{ostis_library_component_manager}~\nameref{ostis_library_component_manager}}). 

Корректно спроектированные спецификации компонентов позволят построить полную иерархию зависимостей компонентов, а также их структуру, что в свою очередь позволит беспрепятственное использование компонентов и их фрагментов в рамках компонентного проектирования баз знаний.

\section*{Заключение к Главе \ref{chapter_kb_design}}
\label{kb_design_conclusion}

Проектирование и анализ качества \textit{баз знаний} являются важнейшими этапами разработки \textit{интеллектуальных компьютерных систем}, так как они во многом определяют качество всей интеллектуальной системы.

Предложенная методология коллективной разработки базы знаний на основе \textit{Технологии OSTIS}, которая включает в себя модель верификации и контроля качества \textit{базы знаний}, а также \textit{компонентный подход} к проектированию \textit{баз знаний}, позволяет повысить эффективность проектирования \textit{интеллектуальных компьютерных систем} и средств автоматизации разработки таких систем.
%%%%%%%%%%%%%%%%%%%%%%%%% referenc.tex %%%%%%%%%%%%%%%%%%%%%%%%%%%%%%
% sample references
% %
% Use this file as a template for your own input.
%
%%%%%%%%%%%%%%%%%%%%%%%% Springer-Verlag %%%%%%%%%%%%%%%%%%%%%%%%%%
%
% BibTeX users please use
% \bibliographystyle{}
% \bibliography{}
%
\biblstarthook{In view of the parallel print and (chapter-wise) online publication of your book at \url{www.springerlink.com} it has been decided that -- as a genreral rule --  references should be sorted chapter-wise and placed at the end of the individual chapters. However, upon agreement with your contact at Springer you may list your references in a single seperate chapter at the end of your book. Deactivate the class option \texttt{sectrefs} and the \texttt{thebibliography} environment will be put out as a chapter of its own.\\\indent
References may be \textit{cited} in the text either by number (preferred) or by author/year.\footnote{Make sure that all references from the list are cited in the text. Those not cited should be moved to a separate \textit{Further Reading} section or chapter.} If the citatiion in the text is numbered, the reference list should be arranged in ascending order. If the citation in the text is author/year, the reference list should be \textit{sorted} alphabetically and if there are several works by the same author, the following order should be used:
\begin{enumerate}
\item all works by the author alone, ordered chronologically by year of publication
\item all works by the author with a coauthor, ordered alphabetically by coauthor
\item all works by the author with several coauthors, ordered chronologically by year of publication.
\end{enumerate}
The \textit{styling} of references\footnote{Always use the standard abbreviation of a journal's name according to the ISSN \textit{List of Title Word Abbreviations}, see \url{http://www.issn.org/en/node/344}} depends on the subject of your book:
\begin{itemize}
\item The \textit{two} recommended styles for references in books on \textit{mathematical, physical, statistical and computer sciences} are depicted in ~\cite{science-contrib, science-online, science-mono, science-journal, science-DOI} and ~\cite{phys-online, phys-mono, phys-journal, phys-DOI, phys-contrib}.
\item Examples of the most commonly used reference style in books on \textit{Psychology, Social Sciences} are~\cite{psysoc-mono, psysoc-online,psysoc-journal, psysoc-contrib, psysoc-DOI}.
\item Examples for references in books on \textit{Humanities, Linguistics, Philosophy} are~\cite{humlinphil-journal, humlinphil-contrib, humlinphil-mono, humlinphil-online, humlinphil-DOI}.
\item Examples of the basic Springer style used in publications on a wide range of subjects such as \textit{Computer Science, Economics, Engineering, Geosciences, Life Sciences, Medicine, Biomedicine} are ~\cite{basic-contrib, basic-online, basic-journal, basic-DOI, basic-mono}. 
\end{itemize}
}

\begin{thebibliography}{99.}%
% and use \bibitem to create references.
%
% Use the following syntax and markup for your references if 
% the subject of your book is from the field 
% "Mathematics, Physics, Statistics, Computer Science"
%
% Contribution 
\bibitem{science-contrib} Broy, M.: Software engineering --- from auxiliary to key technologies. In: Broy, M., Dener, E. (eds.) Software Pioneers, pp. 10-13. Springer, Heidelberg (2002)
%
% Online Document
\bibitem{science-online} Dod, J.: Effective substances. In: The Dictionary of Substances and Their Effects. Royal Society of Chemistry (1999) Available via DIALOG. \\
\url{http://www.rsc.org/dose/title of subordinate document. Cited 15 Jan 1999}
%
% Monograph
\bibitem{science-mono} Geddes, K.O., Czapor, S.R., Labahn, G.: Algorithms for Computer Algebra. Kluwer, Boston (1992) 
%
% Journal article
\bibitem{science-journal} Hamburger, C.: Quasimonotonicity, regularity and duality for nonlinear systems of partial differential equations. Ann. Mat. Pura. Appl. \textbf{169}, 321--354 (1995)
%
% Journal article by DOI
\bibitem{science-DOI} Slifka, M.K., Whitton, J.L.: Clinical implications of dysregulated cytokine production. J. Mol. Med. (2000) doi: 10.1007/s001090000086 
%
\bigskip

% Use the following (APS) syntax and markup for your references if 
% the subject of your book is from the field 
% "Mathematics, Physics, Statistics, Computer Science"
%
% Online Document
\bibitem{phys-online} J. Dod, in \textit{The Dictionary of Substances and Their Effects}, Royal Society of Chemistry. (Available via DIALOG, 1999), 
\url{http://www.rsc.org/dose/title of subordinate document. Cited 15 Jan 1999}
%
% Monograph
\bibitem{phys-mono} H. Ibach, H. L\"uth, \textit{Solid-State Physics}, 2nd edn. (Springer, New York, 1996), pp. 45-56 
%
% Journal article
\bibitem{phys-journal} S. Preuss, A. Demchuk Jr., M. Stuke, Appl. Phys. A \textbf{61}
%
% Journal article by DOI
\bibitem{phys-DOI} M.K. Slifka, J.L. Whitton, J. Mol. Med., doi: 10.1007/s001090000086
%
% Contribution 
\bibitem{phys-contrib} S.E. Smith, in \textit{Neuromuscular Junction}, ed. by E. Zaimis. Handbook of Experimental Pharmacology, vol 42 (Springer, Heidelberg, 1976), p. 593
%
\bigskip
%
% Use the following syntax and markup for your references if 
% the subject of your book is from the field 
% "Psychology, Social Sciences"
%
%
% Monograph
\bibitem{psysoc-mono} Calfee, R.~C., \& Valencia, R.~R. (1991). \textit{APA guide to preparing manuscripts for journal publication.} Washington, DC: American Psychological Association.
%
% Online Document
\bibitem{psysoc-online} Dod, J. (1999). Effective substances. In: The dictionary of substances and their effects. Royal Society of Chemistry. Available via DIALOG. \\
\url{http://www.rsc.org/dose/Effective substances.} Cited 15 Jan 1999.
%
% Journal article
\bibitem{psysoc-journal} Harris, M., Karper, E., Stacks, G., Hoffman, D., DeNiro, R., Cruz, P., et al. (2001). Writing labs and the Hollywood connection. \textit{J Film} Writing, 44(3), 213--245.
%
% Contribution 
\bibitem{psysoc-contrib} O'Neil, J.~M., \& Egan, J. (1992). Men's and women's gender role journeys: Metaphor for healing, transition, and transformation. In B.~R. Wainrig (Ed.), \textit{Gender issues across the life cycle} (pp. 107--123). New York: Springer.
%
% Journal article by DOI
\bibitem{psysoc-DOI}Kreger, M., Brindis, C.D., Manuel, D.M., Sassoubre, L. (2007). Lessons learned in systems change initiatives: benchmarks and indicators. \textit{American Journal of Community Psychology}, doi: 10.1007/s10464-007-9108-14.
%
%
% Use the following syntax and markup for your references if 
% the subject of your book is from the field 
% "Humanities, Linguistics, Philosophy"
%
\bigskip
%
% Journal article
\bibitem{humlinphil-journal} Alber John, Daniel C. O'Connell, and Sabine Kowal. 2002. Personal perspective in TV interviews. \textit{Pragmatics} 12:257--271
%
% Contribution 
\bibitem{humlinphil-contrib} Cameron, Deborah. 1997. Theoretical debates in feminist linguistics: Questions of sex and gender. In \textit{Gender and discourse}, ed. Ruth Wodak, 99--119. London: Sage Publications.
%
% Monograph
\bibitem{humlinphil-mono} Cameron, Deborah. 1985. \textit{Feminism and linguistic theory.} New York: St. Martin's Press.
%
% Online Document
\bibitem{humlinphil-online} Dod, Jake. 1999. Effective substances. In: The dictionary of substances and their effects. Royal Society of Chemistry. Available via DIALOG. \\
http://www.rsc.org/dose/title of subordinate document. Cited 15 Jan 1999
%
% Journal article by DOI
\bibitem{humlinphil-DOI} Suleiman, Camelia, Daniel C. O'Connell, and Sabine Kowal. 2002. `If you and I, if we, in this later day, lose that sacred fire...': Perspective in political interviews. \textit{Journal of Psycholinguistic Research}. doi: 10.1023/A:1015592129296.
%
%
%
\bigskip
%
%
% Use the following syntax and markup for your references if 
% the subject of your book is from the field 
% "Computer Science, Economics, Engineering, Geosciences, Life Sciences"
%
%
% Contribution 
\bibitem{basic-contrib} Brown B, Aaron M (2001) The politics of nature. In: Smith J (ed) The rise of modern genomics, 3rd edn. Wiley, New York 
%
% Online Document
\bibitem{basic-online} Dod J (1999) Effective Substances. In: The dictionary of substances and their effects. Royal Society of Chemistry. Available via DIALOG. \\
\url{http://www.rsc.org/dose/title of subordinate document. Cited 15 Jan 1999}
%
% Journal article by DOI
\bibitem{basic-DOI} Slifka MK, Whitton JL (2000) Clinical implications of dysregulated cytokine production. J Mol Med, doi: 10.1007/s001090000086
%
% Journal article
\bibitem{basic-journal} Smith J, Jones M Jr, Houghton L et al (1999) Future of health insurance. N Engl J Med 965:325--329
%
% Monograph
\bibitem{basic-mono} South J, Blass B (2001) The future of modern genomics. Blackwell, London 
%
\end{thebibliography}
