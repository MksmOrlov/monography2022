\chapter{Интеллектуальные компьютерные системы нового поколения}
\chapauthortoc{Голенков В.~В.\\Шункевич Д.~В.\\Ковалёв М.~В.\\Садовский М.~Е.}
\label{chapter_new_generation_systems} 

\vspace{-7\baselineskip}

\begin{SCn}
\begin{scnrelfromlist}{автор}
	\scnitem{Голенков В.~В.}
	\scnitem{Шункевич Д.~В.}
	\scnitem{Ковалёв М.~В.}
	\scnitem{Садовский М.~Е.}
\end{scnrelfromlist}

\bigskip

\scntext{аннотация}{В главе рассмотрены принципы построения \textit{интеллектуальных компьютерных систем нового поколения}. В качестве ключевых свойств \textit{интеллектуальных систем нового поколения} выделяются их \textit{самообучаемость}, \textit{интероперабельность} и \textit{семантическая совместимость}. В главе рассматривается подход к обеспечению указанных свойств на основе \textit{смыслового представления информации} и \textit{многоагентных моделей обработки информации}.}

\bigskip

\begin{scnrelfromlist}{подраздел}
	\scnitem{\ref{sec_ngics_req}~\nameref{sec_ngics_req}}
	\scnitem{\ref{sec_ngics_sense_principles}~\nameref{sec_ngics_sense_principles}}
	\scnitem{\ref{sec_ngics_ps}~\nameref{sec_ngics_ps}}
	\scnitem{\ref{sec_ngics_ui}~\nameref{sec_ngics_ui}}
\end{scnrelfromlist}

\bigskip

\begin{scnrelfromlist}{ключевое понятие}
	\scnitem{интеллектуальная компьютерная система нового поколения}
	\scnitem{интероперабельная интеллектуальная компьютерная система}
	\scnitem{самообучаемая интеллектуальная компьютерная система}
	\scnitem{семантическая сеть}
	\scnitem{многоагентная система обработки информации в общей памяти}
\end{scnrelfromlist}

\begin{scnrelfromlist}{ключевой знак}
	\scnitem{Технология OSTIS}
\end{scnrelfromlist}

\begin{scnrelfromlist}{библиографическая ссылка}
	\scnitem{\scncite{Yaghoobirafi2022}}
	\scnitem{\scncite{Ouksel1999}}
	\scnitem{\scncite{Lanzenberger2008}}
	\scnitem{\scncite{Neiva2016}}
	\scnitem{\scncite{Pohl2004}}
	\scnitem{\scncite{Waters2009}}
	\scnitem{\scncite{Lopes2022}}
	\scnitem{\scncite{Hamilton2006}}
	\scnitem{\scncite{Barinov2021}}
	\scnitem{\scncite{Kharlamov2011}}
	\scnitem{\scncite{Moskin2011}}
	\scnitem{\scncite{Timchenko2013}}
	\scnitem{\scncite{Massel2013}}
	\scnitem{\scncite{Efimenko2014}}
	\scnitem{\scncite{Tarasov2014}}
	\scnitem{\scncite{Golenkov2014}}
	\scnitem{\scncite{Zagorulko2014}}
	\scnitem{\scncite{Golenkov2019}}
	\scnitem{\scncite{Golenkov2020}}
	\scnitem{\scncite{Martynov}}
	\scnitem{\scncite{Letichevskij2003}}
\end{scnrelfromlist}
	
\end{SCn}

\section*{Введение в Главу \ref{chapter_new_generation_systems}}

Важнейшим направлением повышения уровня \textit{интеллекта} \textit{индивидуальных интеллектуальных кибернетических систем} является переход к \textit{коллективам индивидуальных интеллектуальных кибернетических систем} и далее к \textit{иерархическим коллективам интеллектуальных кибернетических систем}, членами которых являются как \textit{индивидуальные интеллектуальные кибернетические системы}, так и \textit{коллективы индивидуальных интеллектуальных кибернетических систем}, а также \textit{иерархические коллективы интеллектуальных кибернетических систем}. 

Аналогичным образом необходимо повышать уровень интеллекта и \textit{индивидуальных интеллектуальных \myuline{компью}-\myuline{терных} систем} (искусственных \textit{кибернетических систем}). Но при этом надо помнить, что далеко не каждое объединение \textit{\myuline{интеллектуальных} кибернетических систем} (в том числе и \textit{компьютерных систем}) становится \textit{интеллектуальным коллективом}. Для этого необходимо соблюдение дополнительных требований, предъявляемых \myuline{ко всем} членам \textit{интеллектуальных коллективов}. Важнейшим из них является требование высокого уровня \textbf{\textit{интероперабельности}}, то есть способности к эффективному взаимодействию с другими членами коллектива. Переход от современных \textit{интеллектуальных компьютерных систем} к \textit{интероперабельным интеллектуальным компьютерным системам} является ключевым фактором перехода к \textbf{\textit{интеллектуальным компьютерным системам нового поколения}}, обеспечивающим существенное повышение уровня автоматизации человеческой деятельности.

Расширение областей применения \textit{интеллектуальных компьютерных систем} требует перехода к решению \myuline{комплексных} задач --- задач, решение которых невозможно с помощью одной модели решения задач, одного вида знаний, одной интеллектуальной компьютерной системы.

Для решения комплексных задач необходим:
\begin{textitemize}
	\item Переход к \textit{гибридным} индивидуальным интеллектуальным компьютерным системам, в которых осуществляется конвергенция и интеграция различных моделей решения задач и различных видов знаний;	
	\item Переход к \myuline{коллективам} семантически совместимых самостоятельных интеллектуальных компьютерных систем, в которых обеспечивается:
	\begin{textitemize}
		\item \textit{интероперабельность} объединяемых интеллектуальных компьютерных систем,
		\item конвергенция объединяемых интеллектуальных компьютерных систем при сохранении их самостоятельности.
	\end{textitemize}
\end{textitemize}

\section{Требования, предъявляемые к интеллектуальным компьютерным системам нового поколения}
\label{sec_ngics_req}
\begin{SCn}
\begin{scnrelfromlist}{ключевое понятие}
	\scnitem{интеллектуальная компьютерная система нового поколения}
	\scnitem{интероперабельная интеллектуальная компьютерная система}
	\scnitem{гибридная интеллектуальная компьютерная система}
\end{scnrelfromlist}

\begin{scnrelfromlist}{ключевое отношение}
	\scnitem{соединение интеллектуальных компьютерных систем*}
	\begin{scnindent}
		\scnidtf{преобразование множества интеллектуальных компьютерных систем в коллектив, членами (агентами) которого являются эти системы*}
	\end{scnindent}
	\scnitem{глубокая интеграция интеллектуальных компьютерных систем*}
	\begin{scnindent}
		\scnidtf{быть результатом преобразования множества индивидуальных интеллектуальных компьютерных систем в одну интегрированную индивидуальную интеллектуальную компьютерную систему*}
	\end{scnindent}
\end{scnrelfromlist}

\begin{scnrelfromlist}{ключевой параметр}
	\scnitem{интероперабельность интеллектуальных компьютерных систем\scnsupergroupsign}
	\scnitem{семантическая совместимость пар интеллектуальных компьютерных систем\scnsupergroupsign}
\end{scnrelfromlist}

\begin{scnrelfromlist}{ключевое знание}
	\scnitem{Требования, предъявляемые к интеллектуальным компьютерным системам нового поколения}
	\scnitem{Принципы, лежащие в основе интеллектуальных компьютерных систем нового поколения}
	\scnitem{Отличие данных от знаний}
\end{scnrelfromlist}

\begin{scnrelfromlist}{библиографическая ссылка}
	\scnitem{\scncite{Yaghoobirafi2022}}
	\scnitem{\scncite{Ouksel1999}}
	\scnitem{\scncite{Lanzenberger2008}}
	\scnitem{\scncite{Neiva2016}}
	\scnitem{\scncite{Pohl2004}}
	\scnitem{\scncite{Waters2009}}
	\scnitem{\scncite{Lopes2022}}
	\scnitem{\scncite{Hamilton2006}}
	\scnitem{\scncite{Barinov2021}}
\end{scnrelfromlist}
\end{SCn}

Создание различных комплексов взаимодействующих \textit{интеллектуальных компьютерных систем} \myuline{требует} повышения качества не только самих этих систем, но также и качества их взаимодействия. \textit{Интеллектуальные компьютерные системы нового поколения} должны иметь высокий уровень \textbf{\textit{интероперабельности}}, то есть высокий уровень способности к эффективному, целенаправленному взаимодействию с себе подобными и с пользователями в процессе коллективного (распределённого) и децентрализованного решения сложных задач (см.~\scncite{Yaghoobirafi2022,Ouksel1999,Lanzenberger2008,Neiva2016,Pohl2004,Waters2009}). Уровень \textbf{\textit{интероперабельности}} \textit{интеллектуальных компьютерных систем} --- это, образно говоря, уровень их \scnqq{социализации}, полезности в рамках различных априори неизвестных сообществ (коллективов) \textit{интеллектуальных систем}.

Уровень \textbf{\textit{интероперабельности}} \textit{интеллектуальных компьютерных систем} --- это уровень их коммуникационной (социальной) совместимости, позволяющей им \myuline{самостоятельно} формировать коллективы \textit{интеллектуальных компьютерных систем} и их пользователей, а также \myuline{самостоятельно} согласовывать и координировать свою деятельность в рамках этих коллективов при решении сложных задач в частично предсказуемых условиях. Повышение уровня \textit{интероперабельности} интеллектуальных компьютерных систем определяет переход к \textbf{\textit{интеллектуальным компьютерным системам нового поколения}}, без которых невозможна реализация таких проектов, как \textit{интеллектуальное-предприятие}, \textit{интеллектуальная-больница}, \textit{интеллектуальная-школа}, \textit{интеллектуальный-университет}, \textit{интеллектуальная-кафедра}, \textit{интеллектуальный-дом}, \textit{интеллектуальный-город}, \textit{интеллектуаль\-ное-общество} (см. \scncite{Lopes2022,Hamilton2006}).

\begin{SCn}
\scnheader{интеллектуальная компьютерная система}
\scnidtf{интеллектуальная искусственная кибернетическая система}
\begin{scnrelfromset}{разбиение}
	\scnitem{индивидуальная интеллектуальная компьютерная система}
	\scnitem{интеллектуальный коллектив интеллектуальных компьютерных систем}
	\begin{scnindent}
		\scnidtf{интеллектуальная \textit{многоагентная система}, агенты которой являются \textit{интеллектуальными компьютерными системами}}
		\scntext{примечание}{Не каждый \textit{коллектив интеллектуальных компьютерных систем} может оказаться интеллектуальным, поскольку уровень интеллекта такого коллектива определяется не только уровнем интеллекта его членов, но также и эффективностью (качеством) \myuline{их взаимодействия}.}
		\begin{scnrelfromset}{разбиение}
			\scnitem{интеллектуальный коллектив \myuline{индивидуальных} интеллектуальных компьютерных систем}
			\scnitem{иерархический интеллектуальный коллектив интеллектуальных компьютерных систем}
			\begin{scnindent}
				\scnidtf{\textit{интеллектуальный коллектив интеллектуальных компьютерных систем}, по крайней мере одним из членов которого является \textit{интеллектуальный коллектив интеллектуальных компьютерных систем}}
			\end{scnindent}
		\end{scnrelfromset}
	\end{scnindent}
\end{scnrelfromset}
\end{SCn}

\begin{SCn}
\scnheader{интеллектуальные компьютерные системы нового поколения}
\begin{scnrelfromlistcustom}{предъявляемые требования}
	\scnitemcustom{высокий уровень \textit{интероперабельности}}
	\scnitemcustom{высокий уровень \textit{обучаемости}}
	\scnitemcustom{высокий уровень \textit{гибридности}}
	\scnitemcustom{высокий уровень способности решать \textit{интеллектуальные задачи} (то есть \textit{задачи}, \textit{методы} решения которых и/или требуемая для их решения исходная информация априори неизвестны)}
	\scnitemcustom{высокий уровень \textit{синергетичности}}
\end{scnrelfromlistcustom}

\scnheader{интероперабельность\scnsupergroupsign}
\scnidtf{способность к эффективному (целенаправленному) взаимодействию с другими самостоятельными субъектами}
\scnidtf{способность к партнёрскому взаимодействию в решении \textit{комплексных задач}, требующих \textit{коллективной деятельности}}
\scnidtf{способность работать в коллективе (в команде)}
\scnidtf{уровень социализации}
\scnidtf{social skills}

\scnheader{высокий уровень интероперабельности}
\begin{scnrelfromlistcustom}{обеспечивается}
	\scnitemcustom{высоким уровнем \textit{взаимопонимания}}
	\begin{scnindent}
		\begin{scnrelfromlistcustom}{обеспечивается}
			\scnitemcustom{высоким уровнем \textbf{\textit{семантической совместимости}} заданного субъекта с другими субъектами заданного коллектива}
			\scnitemcustom{высоким уровнем \textit{способности понимать} сообщения и поведение партнеров}
			\scnitemcustom{высоким уровнем \textit{способности быть понятной} для партнеров:
				\begin{itemize}[labelsep=\tabsize-\bulletsize,leftmargin=\tabsize,label=$\bullet$]
					\item способности понятно и обоснованно формулировать свои предложения и информацию, полезную для решения текущих задач
					\item способности действовать и комментировать свои действия так, чтобы они и их мотивы были понятны партнерам
			\end{itemize}}
			
			\scnitemcustom{высоким уровнем \textit{способности к повышению уровня семантической совместимости} со своими партнёрами}
		\end{scnrelfromlistcustom}
	\end{scnindent}
	\scnitemcustom{высоким уровнем \textit{договороспособности}, то есть способности согласовывать с партнёрами свои планы и намерения в целях своевременного обеспечения высокого качества коллективного результата}
	\scnitemcustom{высоким уровнем \textit{способности к децентрализованной координации} своих действий с действиями партнёров в непредсказуемых (нештатных) обстоятельствах}
	\scnitemcustom{высоким уровнем способности разделять ответственность с партнёрами}
	\scnitemcustom{высоким уровнем \textit{способности к минимизации негативных последствий конфликтных ситуаций} с другими субъектами}
	\begin{scnindent}
		\begin{scnrelfromlistcustom}{обеспечивается}
			\scnitemcustom{высоким уровнем \textit{способности к предотвращению возникновения конфликтных ситуаций}}
			\scnitemcustom{\textit{соблюдением этических норм} и правил, препятствующих возникновению разрушительных последствий конфликтных ситуаций}
			\scnitemcustom{высоким уровнем \textit{способности разделять ответственность} с партнерами за своевременное и качественное достижение общей цели}
		\end{scnrelfromlistcustom}
	\end{scnindent}
\end{scnrelfromlistcustom}

\scnheader{семантическая совместимость\scnsupergroupsign}
\scnidtf{степень согласованности (совпадения) систем \textit{понятий} и других \textit{ключевых знаков}, используемых заданными взаимодействующими субъектами}
\scntext{примечание}{Обеспечение \textit{семантической совместимости} требует формализации \textit{смыслового представления информации}.}
\end{SCn}

\noindent
\textbf{\textit{способность разделять ответственность}} с партнерами, являющаяся необходимым условием децентрализованного управления коллективной деятельностью
\begin{SCn}
\begin{scnrelfromlistcustom}{обеспечивается}
	\scnitemcustom{\textit{способностью к мониторингу} и анализу коллективно выполняемой деятельности}
	\scnitemcustom{\textit{способностью оперативно информировать партнеров} о неблагоприятных ситуациях, событиях, тенденциях, а также инициировать соответствующие коллективные действия}
\end{scnrelfromlistcustom}

\begin{SCn}
	\scnheader{высокий уровень обучаемости интеллектуальной компьютерной системы нового поколения}
	\scnexplanation{Важнейшим направлением повышения уровня автоматизации человеческой деятельности является повышение уровня автоматизации не только проектирования интеллектуальной компьютерной системы, но и комплексной поддержки всех остальных этапов жизненного цикла \textit{интеллектуальной компьютерной системы}. В частности, это касается модернизации (совершенствования, реинжиниринга) интеллектуальной компьютерной системы непосредственно в ходе их эксплуатации. Для того, чтобы обеспечить высокий уровень автоматизации такой модернизации, необходимо существенно повысить \textbf{\textit{уровень самообучаемости}} \textit{интеллектуальной компьютерной системы} для того, что они сами (самостоятельно) могли себя модернизировать (самосовершенствовать) входе своего целевого функционирования.}
\end{SCn}

\scnheader{высокий уровень обучаемости}
\begin{scnrelfromlistcustom}{обеспечивается}
	\scnitemcustom{высоким уровнем \textit{гибкости информации}, хранимой в памяти интеллектуальной системы}
	\scnitemcustom{высоким уровнем \textit{качества} \textit{стратификации информации}, хранимой в памяти интеллектуальной системы (стратифицированностью \textit{базы знаний})}
	\scnitemcustom{высоким уровнем \textit{рефлексивности} интеллектуальной системы}
	\scnitemcustom{высоким уровнем \textit{способности исправлять свои ошибки} (в том числе устранять противоречия в своей \textit{базе знаний})}
	\scnitemcustom{высоким уровнем \textit{познавательной активности}}
	\scnitemcustom{низким уровнем \textit{ограничений на вид приобретаемых знаний и навыков} (отсутствие таких ограничений означает потенциальную \textit{универсальность} интеллектуальной системы и предполагает высокий уровень её гибридности)}
\end{scnrelfromlistcustom}

\scnheader{обучаемость\scnsupergroupsign}
\scnidtf{способность быстро и качественно приобретать новые \textit{знания} и \textit{навыки}, а также совершенствовать уже приобретённые \textit{знания} и \textit{навыки}}

\scnheader{гибридность\scnsupergroupsign}
\scnidtf{степень многообразия используемых \textit{видов знаний} и \textit{моделей решения задач} и уровень эффективности их совместного использования}
\scnidtf{индивидуальная способность решать \textit{комплексные задачи}, требующие использования различных \textit{видов знаний}, а также различных комбинаций различных \textit{моделей решения задач}}

\scnheader{высокий уровень гибридности}
\begin{scnrelfromlistcustom}{обеспечивается}
	\scnitemcustom{высокой степенью многообразия используемых \textit{видов знаний} и \textit{моделей решения задач}}
	\scnitemcustom{высокой степенью \textit{конвергенции} и глубокой \textit{интеграции} (степенью взаимопроникновения) различных \textit{видов знаний} и \textit{моделей решения задач}}
	\scnitemcustom{способностью неограниченно расширять уровень своей \textit{гибридности}}
\end{scnrelfromlistcustom}
\end{SCn}

Подчеркнем, что \textit{гибридность} и \textit{интероперабельность} \textit{интеллектуальных компьютерных систем нового поколения} предполагает отказ от известной парадигмы \scnqq{черных ящиков}, поскольку:

\begin{textitemize}
	\item
	всё многообразие моделей решения задач \textit{гибридной интеллектуальной компьютерной системы} должно интерпретироваться на одной общей \textit{универсальной платформе};
	\item
	доступность информации о том, как устроен каждый используемый метод, модель решения задач, каждый субъект существенно повышает качество их \textit{координации} при \textit{совместном решении комплексных задач};
	\item
	появляется возможность некоторые методы, модели решения задач и целые субъекты (например, \textit{интеллектуальные компьютерные системы}) использовать для совершенствования (повышения качества) других методов, моделей и субъектов.
\end{textitemize}

Особо необходимо отметить следующие характеристики \textit{интеллектуальных компьютерных систем нового поколения}:
\begin{textitemize}
	\item \textbf{\textit{степень}} \textbf{\textit{конвергенции}}, унификации и стандартизации \textit{интеллектуальных компьютерных систем} и их компонентов и соответствующая этому \textbf{\textit{степень интеграции}} (глубина интеграции) \textit{интеллектуальных компьютерных систем} и их компонентов;
	\item \textbf{\textit{семантическая совместимость}} между \textit{интеллектуальными компьютерными системами} в целом и \textit{семантическая совместимость} между компонентами каждой \textit{интеллектуальной компьютерной системы} (в частности, совместимость между различными \textit{видами знаний} и различными \textit{моделями обработки знаний}), которые являются основными показателями степени \textbf{\textit{конвергенции}} (сближения) между \textit{интеллектуальными компьютерными системами} и их компонентами.
\end{textitemize}

Особенность указанных характеристик \textit{интеллектуальных компьютерных систем} их компонентов заключается в том, что они играют важную роль при решении всех ключевых задач современного этапа развития \textit{Искусственного интеллекта} и тесно связаны друг с другом.

Заметим также, что перечисленные требования, предъявляемые к \textit{интеллектуальным компьютерным системам нового поколения}, направлены на преодоление проклятия \textit{вавилонского столпотворения} как внутри \textit{интеллектуальных компьютерных систем нового поколения} (между внутренними \textit{информационными процессами} решения различных задач), так и между взаимодействующими самостоятельными \textit{интеллектуальными компьютерными системами нового поколения} в процессе коллективного решения \textit{комплексных задач}.

На современном этапе эволюции \textit{интеллектуальных компьютерных систем} для существенного расширения областей их применения и качественного повышения уровня автоматизации человеческой деятельности:
\begin{textitemize}
	\item
	необходим переход к созданию \myuline{семантически} \myuline{совместимых} \textbf{\textit{интеллектуальных компьютерных систем \myuline{нового поколения}}}, ориентированных не только на индивидуальное, но и на \myuline{коллективное} (совместное) решение \textit{комплексных задач}, требующих скоординированной деятельности нескольких самостоятельных интеллектуальных компьютерных систем и использования различных моделей и методов в непредсказуемых комбинациях, что необходимо для существенного расширения сфер применения \textit{интеллектуальных компьютерных систем}, для перехода от автоматизации локальных видов и областей \textit{человеческой деятельности} к комплексной автоматизации более крупных (объединённых) видов и областей этой деятельности;
	\item
	необходима разработка \textbf{\textit{Общей формальной теории и стандарта интеллектуальных компьютерных систем нового поколения}};
	\item
	необходима разработка \textbf{\textit{Технологии комплексной поддержки жизненного цикла интеллектуальных компьютерных систем нового поколения}}, которая включает в себя поддержку \textit{проектирования} этих систем (как начального этапа их жизненного цикла) и обеспечение их \textit{совместимости} на всех этапах их жизненного цикла;
	\item
	необходима \textbf{\textit{конвергенция}} и \textbf{\textit{унификация}} \textit{интеллектуальных компьютерных систем нового поколения} и их компонентов;
	\item
	необходима реализация \scnqq{бесшовной}, \scnqq{диффузной}, взаимопроникающей, \textbf{\textit{глубокой интеграции семантически смежных компонентов интеллектуальных компьютерных систем}}, то есть интеграции, при которой отсутствуют чёткие границы (\scnqq{швы}) интегрируемых (соединяемых) компонентов, и которая может осуществляться \myuline{автоматически}. Это означает переход к \textbf{\textit{\myuline{гибридным} интеллектуальным компьютерным системам}};
	\item
	необходимо соблюдение \textbf{\textit{Принципа бритвы Оккама}} --- максимально возможное структурное упрощение \textit{интеллектуальных компьютерных систем нового поколения}, исключение \myuline{эклектичных} решений;
	\item
	необходима ориентация на потенциально \textbf{\textit{универсальные}} (то есть способные быстро приобретать \myuline{любые} знания и навыки), \textbf{\textit{синергетические}} \textit{интеллектуальные компьютерные системы} с \scnqq{сильным} интеллектом.
\end{textitemize}

\begin{SCn}
	\scnheader{интеллектуальная компьютерная система нового поколения}
	\begin{scnrelfromlistcustom}{принципы, лежащие в основе}
		\scnitemcustom{\textit{смысловое представление знаний} в памяти \textit{интеллектуальных компьютерных систем}, предполагающее отсутствие \textit{омонимических знаков}, которые в разных контекстах обозначают разные сущности, а также отсутствие \textit{синонимии}, то есть пар синонимичных \textit{знаков}, которые обозначают одну и ту же сущность}	
		\scnitemcustom{смысловое представление информационной конструкции в общем случае имеет нелинейный (графовый) характер представления информации, который является \textit{рафинированной семантической сетью}}
		\scnitemcustom{фрактальный характер (масштабируемое самоподобие) структуризации представляемых знаний в базах знаний}
		\scnitemcustom{использование \myuline{общего} для всех интеллектуальных компьютерных систем \textit{универсального языка смыслового представления знаний} в памяти \textit{интеллектуальных компьютерных систем}, обладающий максимально простым \textit{синтаксисом}, обеспечивающий представление любых \textit{видов знаний} и имеющий неограниченные возможности перехода от \textit{знаний} к \textit{метазнаниям}. Простота синтаксиса \textit{информационных конструкций} указанного \textit{языка} позволяет называть эти конструкции \textit{рафинированными семантическими сетями}}
		\scnitemcustom{\textit{структурно-перестраиваемая (графодинамическая) организация памяти} интеллектуальных компьютерных систем, при которой обработка знаний сводится не столько к изменению состояния хранимых \textit{знаков}, сколько к изменению конфигурации связей между этими \textit{знаками}}
		\scnitemcustom{\textit{семантически неограниченный ассоциативный доступ к информации}, хранимой в памяти \textit{интеллектуальных компьютерных систем}, по заданному образцу произвольного размера и произвольной конфигурации}
		\scnitemcustom{универсальная ситуационная многоагентная модель обработки знаний, ориентированная на обработку смыслового представления информации в ассоциативной графодинамической памяти, \textit{децентрализованное ситуационное управление информационными процессами} в памяти \textit{интеллектуальных компьютерных систем}, реализованное с помощью \textit{агентно-ориентированной модели обработки баз знаний}, в котором \textit{инициирование} новых \textit{информационных процессов} осуществляется не путём передачи управления соответствующим априори известным процедурам, а в результате возникновения соответствующих \textit{ситуаций} или \textit{событий} \textit{в памяти интеллектуальной компьютерной системы}, поскольку <<основная проблема компьютерных систем состоит не в накоплении знаний, а в умении активизировать нужные знания в процессе решения задач>> (Поспелов Д.~А.). Такой многоагентный процесс обработки информации представляет собой \textit{деятельность}, выполняемую некоторым коллективом \myuline{самостоятельных} \textit{информационных агентов} (агентов обработки информации), условием инициирования каждого из которых является появление в текущем состоянии \textit{базы знаний} соответствующей этому агенту \textit{ситуации} и/или \textit{события}.			
			<<Выбор многоагентных технологий объясняется тем, что в настоящее время любая сложная производственная, логистическая или другая система может быть представлена набором взаимодействий более простых систем до любого уровня детальности, что обеспечивает фрактально-рекурсивный принцип построения многоярусных систем, построенных как открытые цифровые колонии и экосистемы ИИ. В основе многоагентных технологий лежит распределенный или децентрализованный подход к решению задач, при котором динамически обновляющаяся информация в распределенной сети интеллектуальных агентов обрабатывается непосредственно у агентов вместе с локально доступной информацией от \scnqq{соседей}. При этом существенно сокращаются как ресурсные и временные затраты на коммуникации в сети, так и время на обработку и принятие решений в центре системы (если он все-таки есть).>>
		}
		\begin{scnindent}
			\scnrelto{цитата}{\scncite{Barinov2021}~стр.~270}
		\end{scnindent}
		\scnitemcustom{агентно-ориентированная модель обработки знаний в памяти интеллектуальной компьютерной системы, обеспечивающая высокую степень \textit{интероперабельности} между внутренними агентами индивидуальной интеллектуальной компьютерной системы, взаимодействующими через общую память (это, фактически, \scnqq{внутренняя} интероперабельность интеллектуальной компьютерной системы нового поколения)} 
		\scnitemcustom{Переход к \textit{семантическим} \textit{моделям решения задач}, в основе которых лежит учёт не только синтаксических (структурных) аспектов обрабатываемой информации, но также и \myuline{семантических} (смысловых) аспектов этой информации --- \scnqqi{From data science to knowledge science}}
		\scnitemcustom{\textbf{\textit{онтологическая модель баз знаний}} \textit{интеллектуальных компьютерных систем}, то есть онтологическая структуризация всей информации, хранимой в памяти \textit{интеллектуальной компьютерной системы}, предполагающая четкую \textit{стратификацию базы знаний} в виде иерархической системы \textit{предметных областей} и соответствующих им \textit{онтологий}, каждая из которых обеспечивает семантическую \textit{спецификацию} всех \textit{понятий}, являющихся ключевыми в рамках соответствующей \textit{предметной области}}
		\scnitemcustom{\textbf{\textit{онтологическая локализация решения задач}} в \textit{интеллектуальных компьютерных системах}, предполагающая \myuline{локализацию} \textit{области действия} каждого хранимого в памяти \textit{метода} и каждого \textit{информационного агента} в соответствии с \textit{онтологической моделью} обрабатываемой \textit{базы знаний}. Чаще всего, такой \textit{областью действия} является одна из \textit{предметных областей} либо одна из \textit{предметных областей} вместе с соответствующей ей \textit{онтологии}}
		\scnitemcustom{\textbf{\textit{онтологическая модель интерфейса}} \textit{интеллектуальной компьютерной системы,} в состав которой входит:
			\begin{itemize}[labelsep=\tabsize-\bulletsize,leftmargin=\tabsize,label=$\bullet$]
				\item онтологическое описание \textit{синтаксиса} всех \textit{языков}, используемых \textit{интеллектуальной компьютерной системой} для \textit{общения} с внешними \textit{субъектами}
				\item онтологическое описание \textit{денотационной семантики} каждого \textit{языка}, используемого \textit{интеллектуальной компьютерной системой} для \textit{общения} с внешними \textit{субъектами}
				\item семейство \textit{информационных агентов}, обеспечивающих \textit{синтаксический анализ}, \textit{семантический анализ} (перевод на внутренний смысловой язык) и \textit{понимание} (погружение в \textit{базу знаний}) любого введенного \textit{сообщения}, принадлежащего любому \textit{внешнему языку}, полное онтологическое описание которого находится в базе знаний \textit{интеллектуальной компьютерной системы}
				\item семейство \textit{информационных агентов}, обеспечивающих \textit{синтез сообщений}, которые (1) адресуются внешним субъектам, с которыми общается \textit{интеллектуальная компьютерная система}, (2) \textit{семантически эквивалентны} заданным \textit{фрагментам базы знаний} интеллектуальной компьютерной системы, определяющим \textit{смысл} передаваемых \textit{сообщений}, (3) принадлежат одному из \textit{внешних языков}, полное онтологическое описание которого находится в \textit{базе знаний} интеллектуальной компьютерной системы
			\end{itemize}
		}
		
		\scnitemcustom{\textit{семантически дружественный характер пользовательского интерфейса}, обеспечиваемый (1) формальным описание в базе знаний средства управления пользовательским интерфейсом и (2) введением в состав \textit{интеллектуальной компьютерной системы} соответствующих help-подсистем, обеспечивающих существенное снижение языкового барьера между пользователями и \textit{интеллектуальными компьютерными системами}, что существенно повысит эффективность \textit{эксплуатации интеллектуальных компьютерных систем}}
		\scnitemcustom{\textit{минимизация негативного влияния человеческого фактора} на эффективность \textit{эксплуатации} \textit{интеллектуальных компьютерных систем} благодаря реализации интероперабельного (партнерского) стиля взаимодействия не только между самими \textit{интеллектуальными компьютерными системами}, но также и между \textit{интеллектуальными компьютерными системами} и их пользователями. Ответственность за качество совместной деятельности должно быть распределено между всеми партнёрами}
		\scnitemcustom{\textbf{\textit{мультимодальность}} (гибридный характер) \textit{интеллектуальной компьютерной системы}, что предполагает:
			\begin{itemize}[labelsep=\tabsize-\bulletsize,leftmargin=\tabsize,label=$\bullet$]
				\item многообразие \textit{видов знаний}, входящих в состав \textit{базы знаний} интеллектуальной компьютерной системы
				\item многообразие \textit{моделей решения задач,} используемы\textit{х решателем задач} интеллектуальной компьютерной системы
				\item многообразие \textit{сенсорных каналов}, обеспечивающих \textit{мониторинг} состояния \textit{внешней среды} интеллектуальной компьютерной системы
				\item многообразие \textit{эффекторов}, осуществляющих \textit{воздействие} на \textit{внешнюю среду}
				\item многообразие \textit{языков общения} с другими субъектами (с пользователями, с интеллектуальными компьютерными системами)
			\end{itemize}
		}
		
		\scnitemcustom{\textbf{\textit{внутренняя семантическая совместимость}} между компонентами \textit{интеллектуальной компьютерной системы} (то есть максимально возможное введение общих, совпадающих \textit{понятий} для различных фрагментов хранимой \textit{базы знаний}), являющаяся формой \textbf{\textit{конвергенции}} и \textit{глубокой интеграции} внутри \textit{интеллектуальной компьютерной системы} для различного вида \textit{знаний} и различных \textit{моделей решения задач}, что обеспечивает эффективную реализацию \textit{мультимодальности интеллектуальной компьютерной системы}}
		\scnitemcustom{\textbf{\textit{внешняя семантическая совместимость}} между различными \textit{интеллектуальными компьютерными системами}, выражающаяся не только в общности используемых \textit{понятий}, но и в общности базовых \textit{знаний} и являющаяся необходимым условием обеспечения высокого уровня \textit{интероперабельности} интеллектуальных компьютерных систем}
		\scnitemcustom{ориентация на использование \textit{интеллектуальных компьютерных систем} как \textit{когнитивных агентов} в составе \textbf{\textit{иерархических многоагентных систем}}}
		\scnitemcustom{фрактальный характер (масштабируемое самоподобие) структуризации иерархических коллективов интеллектуальных компьютерных систем нового поколения}
		\scnitemcustom{\textbf{\textit{платформенная независимость} интеллектуальных компьютерных систем}, предполагающая:
			\begin{itemize}[labelsep=\tabsize-\bulletsize,leftmargin=\tabsize,label=$\bullet$]
				\item четкую \textit{стратификацию} каждой \textit{интеллектуальной компьютерной системы} (1) на \textit{логико-семан-тическую модель}, представленную ее \textit{базой знаний}, которая содержит не только \textit{декларативные знания}, но и знания, имеющие \textit{операционную семантику}, и (2) на \textit{платформу}, обеспечивающую \textit{интерпретацию} указанной \textit{логико-семантической модели}
				\item универсальность указанной \textit{платформы} интерпретации \textit{логико-семантической модели интеллектуальной компьютерной системы}, что дает возможность каждой такой \textit{платформе} обеспечивать интерпретацию любой \textit{логико-семантической модели интеллектуальной компьютерной системы}, если эта модель представлена на том же \textit{универсальном языке смыслового представления информации}
				\item многообразие вариантов реализации \textit{платформ} \textit{интерпретации логико-семантических моделей интеллектуальных компьютерных систем} --- как вариантов, программно реализуемых на \textit{современных компьютерах}, так и вариантов, реализуемых в виде \textit{универсальных компьютеров нового поколения}, ориентированных на использование в \textit{интеллектуальных компьютерных системах} нового поколения (такие компьютеры мы назвали \textit{ассоциативными семантическими компьютерами})
				\item легко реализуемую возможность переноса (переустановки) логико-семантической модели (\textit{базы знаний}) любой \textit{интеллектуальной компьютерной системы} на любую другую \textit{платформу интерпретации логико-семантических моделей}
				\end{itemize}
		}	
		
		\scnitemcustom{изначальная ориентация \textit{интеллектуальных компьютерных систем нового поколения} на использование \textbf{\textit{универсальных ассоциативных семантических компьютеров}} (компьютеров нового поколения) в качестве \textit{платформы интерпретации логико-семантических моделей} (баз знаний) \textit{интеллектуальных компьютерных систем}}
	\end{scnrelfromlistcustom}
\end{SCn}

В настоящее время разработано большое количество различного вида \textit{моделей решения задач}, моделей представления и обработки знаний различного вида. Но в разных \textit{интеллектуальных компьютерных системах} могут быть востребованы разные комбинации этих моделей. При разработке и реализации различных \textit{интеллектуальных компьютерных систем} соответствующие методы и средства должны гарантировать \textit{логико-семантическую совместимость} разрабатываемых компонентов и, в частности, их способность использовать общие \textit{информационные ресурсы}. Для этого, очевидно, необходима \textit{унификация} указанных моделей.

\myuline{Многообразие} различных видов интеллектуальных компьютерных систем и, соответственно, многообразие используемых ими комбинаций моделей представления знаний и решения задач определяется:
\begin{textitemize}
	\item многообразием назначения интеллектуальных компьютерных систем и вида окружающей их среды;
	\item многообразием различных видов хранимых знаний;
	\item многообразием моделей обработки знаний и решений задач;
	\item многообразием различных видов сенсорных и эффекторных подсистем.
\end{textitemize}

Следует выделить следующие аспекты \textit{совместимости} моделей представления и обработки знаний в \textit{интеллектуальных компьютерных системах}:

\begin{textitemize}
	\item синтаксический;
	\item семантический (согласованность систем понятий, их денотационной семантики);
	\item функциональный (операционный).
\end{textitemize}

Следует также отличать:

\begin{textitemize}
	\item \textit{совместимость} между компонентами \textit{интеллектуальных компьютерных систем};
	\item \textit{совместимость} между верхним логико-семантическим уровнем используемых моделей представления и обработки знаний и различными уровнями их интерпретации вплоть до аппаратного уровня;
	\item \textit{совместимость} между индивидуальными интеллектуальными компьютерными системами;
	\item \textit{совместимость} между индивидуальными интеллектуальными компьютерными системами и их пользователями;
	\item \textit{совместимость} между коллективами интеллектуальных компьютерных системам.
\end{textitemize}

\begin{SCn}
	\scnheader{следует отличать*}
	\begin{scnhaselementset}
		\scnitem{данные}
		\begin{scnindent}
			\scnidtf{информационная конструкция, обрабатываемая с помощью программы традиционного языка программирования}
		\end{scnindent}
		\scnitem{знание}
		\begin{scnindent}
			\scnidtf{семантически целостный фрагмент базы знаний}
		\end{scnindent}
	\end{scnhaselementset}
	\begin{scnindent}
		\scntext{отличие}{Для каждого знания всегда известен язык, на котором это знание представлено и денотационная семантика которого задана. При этом указанный язык имеет достаточно большую семантическую мощность, а в идеале является универсальным языком. В отличие от этого структуризация данных для традиционных программ осуществляется в целях упрощения самих этих программ и, следовательно, для разных программ в общем случае осуществляется по-разному. Таким образом, при разработке традиционных программ представление обрабатываемых данных осуществляется в общем случае на разных языках, денотационная семантика которых нигде не документируется и известна только разработчикам программ. Другими словами, данные для разных программ имеют денотационную семантику не только разную, но еще и априори неизвестную. По сути это форма проявления \textit{вавилонского столпотворения} в традиционных языках программирования, которые образно говоря \scnqq{хромают на одну ногу}, формализуя методы обработки информации, но не формализуя семантику обрабатываемой информации.}
	\end{scnindent} 
\end{SCn}

\section{Принципы, лежащие в основе смыслового представления информации}
\label{sec_ngics_sense_principles}

\begin{scnrelfromlist}{подраздел}
	\scnitem{\ref{subsec_directions_evolution_computer_systems}~\nameref{subsec_directions_evolution_computer_systems}}
	\scnitem{\ref{subsec_essence_proposed_approach}~\nameref{subsec_essence_proposed_approach}}
	\scnitem{\ref{subsec_semantic_unification_computer_systems}~\nameref{subsec_semantic_unification_computer_systems}}
\end{scnrelfromlist}

\begin{scnrelfromlist}{библиографическая ссылка}
	\scnitem{\scncite{Kharlamov2011}}
	\scnitem{\scncite{Moskin2011}}
	\scnitem{\scncite{Timchenko2013}}
	\scnitem{\scncite{Massel2013}}
	\scnitem{\scncite{Efimenko2014}}
	\scnitem{\scncite{Tarasov2014}}
	\scnitem{\scncite{Golenkov2014}}
	\scnitem{\scncite{Zagorulko2014}}
	\scnitem{\scncite{Golenkov2019}}
	\scnitem{\scncite{Golenkov2020}}
	\scnitem{\scncite{Martynov}}
\end{scnrelfromlist}

Предлагаемый подход к решению проблем, препятствующих дальнейшей эволюции компьютерных систем и технологий --- стандартизация моделей представления и обработки информации

Анализ проблем эволюции компьютерных систем разного уровня сложности, разного уровня обучаемости и интеллектуальности, разного назначения показывает, что проклятие \scnqq{вавилонского столпотворения} и, как следствие, несовместимость, дублирование и субъективизм согласовываемых информационных ресурсов и моделей их обработки нас преследует везде:
\begin{textitemize}
	\item и в развитии традиционных компьютерных систем;
	\item и в развитии технологий искусственного интеллекта;
	\item и в развитии методов и средств информатизации научной и инженерной деятельности.
\end{textitemize}

Рассматривая проблему обеспечения совместимости информационных ресурсов и моделей их обработки, следует говорить о разных аспектах решения этой проблемы:
\begin{textitemize}
	\item об обеспечении совместимости между различными компонентами компьютерных систем, а также между целостными компьютерными системами, входящими в коллективы компьютерных систем;
	\item об обеспечении совместимости, то есть высокого уровня взаимопонимания между различными компьютерными системами и их пользователями;
	\item об обеспечении междисциплинарной совместимости, то есть конвергенции различных областей знаний;
	\item о методах и средствах постоянного мониторинга и восстановления совместимости в условиях интенсивной эволюции компьютерных систем и их пользователей, которая часто нарушает достигнутую совместимость (согласованность) и требует дополнительных усилий на ее восстановление.
\end{textitemize}

\subsection{Направления эволюции компьютерных систем}
\label{subsec_directions_evolution_computer_systems}
В эволюции компьютерных систем можно выделить два общих направления.

\textbf{Первое направление} --- это 
\begin{textitemize}
    \item \textbf{расширение множества и многообразия задач}, решаемых компьютерной системой; 
    \item повышение \textbf{сложности этих задач} вплоть до трудно формализуемых (трудно решаемых) задач, интеллектуальных задач, решаемых в условиях неполноты, неточности, нечеткости и так далее;
    \item повышение \textbf{качества решения задач} либо путем более эффективного использования известных моделей решения задач (например, путем разработки более качественных алгоритмов), либо путем использования принципиально новых моделей решения задач;
    \item расширение \textbf{многообразия используемых видов информации} (знаний);
    \item расширение \textbf{многообразия используемых моделей решения задач}.
\end{textitemize}

Очевидно, что расширение множества решаемых задач в условиях пусть и большой, но всегда конечной памяти компьютерной системы делает все более и более актуальным переход от частных методов и моделей решения задач к их обобщениям (или, как отмечал Д.~А. Поспелов, от связки \scnqq{ключей} к набору \scnqq{отмычек}).

Очевидно также, что многообразие видов задач, решаемых компьютерными системами, многообразие используемых моделей решения задач приводит: 
\begin{textitemize}
    \item к интегрированным информационным ресурсам;
    \item к интегрированным решателям задач;
    \item к интегрированным компьютерным системам;
    \item к коллективам компьютерных систем.
\end{textitemize}

Проблема здесь заключается не в самой интеграции, а в ее качестве. Интеграция может быть \textbf{эклектичной}, если не обеспечить совместимость интегрируемых компонентов, а в случае такой совместимости интеграция может привести к новому качеству, к дополнительному расширению множества решаемых задач. Это будет означать переход от эклектичности к гибридности, синергетичности.

\textbf{Второе общее направление} эволюции компьютерных систем --- это повышение уровня их \textbf{обучаемости} и, как следствие, темпов их эволюции.

\textbf{Обучаемость} компьютерной системы определяется:
\begin{textitemize}
    \item \textbf{трудоемкостью} и темпами приобретения (расширения) и совершенствования активно используемых знаний и навыков;
    \item \textbf{уровнем ограничений}, накладываемых на вид приобретаемых и используемых знаний и навыков (фактически, это ограничения на множество всех тех задач, которые принципиально могут быть решены данной компьютерной системой).
\end{textitemize}

В свою очередь, \textbf{трудоемкость и темпы расширения и совершенствования} знаний и навыков компьютерной системы определяется:
\begin{textitemize}
    \item \textbf{гибкостью} --- многообразием и трудоемкостью возможных изменений, вносимых в систему в процессе пополнения системы новыми знаниями и навыками и совершенствования уже приобретенных знаний и навыков;
    \item \textbf{стратифицированностью} --- четким разделением системы на достаточно независящие друг от друга уровни иерархии, то есть возможностью локализации фрагментов компьютерной системы, не выходя за пределы которых, \myuline{априори достаточно} проводить анализ последствий тех или иных вносимых в систему изменений;
    \item \textbf{рефлексивностью} --- способностью анализировать собственное состояние и свою деятельность;
    \item \textbf{гибридностью} --- способностью приобретать и использовать широкое (а в идеале --- неограниченное) многообразие знаний и навыков;
    \item \textbf{уровнем самообучаемости} --- уровнем активности, самостоятельности, целеустремленности в процессе своего обучения, то есть уровнем способности к обучению \myuline{без учителя}, уровнем автоматизации приобретения новых знаний и навыков, а также совершенствования уже приобретенных знаний и навыков;
    \item \textbf{совместимостью} --- трудоемкостью интеграции;
    \item \textbf{способностью к постоянному мониторингу и поддержке своей совместимости} как с другими компьютерными системами, так и со своими пользователями в условиях интенсивной эволюции этих компьютерных систем и их пользователей. 
\end{textitemize}

\textbf{Совместимость} (трудоемкость интеграции) компьютерных систем может рассматриваться в двух аспектах:
\begin{textitemize}
    \item в аспекте \textbf{глубокой интеграции} компьютерных систем, что предполагает преобразование нескольких компьютерных систем в одну целостную компьютерную систему путем объединения информационных и функциональных ресурсов интегрируемых компьютерных систем;
    \item в аспекте преобразования нескольких компьютерных систем в \textbf{коллектив взаимодействующих компьютерных систем}, способных к совместному корпоративному решению сложных задач.
\end{textitemize}

Совместимость (трудоемкость интеграции) компьютерных систем определяется:
\begin{textitemize}
    \item совместимостью различного вида информации (знаний), хранимой в памяти компьютерной системы;
    \item совместимостью различных моделей решения задач;
    \item совместимостью встроенных (в том числе типовых) подсистем, входящих в состав компьютерных систем;
    \item совместимостью внешней информации, поступающей на вход компьютерной системе, с информацией, хранимой в памяти компьютерной системы (трудоемкостью понимания внешней информации --- трансляции, погружения, выравнивания понятий);
    \item коммуникационной (в том числе семантической) совместимостью с пользователями и с другими компьютерными системами.
\end{textitemize}

Важнейшая форма обучения компьютерной системы это приобретение новых знаний и навыков в \scnqq{готовом} виде, то есть в виде некоторых знаковых конструкций, вводимых в память компьютерной системы, поскольку приобретение знаний и навыков из внешних достоверных источников требует существенно меньшего времени по сравнению с их приобретением собственными силами, на основе собственного опыта и собственных ошибок. Но для того, чтобы указанная форма обучения была эффективной, необходимо максимально возможным образом упростить и формализовать механизм (процедуру) погружения новых знаний в память компьютерной системы.

Для решения этой задачи ключевое значение имеет создание удобного для этой цели способа кодирования различного вида информации в памяти компьютерной системы.

Поскольку основным каналом обучения компьютерных систем является приобретение ими знаний и навыков от других субъектов --- от других компьютерных систем и от пользователей (от разработчиков-учителей и от конечных пользователей), важнейшим фактором обучаемости компьютерной системы является превращение компьютерной системы в коммуникативную систему, способную эффективно общаться с внешними субъектами. Следовательно, уровень обучаемости компьютерных систем определяется также уровнем ее совместимости с самими этими внешними субъектами, с приобретаемыми ею знаниями и навыками, то есть степенью того, как компьютерная система вместе с теми субъектами, с которыми она обменивается информацией, решает проблему \scnqq{вавилонского столпотворения}.

\subsection{Суть предлагаемого подхода}
\label{subsec_essence_proposed_approach}
Суть предлагаемого нами подхода к решению проблем эволюции компьютерных систем заключается, во-первых, в объединении всех указанных выше направлений эволюции компьютерных систем (как общих направлений, так и частных) и, во-вторых, в трактовке проблемы обеспечения \textbf{совместимости} различных видов знаний, различных моделей решения задач, различных компьютерных систем как \textbf{ключевой проблемы} эволюции компьютерных систем, решение которой существенно упростит решение и многих других проблем.

Так, например, без обеспечения совместимости информационных ресурсов, используемых в разных компьютерных системах, а также информационных ресурсов, представляющих знания различного семантического вида невозможно:
\begin{textitemize}
	\item ни создавать \textbf{коллективы компьютерных систем}, способные координировать свои действия при кооперативном расширении сложных задач;
	\item ни создавать \textbf{гибридные компьютерные системы}, которые способны при решении сложных комплексных задач использовать всевозможные сочетания разных видов знаний и разных моделей решения задач;
	\item ни использовать \textbf{компонентную методику проектирования} компьютерных систем \textbf{на всех уровнях} иерархии проектируемых систем.
\end{textitemize}

О какой информационной совместимости и взаимопонимании (в том числе между специалистами) можно говорить при наличии ужасающей понятийной и терминологической неряшливости, терминологического псевдотворчества, в том числе, в области информатики.

Говоря о \textbf{совместимости} компьютерных систем и их компонентов, а также совместимости компьютерных систем с пользователями, следует отметить неоднозначность трактовки термина \scnqqi{совместимость}. В этой связи следует отличать:
\begin{textitemize}
    \item совместимость как один из факторов обучаемости, как \textbf{способность} к быстрому повышению уровня согласованности (интеграции, взаимопонимания).
    Сравните обучаемость как \textbf{способность} к быстрому расширению знаний и навыков, но никак не характеристика объема и качества приобретенных знаний и навыков;
    \item совместимость как характеристика достигнутого уровня согласованности (интеграции, взаимопонимания).
\end{textitemize}

Аналогичным образом интеллект компьютерной системы с одной стороны можно трактовать как \textbf{уровень} (объем и качество) приобретенных знаний и навыков, а с другой стороны как \textbf{способность} к быстрому расширению и совершенствованию знаний и навыков, то есть как \textbf{скорость} повышения уровня знаний и навыков.

Кроме того, следует говорить не только о \textbf{способности} к быстрому повышению уровня согласованности и не только о достигнутом уровне согласованности, но и о самом \textbf{процессе} повышения уровня согласованности и, прежде всего, о перманентном процессе восстановления (поддержки, сохранения) достигнутого уровня согласованности, поскольку в ходе эволюции компьютерных систем и их пользователей (то есть в ходе расширения и повышения качества их знаний и навыков) уровень их согласованности может понижаться.

\subsection{Семантическая унификация компьютерных систем}
\label{subsec_semantic_unification_computer_systems}
Главным фактором обеспечения совместимости различных видов знаний, различных моделей решения задач и различных компьютерных систем в целом является 
\begin{textitemize}
    \item унификация (стандартизация) представления информации в памяти компьютерных систем;
    \item унификация принципов организации обработки информации в памяти компьютерных систем.
\end{textitemize}

Унификация представления информации, используемой в компьютерных системах, предполагает:
\begin{textitemize}
    \item синтаксическую унификацию используемой информации --- унификацию формы представления (кодирования) этой информации. При этом следует отличать:
    \begin{textitemize}
	    	\item кодирование информации в памяти компьютерной системы (внутреннее представление информации);
	    	\item внешнее представление информации, обеспечивающее однозначность интерпретации (понимания, трактовки) этой информации разными пользователями и разными компьютерными системами;
	    \end{textitemize}
    \item семантическую унификацию используемой информации в основе которой лежит согласование и точная спецификация всех (!) используемых понятий (концептов) с помощью иерархической системы формальных онтологий.
\end{textitemize}

Важно отметить, что грамотная унификация (стандартизация) должна не ограничивать творческую свободу разработчика, а гарантировать \textbf{совместимость} его результатов с результатами других разработчиков. Подчеркнем также, что текущая версия любого \textbf{стандарта} --- это не догма, а только опора для дальнейшего его совершенствования.

Целью качественного стандарта является не только обеспечения совместимости технических решений, но и минимализация дублирования (повторения) таких решений. Один из важных критериев качества стандарта --- ничего лишнего.

\begin{SCn}
	\scnheader{стандарт}
	\scnidtf{знания о структуре и принципах функционирования искусственных систем соответствующего класса}
	\scnidtf{онтология искусственных систем некоторого класса}
	\scnidtf{теория искусственных систем некоторого класса}
\end{SCn}

Стандарты, как и другие важные для человечества знания, должны быть формализованы и должны постоянно совершенствоваться с помощью специальных интеллектуальных компьютерных систем, поддерживающих процесс эволюции стандартов путем согласования различных точек зрения.

\begin{SCn}
\begin{scnrelfromlist}{ключевое понятие}
	\scnitem{смысловое представление информации}
	\begin{scnindent}
		\scnidtf{смысл}
	\end{scnindent}
	\scnitem{семантическая сеть}
	\scnitem{рафинированная семантическая сеть}
	\scnitem{граф знаний}
	\begin{scnindent}
		\scnidtf{представление сложноструктурированного знания в виде графовой структуры}
	\end{scnindent}
	\scnitem{универсальный язык семантических сетей}
	\begin{scnindent}
		\scnidtf{универсальный язык, информационными конструкциями которого являются семантические сети}
	\end{scnindent}
\end{scnrelfromlist}

\begin{scnrelfromlist}{ключевое знание}
	\scnitem{Принципы, лежащие в основе смыслового представления информации}
\end{scnrelfromlist}	 

\scnheader{смысловое представление информации}
\scnidtf{запись (представление) информационной конструкции на смысловом уровне}
\scnidtf{информационная конструкция синтаксическая структура которой близка её смыслу, то есть близка описываемой конфигурации связей между описываемыми сущностями}
\scnidtf{смысловое представление информационной конструкции}
\scnsuperset{семантическая сеть}
\begin{scnindent}
	\scnsuperset{рафинированная семантическая сеть}
\end{scnindent}

\scnheader{рафинированная семантическая сеть}
\begin{scnrelfromlistcustom}{принципы, лежащие в основе}
	\scnitemcustom{Каждый элемент (синтаксически атомарный фрагмент) рафинированной семантической сети является знаком одной из описываемых сущностей}
	\scnitemcustom{Каждая сущность, описываемая рафинированной семантической сетью, должна быть представлена своим знаком, который является элементом этой сети}
	\scnitemcustom{В рамках каждой отдельной рафинированной семантической сети отсутствует синонимия разных знаков, а также отсутствуют омонимичные знаки}
	\scnitemcustom{Многообразие сущностей, описываемых рафинированными семантическими сетями, ничем не ограничивается. Соответственно этому, семантическая типология элементов рафинированных семантических сетей является весьма богатой}
	\scnitemcustom{Особым видом элементов рафинированных семантических систем являются знаки связей между другими элементами этих сетей. При этом, связываемыми элементами (то есть элементами, которые инцидентны указанным знакам связей) могут быть также и знаки других связей. Чаще всего знак связи между элементами рафинированной семантической сети является \myuline{отражением} связи между сущностями, которые обозначаются указанными элементами. Но в некоторых случаях знак связи между элементами рафинированной семантической сети может быть отражением, например, связи между одной описываемой сущностью и \myuline{знаком} другой описываемой сущности}
\end{scnrelfromlistcustom}
\end{SCn}

\section{Принципы, лежащие в основе многоагентных моделей решателей задач интеллектуальных компьютерных систем нового поколения}
\sectionmark{Принципы, лежащие в основе многоагентных моделей решателей задач и.к.с.}
\label{sec_ngics_ps}

\begin{SCn}
\begin{scnrelfromlist}{ключевое понятие}
	\scnitem{смысловая память}
	\scnitem{графодинамическая память}
	\scnitem{ассоциативная память с информационным доступом по образцу произвольного размера и конфигурации}
	\scnitem{система ситуационного децентрализованного управления информационными процессами}
	\scnitem{многоагентная система обработки информации в общей памяти}
	\scnitem{язык смыслового представления задач}
	\scnitem{универсальный язык смыслового представления знаний}
	\scnitem{язык смыслового представления методов}
	\begin{scnindent}
		\scnidtf{интегрированный язык смыслового представления различного вида программ}
	\end{scnindent}
	\scnitem{инсерционная программа}
\end{scnrelfromlist}

\begin{scnrelfromlist}{ключевое знание}
	\scnitem{Принципы, лежащие в основе решателей задач индивидуальных интеллектуальных компьютерных систем нового поколения}
\end{scnrelfromlist}

\begin{scnrelfromlist}{библиографическая ссылка}
	\scnitem{\scncite{Letichevskij2003}}
\end{scnrelfromlist}

\scnheader{решатель задач интеллектуальных компьютерных систем нового поколения}
\begin{scnrelfromlistcustom}{предъявляемые требования}
	\scnitemcustom{решатель задач интеллектуальных компьютерных систем нового поколения должен уметь решать интеллектуальные задачи, к числу которых относятся следующие виды задач:
		\begin{scnitemize}
			\item некачественно сформулированная задача\\
			\scnidtf{задача, формулировка которой содержит различные не-факторы (неполнота, нечеткость, противоречивость (некорректность) и так далее)}
			\item задача, для решения которой, кроме самой формулировки задачи и соответствующего метода её решения необходима дополнительная, но априори неизвестно какая информация об объектах, указанных в формулировке (постановке) задачи. При этом указанная дополнительная информация может присутствовать, а может и отсутствовать в текущем состоянии базы знаний интеллектуальных компьютерных систем. Кроме того, для некоторых задач может быть задана (указана) та область базы знаний, использования которой достаточно для поиска или генерации (в частности, логического вывода) указанной дополнительной требуемой информации. Такую область базы знаний будем называть областью решения соответствующей задачи
			\item задача, для которой соответствующий метод ее решения в текущий момент не известен.			
			Для решения такой задачи можно:
			\begin{scnitemize}
				\item переформулировать задачу, то есть сгенерировать (логически вывести) логически эквивалентную формулировку исходной задачи, для которой метод её решения в текущий момент является известным
				\item свести исходную задачу к семейству подзадач, для которых методы их решения в текущий момент известны.
			\end{scnitemize}
		\end{scnitemize}
	}
	
	\scnitemcustom{процесс решения задач в интеллектуальных компьютерных системах нового поколения реализуется коллективом информационных агентов, обрабатывающих базу знаний интеллектуальных компьютерных систем}
	\scnitemcustom{управление информационными процессами в памяти интеллектуальных компьютерных систем нового поколения осуществляется децентрализованным образом по принципам ситуационного управления}
\end{scnrelfromlistcustom}




\scnheader{ситуационное управление}
\scnidtf{ситуационно-событийное управление}
\scntext{пояснение}
{управление последовательностью выполнения действий, при котором условием (\scnqq{триггером}) инициирования указанных действий является:
	\begin{textitemize}
		\item возникновение некоторых ситуаций (условий, состояний);
		\item и/или возникновение некоторых \textit{событий}.
	\end{textitemize}
}

\scnheader{ситуация}
\scnidtf{структура, описывающая некоторую временно существующую конфигурацию связей между некоторыми сущностями}
\scnidtf{описание временно существующего состояния некоторого фрагмента (некоторой части) некоторой динамической системы}

\scnheader{событие}
\scnsuperset{возникновение временной сущности}
\begin{scnindent}
	\scnidtf{появление, рождение, начало существования некоторой временной сущности}
\end{scnindent}
\scnsuperset{исчезновение временной сущности}
\begin{scnindent}
	\scnidtf{прекращение, завершение существования некоторой временной сущности}
\end{scnindent}
\scnsuperset{переход от одной ситуации к другой}
\begin{scnindent}
	\scntext{примечание}{Здесь учитывается не только факт возникновения новой ситуации, но и её предыстория --- то есть та ситуация, которая ей непосредственно предшествует. Так, например, реагируя на аномальное значение какого-либо параметра, нам важно знать:
	\begin{textitemize}
			\item какова динамика изменения этого параметра (увеличивается он или уменьшается и с какой скоростью);
			\item какие меры были предприняты ранее для ликвидации этой аномалии.
	\end{textitemize}}
\end{scnindent}

\scnheader{решатель задач индивидуальной интеллектуальной компьютерной системы нового поколения}
\begin{scnrelfromlistcustom}{принципы, лежащие в основе}
	\scnitemcustom{смысловое представление обрабатываемых знаний}
	\scnitemcustom{семантически неограниченный ассоциативный доступ к различным фрагментам знаний, хранимым в памяти интеллектуальных компьютерных систем нового поколения (доступ по заданному образцу произвольного размера и произвольной конфигурации)}
	\scnitemcustom{графодинамический характер обработки знаний в памяти, при котором обработка знаний сводится не только к изменению состояния атомарных фрагментов (ячеек) памяти, но и к изменению конфигурации связей между этими атомарными фрагментами}
	\scnitemcustom{ситуационное децентрализованное управление процессом обработки знаний, а также процессом организации взаимодействия интеллектуальных компьютерных систем с внешней средой}
	\scnitemcustom{использование семантически мощного языка задач, обеспечивающего представление формулировок самых различных задач, которые могут решаться либо в рамках памяти интеллектуальной компьютерной системы, либо во внешней среде и которые осуществляют инициирование соответствующих процессов решения задач}
	\scnitemcustom{многоагентный характер реализации процессов решения инициированных задач, в основе которого лежит иерархическая система агентов, каждый из которых активизируются при возникновении в памяти интеллектуальной компьютерной системы соответствующий ситуации или соответствующего события}
\end{scnrelfromlistcustom}

\end{SCn}

\section{Принципы, лежащие в основе онтологических моделей мультимодальных интерфейсов интеллектуальных компьютерных систем нового поколения}
\sectionmark{Принципы, лежащие в основе онтологических моделей мультимодальных интерфейсов и.к.с.}
\label{sec_ngics_ui}

\begin{SCn}
\begin{scnrelfromlist}{ключевое понятие}
	\scnitem{мультимодальный интерфейс}
	\scnitem{вербальный интерфейс}
	\scnitem{естественно-языковой интерфейс}
	\scnitem{внешний язык}
	\begin{scnindent}
		\scnidtf{язык обмена сообщениями}
	\end{scnindent}
	\scnitem{внутренний язык}
	\begin{scnindent}
		\scnidtf{язык представления информации в памяти кибернетической системы}
	\end{scnindent}
	\scnitem{синтаксис внешнего языка}
	\scnitem{денотационная семантика внешнего языка}
	\scnitem{интерфейсная задача}
	\scnitem{понимание сообщения}
	\scnitem{синтез сообщения}
	\scnitem{невербальный интерфейс}
	\scnitem{сенсор}
	\begin{scnindent}
		\scnidtf{рецептор}
	\end{scnindent}
	\scnitem{сенсорная подсистема}
	\scnitem{мультисенсорная подсистема}
	\scnitem{сенсорная информация}
	\scnitem{эффектор}
	\scnitem{мультиэффекторная подсистема}
	\scnitem{сенсо-моторная координация}
\end{scnrelfromlist}

\begin{scnrelfromlist}{ключевое знание}
	\scnitem{Принципы, лежащие в основе интерфейсов интеллектуальных компьютерных систем нового поколения}
\end{scnrelfromlist}
	
	
\scnheader{интерфейс интеллектуальной компьютерной системы нового поколения}
\begin{scnrelfromlistcustom}{принципы, лежащие в основе}
	\scnitemcustom{интерфейс \textit{интеллектуальной компьютерной системы нового поколения} рассматривается как решатель задач частного вида --- \textit{интерфейсных задач}, основными из которых являются:
	\begin{scnitemize}
		\item задачи понимания вербальной информации, приобретаемой интеллектуальной компьютерной системой (синтаксический анализ, семантический анализ и погружение в базу знаний интеллектуальной компьютерной системы)
		\item задачи понимания невербальной информации, воспринимаемой сенсорными подсистемами интеллектуальной компьютерной системы (анализ изображений, анализ аудио-сигналов, погружение результатов анализа в базу знаний интеллектуальной компьютерной системы)
		\item задачи синтеза сообщений, адресуемых внешним субъектам (кибернетическим системам)
	\end{scnitemize}}
	
	\scnitemcustom{тот факт, что интерфейс \textit{интеллектуальной компьютерной системы нового поколения} является решателем частного вида \textit{задач интеллектуальной компьютерной системы нового поколения}, свойства, лежащие в основе решателей \textit{задач интеллектуальной компьютерной систем нового поколения}, наследуются интерфейсами \textit{интеллектуальной компьютерной систем нового поколения}. Из этого следует, что в основе интерфейса \textit{интеллектуальной компьютерной систем нового поколения} лежит:
	\begin{scnitemize}
		\item смысловое представление накапливаемых (приобретаемых знаний)
		\item трактовка семантического анализа приобретаемой вербальной информации как процесса перевода этой информации на внутренний язык смыслового представления знаний с последующим погружением (вводом, интеграцией) результата этого перевода в состав текущего состояния базы знаний \textit{интеллектуальной компьютерной системы нового поколения}
		\item трактовка синтеза сообщений, адресуемых внешними субъектами как процесса обратного перевода некоторого фрагмента базы знаний с внутреннего языка смыслового представления информации на внешний язык, используемый для общения с заданным субъектом
		\item агентно-ориентированная организация решения интерфейсных задач, реализуемая соответствующим коллективов внутренних агентов интерфейса \textit{интеллектуальных компьютерных систем нового поколения}, взаимодействующих через общедоступную для них базу знаний \textit{интеллектуальной компьютерной системы нового поколения}
		\end{scnitemize}}
	
	\scnitemcustom{интерфейс \textit{интеллектуальной компьютерной системы нового поколения} трактуется как специализированная встроенная \textit{интеллектуальная компьютерная система нового поколения}, входящая в состав указанной выше интеллектуальной компьютерной системы, база знаний которой включает в себя:
	\begin{scnitemize}
		\item онтологию синтаксиса внутреннего языка смыслового преставления информации
		\item онтологию денотационной семантики внутреннего языка смыслового представления информации
		\item онтологию синтаксиса всех внешних языков, используемых для общения с внешними субъектами
		\item онтологии денотационной семантики всех внешних языков, используемых для общения с внешними субъектами (каждая такая онтология с формальной точки зрения является описанием соответствия между текстами внешних языков и семантически эквивалентными им текстами внутреннего языка смыслового представления информации.
	\end{scnitemize}
	Подчеркнем при этом, что все указанные онтологии, входящие в состав базы знаний интерфейса интеллектуальных компьютерных систем нового поколения, как и вся остальная информация, входящая в состав этой базы знаний, представляется на внутреннем языке смыслового представления информации, который, соответственно используется в данном случае как метаязык.
	}
\end{scnrelfromlistcustom}
\end{SCn}

\begin{SCn}
\scnheader{интерфейс индивидуальной интеллектуальной компьютерной системы нового поколения}
\begin{scnrelfromlistcustom}{принципы, лежащие в основе}
	\scnitemcustom{интерфейс индивидуальной интеллектуальной компьютерной системы нового поколения является специализированным компонентом решателя задач интеллектуальной компьютерной системы нового поколения, то есть специализированной \myuline{встроенной} (в индивидуальную интеллектуальную компьютерную систему нового поколения) интеллектуальной компьютерной системой нового поколения, ориентированной на решение интерфейсных задач, к которым относятся:
	\begin{scnitemize}
		\item понимание принятых сообщений (их перевод на язык внутреннего смыслового представления информации и погружения в текущее состояние базы знаний)
		\item синтез передаваемых сообщений (перевод сформированного сообщения с внутреннего языка смыслового представления на используемый внешний язык)
		\item первичный анализ приобретаемой сенсорной информации, предполагающий распознавание некоторого семейства первичных образов и сцен
		\item сенсомоторная координация действий, выполняемых эффекторами интеллектуальной компьютерной системы
		\end{scnitemize}}
		
	\scnitemcustom{мультимодальный характер интерфейса --- многообразие внешних языков, видов сенсоров и эффекторов}
	\scnitemcustom{формальное онтологическое описание на языке внутреннего смыслового представления информации
	\begin{scnitemize}
		\item синтаксиса и денотационной семантики всех используемых внешних языков
		\item первичных образов и сцен (ситуаций), являющихся результатом первичного анализа приобретаемой сенсорной информации
		\item методов низкого уровня, непосредственно интерпретируемых эффекторами интеллектуальной компьютерной системы
	\end{scnitemize}}

\end{scnrelfromlistcustom}
\end{SCn}

Разговоры о дружественном и, в частности, адаптивном \textit{пользовательском интерфейсе} ведутся давно, но это, чаще всего, касается формы (\scnqq{синтаксической} стороны) \textit{пользовательского интерфейса}, а не смыслового содержания взаимодействия с пользователями. В настоящее время \textit{пользовательские интерфейсы} компьютерных систем (в том числе и \textit{интеллектуальных компьютерных систем}) для широкого контингента пользователей не являются семантически (содержательно) дружественными (семантически комфортными). Организация взаимодействия пользователей с компьютерными системами (в том числе и с \textit{интеллектуальными компьютерными системами}) является \scnqq{узким местом}, оказывающим существенное влияние на эффективность \textit{автоматизации человеческой деятельности}. В основе современной организации взаимодействия пользователя с компьютерной системой лежит парадигма \myuline{грамотного} пользователя, который знает, чего он хочет от используемого им инструмента и несёт полную ответственность за качество взаимодействия с этим инструментом. Эта парадигма лежит в основе деятельности лесоруба во взаимодействии с топором, всадника во взаимодействии с лошадью, автоводителя, летчика во взаимодействии с соответствующим транспортным средством, оператора атомной электростанции, железнодорожного диспетчера и так далее.

На современном этапе развития \textit{Искусственного интеллекта} для повышения эффективности взаимодействия необходим переход \myuline{от парадигмы} \myuline{грамотного} \myuline{управления} используемым инструментом \myuline{к парадигме} \myuline{равноправного} \myuline{сотрудничества}, партнёрскому взаимодействию \textit{интеллектуальной компьютерной системы} со своим пользователем. \textit{Интеллектуальная компьютерная система} должна повернуться \scnqq{лицом} к пользователю.

\textbf{\textit{Семантическая дружественность пользовательского интерфейса}} должна заключаться в адаптивности к особенностям и квалификации пользователя, исключении любых проблем для пользователя в процессе диалога с \textit{интеллектуальной компьютерной системой}, в перманентной заботе о совершенствовании коммуникационных навыков пользователя.

При организации взаимодействия пользователя с \textit{Глобальной сетью} компьютерным системам необходимо перейти от парадигмы \scnqq{многооконного} интерфейса, в каждом \scnqq{окне} которого свои \scnqq{правила игры} к парадигме \scnqq{одного окна}. Пользователь не должен знать, какое \scnqq{окно} ему надо \scnqq{открыть} (в какую систему ему надо войти) для удовлетворения той или иной его потребности.

Пользователь не должен знать, какая конкретно система будет решать его задачу. Пользователь должен уметь с помощью \myuline{универсальных} средств сформулировать свою задачу, а соответствующая компьютерная система, входящая в \textit{Глобальную сеть} и способная решить эту задачу, должна сама инициироваться, реагируя на факт появления указанной задачи. Таким образом пользовательский интерфейс должен быть интерфейсом пользователя не с конкретной компьютерной системой, а в целом со всей \textit{Глобальной сетью компьютерных систем}.

\section*{Заключение к Главе~\ref{chapter_new_generation_systems} Достоинства предлагаемых принципов, лежащих в основе интеллектуальных компьютерных систем нового поколения}
\sectionmark{Достоинства предлагаемых принципов, лежащих в основе и.к.с.}
\label{sec_ngics_advant}

\textbf{\textit{Смысловое представление информации}} в памяти \textit{интеллектуальных компьютерных систем} обеспечивает устранение дублирования информации, хранимой в памяти \textit{интеллектуальной компьютерной системы}, то есть устранение многообразия форм представления одной и той же информации, запрещение появления в одной памяти \textit{семантически эквивалентных информационных конструкций} и, в том числе, синонимичных \textit{знаков}. Это существенно снижает сложность и повышает качество:

\begin{textitemize}
	\item разработки различных \textit{моделей обработки знаний} (так как нет необходимости учитывать многообразие форм представления одного и того же знания);
	\item \textit{семантического анализа} и \textit{понимания} информации, поступающей (передаваемой) от различных внешних субъектов (от пользователей, от разработчиков, от других \textit{интеллектуальных компьютерных систем});
	\item \textit{конвергенции} и \textit{интеграции} различных видов знаний в рамках каждой \textit{интеллектуальной компьютерной системы};
	\item обеспечения \textit{семантической совместимости} и \textit{взаимопонимания} между различными \textit{интеллектуальными компьютерными системами}, а также между \textit{интеллектуальными компьютерными системами} и их пользователями.
\end{textitemize}

Понятие \textit{семантической сети} нами рассматривается не как красивая метафора сложноструктурированных \textit{знаковых конструкций}, а как формальное уточнение понятия \textit{смыслового представления информации}, как принцип представления информации, лежащей в основе нового поколения \textit{компьютерных языков} и самих \textit{компьютерных систем} --- \textit{графовых языков} и \textit{графовых компьютеров}. \textit{Семантическая сеть} --- это нелинейная (графовая) \textit{знаковая конструкция}, обладающая следующими свойствами:

\begin{textitemize}
	\item все элементы (то есть синтаксически элементарные фрагменты) этой \textit{графовой структуры} (узлы и связки) являются \textit{знаками} описываемых сущностей и, в частности, \textit{знаками связей} между этими сущностями;
	\item все \textit{знаки}, входящие в эту \textit{графовую структуру}, не имеют \textit{синонимов} в рамках этой структуры;
	\item \scnqq{внутреннюю} структуру (строение) \textit{знаков}, входящих в \textit{семантическую сеть} не требуется учитывать при ее \textit{семантическом анализе} (понимании);
	\item смысл \textit{семантической сети} определяется \textit{денотационной семантикой} всех входящих в нее \textit{знаков} и конфигурацией \textit{связей инцидентности} этих знаков;
	\item из двух \textit{инцидентных знаков}, входящих в \textit{семантическую сеть}, по крайней мере один является знаком связи.
\end{textitemize}

\textit{Рафинированная семантическая сеть} --- это \textit{семантическая сеть}, имеющая максимально простую \textit{синтаксическую структуру}, в которой, в частности,

\begin{textitemize}
	\item используется \myuline{конечный} \textit{алфавит} элементов \textit{семантической сети}, то есть конечное число синтаксически выделяемых типов (синтаксических меток), приписываемых этим элемента;
	\item внешние идентификаторы (в частности, имена), приписываемые элементам \textit{семантической сети} используются \myuline{только} для ввода/вывода информации.
\end{textitemize}

\textit{Агентно-ориентированная модель обработки информации} в сочетании с \textit{децентрализованным ситуационным управлением процессом обработки информации}, а также со \textit{смысловым представлением информации} в памяти \textit{интеллектуальной компьютерной системы} существенно снижает сложность и повышает качество интеграции различных \textit{моделей решения задач} при обработке \myuline{общей} \textit{базы знаний}. Имеется в виду одновременное использование различных \textit{моделей решения задач} при обработке одних и тех же \textit{знаний}, в частности, при решении одной и той же \textit{задачи}.

Высокий уровень \textit{семантической гибкости информации}, хранимой в памяти \textit{интеллектуальной компьютерной системы нового поколения}, обеспечивается тем, что каждое удаление или добавление синтаксически элементарного фрагмента хранимой информации, а также удаление или добавление каждой \textit{связи инцидентности} между такими элементами имеет четкую семантическую интерпретацию.

Высокий уровень \textit{стратифицированности информации}, хранимой в памяти \textit{интеллектуальной компьютерной системы нового поколения}, обеспечивается онтологически ориентированной структуризацией \textit{базы знаний интеллектуальной компьютерной системы нового поколения}.

Высокий уровень \textit{индивидуальной обучаемости} интеллектуальных компьютерных систем нового поколения (то есть их способности к быстрому расширению своих \textit{знаний} и \textit{навыков}) обеспечивается:

\begin{textitemize}
	\item \textit{семантической гибкостью информации}, хранимой в их \textit{памяти};
	\item \textit{стратифицированностью} этой информации;
	\item \textit{рефлексивностью} интеллектуальных компьютерных систем нового поколения.
\end{textitemize}

Высокий уровень \textit{коллективной обучаемости} интеллектуальных компьютерных систем нового поколения обеспечивается высоким уровнем их интероперабельности (их \textit{социализации}, способности к эффективному участию в деятельности различных коллективов, состоящих из \textit{интеллектуальных компьютерных систем нового поколения} и людей) и, прежде всего, высоким уровнем их \textit{взаимопонимания}.

Высокий уровень \textit{интероперабельности} интеллектуальных компьютерных систем нового поколения принципиально меняет характер взаимодействия \textit{компьютерных систем} с людьми, автоматизацию деятельности которых они осуществляют, --- от управления этими средствами автоматизации к \textit{равноправным партнерским осмысленным взаимоотношениям}.

Каждая \textit{интеллектуальная компьютерная система нового поколения} способна:

\begin{textitemize}
	\item самостоятельно или по приглашению войти в состав коллектива, состоящего из \textit{интеллектуальных компьютерных систем нового поколения} и/или людей. Такие коллективы создаются на временной или постоянной основе для коллективного решения сложных \textit{задач};
	\item участвовать в распределении (в том числе в согласовании распределения) \textit{задач} --- как \scnqq{разовых} задач, так и долгосрочных задач (обязанностей);
	\item мониторить состояние всего процесса коллективной деятельности и координировать свою деятельность с деятельностью других членов коллектива при возможных непредсказуемых изменениях условий (состояния) соответствующей среды.
\end{textitemize}

Высокий уровень интеллекта \textit{интеллектуальных компьютерных систем нового поколения} и, соответственно, высокий уровень их самостоятельности и целенаправленности позволяет им быть полноправными членами самых различных сообществ, в рамках которых \textit{интеллектуальные компьютерные системы нового поколения} получают права самостоятельно инициировать (на основе детального анализа текущего положения дел и, в том числе, текущего состояния плана действий сообщества) широкий спектр действий (задач), выполняемых другими членами сообщества, и тем самым участвовать в согласовании и координации деятельности членов сообщества. Способность \textit{интеллектуальной компьютерной системы нового поколения} согласовывать свою деятельность с другими подобными системами, а также корректировать деятельность всего \textit{коллектива интеллектуальных компьютерных систем нового поколения,} адаптируясь к различного вида изменениям среды (условий), в которой эта деятельность осуществляется, позволяет существенно автоматизировать деятельность \textit{системного интегратора} как на этапе создания \textit{коллектива интеллектуальных компьютерных систем нового поколения}, так и на этапе его обновления (реинжиниринга).

Достоинства \textit{интеллектуальных компьютерных систем нового поколения} обеспечиваются:

\begin{textitemize}
	\item достоинствами языка внутреннего \textit{смыслового кодирования информации}, хранимой в памяти этих систем;
	\item достоинствами организации графодинамической ассоциативной смысловой памяти \textit{интеллектуальных компьютерных систем нового поколения};
	\item достоинствами \textit{смыслового представления баз знаний} интеллектуальных компьютерных систем нового поколения и \textit{средствами онтологической структуризации баз знани}й этих систем;
	\item достоинствами \textit{агентно-ориентированных моделей решения задач}, используемых в \textit{интеллектуальных компьютерных системах нового поколения} в сочетании с децентрализованным управлением процессом обработки информации.
\end{textitemize}

Кратко перечислим основные положения Главы~\ref{chapter_new_generation_systems}:

\begin{textitemize}
	\item основным практически значимым направлением развития современных \textit{интеллектуальных компьютерных систем} является переход к \textit{интероперабельным} \textit{интеллектуальным компьютерным системам}, способным к эффективному взаимодействию между собой и с пользователями, что:
	\begin{textitemize}
		\item обеспечивает автоматизацию решения сложных комплексных задач, для которых требуется создание временных или постоянных \myuline{коллективов};
		\item превращает \textit{интеллектуальные компьютерные системы} в \myuline{самостоятельные} активные \textit{субъекты}, способные инициировать различные комплексные задачи и, собственно, инициировать для этого работоспособные коллективы, состоящие из людей и \textit{интероперабельных интеллектуальных компьютерных систем} требуемой квалификации.
	\end{textitemize}
	\item коллективы, состоящие из самостоятельных \textit{интероперабельных интеллектуальных компьютерных систем} и людей, имеют хорошие перспективы стать \textit{синергетическими} системами.\\
	\item \textit{интероперабельность интеллектуальных компьютерных систем} обеспечивается:
	\begin{textitemize}
		\item высоким уровнем взаимопонимания и, соответственно, семантической совместимостью;
		\item высоким уровнем договороспособности, то есть способности предварительно согласовывать свои действия с действиями других субъектов;
		\item высоким уровнем способности оперативно координировать свои действия с действиями других субъектов в ходе их выполнения (см. \ref{chap_intro}~\nameref{chap_intro}).
	\end{textitemize}
	\item к числу принципов, лежащих в основе построения \textit{интероперабельных интеллектуальных компьютерных систем}, относятся:
	\begin{textitemize}
		\item смысловое представление знаний в памяти \textit{интеллектуальных компьютерных систем} в виде рафинированных семантических сетей;
		\item использование универсального языка внутреннего смыслового представления знаний (см. \ref{chapter_sc_code}~\nameref{chapter_sc_code});
		\item графодинамическая организация обработки знаний;
		\item агентно-ориентированные модели решения задач (см. \ref{chapter_situation_management}~\nameref{chapter_situation_management});
		\item структуризация и стратификация баз знаний в виде иерархической системы формальных онтологий (см. \ref{chapter_kb}~\nameref{chapter_kb});
		\item семантически дружественный пользовательский интерфейс (см. \ref{chapter_interfaces}~\nameref{chapter_interfaces}).
	\end{textitemize}
	\item для разработки большого количества интероперабельных семантически совместимых \textit{интеллектуальных компьютерных систем}, обеспечивающих переход на принципиально новый уровень автоматизации \textit{человеческой деятельности}, необходимо создание технологии, обеспечивающей массовое производство таких \textit{интеллектуальных компьютерных систем}, участие в котором доступно широкому контингенту разработчиков (в том числе разработчиков средней квалификации и начинающих разработчиков). Основными положениями такой технологии являются
	\begin{textitemize}
		\item стандартизация интероперабельных \textit{интеллектуальных компьютерных систем};
		\item широкое использование \textit{компонентного проектирования} на основе мощной библиотеки семантически совместимых многократно используемых (типовых) компонентов \textit{интероперабельных интеллектуальных компьютерных систем} (см. \ref{chapter_library}~\nameref{chapter_library}).
	\end{textitemize}
	\item эффективная эксплуатация \textit{интероперабельных интеллектуальных компьютерных систем} требует создания не только \textit{технологии проектирования} таких систем, но также и семейства технологий поддержки всех остальных этапов их жизненного цикла. Особенно это касается технологии перманентной поддержки \textit{семантической совместимости} \textit{всех} взаимодействующих \textit{интероперабельных интеллектуальных компьютерных систем} в ходе их эксплуатации (см. \ref{chapter_ostis_tech}~\nameref{chapter_ostis_tech}).
\end{textitemize}

%%%%%%%%%%%%%%%%%%%%%%%%% referenc.tex %%%%%%%%%%%%%%%%%%%%%%%%%%%%%%
% sample references
% %
% Use this file as a template for your own input.
%
%%%%%%%%%%%%%%%%%%%%%%%% Springer-Verlag %%%%%%%%%%%%%%%%%%%%%%%%%%
%
% BibTeX users please use
% \bibliographystyle{}
% \bibliography{}
%
\biblstarthook{In view of the parallel print and (chapter-wise) online publication of your book at \url{www.springerlink.com} it has been decided that -- as a genreral rule --  references should be sorted chapter-wise and placed at the end of the individual chapters. However, upon agreement with your contact at Springer you may list your references in a single seperate chapter at the end of your book. Deactivate the class option \texttt{sectrefs} and the \texttt{thebibliography} environment will be put out as a chapter of its own.\\\indent
References may be \textit{cited} in the text either by number (preferred) or by author/year.\footnote{Make sure that all references from the list are cited in the text. Those not cited should be moved to a separate \textit{Further Reading} section or chapter.} If the citatiion in the text is numbered, the reference list should be arranged in ascending order. If the citation in the text is author/year, the reference list should be \textit{sorted} alphabetically and if there are several works by the same author, the following order should be used:
\begin{enumerate}
\item all works by the author alone, ordered chronologically by year of publication
\item all works by the author with a coauthor, ordered alphabetically by coauthor
\item all works by the author with several coauthors, ordered chronologically by year of publication.
\end{enumerate}
The \textit{styling} of references\footnote{Always use the standard abbreviation of a journal's name according to the ISSN \textit{List of Title Word Abbreviations}, see \url{http://www.issn.org/en/node/344}} depends on the subject of your book:
\begin{itemize}
\item The \textit{two} recommended styles for references in books on \textit{mathematical, physical, statistical and computer sciences} are depicted in ~\cite{science-contrib, science-online, science-mono, science-journal, science-DOI} and ~\cite{phys-online, phys-mono, phys-journal, phys-DOI, phys-contrib}.
\item Examples of the most commonly used reference style in books on \textit{Psychology, Social Sciences} are~\cite{psysoc-mono, psysoc-online,psysoc-journal, psysoc-contrib, psysoc-DOI}.
\item Examples for references in books on \textit{Humanities, Linguistics, Philosophy} are~\cite{humlinphil-journal, humlinphil-contrib, humlinphil-mono, humlinphil-online, humlinphil-DOI}.
\item Examples of the basic Springer style used in publications on a wide range of subjects such as \textit{Computer Science, Economics, Engineering, Geosciences, Life Sciences, Medicine, Biomedicine} are ~\cite{basic-contrib, basic-online, basic-journal, basic-DOI, basic-mono}. 
\end{itemize}
}

\begin{thebibliography}{99.}%
% and use \bibitem to create references.
%
% Use the following syntax and markup for your references if 
% the subject of your book is from the field 
% "Mathematics, Physics, Statistics, Computer Science"
%
% Contribution 
\bibitem{science-contrib} Broy, M.: Software engineering --- from auxiliary to key technologies. In: Broy, M., Dener, E. (eds.) Software Pioneers, pp. 10-13. Springer, Heidelberg (2002)
%
% Online Document
\bibitem{science-online} Dod, J.: Effective substances. In: The Dictionary of Substances and Their Effects. Royal Society of Chemistry (1999) Available via DIALOG. \\
\url{http://www.rsc.org/dose/title of subordinate document. Cited 15 Jan 1999}
%
% Monograph
\bibitem{science-mono} Geddes, K.O., Czapor, S.R., Labahn, G.: Algorithms for Computer Algebra. Kluwer, Boston (1992) 
%
% Journal article
\bibitem{science-journal} Hamburger, C.: Quasimonotonicity, regularity and duality for nonlinear systems of partial differential equations. Ann. Mat. Pura. Appl. \textbf{169}, 321--354 (1995)
%
% Journal article by DOI
\bibitem{science-DOI} Slifka, M.K., Whitton, J.L.: Clinical implications of dysregulated cytokine production. J. Mol. Med. (2000) doi: 10.1007/s001090000086 
%
\bigskip

% Use the following (APS) syntax and markup for your references if 
% the subject of your book is from the field 
% "Mathematics, Physics, Statistics, Computer Science"
%
% Online Document
\bibitem{phys-online} J. Dod, in \textit{The Dictionary of Substances and Their Effects}, Royal Society of Chemistry. (Available via DIALOG, 1999), 
\url{http://www.rsc.org/dose/title of subordinate document. Cited 15 Jan 1999}
%
% Monograph
\bibitem{phys-mono} H. Ibach, H. L\"uth, \textit{Solid-State Physics}, 2nd edn. (Springer, New York, 1996), pp. 45-56 
%
% Journal article
\bibitem{phys-journal} S. Preuss, A. Demchuk Jr., M. Stuke, Appl. Phys. A \textbf{61}
%
% Journal article by DOI
\bibitem{phys-DOI} M.K. Slifka, J.L. Whitton, J. Mol. Med., doi: 10.1007/s001090000086
%
% Contribution 
\bibitem{phys-contrib} S.E. Smith, in \textit{Neuromuscular Junction}, ed. by E. Zaimis. Handbook of Experimental Pharmacology, vol 42 (Springer, Heidelberg, 1976), p. 593
%
\bigskip
%
% Use the following syntax and markup for your references if 
% the subject of your book is from the field 
% "Psychology, Social Sciences"
%
%
% Monograph
\bibitem{psysoc-mono} Calfee, R.~C., \& Valencia, R.~R. (1991). \textit{APA guide to preparing manuscripts for journal publication.} Washington, DC: American Psychological Association.
%
% Online Document
\bibitem{psysoc-online} Dod, J. (1999). Effective substances. In: The dictionary of substances and their effects. Royal Society of Chemistry. Available via DIALOG. \\
\url{http://www.rsc.org/dose/Effective substances.} Cited 15 Jan 1999.
%
% Journal article
\bibitem{psysoc-journal} Harris, M., Karper, E., Stacks, G., Hoffman, D., DeNiro, R., Cruz, P., et al. (2001). Writing labs and the Hollywood connection. \textit{J Film} Writing, 44(3), 213--245.
%
% Contribution 
\bibitem{psysoc-contrib} O'Neil, J.~M., \& Egan, J. (1992). Men's and women's gender role journeys: Metaphor for healing, transition, and transformation. In B.~R. Wainrig (Ed.), \textit{Gender issues across the life cycle} (pp. 107--123). New York: Springer.
%
% Journal article by DOI
\bibitem{psysoc-DOI}Kreger, M., Brindis, C.D., Manuel, D.M., Sassoubre, L. (2007). Lessons learned in systems change initiatives: benchmarks and indicators. \textit{American Journal of Community Psychology}, doi: 10.1007/s10464-007-9108-14.
%
%
% Use the following syntax and markup for your references if 
% the subject of your book is from the field 
% "Humanities, Linguistics, Philosophy"
%
\bigskip
%
% Journal article
\bibitem{humlinphil-journal} Alber John, Daniel C. O'Connell, and Sabine Kowal. 2002. Personal perspective in TV interviews. \textit{Pragmatics} 12:257--271
%
% Contribution 
\bibitem{humlinphil-contrib} Cameron, Deborah. 1997. Theoretical debates in feminist linguistics: Questions of sex and gender. In \textit{Gender and discourse}, ed. Ruth Wodak, 99--119. London: Sage Publications.
%
% Monograph
\bibitem{humlinphil-mono} Cameron, Deborah. 1985. \textit{Feminism and linguistic theory.} New York: St. Martin's Press.
%
% Online Document
\bibitem{humlinphil-online} Dod, Jake. 1999. Effective substances. In: The dictionary of substances and their effects. Royal Society of Chemistry. Available via DIALOG. \\
http://www.rsc.org/dose/title of subordinate document. Cited 15 Jan 1999
%
% Journal article by DOI
\bibitem{humlinphil-DOI} Suleiman, Camelia, Daniel C. O'Connell, and Sabine Kowal. 2002. `If you and I, if we, in this later day, lose that sacred fire...': Perspective in political interviews. \textit{Journal of Psycholinguistic Research}. doi: 10.1023/A:1015592129296.
%
%
%
\bigskip
%
%
% Use the following syntax and markup for your references if 
% the subject of your book is from the field 
% "Computer Science, Economics, Engineering, Geosciences, Life Sciences"
%
%
% Contribution 
\bibitem{basic-contrib} Brown B, Aaron M (2001) The politics of nature. In: Smith J (ed) The rise of modern genomics, 3rd edn. Wiley, New York 
%
% Online Document
\bibitem{basic-online} Dod J (1999) Effective Substances. In: The dictionary of substances and their effects. Royal Society of Chemistry. Available via DIALOG. \\
\url{http://www.rsc.org/dose/title of subordinate document. Cited 15 Jan 1999}
%
% Journal article by DOI
\bibitem{basic-DOI} Slifka MK, Whitton JL (2000) Clinical implications of dysregulated cytokine production. J Mol Med, doi: 10.1007/s001090000086
%
% Journal article
\bibitem{basic-journal} Smith J, Jones M Jr, Houghton L et al (1999) Future of health insurance. N Engl J Med 965:325--329
%
% Monograph
\bibitem{basic-mono} South J, Blass B (2001) The future of modern genomics. Blackwell, London 
%
\end{thebibliography}
