\section{Эволюция традиционных и интеллектуальных компьютерных систем}
{\label{sec_comp_syst}} 

\begin{SCn}

\bigskip

\begin{scnrelfromlist}{ключевое понятие}
    \scnitem{компьютерная система} 
    \scnitem{память компьютерной системы}
    \scnitem{информационный процесс в памяти компьютерной системы}
    \scnitem{компьютер}
    \scnitem{интерфейс компьютерной системы}
    \scnitem{пользовательский интерфейс компьютерной системы}
    \scnitem{традиционная компьютерная система}
    \scnitem{алгоритм}
    \scnitem{абстрактная машина Тьюринга}
    \scnitem{абстрактная машина фон-Неймана}
    \scnitem{параллельная программа}
        \begin{scnindent}
        \scnidtf{программа, при выполнении (интерпретации) которой инициируются одновременно выполняемые информационные процессы}
        \end{scnindent}
    \scnitem{объектно-ориентированная программа}
    \scnitem{интерпретация программы}
    \scnitem{компиляция программы}
    \scnitem{уровень языка программирования}
    \scnitem{база данных}
    \scnitem{интеллектуальная компьютерная система}
    \scnitem{логическая программа}
    \scnitem{функциональная программа}
    \scnitem{продукционная программа}
    \scnitem{искусственная нейронная сеть}
    \scnitem{база знаний}
    \scnitem{интеллектуальный решатель задач}
\end{scnrelfromlist}

\bigskip

\begin{scnrelfromlist}{ключевое знание}
    \scnitem{Классификация компьютерных систем}
    \scnitem{Обобщённая архитектура компьютерных систем}
    \scnitem{Классификация традиционных компьютерных систем}
    \scnitem{Эволюция традиционных компьютерных систем}
    \scnitem{Эволюция языков программирования}
    \scnitem{Эволюция систем программирования}
    \scnitem{Классификация интеллектуальных компьютерных систем}
    \scnitem{Обобщённая архитектура интеллектуальных компьютерных систем}
    \scnitem{Отличия интеллектуальных компьютерных систем от традиционных компьютерных систем}
    \scnitem{Эволюция интеллектуальных компьютерных систем}
\end{scnrelfromlist}

\bigskip

\begin{scnrelfromlist}{библиографическая ссылка}
	\scnitem{\scncite{Cho2019}}
	\scnitem{\scncite{Sherif1988}}
	\scnitem{\scncite{Laird2009}}
	\scnitem{\scncite{Gao2002}}
\end{scnrelfromlist}

\bigskip

\end{SCn}

Были предложены различные системные показатели для измерения качества компьютерных систем. Поскольку компьютерные системы становятся все более сложными и включают множество подсистем или компонентов, измерение их качества в нескольких измерениях становится сложной задачей \scncite{Cho2019}. Метрики качества включают в себя набор мер, которые могут описывать атрибуты системы в терминах, не зависящих от структуры, которая приводит к этим атрибутам; эти меры должны быть выражены количественно и должны иметь значительный уровень точности и надежности \scncite{Sherif1988}.

Целью является построение компьютерных систем, которые демонстрируют весь спектр когнитивных способностей, которые мы обнаруживаем у людей \scncite{Laird2009}.

Исследователи искусственного интеллекта определяют интеллект как неотъемлемое свойство машины \scncite{Gao2002}.