%\subsection{Комплекс свойств, определяющих качество многоагентной системы}
{\label{sec_mas_overall_quality}} 

Переход к \textit{многоагентным системам} является важнейшим фактором повышения \textit{качества} (и, в частности, уровня интеллекта) \textit{кибернетических систем}, так как уровень \textit{интеллекта многоагентной системы} может быть значительно выше уровня интеллекта каждого входящего в неё агента. 
Это происходит далеко не всегда, поскольку важнейшим фактором \textit{качества многоагентных систем} является не только качество входящих в неё агентов, но и организация взаимодействия агентов и, в частности, переход от централизованного к децентрализованному управлению. 
Количество не всегда переходит в новое качество.

\textit{качество индивидуальных кибернетических систем} определяется, кроме всего прочего, тем, насколько большой вклад \textit{индивидуальная кибернетическая система} вносит в повышение качества тех коллективов, в состав которых она входит.
Указанное свойство \textit{индивидуальных кибернетических систем} будем называть уровнем их \textit{интероперабельности} (см \scncite{Ouksel1999}, \ref{chapter_ostis_tech}~\nameref{chapter_ostis_tech}).

\textbf{\textit{синергетическая кибернетическая система}} --- \textit{многоагентная система}, обладающая высоким уровнем коллективного интеллекта, атомарными агентами которой являются \textit{индивидуальные интеллектуальные системы}, имеющие высокий уровень \textit{интероперабельности} (см. \scncite{Lopes2022}, \scncite{Hamilton2006}).
Примером \textit{синергетической кибернетической системы} является \textit{творческий коллектив}, реализующий сложный наукоемкий проект.

Эффективность \textit{творческого коллектива} (например в области \textit{научно-технической деятельности}) определяется:
\begin{textitemize}
    \item согласованностью мотивации, целевой установки всего коллектива и каждого его члена (не должно быть противоречий между целью коллектива и творческой самореализацией каждого его члена);
    \item эффективной организацией децентрализованного управления деятельностью членов сообщества;
    \item четкой, оперативной и доступной всем фиксацией документации текущего состояния содеянного и направлений его дальнейшего развития;
    \item уровнем трудоемкости оперативности фиксации индивидуальных результатов в рамках коллективно создаваемого общего результата;
    \item уровнем структурированности и, прежде всего, стратифицированности обобщенной документации (базы знаний);
    \item эффективностью ассоциативного доступа к фрагментам документации;
    \item гибкостью коллективно создаваемой базы;
    \item автоматизацией анализа содеянного и управления проектом.
\end{textitemize}

Уровень \textit{интеллекта многоагентной системы} может быть значительно ниже уровня интеллекта самого "глупого"{} члена этого коллектива, но может быть и значительно выше уровня интеллекта самого "умного"{} члена указанного коллектива.
Для того, чтобы количество \textit{интеллектуальных систем} переходило в существенно более интеллектуальное качество коллектива таких систем, все объединяемые в коллектив \textit{интеллектуальные системы} должны иметь высокий уровень \textit{интероперабельности}, что накладывает дополнительные требования, предъявляемые к \textit{информации, хранимой в памяти интеллектуальных систем}, а также к \textit{решателям задач интеллектуальных систем}, объединяемых в коллектив.

\textbf{\textit{интероперабельность кибернетической системы}} --- способность \textit{кибернетической системы} взаимодействовать с другими кибернетическими системами в целях создания коллектива \textit{кибернетических систем} (многоагентных систем), уровень качества и, в частности, уровень \textit{интеллекта} которого выше уровня \textit{качества} каждой \textit{кибернетической системы}, входящей в состав этого коллектива.

Для того, чтобы количество членов \textit{коллектива кибернетической системы} перешло в более высокое качество самого коллектива, члены коллектива должны обладать дополнительными способностями, которые будем называть свойствами \textit{интероперабельности}.
Основными такими свойствами являются способность устанавливать и поддерживать достаточный уровень \textit{семантической совместимости} (взаимопонимания) с другими \textit{кибернетическими системами} и \textit{договороспособности} между ними (способность согласовывать свои действия с другими) (см. \scncite{Neiva2016}).

Целенаправленный обмен информацией между \textit{кибернетическими системами} существенно ускоряет процесс их обучения (процесс накопления знаний и навыков). Следовательно, способность эффективно использовать указанный канал накопления знаний и навыков существенно повышает уровень \textit{обучаемости кибернетических систем}. Повышение уровня \textit{интероперабельности кибернетической системы} является, с одной стороны, дополнительным повышением уровня \textit{интеллекта} самой этой \textit{кибернетической системы}, а также фактором повышения уровня \textit{интеллекта} тех \textit{коллективов}, тех \textit{многоагентных систем}, в состав которых эта \textit{кибернетическая система} входит.

\begin{SCn}
\scnheader{интероперабельность кибернетической системы}
\begin{scnrelfromlist}{cвойство-предпосылка}
    \scnitem{договороспособность кибернетической системы}
    \scnitem{социальная ответственность кибернетической системы}
    \scnitem{социальная активность кибернетической системы}
\end{scnrelfromlist}
\end{SCn}

Свойства-предпосылки уровня \textit{договороспособности кибернетической системы} представлены ниже:

\begin{SCn}
\scnheader{договороспособность кибернетической системы}
\begin{scnrelfromlist}{cвойство-предпосылка}
    \scnitem{способность кибернетической системы к пониманию принимаемых сообщений}
    \scnitem{способность кибернетической системы к формированию передаваемых сообщений, понятных адресатам}
    \scnitem{способность кибернетической системы к обеспечению семантической совместимости с партнёрами}
    \scnitem{коммуникабельность кибернетической системы}
    \scnitem{способность кибернетической системы к обсуждению и согласованию целей и планов коллективной деятельности}
    \scnitem{способность кибернетической системы брать на себя выполнение актуальных задач в рамках согласованных планов коллективной деятельности}
\end{scnrelfromlist}
\end{SCn}

Понимание информации, поступающей извне, включает в себя:
\begin{textitemize}
    \item перевод этой информации на внутренний язык кибернетической системы;
    \item локальную верификацию вводимой информации;
    \item погружение (конвергенцию, размещение) текста, являющегося результатом указанного перевода в состав хранимой информации (в частности, в состав базы знаний).
\end{textitemize}

Погружение вводимой информации в состав \textit{базы знаний} \textit{кибернетической системы} сводится к выявлению и устранению противоречий, возникающих между погружаемым текстом и текущего состояния \textit{базы знаний}. Сложность проблемы понимания вводимой вербальной информации заключается не только в сложности непротиворечивого погружения вводимой информации в текущее состояние \textit{базы знаний}, но и в сложности трансляции этой информации с \textit{внешнего языка} на \textit{внутренний язык кибернетической системы}, то есть в сложности генерации текста внутреннего языка, семантически эквивалентного вводимому тексту \textit{внешнего языка}.
Для \textit{естественных языков} указанная трансляция является сложной задачей, так как в настоящее время проблема формализации \textit{синтаксиса} и \textit{семантики естественных языков} не решена.

\textbf{\textit{семантическая совместимость двух заданных кибернетических систем}} определяется \textit{согласованностью систем понятий}, используемых обеими взаимодействующими \textit{кибернетическими системами}.
Проблема обеспечения перманентной поддержки \textit{семантической совместимости} взаимодействующих \textit{кибернетических систем} является необходимым условием обеспечения высокого уровня \textit{взаимопонимания кибернетических систем} и, как следствие, эффективного их взаимодействия.

\textbf{\textit{коммуникабельность кибернетической системы}} --- способность \textit{кибернетической системы} к установлению взаимовыгодных контактов с другими \textit{кибернетическими системами} (в том числе, с коллективами интеллектуальных систем) путем честного выявления взаимовыгодных общих целей (интересов).

Свойства-предпосылки уровня с\textit{оциальной ответственности кибернетической системы} представлены ниже:

\begin{SCn}
\scnheader{социальная ответственность кибернетической системы}
\begin{scnrelfromlist}{cвойство-предпосылка}
    \scnitem{способность кибернетической системы выполнять качественно и в срок взятые на себя обязательства в рамках соответствующих коллективов}
    \scnitem{способность кибернетической системы адекватно оценивать свои возможности при распределении коллективной деятельности}
    \scnitem{альтруизм/эгоизм кибернетической системы}
    \scnitem{отсутствие/наличие действий, которые по безграмотности кибернетической системы снижают качество коллективов, в состав которых она входит}
    \scnitem{отсутствие/наличие "осознанных"{}, мотивированных действий, снижающих качество коллективов, в состав которых кибернетическая система входит}
\end{scnrelfromlist}
\end{SCn}

Свойства-предпосылки уровня \textit{социальной активности кибернетической системы} представлены ниже:

\begin{SCn}
\scnheader{социальная активность кибернетической системы}
\begin{scnrelfromlist}{cвойство-предпосылка}
    \scnitem{способность кибернетической системы к генерации предлагаемых целей и планов коллективной деятельности}
    \scnitem{активность кибернетической системы в экспертизе результатов других участников коллективной деятельности}
    \scnitem{способность кибернетической системы к анализу качества всех коллективов, в состав которых она входит, а также всех членов этих коллективов}
    \scnitem{способность кибернетической системы к участию в формировании новых коллективов}
    \scnitem{количество и качество тех коллективов, в состав которых кибернетическая система входит или входила}
\end{scnrelfromlist}
\end{SCn}

Формирование специализированного \textit{коллектива кибернетических систем} сводится к тому, что в \textit{памяти} каждой \textit{кибернетической системы}, входящей в коллектив, генерируется спецификация этого коллектива, включающая в себя:
\begin{textitemize}
    \item перечень весь членов коллектива;
    \item способности каждого из членов коллектива;
    \item их обязанности в рамках коллектива;
    \item спецификацию всего множества задач (вида деятельности), для решения (выполнения) которых сформирован данный коллектив кибернетических систем.
\end{textitemize}

Каждая \textit{кибернетическая система} может входить в состав большого количества коллективов, выполняя при этом в разных коллективах в общем случае разные "должностные обязанности"{}, разные "бизнес-процессы"{}.
