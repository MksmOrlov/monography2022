\subsection{Комплекс свойств, определяющих качество многоагентной системы}
{\label{sec_mas_overall_quality}} 

Переход от кибернетических систем к коллективам взаимодействующих между собой кибернетических систем, т.е. к социальной организации кибернетических систем, является важнейшим фактором эволюции кибернетических систем. 
Кибернетическую систему, представляющую собой коллектив взаимодействующих кибернетических систем, обладающих определенной степенью самостоятельности (самодостаточности, свободы выбора), будем называть многоагентной системой.

\begin{SCn}
\scnheader{кибернетическая система}
\begin{scnrelfromset}{разбиение}
    \scnitem{индивидуальная кибернетическая система}
    \scnitem{кибернетическая система, являющаяся минимальным компонентом индивидуальной кибернетической системы}
    \scnitem{кибернетическая система, являющаяся комплексом компонентов соответствующей индивидуальной кибернетической системы}
    \scnitem{сообщество индивидуальных кибернетических систем}
    \begin{scnindent}
    \begin{scnrelfromset}{разбиение}
        \scnitem{простое сообщество индивидуальных кибернетических систем}
        \scnitem{иерархическое сообщество индивидуальных кибернетических систем}
    \end{scnrelfromset}
    \end{scnindent}
\end{scnrelfromset}
\end{SCn}

Агенты многоагентной системы могут (но вовсе не обязательно должны) быть интеллектуальными системами. 
Так, например, агенты интеллектуального решателя задач, имеющего агентно-ориентированную архитектуру, не являются интеллектуальными системами. 
Агентом иерархической многоагентной системы может быть другая многоагентная система.

В многоагентной системе с централизованным управлением специально выделяются агенты, которые принимают решения в определенной области деятельности многоагентной системы и обеспечивают выполнение этих решений путем управления деятельностью остальных агентов, входящих в состав этой системы.

В многоагентной системе с децентрализованным управлением решения принимаются коллегиально и "автоматически"{} (решения о признании новой кем-то предложенной информации – в том числе, об инициировании некоторой задачи, решения о коррекции (уточнении) уже ранее признанной, согласованной информации) на основе четко продуманной и постоянно совершенствуемой методики, а также на основе активного участия всех агентов в формировании новых предложений, подлежащих признанию или согласованию. 
В такой многоагентной системе все агенты участвуют в управлении этой системы. 
Примером такой системы является оркестр, способный играть без дирижера.

Переход к многоагентным системам является важнейшим фактором повышения качества (и, в частности, уровня интеллекта) кибернетических систем, т.к. уровень интеллекта многоагентной системы может быть значительно выше уровня интеллекта каждого входящего в неё агента. 
Это происходит далеко не всегда, поскольку важнейшим фактором качества многоагентных систем является не только качество входящих в неё агентов, но и организация взаимодействия агентов и, в частности, переход от централизованного к децентрализованному управлению. 
Количество не всегда переходит в новое качество.

Качество индивидуальных кибернетических систем определяется, кроме всего прочего тем, насколько большой вклад индивидуальная кибернетическая система вносит в повышение качества тех коллективов, в состав которых она входит.
Указанное свойство индивидуальных кибернетических систем будем называть уровнем их интероперабельности \cite{Ouksel1999interoperability}.

Синергетическая кибернетическая система есть многоагентная система, обладающая высоким уровнем коллективного интеллекта, атомарными агентами которой являются индивидуальные интеллектуальные системы, имеющие высокий уровень интероперабельности \cite{Lopes2022semantic} \cite{Hamilton2006Interoperability}.
Примером синергетической кибернетической системы является творческий коллектив, реализующий сложный наукоемкий проект.

Эффективность творческого коллектива (например в области научно-технической деятельности) определяется:
\begin{itemize}
    \item{согласованностью мотивации, целевой установки всего коллектива и каждого его члена (не должно быть противоречий между целью коллектива и творческой самореализацией каждого его члена);}
    \item{эффективной организацией децентрализованного управления деятельностью членов сообщества;}
    \item{четкой, оперативной и доступной всем фиксацией документации текущего состояния содеянного и направлений его дальнейшего развития;}
    \item{уровнем трудоемкости оперативности фиксации индивидуальных результатов в рамках коллективно создаваемого общего результата;}
    \item{уровнем структурированности и, прежде всего, стратифицированности обобщенной документации (базы знаний);}
    \item{эффективностью ассоциативного доступа к фрагментам документации;}
    \item{гибкостью коллективно создаваемой базы;}
    \item{автоматизацией анализа содеянного и управления проектом.}
\end{itemize}

Уровень интеллекта многоагентной системы может быть значительно ниже уровня интеллекта самого "глупого"{} члена этого коллектива, но может быть и значительно выше уровня интеллекта самого "умного"{} члена указанного коллектива.
Для того, чтобы количество интеллектуальных систем переходило в существенно более интеллектуальное качество коллектива таких систем, все объединяемые в коллектив интеллектуальные системы должны иметь высокий уровень интероперабельности, что накладывает дополнительные требования, предъявляемые к информации, хранимой в памяти, а также к решателям задач интеллектуальных систем, объединяемых в коллектив.

Интероперабельность кибернетической системы есть способность кибернетической системы взаимодействовать с другими кибернетическими системами в целях создания коллектива кибернетических систем (многоагентных систем), уровень качества и, в частности, уровень интеллекта которого выше уровня качества каждой кибернетической системы, входящей в состав этого коллектива.

Для того, чтобы количество членов коллектива кибернетической системы перешло в более высокое качество самого коллектива, члены коллектива должны обладать дополнительными способностями, которые будем называть свойствами интероперабельности.
Основными такими свойствами являются способность устанавливать и поддерживать достаточный уровень семантической совместимости (взаимопонимания) с другими кибернетическими системами и договороспособность (способность согласовывать свои действия с другими) \cite{NEIVA2016Interoperability}.

Целенаправленный обмен информацией между кибернетическими системами существенно ускоряет процесс их обучения (процесс накопления знаний и навыков).
Следовательно, способность эффективно использовать указанный канал накопления знаний и навыков существенно повышает уровень обучаемости кибернетических систем.
Повышение уровня интероперабельности кибернетической системы является, с одной стороны, дополнительным повышением уровня интеллекта самой этой кибернетической системы, а также фактором повышения уровня интеллекта тех коллективов, тех многоагентных систем, в состав которых эта кибернетическая система входит.

\begin{SCn}
\scnheader{интероперабельность кибернетической системы}
\begin{scnrelfromlist}{cвойство-предпосылка}
    \scnitem{договороспособность кибернетической системы}
    \scnitem{социальная ответственность кибернетической системы}
    \scnitem{социальная активность кибернетической системы}
\end{scnrelfromlist}
\end{SCn}

Свойства-предпосылки уровня договороспособности кибернетической системы представлены ниже:

\begin{SCn}
\scnheader{договороспособность кибернетической системы}
\begin{scnrelfromlist}{cвойство-предпосылка}
    \scnitem{способность кибернетической системы к пониманию принимаемых сообщений}
    \scnitem{способность кибернетической системы к формированию передаваемых сообщений, понятных адресатам}
    \scnitem{способность кибернетической системы к обеспечению семантической совместимости с партнёрами}
    \scnitem{коммуникабельность кибернетической системы}
    \scnitem{способность кибернетической системы к обсуждению и согласованию целей и планов коллективной деятельности}
    \scnitem{способность кибернетической системы брать на себя выполнение актуальных задач в рамках согласованных планов коллективной деятельности}
\end{scnrelfromlist}
\end{SCn}

Понимание информации, поступающей извне, включает в себя:
\begin{itemize}
    \item{перевод этой информации на внутренний язык кибернетической системы;}
    \item{локальную верификацию вводимой информации;}
    \item{погружение (конвергенцию, размещение) текста, являющегося результатом указанного перевода в состав хранимой информации (в частности, в состав базы знаний).}
\end{itemize}

Погружение вводимой информации в состав базы знаний кибернетической системы сводится к выявлению и устранению противоречий, возникающих между погружаемым текстом и текущего состояния базы знаний.
Сложность проблемы понимания вводимой вербальной информации заключается не только в сложности непротиворечивого погружения вводимой информации в текущее состояние базы знаний, но и в сложности трансляции этой информации с внешнего языка на внутренний язык кибернетической системы, т. е. в сложности генерации текста внутреннего языка, семантически эквивалентного вводимому тексту внешнего языка.
Для естественных языков указанная трансляция является сложной задачей, так как в настоящее время проблема формализации синтаксиса и семантики естественных языков не решена.

Семантическая совместимость двух заданных кибернетических систем определяется согласованностью систем понятий, используемых обеими взаимодействующими кибернетическими системами.
Проблема обеспечения перманентной поддержки семантической совместимости взаимодействующих кибернетических систем является необходимым условием обеспечения высокого уровня взаимопонимания кибернетических систем и, как следствие, эффективного их взаимодействия.

Коммуникабельность кибернетической системы есть способность кибернетической системы к установлению взаимовыгодных контактов с другими кибернетическими системами (в том числе, с коллективами интеллектуальных систем) путем честного выявления взаимовыгодных общих целей (интересов).

Свойства-предпосылки уровня социальной ответственности кибернетической системы представлены ниже:

\begin{SCn}
\scnheader{социальная ответственность кибернетической системы}
\begin{scnrelfromlist}{cвойство-предпосылка}
    \scnitem{способность кибернетической системы выполнять качественно и в срок взятые на себя обязательства в рамках соответствующих коллективов}
    \scnitem{способность кибернетической системы адекватно оценивать свои возможности при распределении коллективной деятельности}
    \scnitem{альтруизм/эгоизм кибернетической системы}
    \scnitem{отсутствие/наличие действий, которые по безграмотности кибернетической системы снижают качество коллективов, в состав которых она входит}
    \scnitem{отсутствие/наличие "осознанных"{}, мотивированных действий, снижающих качество коллективов, в состав которых кибернетическая система входит}
\end{scnrelfromlist}
\end{SCn}

Свойства-предпосылки уровня социальной активности кибернетической системы представлены ниже:

\begin{SCn}
\scnheader{социальная активность кибернетической системы}
\begin{scnrelfromlist}{cвойство-предпосылка}
    \scnitem{способность кибернетической системы к генерации предлагаемых целей и планов коллективной деятельности}
    \scnitem{активность кибернетической системы в экспертизе результатов других участников коллективной деятельности}
    \scnitem{способность кибернетической системы к анализу качества всех коллективов, в состав которых она входит, а также всех членов этих коллективов}
    \scnitem{способность кибернетической системы к участию в формировании новых коллективов}
    \scnitem{количество и качество тех коллективов, в состав которых кибернетическая система входит или входила}
\end{scnrelfromlist}
\end{SCn}

Формирование специализированного коллектива кибернетических систем сводится к тому, что в памяти каждой кибернетической системы, входящей в коллектив, генерируется спецификация этого коллектива, включающая в себя:
\begin{itemize}
    \item{перечень весь членов коллектива;}
    \item{способности каждого из членов коллектива;}
    \item{их обязанности в рамках коллектива;}
    \item{спецификацию всего множества задач (вида деятельности), для решения (выполнения) которых сформирован данный коллектив кибернетических систем.}
\end{itemize}

Каждая кибернетическая система может входить в состав большого количества коллективов, выполняя при этом в разных коллективах в общем случае разные "должностные обязанности"{}, разные "бизнес-процессы"{}.
