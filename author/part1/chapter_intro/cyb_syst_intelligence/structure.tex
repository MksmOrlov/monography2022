\subsection{Структура кибернетической системы}
{\label{sec_cyb_syst_structure}} 

Рассмотрим структуру кибернетической системы.
\begin{SCn}
	\scnheader{кибернетическая система}
	\begin{scnrelfromset}{обобщенная декомпозиция}
		\scnitem{информация, хранимая в памяти кибернетической системы}
		\scnitem{абстрактная память кибернетической системы}
		\scnitem{решатель задач кибернетической системы}
		\scnitem{физическая оболочка кибернетической системы}
	\end{scnrelfromset}
\end{SCn}

Информация, хранимая в памяти \textit{кибернетической системы}, представляет собой информационную модель среды, в которой действует (существует, функционирует) эта \textit{кибернетическая система}, текущее состояние памяти кибернетической системы.

Абстрактная память кибернетической системы есть внутренняя абстрактная информационная среда кибернетической системы, представляющая собой динамическую информационную  конструкцию, каждое состояние которой есть не что иное, как информация, хранимая в памяти кибернетической системы в соответствующий момент времени. Абстрактная память является подмножеством динамической информационной конструкции.

Решателем задач кибернетической системы называют совокупность всех навыков (умений), приобретенных кибернетической системой к рассматриваемому моменту. Встроенный в кибернетическую систему субъект, способный выполнять целенаправленные ("осознанные") действия во внешней среде этой кибернетической системы, а также в её внутренней среде (в абстрактной памяти).


% Возможно фрагмент излишний, стоит переместить в другие разделы, например понятие действия в sc-памяти
Рассмотрим подробнее понятие действия кибернетической системы.

\begin{SCn}
\scnheader{действие кибернетической системы}
\scnsubset{действие}
\scnidtf{целенаправленное действие, выполняемое кибернетической системой, а точнее, её решателем задач}
\begin{scnrelfromset}{разбиение}
	\scnitem{внешнее действие кибернетической системы}
	\begin{scnindent}
		\scnidtf{действие, выполняемое кибернетической системой в её внешней среде}
		\scnidtf{поведенческое действие}
	\end{scnindent}

	\scnitem{действие кибернетической системы, выполняемое в собственной физической оболочке}
	\scnitem{действие кибернетической системы, выполняемое в собственной абстрактной памяти}
\end{scnrelfromset}
\end{SCn}

Говоря о действиях кибернетической системы, выполняемых в собственной абстрактной памяти, подразумеваются действия, направленные на преобразование информации, хранимой в памяти, но никак не на преобразование физической памяти (физической оболочки абстрактной памяти).

Каждое \uline{сложное} действие, выполняемое кибернетической системой вне собственный абстрактной памяти, включает в себя поддействия, выполняемые в указанной абстрактной памяти. 
Это означает, что все внешние действия кибернетической системы \uline{управляются} внутренними её действиями (действиями в абстрактной памяти).

% понятие задачи
% типология задач, решаемых кибернетической системой
% понятие навыка

Интерфейс кибернетической системы --- условно выделяемый компонент \textit{решателя задач кибернетической системы}, обеспечивающий решение \textit{интерфейсных задач}, направленных на \uline{непосредственную} реализацию взаимодействия \textit{кибернетической системы} с её \textit{внешней средой}. Решатель интерфейсных задач кибернетической системы. Стоит отличать понятие интерфейса кибернетической системы и понятие физического обеспечения интерфейса кибернетической системы.

Физическая оболочка кибернетической системы --- часть кибернетической системы, являющаяся "посредником"{} между её внутренней средой (памятью, в которой хранится и обрабатывается информация кибернетической системы) и её внешней средой.

\begin{SCn}
	\scnheader{кибернетическая система}
	\begin{scnrelfromset}{обобщенная декомпозиция}
		\scnitem{память кибернетической системы}
		\scnitem{процессор кибернетической системы}
		\scnitem{физическое обеспечение интерфейса кибернетической системы}
		\scnitem{корпус кибернетической системы}
	\end{scnrelfromset}
\end{SCn}

% физическое обеспечение интерфейса кибернетической системы

\begin{SCn}
	\scnheader{физическое обеспечение интерфейса кибернетической системы}
	\begin{scnrelfromset}{обобщенная декомпозиция}
		\scnitem{сенсорная подсистема физической оболочки кибернетической системы}
		\scnitem{эффекторная подсистема физической оболочки кибернетической системы}
	\end{scnrelfromset}
\end{SCn}

Память кибернетической системы --- физическая оболочка (реализация) абстрактной \textit{памяти кибернетической системы}, внутренней среды \textit{кибернетической системы}, в рамках которой \textit{кибернетическая система} формирует и использует (обрабатывает) информационную модель своей внешней среды.

Не каждая \textit{кибернетическая система} имеет \textit{память}. 
В \textit{кибернетических системах}, которые не имеют \textit{памяти}, обработка информации сводится к обмену сигналами между компонентами этих систем. Появление в \textit{кибернетических системах} памяти как среды для "централизованного"{} хранения и обработки \textit{информации} является важнейшим этапом их эволюции. Дальнейшая эволюция \textit{кибернетических систем} во многом определяется:
\begin{textitemize}
	\item \textit{качеством памяти} как среды для хранения и обработки информации;
	\item качеством информации (информационной модели), хранимой в памяти кибернетической системы;
\end{textitemize}

Компонент кибернетической системы, в рамках которого \textit{кибернетическая система} осуществляет отображение (формирование информационной модели) среды своего существования, а также использование этой информационной модели для управления собственным поведением в указанной среде

Сам факт появления в кибернетической системе памяти, которая (1) обеспечивает представление различного виды информации о среде, в рамках которой кибернетическая система решает различные задачи (выполняет различные действия), (2) обеспечивает хранение достаточно полной информационной модели указанной среды (достаточно полной для реализации своей деятельности), (3) обеспечивает высокую степень гибкости указанной хранимой в памяти информационной модели среды жизнедеятельности (то есть лёгкость внесения изменений в эту информационную модель), существенно повышает уровень адаптивности кибернетической системы к различным изменениям своей среды.
Появление{} \uline{\textit{памяти}} в кибернетических системах является основным признаком перехода от "простых"{} автоматов к компьютерным системам, от роботов 1-го поколения к роботам следующих поколений.

Принципы организации памяти кибернетической системы могут быть разными (ассоциативная, адресная, структурно фиксированная/структурно перестраиваемая, нелинейная/линейная). 
От организации памяти во многом зависит её качество.

% Уровни структурной эволюции кибернетических систем

Процессором кибернетической системы называют физически реализованный интерпретатор хранимых в памяти кибернетической системы программ, соответствующих базовой модели решения задач для данной кибернетической системы. Такая модель решения задач для данной кибернетической системы является моделью решения задач самого нижнего уровня и не может быть интерпретирована с помощью другой модели решения задач, используемой этой же кибернетической системой. Она может быть проинтерпретирована либо путем аппаратной реализации такого интерпретатора, либо путём его программной реализации, например, на современных компьютерах. В последнем случае, кроме собственного интерпретатора, необходимо также построить модель памяти реализуемой кибернетической системы.

Развитие рынка интеллектуальных компьютерных систем существенно сдерживается неприспособленностью современного поколения компьютеров к реализации на их основе интеллектуальных компьютерных систем. Попытки создания компьютеров, приспособленных к реализации интеллектуальных компьютерных систем, не привели к успеху, т.к. эти проекты были направлены на выполнение частных требований, предъявляемых к аппаратному уровню интеллектуальных систем, что неминуемо приводило к приспособленности создаваемых компьютеров к реализации не всего многообразия интеллектуальных компьютерных систем, а только некоторых подмножеств таких систем. Указанные подмножества интеллектуальных компьютерных систем в основном определялись ориентацией на конкретные используемые модели решения интеллектуальных задач, тогда как важнейшим фактором, определяющим уровень интеллекта кибернетических систем, является их универсальность в плане многообразия используемых моделей решения задач. Следовательно, компьютер для интеллектуальных компьютерных систем должен быть эффективным аппаратным интерпретатором любых моделей решения задач, как интеллектуальных, так и достаточно простых.

Таким образом, выделено следующее семейство отношений, заданных на множестве кибернетических систем.

\begin{SCn}
\scnheader{отношение, заданное на множестве кибернетических систем}
\scnhaselement{память кибернетической системы*}
\scnhaselement{процессор кибернетической системы*}
\scnhaselement{член коллектива*}
\scnhaselement{внешняя среда кибернетической системы*}
\scnhaselement{сенсор кибернетической системы*}
\scnhaselement{эффектор кибернетической системы*}
\scnhaselement{физическая оболочка кибернетической системы*}
\scnhaselement{информация, хранимая в памяти кибернетической системы*}
\scnhaselement{абстрактная память кибернетической системы*}
\scnhaselement{часть*}
\begin{scnindent}
	\scnsuperset{встроенная кибернетическая система*}
\end{scnindent}
\end{SCn}

Понятие \textit{внешней среды кибернетической системы*} является понятием относительным, т.к. (1) разные кибернетические системы могут иметь разную внешнюю среду и (2) одна кибернетическая система может входить в состав внешней среды другой кибернетической системы. 
В общем случае среда жизнедеятельности \textit{кибернетической системы} включает в себя (1) \textit{внешнюю среду*} этой системы, (2) \textit{физическую оболочку*} этой системы и (3) её \textit{абстрактную память}, т.е. внутреннюю среду*, которая является хранилищем информационной модели всей среды.
