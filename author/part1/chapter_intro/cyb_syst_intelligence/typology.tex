\subsection{Типология кибернетических систем}
{\label{sec_cyb_syst_typology}} 

\begin{SCn}
\scnheader{кибернетическая система}
\scnrelfrom{разбиение}{Признак естественности или искусственности кибернетических систем}
\begin{scnindent}
	\begin{scneqtoset}
		\scnitem{естественная кибернетическая система}
		\begin{scnindent}
			\scnidtf{кибернетическая система естественного происхождения}
   			\scnsuperset{человек}
    	\end{scnindent}
		\scnitem{компьютерная система}
		\begin{scnindent}
			\scnidtf{искусственная кибернетическая система}
			\scnidtf{кибернетическая система искусственного происхождения}
			\scnidtf{технически реализованная кибернетическая система}
		\end{scnindent}
		\scnitem{симбиоз естественных и искусственных кибернетических систем}
		\begin{scnindent}
			\scnidtf{кибернетическая система, в состав которой входят компоненты как естественного, так и
			искусственного происхождения}
			\scnsuperset{сообщество компьютерных систем и людей}
		\end{scnindent}
	\end{scneqtoset}
\end{scnindent}
\end{SCn}

Особенностью \textit{компьютерных систем} является то, что они могут выполнять "роль" не только продуктов соответствующих действий по реализации этих систем, но и сами являются \textit{субъектами*}, способными выполнять (автоматизировать) широкий спектр действий.
При этом интеллектуализация этих систем существенно расширяет этот спектр.

\begin{SCn}
\scnheader{кибернетическая система}
\scnrelfrom{разбиение}{Структурная классификация кибернетических систем}
\begin{scnindent}
	\begin{scneqtoset}
		\scnitem{простая кибернетическая система}
		\scnitem{индивидуальная кибернетическая система}
		\scnitem{многоагентая система}
	\end{scneqtoset}
\end{scnindent}
\scnrelfrom{разбиение}{Классификация кибернетических систем по признаку наличия надсистемы и роли в рамках этой надсистемы}
\begin{scnindent}
	\begin{scneqtoset}
		\scnitem{кибернетическая система, не являющаяся частью никакой другой кибернетической системы}
		\begin{scnindent}
			\scnidtf{кибернетическая система, не имеющая надсистем}	
		\end{scnindent}
		\scnitem{кибернетическая система, встроенная в индивидуальную кибернетическую систему}
		\scnitem{агент многоагентной системы}
		\begin{scnindent}
			\scnidtf{кибернетическая система, являющаяся агентом одной или нескольких многоагентных систем}
		\end{scnindent}
	\end{scneqtoset}
\end{scnindent}
\end{SCn}

\textbf{\textit{простая кибернетическая система}} --- это \textit{кибернетическая система}, уровень развития которой находится ниже уровня \textit{индивидуальных кибернетических систем} и которая является \textit{специализированным решателем задач}, реализующим (интерпретирующим) чаще всего один \textit{метод решения задач} и, соответственно, решающим только \textit{задачи} заданного \textit{класса задач}.
\textit{простая кибернетическая система} может быть \textit{компонентом*}, встроенным в \textit{индивидуальную кибернетическую систему}, а также может быть \textit{агентом*} \textit{многоагентной системы}, являющейся коллективом из \textit{простых кибернетических систем}.

\textbf{\textit{индивидуальная кибернетическая система}} --- условно выделенный уровень развития \textit{кибернетических систем}, в основе которого лежит переход от \textit{специализированного решателя задач} к \textit{индивидуальному решателю задач}, обеспечивающему интерпретацию произвольного (нефиксированного) набора \textit{методов} (программ) решения задач при условии, если эти \textit{методы} введены (загружены, записаны) в \textit{память} \textit{кибернетической системы}.
Признаками \textit{индивидуальных кибернетических систем} являются:
\begin{textitemize}
    \item наличие \textit{памяти}, предназначенной для хранения как минимум интерпретируемых \textit{методов}	(программ)  и обеспечивающей корректировку (редактирование) хранимых \textit{методов}, а также их удаление из	\textit{памяти} и ввод (запись) в \textit{память} новых \textit{методов};
    \item легкая возможность "перепрограммировать"{} \textit{кибернетическую систему} на решение других задач, что обеспечивается наличием \textit{универсальной модели решения задач} и, соответственно, \textit{универсальным интерпретатором \uline{любых} моделей}, представленных (записанных) на соответствующем \textit{языке};
    \item наличие пусть даже простых средств коммуникации (обмена информацией) с другими \textit{кибернетическими системами} (например, с людьми);
    \item способность входить в различные \textit{коллективы кибернетических систем}.
\end{textitemize}

Класс \textit{индивидуальных кибернетических систем} — это определенный этап эволюции кибернетических систем, означающий переход к кибернетическим системам, которые способны самостоятельно "выживать". \textit{индивидуальная кибернетическая система} может быть агентом (членом) \textit{многоагентной системы} (членом коллектива индивидуальных кибернетических систем), но некоторые многоагентные системы могут состоять из агентов, не являющихся \textit{индивидуальными кибернетическими системами}, представляющих собой простые специализированные кибернетические системы, выполняющие достаточно простые действия (\scncite{Stefanuk}, \scncite{fonNeuman}).

Примерами \textit{кибернетической системы}, встроенной в \textit{индивидуальную кибернетическую систему}, являются \textit{sc-агент ostis-системы} и \textit{решатель задач ostis-системы}.

\textbf{\textit{многоагентная система}} представляет собой коллектив взаимодействующих автономных \textit{кибернетических систем}, имеющих общую среду обитания (жизнедеятельности). 

\begin{SCn}
\scnheader{многоагентая система}
\begin{scnrelfromset}{разбиение}
	\scnitem{коллектив из простых кибернетических систем}
	\scnitem{коллектив индивидуальных кибернетических систем}
	\scnitem{коллектив индивидуальных и простых кибернетических систем}
\end{scnrelfromset}
\begin{scnrelfromset}{разбиение}
	\scnitem{одноуровневый коллектив кибернетических систем}
	\begin{scnindent}
		\scnidtf{многоагентная система, агентами которой не могут быть многоагентные системы}
	\end{scnindent}
	\scnitem{иерархический коллектив кибернетических систем}
	\begin{scnindent}
		\scnidtf{многоагентная система, по крайней мере одним  агентом которой является многоагентная система}
	\end{scnindent}
\end{scnrelfromset}
\end{SCn}


\textbf{\textit{коллектив индивидуальных кибернетических систем}} --- \textbf{\textit{многоагентная система}}, агентами (членами) которой являются \textit{\uline{индивидуальные} кибернетические системы}. Примерами \textit{коллектива индивидуальных кибернетических систем} могут быть как коллективы людей, так и коллективы компьютерных систем и людей.

\textbf{\textit{одноуровневый коллектив кибернетических систем}} определён как специализированное средство решения задач, реализующее либо \uline{одну} модель параллельного (распределенного) решения задач соответствующего класса, либо комбинацию \uline{фиксированного числа} разных и параллельно реализованных \textit{моделей решения задач}.

Агентами \textbf{\textit{иерархического коллектива кибернетических систем}} могут быть:
\begin{textitemize}
    \item \textit{индивидуальные кибернетические системы};
    \item \textit{коллективы индивидуальных кибернетических систем};
    \item \textit{коллективы, состоящие из индивидуальных кибернетических систем и коллективов индивидуальных кибернетических систем} и так далее.
\end{textitemize}
