\section{Понятие интеллектуальной многоагентной системы}
{\label{sec_mas}} 

Переход от \textit{кибернетических систем} к \textit{коллективам} взаимодействующих между собой \textit{кибернетических систем}, то есть к социальной организации \textit{кибернетических систем}, является важнейшим фактором эволюции \textit{кибернетических систем}. 
\textit{кибернетическую систему}, представляющую собой \textit{коллектив} взаимодействующих \textit{кибернетических систем}, обладающих определенной степенью самостоятельности (самодостаточности, свободы выбора), будем называть \textbf{\textit{многоагентной системой}} (см. \scncite{Dorri2018}).

\begin{SCn}
\scnheader{кибернетическая система}
\begin{scnrelfromset}{разбиение}
    \scnitem{индивидуальная кибернетическая система}
    \scnitem{кибернетическая система, являющаяся минимальным компонентом индивидуальной кибернетической системы}
    \scnitem{кибернетическая система, являющаяся комплексом компонентов соответствующей индивидуальной кибернетической системы}
    \scnitem{сообщество индивидуальных кибернетических систем}
    \begin{scnindent}
    \begin{scnrelfromset}{разбиение}
        \scnitem{простое сообщество индивидуальных кибернетических систем}
        \scnitem{иерархическое сообщество индивидуальных кибернетических систем}
    \end{scnrelfromset}
    \end{scnindent}
\end{scnrelfromset}
\end{SCn}

\begin{SCn}
	\scnheader{Теория многоагентных систем}
	\scnexplanation{Теория многоагентных систем --- это теория перехода количества \textit{кибернетических систем} в \textit{кибернетическую систему} нового качества, --- это выявление принципов и методов, позволяющие множество \textit{кибернетических систем соединить} в \textit{коллективную кибернетическую систему}, обладающую существенно более высоким уровнем качества (в том числе интеллекта) по сравнению с \textit{качеством кибернетической системы}, входящих в этот коллектив.}
\end{SCn}


Агенты \textit{многоагентной системы} могут (но вовсе не обязательно должны) быть \textit{интеллектуальными системами}. 
Так, например, агенты \textit{интеллектуального решателя задач}, имеющего агентно-ориентированную архитектуру, не являются \textit{интеллектуальными системами}. Агентом \textit{иерархической многоагентной системы} может быть другая \textit{многоагентная система} (см. \scncite{Ferber2003}).

\textit{многоагентная система} --- \textit{коллектив} взаимодействующих автономных \textit{кибернетических систем}, имеющих общую среду обитания, жизнедеятельности (см. \scncite{Hadzic2009}). 

Классификация \textit{многоагентных систем} приведена ниже:

\begin{SCn}
\scnheader{многоагентная система}
\begin{scnrelfromset}{разбиение}
    \scnitem{одноуровневая многоагентная система}
    \scnitem{иерархическая многоагентная система}
\end{scnrelfromset}
\begin{scnrelfromset}{разбиение}
	\scnitem{многоагентная система с централизованным управлением}
	\scnitem{многоагентная система с децентрализованным управлением}
\end{scnrelfromset}
\end{SCn}

\textbf{\textit{одноуровневая многоагентная система}} реализует либо одну модель параллельного (распределенного) решения задач соответствующего класса, либо комбинацию фиксированного числа разных и параллельно реализованных \textit{моделей решения задач}. 
\textbf{\textit{иерархическая многоагентная система}} состоит из агентов, которые могут быть \textit{индивидуальными кибернетическими системами}, \textit{коллективами индивидуальных кибернетических систем}, а также \textit{коллективами, состоящими из индивидуальных кибернетических систем и коллективов индивидуальных кибернетических систем}.

В \textbf{\textit{многоагентной системе с централизованным управлением}} специально выделяются агенты, которые принимают решения в определенной области деятельности \textit{многоагентной системы} и обеспечивают выполнение этих решений путем управления деятельностью остальных агентов, входящих в состав этой системы.

В \textbf{\textit{многоагентной системе с децентрализованным управлением}} решения принимаются коллегиально и "автоматически"{} (решения о признании новой кем-то предложенной информации --- в том числе, об инициировании некоторой задачи, решения о коррекции (уточнении) уже ранее признанной, согласованной информации) на основе четко продуманной и постоянно совершенствуемой методики, а также на основе активного участия всех агентов в формировании новых предложений, подлежащих признанию или согласованию (см. \scncite{Balaji2010}). 
В такой \textit{многоагентной системе} все агенты участвуют в управлении этой системы. 
Примером такой системы является оркестр, способный играть без дирижера.

\input{author/part1/chapter_intro/mas_intelligence/overall_quality.tex}
