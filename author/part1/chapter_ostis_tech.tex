\chapter{Принципы, лежащие в основе технологии комплексной поддержки жизненного цикла интеллектуальных компьютерных систем нового поколения}
\chapauthortoc{Голенков В.В.\\Шункевич Д.В.\\Ковалёв М.В.\\Садовский М.Е.}
\label{chapter_ostis_tech} 

\vspace{-7\baselineskip}

\begin{SCn}
\begin{scnrelfromlist}{автор}
	\scnitem{Голенков В.В.}
	\scnitem{Шункевич Д.В.}
	\scnitem{Ковалёв М.В.}
	\scnitem{Садовский М.Е.}
\end{scnrelfromlist}

\bigskip

\scntext{аннотация}{В главе рассмотрены принципы построения комплексной технологии разработки и поддержки жизненного цикла интеллектуальных компьютерных систем нового поколения -- Технологии OSTIS.}

\bigskip

\begin{scnrelfromlist}{подраздел}
	\scnitem{\ref{sec_ostis_tech}~\nameref{sec_ostis_tech}}
	\scnitem{\ref{sec_sem_compatible_os}~\nameref{sec_sem_compatible_os}}
\end{scnrelfromlist}

\end{SCn}

\section{Технология OSTIS (Open Semantic Technology for Intelligent Systems)}
\label{sec_ostis_tech}

\begin{SCn}
\begin{scnrelfromlist}{ключевой знак}
	\scnitem{Технология OSTIS}
	\scnitem{ostis-система}
	\scnitem{Стандарт ostis-систем}
	\scnitem{Метасистема OSTIS}
	\scnitem{Стандарт OSTIS}
	\scnitem{платформа ostis-систем}
	\scnitem{Экосистема OSTIS}
\end{scnrelfromlist}

\begin{scnrelfromlist}{ключевое знание}
	\scnitem{Обобщенный жизненный цикл ostis-систем}
	\scnitem{Принципы, лежащие в основе Технологии OSTIS}
\end{scnrelfromlist}	 

\scnheader{жизненный цикл интеллектуальной компьютерной системы нового поколения}
\begin{scnrelfromlistcustom}{включает в себя}
	\scnitemcustom{проектирование интеллектуальной компьютерной системы нового поколения}
	\begin{scnindent}
		\begin{scnrelfromlistcustom}{включает в себя}
			\scnitemcustom{проектирование базы знаний интеллектуальной компьютерной системы нового поколения}
			\scnitemcustom{проектирование решателя задач интеллектуальной компьютерной системы нового поколения}
			\scnitemcustom{проектирование интерфейса интеллектуальной компьютерной системы нового поколения}
		\end{scnrelfromlistcustom}
	\end{scnindent}
	\scnitemcustom{реализацию интеллектуальной компьютерной системы нового поколения}
	\scnitemcustom{начальное обучение интеллектуальной компьютерной системы нового поколения}
	\scnitemcustom{мониторинг качества интеллектуальной компьютерной системы нового поколения}
	\scnitemcustom{восстановление требуемого уровня интеллектуальной компьютерной системы нового поколения}
	\scnitemcustom{реинжиниринг интеллектуальной компьютерной системы нового поколения}
	\scnitemcustom{обеспечение безопасности интеллектуальной компьютерной системы нового поколения}
	\scnitemcustom{эксплуатация интеллектуальной компьютерной системы нового поколения конечными пользователями}
\end{scnrelfromlistcustom}
\end{SCn}

Построение \textit{Технологии} \textit{комплексной поддержки жизненного цикла интеллектуальных компьютерных систем нового поколения} предполагает:

\begin{textitemize}
	\item
	Четкое описание текущей версии \textit{стандарта интеллектуальных компьютерных систем нового поколения}, обеспечивающего семантическую совместимость разрабатываемых систем;
	\item
	Создание мощных библиотек семантически совместимых и многократно (повторно) используемых компонентов разрабатываемых \textit{интеллектуальных компьютерных систем};
	\item
	Уточнение требований, предъявляемых к создаваемой комплексной технологии и обусловленных особенностями \textit{интеллектуальных компьютерных систем нового поколения}, разрабатываемых и эксплуатируемых с помощью указанной технологии.
\end{textitemize}

Создание инфраструктуры, обеспечивающей интенсивное перманентное развитие \textit{Технологии} \textit{комплексной поддержки жизненного цикла интеллектуальных компьютерных систем нового поколения} предполагает:

\begin{textitemize}
	\item
	Обеспечение низкого порога вхождения в \textit{технологию проектирования интеллектуальных компьютерных систем} как для пользователей технологии (т.е. разработчиков прикладных или специализированных интеллектуальных компьютерных систем), так и для разработчиков самой технологии;
	\item
	Обеспечение высоких темпов развития \textit{технологии} за счет учета опыта разработки различных приложений путем активного привлечения авторов приложений к участию в развитии (совершенствовании) \textit{технологии}.
\end{textitemize}

В основе создания предлагаемой нами \textbf{\textit{технологии комплексной поддержки жизненного цикла интеллектуальных компьютерных систем нового поколения}} лежат следующие положения:
\begin{textitemize}
	\item реализация предлагаемой \textit{технологии} разработки и сопровождения \textit{интеллектуальных компьютерных систем нового поколения} в виде \textbf{\textit{интеллектуальной компьютерной метасистемы}}, которая полностью соответствует \textit{стандартам} предлагаемых \textit{интеллектуальных компьютерных систем нового поколения}, разрабатываемым по предлагаемой \textit{технологии}. В состав такой \textit{интеллектуальной компьютерной метасистемы}, реализующей предлагаемую технологию входит:
	
	\begin{textitemize}
		\item  формальное онтологическое описание текущей версии \textit{стандарта интеллектуальных компьютерных систем нового поколения};
		\item  формальное онтологическое описание текущей версии \textit{методов и средств проектирования, реализации, сопровождения, реинжиниринга и эксплуатации интеллектуальных компьютерных систем нового поколения}.
	\end{textitemize}
	
	Благодаря этому технология проектирования и реинжиниринга интеллектуальных компьютерных систем нового поколения и технология проектирования и реинжиниринга самой указанной технологии (т.е. интеллектуальной компьютерной метасистемы) суть одно и тоже;
	
	\item \textbf{\textit{унификация}} и \textbf{\textit{стандартизация} интеллектуальных компьютерных систем нового поколения}, а также \textit{методов} их \textit{проектирования, реализации, сопровождения, реинжиниринга и эксплуатации};
	\item перманентная эволюция \textbf{\textit{стандарта интеллектуальных компьютерных систем нового поколения}}, а также \textit{методов} их \textit{проектирования, реализации, сопровождения, реинжиниринга и эксплуатации;}
	\item \textbf{\textit{онтологическое проектирование} интеллектуальных компьютерных систем нового поколения}, предполагающее:
	
	\begin{textitemize}
		\item  четкое согласование и оперативную формализованную фиксацию (в виде \textit{формальных онтологий}) утвержденного \textit{текущего состояния} иерархической системы всех \textit{понятий}, лежащих в основе перманентно эволюционируемого \textit{стандарта интеллектуальных компьютерных систем нового поколения}, а также в основе каждой разрабатываемой \textit{интеллектуальной компьютерной системы};
		
		\item  достаточно полное и оперативное документирование текущего состояния каждого проекта;
		
		\item  использование \textit{методики проектирования} \textit{"сверху-вниз"{}}.
	\end{textitemize}
	
	\item \textbf{\textit{компонентное проектирование} интеллектуальных компьютерных систем нового поколения}, т.е. проектирование, ориентированное на сборку \textit{интеллектуальных компьютерных систем} из готовых компонентов на основе постоянно расширяемых библиотек \textit{многократно используемых компонентов};
	
	\item  \textbf{\textit{комплексный характер}} предлагаемой \textit{технологии}, осуществляющей:
	
	\begin{textitemize}
		\item  поддержку \textit{проектирования} не только \textit{компонентов} \textit{интеллектуальных компьютерных систем нового поколения} (различных \textit{фрагментов баз знаний, баз знаний} в целом, различных \textit{методов решения задач}, различных \textit{внутренних информационных агентов, решателей задач} в целом, формальных онтологических описаний различных \textit{внешних языков}, \textit{интерфейсов} в целом), но также и \textit{интеллектуальных компьютерных систем} в целом как самостоятельных \textit{объектов проектирования} с учетом специфики тех классов, которым принадлежат проектируемые \textit{интеллектуальных компьютерных системы};
		
		\item  поддержку не только \textit{комплексного} \textit{проектирования} \textit{интеллектуальных компьютерных систем} \textit{нового поколения}, но также и поддержку их реализации (сборки, воспроизводства), сопровождения, реинжиниринга в ходе эксплуатации и непосредственно самой эксплуатации.
		
	\end{textitemize}
\end{textitemize}

Для создания \textit{технологии} комплексного проектирования и комплексной поддержки последующих этапов жизненного цикла \textit{интеллектуальных компьютерных систем нового поколения} необходимо:

\begin{textitemize}
	\item  Унифицировать формализацию различных моделей представления различного вида используемой информации, хранимой в памяти \textit{интеллектуальных компьютерных систем} и различных моделей решения интеллектуальных задач для обеспечения \textit{семантической совместимости} и простой автоматизируемой интегрируемости различных видов \textit{знаний} и \textit{моделей решения задач} в \textit{интеллектуальных компьютерных системах}. Для этого необходимо разработать базовую \textit{универсальную} абстрактную модель представления и обработки знаний, обеспечивающую реализацию всевозможных моделей решения задач.
	
	\item  Унифицировать структуризацию \textit{баз знаний} интеллектуальных компьютерных систем в виде иерархической системы онтологий разного уровня.
	
	\item  Унифицировать систему используемых \textit{понятий}, специфицируемых соответствующими \textit{онтологиями} для обеспечения \textit{семантической совместимости} и \textit{интероперабельности} различных \textit{интеллектуальных компьютерных систем}.
	
	\item  Унифицировать архитектуру \textit{интеллектуальных компьютерных систем}, обеспечивающую \textit{семантическую совместимость}:
	
	\begin{textitemize}
		\item между \textit{интеллектуальными компьютерными системами} и их пользователями
		\item между \textit{индивидуальными интеллектуальными компьютерными системами};
		\item между \textit{коллективными интеллектуальными компьютерными системами},
	\end{textitemize}
	
	а также обеспечивающую \textit{интероперабельность} сообществ, состоящих из:
	
	\begin{textitemize}
		\item \textit{индивидуальных интеллектуальных компьютерных систем};
		\item \textit{коллективных интеллектуальных компьютерных систем};
		\item пользователей интеллектуальных компьютерных систем
	\end{textitemize}
	
	\item Разработать \textit{базовую модель интерпретации} всевозможных формальных моделей решения задач в интеллектуальных компьютерных системах с ориентацией на максимально возможное упрощение такой интерпретации в \textit{компьютерах нового поколения}, которые специально предназначены для реализации индивидуальных \textit{интеллектуальных компьютерных систем}.
	\item Разработать \textit{компьютеры нового поколения}, принципы функционирования которых максимально близки к базовой абстрактной модели, обеспечивающей интеграцию всевозможных моделей представления знаний и моделей решения задач. При этом базовая машина обработки информации, лежащая в основе указанных компьютеров, должна существенно отличаться от фон-Неймановской машины и должна быть близка к базовой модели решения задач в интеллектуальных компьютерных системах для того, чтобы существенно снизить сложность интерпретации всего многообразия моделей решения задач в интеллектуальных компьютерных системах.
	
\end{textitemize}

Реализация всех перечисленных этапов развития \textit{технологий Искусственного интеллекта} представляет собой переход на принципиально новый технологический уклад, обеспечивающий существенное повышение эффективности практического использования результатов работ в области \textit{Искусственного интеллекта} и существенное повышение уровня автоматизации \textit{человеческой деятельности.}

Предложенную нами \textit{технологию комплексной поддержки жизненного цикла интеллектуальных компьютерных систем нового поколения} мы назвали \textbf{\textit{Технологией OSTIS}} (Open Semantic Technology for Intelligent Systems). Соответственно этому \textit{интеллектуальные компьютерные системы нового поколения}, разрабатываемые по этой технологии называются \textbf{\textit{ostis-системами}}. Сама \textit{Технология OSTIS} реализуется нами в форме специальной \textit{ostis-системы}, которая названа нами \textbf{\textit{Метасистемой OSTIS}} и \textit{база знаний} которой содержит:

\begin{textitemize}
	\item Формальную теорию \textit{ostis-систем};
	\item Стандарт \textit{ostis-систем}
	\begin{textitemize}  
		\item Стандарт баз знаний \textit{ostis-систем}
		\begin{textitemize}  
			\item Стандарт внутреннего универсального языка смыслового представления знаний в памяти \textit{ostis-систем}
			\item Стандарт внутреннего представления онтологий верхнего уровня в памяти \textit{ostis-систем}
			\item Стандарт представления исходных текстов баз знаний \textit{ostis-систем}
		\end{textitemize}  
		
		\item Стандарт решателей задач \textit{ostis-систем}
		\begin{textitemize}  
			\item Стандарт базового языка программирования \textit{ostis-систем}
			\item Стандарт языков программирования высокого уровня для \textit{ostis-систем}
			\item Стандарт представления искусственных нейронных сетей в памяти \textit{ostis-систем}
			\item Стандарт внутренних информационных агентов в \textit{ostis-систем}
		\end{textitemize}  
		
		\item Стандарт интерфейсов \textit{ostis-систем}
		\begin{textitemize}  
			\item Стандарт внешних языков \textit{ostis-систем}, близких к внутреннему универсальному языку смыслового представления знаний
		\end{textitemize}  
	\end{textitemize}
	\item Стандарт ostis-систем и Технологии OSTIS (\textbf{\textit{Стандарт OSTIS}}) \cite{Standart2021};
	\item Ядро Библиотеки многократно используемых компонентов \textit{ostis-систем} (\textbf{\textit{Библиотеки OSTIS}});
	\item Методики и \textit{инструментальные средства поддержки жизненного цикла} \textit{ostis-систем} и их компонентов.
\end{textitemize}

\begin{SCn}
	
\scnheader{Технология OSTIS}
\scnidtf{Предлагаемая нами комплексная технология поддержки всех этапов жизненного цикла всех компонентов для всех классов (видов) интеллектуальных компьютерных систем нового поколения при перманентной поддержке их семантической совместимости}
\begin{scnrelfromlistcustom}{принципы, лежащие в основе}
	\scnitemcustom{Комплексный характер технологии, заключающийся в том, что осуществляется поддержка: 
	\begin{itemize}[labelsep=\tabsize-\bulletsize,leftmargin=\tabsize,label=$\bullet$]
	\item всех этапов жизненного цикла создаваемых продуктов
	\item для всех компонентов интеллектуальных компьютерных систем нового поколения
	\item для всех классов интеллектуальных компьютерных систем нового поколения
	\end{itemize}}

	\scnitemcustom{Обеспечивается перманентная поддержка семантической совместимости между всеми создаваемыми интеллектуальными компьютерными системами нового поколения}		
	\scnitemcustom{Ориентация на комплексную автоматизацию всего многообразия человеческой деятельности}		
	\scnitemcustom{Реализация технологии и, соответственно, комплексная автоматизация поддержки жизненного цикла интеллектуальных компьютерных систем нового поколения (со всеми их компонентами и классами) осуществляется в виде семейства интеллектуальных компьютерных систем нового поколения, построенных по той же технологии}
\end{scnrelfromlistcustom}
	
\scnheader{ostis-система}
\begin{scnrelfromset}{разбиение}
	\scnitem{ostis-субъект}
	\begin{scnindent}
		\scnidtf{самостоятельная \textit{ostis-система}}
		\begin{scnrelfromset}{разбиение}
			\scnitem{индивидуальная ostis-система}
			\scnitem{коллективная ostis-система}
		\end{scnrelfromset}
	\end{scnindent}
	\scnitem{встроенная ostis-система}
	\begin{scnindent}
		\scnidtf{\textit{ostis-система}, являющаяся частью некоторой \textit{индивидуальной ostis-системы}}
	\end{scnindent}
\end{scnrelfromset}

\scnheader{индивидуальная ostis-система}
\scnidtf{минимальная самостоятельная \textit{ostis-система}}
\begin{scnrelfromset}{разбиение}
	\scnitem{персональный ostis-ассистент}
	\begin{scnindent}
		\scnidtf{\textit{ostis-система}, осуществляющая комплексное адаптивное обслуживание конкретного пользователя по \textit{всем} вопросам, касающимся его взаимодействия с любыми другими \textit{ostis-системами}, а также представляющая интересы этого пользователя во всей глобальной сети \textit{ostis-систем}}
	\end{scnindent}
	\scnitem{корпоративная ostis-система}
	\begin{scnindent}
		\scnidtf{\textit{ostis-система}, осуществляющая координацию совместной деятельности \textit{ostis-систем} в рамках соответствующего коллектива \textit{ostis-систем}, осуществляющая мониторинг и реинжиниринг соответствующего множества \textit{ostis-систем} и представляющая интересы этого коллектива в рамках других коллективов \textit{ostis-систем}}
	\end{scnindent}
	\scnitem{индивидуальная ostis-система, не являющаяся ни персональным ostis-ассистентом, ни корпоративной ostis-системой}
\end{scnrelfromset}

\scnheader{коллективная ostis-система}
\scnidtf{многоагентная система, представляющая собой коллектив индивидуальных и коллективных \textit{ostis-систем}, деятельность которого координируется соответствующей корпоративной \textit{ostis-системой}}
\scntext{примечание}{В состав коллектива \textit{ostis-систем} могут входить индивидуальные \textit{ostis-системы} могут входить индивидуальные \textit{ostis-системы} любого вида -- в том числе, корпоративные \textit{ostis-системы}, представляющие интересы других коллективов \textit{ostis-систем}}

\end{SCn}

\section{Семантически совместимые ostis-системы}
\label{sec_sem_compatible_os}
%%%%%%%%%%%%%%%%%%%%%%%%% referenc.tex %%%%%%%%%%%%%%%%%%%%%%%%%%%%%%
% sample references
% %
% Use this file as a template for your own input.
%
%%%%%%%%%%%%%%%%%%%%%%%% Springer-Verlag %%%%%%%%%%%%%%%%%%%%%%%%%%
%
% BibTeX users please use
% \bibliographystyle{}
% \bibliography{}
%
\biblstarthook{In view of the parallel print and (chapter-wise) online publication of your book at \url{www.springerlink.com} it has been decided that -- as a genreral rule --  references should be sorted chapter-wise and placed at the end of the individual chapters. However, upon agreement with your contact at Springer you may list your references in a single seperate chapter at the end of your book. Deactivate the class option \texttt{sectrefs} and the \texttt{thebibliography} environment will be put out as a chapter of its own.\\\indent
References may be \textit{cited} in the text either by number (preferred) or by author/year.\footnote{Make sure that all references from the list are cited in the text. Those not cited should be moved to a separate \textit{Further Reading} section or chapter.} If the citatiion in the text is numbered, the reference list should be arranged in ascending order. If the citation in the text is author/year, the reference list should be \textit{sorted} alphabetically and if there are several works by the same author, the following order should be used:
\begin{enumerate}
\item all works by the author alone, ordered chronologically by year of publication
\item all works by the author with a coauthor, ordered alphabetically by coauthor
\item all works by the author with several coauthors, ordered chronologically by year of publication.
\end{enumerate}
The \textit{styling} of references\footnote{Always use the standard abbreviation of a journal's name according to the ISSN \textit{List of Title Word Abbreviations}, see \url{http://www.issn.org/en/node/344}} depends on the subject of your book:
\begin{itemize}
\item The \textit{two} recommended styles for references in books on \textit{mathematical, physical, statistical and computer sciences} are depicted in ~\cite{science-contrib, science-online, science-mono, science-journal, science-DOI} and ~\cite{phys-online, phys-mono, phys-journal, phys-DOI, phys-contrib}.
\item Examples of the most commonly used reference style in books on \textit{Psychology, Social Sciences} are~\cite{psysoc-mono, psysoc-online,psysoc-journal, psysoc-contrib, psysoc-DOI}.
\item Examples for references in books on \textit{Humanities, Linguistics, Philosophy} are~\cite{humlinphil-journal, humlinphil-contrib, humlinphil-mono, humlinphil-online, humlinphil-DOI}.
\item Examples of the basic Springer style used in publications on a wide range of subjects such as \textit{Computer Science, Economics, Engineering, Geosciences, Life Sciences, Medicine, Biomedicine} are ~\cite{basic-contrib, basic-online, basic-journal, basic-DOI, basic-mono}. 
\end{itemize}
}

\begin{thebibliography}{99.}%
% and use \bibitem to create references.
%
% Use the following syntax and markup for your references if 
% the subject of your book is from the field 
% "Mathematics, Physics, Statistics, Computer Science"
%
% Contribution 
\bibitem{science-contrib} Broy, M.: Software engineering --- from auxiliary to key technologies. In: Broy, M., Dener, E. (eds.) Software Pioneers, pp. 10-13. Springer, Heidelberg (2002)
%
% Online Document
\bibitem{science-online} Dod, J.: Effective substances. In: The Dictionary of Substances and Their Effects. Royal Society of Chemistry (1999) Available via DIALOG. \\
\url{http://www.rsc.org/dose/title of subordinate document. Cited 15 Jan 1999}
%
% Monograph
\bibitem{science-mono} Geddes, K.O., Czapor, S.R., Labahn, G.: Algorithms for Computer Algebra. Kluwer, Boston (1992) 
%
% Journal article
\bibitem{science-journal} Hamburger, C.: Quasimonotonicity, regularity and duality for nonlinear systems of partial differential equations. Ann. Mat. Pura. Appl. \textbf{169}, 321--354 (1995)
%
% Journal article by DOI
\bibitem{science-DOI} Slifka, M.K., Whitton, J.L.: Clinical implications of dysregulated cytokine production. J. Mol. Med. (2000) doi: 10.1007/s001090000086 
%
\bigskip

% Use the following (APS) syntax and markup for your references if 
% the subject of your book is from the field 
% "Mathematics, Physics, Statistics, Computer Science"
%
% Online Document
\bibitem{phys-online} J. Dod, in \textit{The Dictionary of Substances and Their Effects}, Royal Society of Chemistry. (Available via DIALOG, 1999), 
\url{http://www.rsc.org/dose/title of subordinate document. Cited 15 Jan 1999}
%
% Monograph
\bibitem{phys-mono} H. Ibach, H. L\"uth, \textit{Solid-State Physics}, 2nd edn. (Springer, New York, 1996), pp. 45-56 
%
% Journal article
\bibitem{phys-journal} S. Preuss, A. Demchuk Jr., M. Stuke, Appl. Phys. A \textbf{61}
%
% Journal article by DOI
\bibitem{phys-DOI} M.K. Slifka, J.L. Whitton, J. Mol. Med., doi: 10.1007/s001090000086
%
% Contribution 
\bibitem{phys-contrib} S.E. Smith, in \textit{Neuromuscular Junction}, ed. by E. Zaimis. Handbook of Experimental Pharmacology, vol 42 (Springer, Heidelberg, 1976), p. 593
%
\bigskip
%
% Use the following syntax and markup for your references if 
% the subject of your book is from the field 
% "Psychology, Social Sciences"
%
%
% Monograph
\bibitem{psysoc-mono} Calfee, R.~C., \& Valencia, R.~R. (1991). \textit{APA guide to preparing manuscripts for journal publication.} Washington, DC: American Psychological Association.
%
% Online Document
\bibitem{psysoc-online} Dod, J. (1999). Effective substances. In: The dictionary of substances and their effects. Royal Society of Chemistry. Available via DIALOG. \\
\url{http://www.rsc.org/dose/Effective substances.} Cited 15 Jan 1999.
%
% Journal article
\bibitem{psysoc-journal} Harris, M., Karper, E., Stacks, G., Hoffman, D., DeNiro, R., Cruz, P., et al. (2001). Writing labs and the Hollywood connection. \textit{J Film} Writing, 44(3), 213--245.
%
% Contribution 
\bibitem{psysoc-contrib} O'Neil, J.~M., \& Egan, J. (1992). Men's and women's gender role journeys: Metaphor for healing, transition, and transformation. In B.~R. Wainrig (Ed.), \textit{Gender issues across the life cycle} (pp. 107--123). New York: Springer.
%
% Journal article by DOI
\bibitem{psysoc-DOI}Kreger, M., Brindis, C.D., Manuel, D.M., Sassoubre, L. (2007). Lessons learned in systems change initiatives: benchmarks and indicators. \textit{American Journal of Community Psychology}, doi: 10.1007/s10464-007-9108-14.
%
%
% Use the following syntax and markup for your references if 
% the subject of your book is from the field 
% "Humanities, Linguistics, Philosophy"
%
\bigskip
%
% Journal article
\bibitem{humlinphil-journal} Alber John, Daniel C. O'Connell, and Sabine Kowal. 2002. Personal perspective in TV interviews. \textit{Pragmatics} 12:257--271
%
% Contribution 
\bibitem{humlinphil-contrib} Cameron, Deborah. 1997. Theoretical debates in feminist linguistics: Questions of sex and gender. In \textit{Gender and discourse}, ed. Ruth Wodak, 99--119. London: Sage Publications.
%
% Monograph
\bibitem{humlinphil-mono} Cameron, Deborah. 1985. \textit{Feminism and linguistic theory.} New York: St. Martin's Press.
%
% Online Document
\bibitem{humlinphil-online} Dod, Jake. 1999. Effective substances. In: The dictionary of substances and their effects. Royal Society of Chemistry. Available via DIALOG. \\
http://www.rsc.org/dose/title of subordinate document. Cited 15 Jan 1999
%
% Journal article by DOI
\bibitem{humlinphil-DOI} Suleiman, Camelia, Daniel C. O'Connell, and Sabine Kowal. 2002. `If you and I, if we, in this later day, lose that sacred fire...': Perspective in political interviews. \textit{Journal of Psycholinguistic Research}. doi: 10.1023/A:1015592129296.
%
%
%
\bigskip
%
%
% Use the following syntax and markup for your references if 
% the subject of your book is from the field 
% "Computer Science, Economics, Engineering, Geosciences, Life Sciences"
%
%
% Contribution 
\bibitem{basic-contrib} Brown B, Aaron M (2001) The politics of nature. In: Smith J (ed) The rise of modern genomics, 3rd edn. Wiley, New York 
%
% Online Document
\bibitem{basic-online} Dod J (1999) Effective Substances. In: The dictionary of substances and their effects. Royal Society of Chemistry. Available via DIALOG. \\
\url{http://www.rsc.org/dose/title of subordinate document. Cited 15 Jan 1999}
%
% Journal article by DOI
\bibitem{basic-DOI} Slifka MK, Whitton JL (2000) Clinical implications of dysregulated cytokine production. J Mol Med, doi: 10.1007/s001090000086
%
% Journal article
\bibitem{basic-journal} Smith J, Jones M Jr, Houghton L et al (1999) Future of health insurance. N Engl J Med 965:325--329
%
% Monograph
\bibitem{basic-mono} South J, Blass B (2001) The future of modern genomics. Blackwell, London 
%
\end{thebibliography}
