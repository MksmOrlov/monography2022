\chapauthor{Садовский М.Е.}
\chapter{Структура интерфейсов интеллектуальных компьютерных систем нового поколения}
\chapauthortoc{Садовский М.Е.}
\label{chapter_interfaces}

\abstract{Аннотация к главе.}

\section{Интерфейсы ostis-систем}

\textit{Интерфейс} -- совокупность технических, программных и методических (протоколов, правил, соглашений) средств, обеспечивающих обмен информацией между пользователем и устройствами и программами, а также между устройствами и другими устройствами и программами. В широком смысле слова, это способ (стандарт) взаимодействия между объектами. Интерфейс в техническом смысле слова задаёт параметры, процедуры и характеристики взаимодействия объектов.

\scnheader{интерфейс}
\begin{scnrelfromset}{разбиение}
	\scnitem{пользовательский интерфейc}
	\scnitem{программный интерфейс}	
	\scnitem{физический интерфейс}
\end{scnrelfromset}

\bigskip
\textit{Пользовательский интерфейс} -- один из наиболее важных компонентов компьютерной системы. Представляет собой совокупность аппаратных и программных средств, обеспечивающих обмен информацией между пользователем и компьютерной системой.

Основными типами пользовательского интерфейса являются командный пользовательский интерфейс, WIMP-интерфейс и SILK-интерфейс:

\scnheader{пользовательский интерфейс}
\scnsuperset{командный пользовательский интерфейс}
\scnsuperset{WIMP-интерфейс}
\begin{scnindent}
	\scnidtf{Window, Image, Menu, Pointer - интерфейс}
	\scnidtf{Окно, Образ, Меню, Указатель - интерфейс}
	\begin{scnindent}
		\scnsuperset{пользовательский интерфейс ostis-системы}
	\end{scnindent}	
\end{scnindent}
\scnsuperset{SILK-интерфейс}
\begin{scnindent}
	\scnidtf{Speech, Image, Language, Knowledge - интерфейс}
	\scnidtf{Речь, Образ, Язык, Знание - интерфейс}
	\scnsuperset{естественно-языковой интерфейc}
	\begin{scnindent}
		\scnsuperset{речевой интерфейc}
	\end{scnindent}
\end{scnindent}

\bigskip
\textit{Командный пользовательский интерфейс} -- пользовательский интерфейс, при котором обмен информацией между компьютерной системой и пользователем осуществляется путем написания текстовых инструкций или команд.
\textit{WIMP-интерфейс} -- пользовательский интерфейс, при котором обмен информацией между компьютерной системой и пользователем осуществляется в форме диалога при помощью окон, меню и других элементов управления.
\textit{SILK-интерфейс} -- пользовательский интерфейс, наиболее приближенный к естественной для человека форме общения. Компьютерная система находит для себя команды, анализируя человеческую речь и находя в ней ключевые фразы. Результат выполнения команд преобразуется в понятную человеку форму, например, в естественно-языковую форму или изображение.
\textit{Естественно-языковой интерфейс} -- SILK-интерфейс, обмен информацией между компьютерной системой и пользователем в котором происходит за счёт диалога. Диалог ведётся на одном из естественных языков.
\textit{Речевой интерфейс} -- SILK-интерфейс, обмен информацией в котором происходит за счёт диалога, в процессе которого компьютерная система и пользователь общаются с помощью речи. Данный вид интерфейса наиболее приближен к естественному общению между людьми.

\textit{Адаптивный интерфейс} -- пользовательский интерфейс, который изменяется на основе потребностей пользователя или контекста.

\textit{Интеллектуальный интерфейс} -- пользовательский интерфейс, который может предположить дальнейшие действия пользователей и представить информацию на основе этого предположения.

\textit{Мультимодальный интерфейс} -- пользовательский интерфейс, предназначенный для обработки двух или более комбинированных режимов пользовательского ввода, таких как речь, перо, касание, ручные жесты и взгляд, скоординированным образом с выводом мультимедийной системы.

\textit{Пользовательский интерфейс ostis-системы} представляет собой специализированную \textit{ostis-систему}, ориентированную на решение интерфейсных задач, и имеющую в своем составе базу знаний и решатель задач пользовательского интерфейса ostis-системы.
Для решения задачи построения пользовательского интерфейса в базе знаний \textit{пользовательского интерфейса ostis-системы} необходимо наличие sc-модели \textit{компонентов пользовательского интерфейса}, \textit{интерфейсных действий пользователей}, а также классификации \textit{пользовательских интерфейсов} вцелом. При проектировании интерфейса используется компонентный подход,который предполагает представление всего интерфейса приложения в виде отдельных специфицированных компонентов, которые могут разрабатываться и совершенствоваться независимо.

\textit{Компонент пользовательского интерфейса} -- знак фрагмента базы знаний, имеющий определённую форму внешнего представления на экране.

\scnheader{Компонент пользовательского интерфейса}
\begin{scnrelfromset}{разбиение}
	\scnitem{атомарный компонент пользовательского интерфейса}
	\scnitem{неатомарный компонент пользовательского интерфейса}	
\end{scnrelfromset}

\bigskip
\textit{Атомарный компонент пользовательского интерфейса} -- компонент пользовательского интерфейса, не содержащий в своём составе других компонентов пользовательского интерфейса.

\textit{Неатомарный компонент пользовательского интерфейса} -- компонент пользовательского интерфейса, состоящий из других компонентов пользовательского интерфейса.

\textit{Визуальная часть пользовательского интерфейса ostis-системы} -- часть базы знаний пользовательского интерфейса ostis-системы, содержащая необходимые для отображения пользовательского интерфейса компоненты.

\scnheader{визуальная часть пользовательского интерфейса ostis-системы}
\scnsubset{неатомарный компонент пользовательского интерфейса}
\bigskip
Компоненты пользовательского интерфейса могут быть отнесены к одной из трех категорий: \textit{компонент пользовательского интерфейса для отображения}, \textit{декоративный компонент пользовательского интерфейса}, \textit{интерактивный компонент пользовательского интерфейса}.

Полная классификация компонентов пользовательского интерфейса приведена далее:
\begin{itemize}
	\item интерактивный компонент пользовательского интерфейса
	\begin{itemize}
		\item компонент ввода данных
		\begin{itemize}
			\item компонент ввода данных с прямой ответной реакцией
			\begin{itemize}
				\item область рисования
				\item ползунок
				\item компонент ввода текста с прямой ответной реакцией (однострочное текстовое поле, многострочное текстовое поле)
				\item компонент выбора (компонент выбора одного значения, компонент выбора нескольких значений)
				\item компонент выбора данных (выбираемый элемент, радиокнопка, переключатель, флаговая кнопка)
			\end{itemize}
			\item компонент ввода данных без прямой ответной реакции
			\begin{itemize}
				\item кнопка-счётчик
				\item компонент ввода движений
				\item компонент речевого ввода
			\end{itemize}
		\end{itemize}
		\item компонент для представления и взаимодействия с пользователем
		\begin{itemize}
			\item активирующий компонент
			\item компонент непрерывной манипуляции
			\begin{itemize}
				\item компонент редактирования размера
				\item полоса прокрутки
			\end{itemize}
		\end{itemize}
		\item компонент запроса действий
		\begin{itemize}
			\item компонент выбора команд
			\begin{itemize}
				\item пункт меню
				\item кнопка
			\end{itemize}
			\item компонент ввода команд
		\end{itemize}
	\end{itemize}
	\item компонент пользовательского интерфейса для отображения
	\begin{itemize}
		\item компонент вывода
		\begin{itemize}
			\item компонент вывода видео
			\item компонент вывода звука
			\item компонент вывода изображения
			\item компонент вывода графической информации
			\begin{itemize}
				\item индикатор выполнения
				\item диаграмма
				\item карта
			\end{itemize}
			\item компонент вывода текста
			\begin{itemize}
				\item сообщение
				\item заголовок
				\item параграф
			\end{itemize}
		\end{itemize}
		\item декоративный компонент пользовательского интерфейса
		\begin{itemize}
			\item пустое пространство
			\item разделитель
		\end{itemize}
		\item контейнер
		\begin{itemize}
			\item списковый контейнер
			\item древовидный контейнер
			\item узловой контейнер
			\item таблично-строковый контейнер
			\item таблично-клеточный контейнер
			\item панель вкладок
			\item панель вращения
			\item меню
			\item строка меню
			\item панель инструментов
			\item строка состояния
			\item панель прокрутки
			\item окно
			\begin{itemize}
				\item модальное окно
				\item немодальное окно
			\end{itemize}
		\end{itemize}
	\end{itemize}
\end{itemize}


\section{Интерфейсные действия пользователей ostis-системы}


Действие, выполняемое пользователем над некоторым \textit{компонентом пользовательского интерфейса}, называется интерфейсным действием. Для связи данного действия с \textit{компонентом пользовательского интерфейса} и необходимым к выполнению \textit{внутренним действием системы} используется отношение \textit{инициируемое пользовательским интерфейсом действие*}.

Классификация интерфейсных действий:
\scnheader{интерфейсное действие пользователя}
\scnsuperset{действие мышью}
\begin{scnindent}
\scnsuperset{прокрутка мышью}
\begin{scnindent}
\scnsuperset{прокрутка мышью вверх}
\scnsuperset{прокрутка мышью вниз}
\end{scnindent}
\scnsuperset{наведение мышью}
\scnsuperset{отпускание мышью}
\scnsuperset{нажатие мыши}
\begin{scnindent}
\scnsuperset{одиночное нажатие мыши}
\scnsuperset{двойное нажатие мыши}			
\end{scnindent}
\scnsuperset{жест мышью}
\scnsuperset{отведение мышью}		
\scnsuperset{перетаскивание мышью}
\end{scnindent}
\scnsuperset{действие голосом}
\scnsuperset{действие клавиатурой}
\begin{scnindent}
\scnsuperset{нажатие функциональной клавиши}
\scnsuperset{нажатие клавиши набора текста}
\end{scnindent}
\scnsuperset{действие осязанием}	
\scnsuperset{действие сенсором}	
\begin{scnindent}
\scnsuperset{нажатие сенсора}
\begin{scnindent}
\scnsuperset{одиночное нажатие сенсора}
\scnsuperset{двойное нажатие сенсора}
\end{scnindent}
\scnsuperset{жест по сенсору}
\begin{scnindent}
\scnsuperset{жест по сенсору одним пальцем}
\scnsuperset{жест по сенсору несколькими пальцами}
\end{scnindent}
\scnsuperset{отпускание сенсором}
\scnsuperset{перетаскивание сенсором}
\end{scnindent}
\scnsuperset{действие пером}	
\begin{scnindent}
\scnsuperset{нажатие функциональной клавиши пером}
\scnsuperset{рисование пером}
\scnsuperset{написание текста пером}
\end{scnindent}

\bigskip
\textit{Прокрутка мышью} -- интерфейсное действие пользователя, соответствующее прокрутке содержимого некоторого компонента пользовательского интерфейса при помощи мыши.

\textit{Наведение мышью} -- интерфейсное действие пользователя, соответствующее появлению курсора мыши в рамках компонента пользовательского интерфейса.

\textit{Отпускание мышью} -- интерфейсное действие пользователя, соответствующее отпусканию некоторого компонента пользовательского интерфейса в рамках другого компонента пользовательского интерфейса при помощи мыши.

\textit{Нажатие мыши} -- интерфейсное действие пользователя, соответствующее выполнению нажатия мыши в рамках некоторого компонента пользовательского интерфейса.

\textit{Отведение мышью} -- интерфейсное действие пользователя, соответствующее выходу курсора мыши за рамки компонента пользовательского интерфейса.

\textit{перетаскивание мышью} -- интерфейсное действие пользователя, соответствующее перетаскиванию компонента пользовательского интерфейса при помощи мыши.

\textit{Нажатие сенсора} -- интерфейсное действие пользователя, соответствующее выполнению нажатия сенсора в рамках некоторого компонента пользовательского интерфейса.

\textit{Жест по сенсору} -- интерфейсное действие пользователя, соответствующее выполнению некоторого жеста, выполняемого при помощи движения пальцев на экране сенсора.

\textit{отпускание сенсором} -- интерфейсное действие пользователя, соответствующее отпусканию некоторого компонента пользовательского интерфейса в рамках другого компонента пользовательского интерфейса при помощи сенсора.

\textit{перетаскивание сенсором} -- интерфейсное действие пользователя, соответствующее перетаскиванию компонента пользовательского интерфейса при помощи сенсора.

\textit{действие пером} -- интерфейсное действие пользователя, осуществляемое при помощи пера на графическом планшете.

\textit{Класс интерфейсных действий пользователя} -- множество, элементами которого являются классы \textit{интерфейсных действий пользователя}.

При взаимодействии пользователя с \textit{компонентом пользовательского интерфейса} могут быть произведены различные интерфейсные действия. В зависимости от выполненного интерфейсного действия и компонента, над которым оно было выполнено, происходит инициирование некоторого \textit{внутреннего действия системы}. Для задания такого инициируемого при взаимодействии с пользовательским интерфейсом действия и используется указанное отношение. Первым компонентом связки отношения \textit{инициируемое пользовательским интерфейсом действие*} является связка, элементами которой являются элемент множества компонентов пользовательского интерфейса и и элемент множества \textit{класс интерфейсных действий пользователя}. Вторым компонентом является элемент множества \textit{класс внутренних действий системы}.

\section{Сообщения, входящие в ostis-систему и выходящие из неё}

Сообщение -- дискретная информационная конструкция, используемая в процессе передачи от отправителя к получателю.

В качестве отправителя сообщения может выступать как пользователь системы, так и сама система. В случае ostis-системы сообщение может быть эффекторным либо рецепторным.

Эффекторное сообщение ostis-системы -- сообщение ostis-системы, формируемое самой ostis-системой при возникновении некоторых ситуаций. К ситуациям, инициирующим возникновение эффекторных сообщений, можно отнести:
\begin{itemize}
	\item ситуации, возникающие при анализе деятельности самого пользователя. Например, задание аргументов, не соответствующих типу инициируемого действия или появление подсказок при использовании компонентов пользовательского интерфейса;
	\item ситуации, возникающие при анализе синтаксиса текстов внешних языков. Например, неполнота сформированного предложения на внешнем языке или использование конструкций, нехарактерных или некорректно использованных в контексте отдельно взятого внешнего языка/
\end{itemize}

Рецепторное сообщение ostis-системы -- сообщение ostis-системы, являющееся реакцией на императивное сообщение (сообщение, побуждающее к какому-либо действию). Возможными реакциями ostis-системы на императивное сообщение пользователя являются:
\begin{itemize}
	\item указание факта завершения выполнения некоторой задачи, что, например, характерно для поведенческих действий;
	\item получение ответа на поставленную задачу, формируемого либо в результате анализа базы знаний	пользовательского интерфейса, либо в результате анализа предметной части базы знаний самой ostis-системы.
\end{itemize}

Сообщение пользователя ostis-системы может быть сформировано как на внешнем языке (языке, внешнем по отношению к ostis-системе, который не используется для коммуникации внутри системы), так и на внутреннем языке (SC-коде).

Любое сообщение может быть атомарным (не содержащем в своем составе другие сообщения) либо неатомарным (сообщение, в состав которого входят другие сообщения).

Типология сообщений представлена в следующем фрагменте:
\scnheader{сообщение}
\begin{scnrelfromset}{разбиение}
\scnitem{сообщение пользователя системы}
\begin{scnindent}
	\scnsubset{сообщение пользователя ostis-системы}
\end{scnindent}
\scnitem{сообщение системы}
\end{scnrelfromset}
\begin{scnrelfromset}{разбиение}
\scnitem{атомарное сообщение}
\scnitem{атомарное сообщение}
\end{scnrelfromset}
\begin{scnrelfromset}{разбиение}
\scnitem{сообщение на естественном языке}
\scnitem{сообщение на искусственном языке}
\end{scnrelfromset}
\scnsubset{графическое сообщение}
\begin{scnindent}
	\scnidtf{сообщение, содержащее графическую информацию}
	\scnsubset{видео-сообщение}
	\begin{scnindent}
		\scnidtf{сообщение, содержащее видео-информацию}
	\end{scnindent}	
\end{scnindent}
\scnsubset{аудио-сообщение}
\begin{scnindent}
	\scnidtf{сообщение, представленное в звуковом формате}
\end{scnindent}
\scnsubset{обонятельное сообщение}
\begin{scnindent}
	\scnidtf{сообщение, содержащее информацию о запахах}
\end{scnindent}
\scnsubset{текстовое сообщение}
\begin{scnindent}
	\scnidtf{сообщение, содержащее текстовую информацию}
\end{scnindent}
\scnsubset{сообщение, требующее трансляции}
\begin{scnindent}
	\scnidtf{сообщение, которое необходимо сформировать системой для дальнейшей передачи его пользователю}
\end{scnindent}
\scnsubset{протранслированное сообщение}
\begin{scnindent}
	\scnidtf{сообщение, которое было сформировано системой для дальнейшей передачи его пользователю}
\end{scnindent}

\section{Действия и внутренние агенты пользовательского интерфейса ostis-системы}

Интерфейсное действие пользователя, как правило, инициирует некоторое внутреннее действие системы. 

\scnheader{внутреннее действие системы}
\scnsuperset{внутреннее действие ostis-системы}

\scnheader{внутреннее действие ostis-системы}
\scnidtf{действие в sc-памяти}
\scnidtf{действие, выполняемое в sc-памяти}
	
Каждое \textit{внутреннее действие ostis-системы} обозначает некоторое преобразование, выполняемое некоторым \textit{sc-агентом} (или коллективом \textit{sc-агентов}) и ориентированное на преобразование \textit{sc-памяти}.

\scnheader{действие в sc-памяти}
\scnsuperset{действие в sc-памяти, инициируемое вопросом}
\scnsuperset{действие редактирования базы знаний ostis-системы}
\scnsuperset{действие установки режима ostis-системы}
\scnsuperset{действие редактирования файла, хранимого в sc-памяти}
\scnsuperset{действие интерпретации программы, хранимой в sc-памяти}

\bigskip
Среди агентов интепретации модели пользовательского интерфейса ostis-систем можно выделить агент генерации интерфейса на основе его модели и агент обработки пользовательских действий.

В качестве входного параметра агент генерации интерфейса на основе его модели принимает экземпляр компонента пользовательского интерфейса для отображения. Результатом работы является графическое представление указанного компонента с учетом используемой реализации платформы интерпретации семантических моделей ostis-систем.

Агент обработки пользовательских действий реагирует на появление в базе знаний системы экземпляра интерфейсного действия пользователя, находит связанный с ним класс внутреннего действия и генерирует экземпляр данного внутреннего действия для последующей обработки.


%%%%%%%%%%%%%%%%%%%%%%%%% referenc.tex %%%%%%%%%%%%%%%%%%%%%%%%%%%%%%
% sample references
% %
% Use this file as a template for your own input.
%
%%%%%%%%%%%%%%%%%%%%%%%% Springer-Verlag %%%%%%%%%%%%%%%%%%%%%%%%%%
%
% BibTeX users please use
% \bibliographystyle{}
% \bibliography{}
%
\biblstarthook{In view of the parallel print and (chapter-wise) online publication of your book at \url{www.springerlink.com} it has been decided that -- as a genreral rule --  references should be sorted chapter-wise and placed at the end of the individual chapters. However, upon agreement with your contact at Springer you may list your references in a single seperate chapter at the end of your book. Deactivate the class option \texttt{sectrefs} and the \texttt{thebibliography} environment will be put out as a chapter of its own.\\\indent
References may be \textit{cited} in the text either by number (preferred) or by author/year.\footnote{Make sure that all references from the list are cited in the text. Those not cited should be moved to a separate \textit{Further Reading} section or chapter.} If the citatiion in the text is numbered, the reference list should be arranged in ascending order. If the citation in the text is author/year, the reference list should be \textit{sorted} alphabetically and if there are several works by the same author, the following order should be used:
\begin{enumerate}
\item all works by the author alone, ordered chronologically by year of publication
\item all works by the author with a coauthor, ordered alphabetically by coauthor
\item all works by the author with several coauthors, ordered chronologically by year of publication.
\end{enumerate}
The \textit{styling} of references\footnote{Always use the standard abbreviation of a journal's name according to the ISSN \textit{List of Title Word Abbreviations}, see \url{http://www.issn.org/en/node/344}} depends on the subject of your book:
\begin{itemize}
\item The \textit{two} recommended styles for references in books on \textit{mathematical, physical, statistical and computer sciences} are depicted in ~\cite{science-contrib, science-online, science-mono, science-journal, science-DOI} and ~\cite{phys-online, phys-mono, phys-journal, phys-DOI, phys-contrib}.
\item Examples of the most commonly used reference style in books on \textit{Psychology, Social Sciences} are~\cite{psysoc-mono, psysoc-online,psysoc-journal, psysoc-contrib, psysoc-DOI}.
\item Examples for references in books on \textit{Humanities, Linguistics, Philosophy} are~\cite{humlinphil-journal, humlinphil-contrib, humlinphil-mono, humlinphil-online, humlinphil-DOI}.
\item Examples of the basic Springer style used in publications on a wide range of subjects such as \textit{Computer Science, Economics, Engineering, Geosciences, Life Sciences, Medicine, Biomedicine} are ~\cite{basic-contrib, basic-online, basic-journal, basic-DOI, basic-mono}. 
\end{itemize}
}

\begin{thebibliography}{99.}%
% and use \bibitem to create references.
%
% Use the following syntax and markup for your references if 
% the subject of your book is from the field 
% "Mathematics, Physics, Statistics, Computer Science"
%
% Contribution 
\bibitem{science-contrib} Broy, M.: Software engineering --- from auxiliary to key technologies. In: Broy, M., Dener, E. (eds.) Software Pioneers, pp. 10-13. Springer, Heidelberg (2002)
%
% Online Document
\bibitem{science-online} Dod, J.: Effective substances. In: The Dictionary of Substances and Their Effects. Royal Society of Chemistry (1999) Available via DIALOG. \\
\url{http://www.rsc.org/dose/title of subordinate document. Cited 15 Jan 1999}
%
% Monograph
\bibitem{science-mono} Geddes, K.O., Czapor, S.R., Labahn, G.: Algorithms for Computer Algebra. Kluwer, Boston (1992) 
%
% Journal article
\bibitem{science-journal} Hamburger, C.: Quasimonotonicity, regularity and duality for nonlinear systems of partial differential equations. Ann. Mat. Pura. Appl. \textbf{169}, 321--354 (1995)
%
% Journal article by DOI
\bibitem{science-DOI} Slifka, M.K., Whitton, J.L.: Clinical implications of dysregulated cytokine production. J. Mol. Med. (2000) doi: 10.1007/s001090000086 
%
\bigskip

% Use the following (APS) syntax and markup for your references if 
% the subject of your book is from the field 
% "Mathematics, Physics, Statistics, Computer Science"
%
% Online Document
\bibitem{phys-online} J. Dod, in \textit{The Dictionary of Substances and Their Effects}, Royal Society of Chemistry. (Available via DIALOG, 1999), 
\url{http://www.rsc.org/dose/title of subordinate document. Cited 15 Jan 1999}
%
% Monograph
\bibitem{phys-mono} H. Ibach, H. L\"uth, \textit{Solid-State Physics}, 2nd edn. (Springer, New York, 1996), pp. 45-56 
%
% Journal article
\bibitem{phys-journal} S. Preuss, A. Demchuk Jr., M. Stuke, Appl. Phys. A \textbf{61}
%
% Journal article by DOI
\bibitem{phys-DOI} M.K. Slifka, J.L. Whitton, J. Mol. Med., doi: 10.1007/s001090000086
%
% Contribution 
\bibitem{phys-contrib} S.E. Smith, in \textit{Neuromuscular Junction}, ed. by E. Zaimis. Handbook of Experimental Pharmacology, vol 42 (Springer, Heidelberg, 1976), p. 593
%
\bigskip
%
% Use the following syntax and markup for your references if 
% the subject of your book is from the field 
% "Psychology, Social Sciences"
%
%
% Monograph
\bibitem{psysoc-mono} Calfee, R.~C., \& Valencia, R.~R. (1991). \textit{APA guide to preparing manuscripts for journal publication.} Washington, DC: American Psychological Association.
%
% Online Document
\bibitem{psysoc-online} Dod, J. (1999). Effective substances. In: The dictionary of substances and their effects. Royal Society of Chemistry. Available via DIALOG. \\
\url{http://www.rsc.org/dose/Effective substances.} Cited 15 Jan 1999.
%
% Journal article
\bibitem{psysoc-journal} Harris, M., Karper, E., Stacks, G., Hoffman, D., DeNiro, R., Cruz, P., et al. (2001). Writing labs and the Hollywood connection. \textit{J Film} Writing, 44(3), 213--245.
%
% Contribution 
\bibitem{psysoc-contrib} O'Neil, J.~M., \& Egan, J. (1992). Men's and women's gender role journeys: Metaphor for healing, transition, and transformation. In B.~R. Wainrig (Ed.), \textit{Gender issues across the life cycle} (pp. 107--123). New York: Springer.
%
% Journal article by DOI
\bibitem{psysoc-DOI}Kreger, M., Brindis, C.D., Manuel, D.M., Sassoubre, L. (2007). Lessons learned in systems change initiatives: benchmarks and indicators. \textit{American Journal of Community Psychology}, doi: 10.1007/s10464-007-9108-14.
%
%
% Use the following syntax and markup for your references if 
% the subject of your book is from the field 
% "Humanities, Linguistics, Philosophy"
%
\bigskip
%
% Journal article
\bibitem{humlinphil-journal} Alber John, Daniel C. O'Connell, and Sabine Kowal. 2002. Personal perspective in TV interviews. \textit{Pragmatics} 12:257--271
%
% Contribution 
\bibitem{humlinphil-contrib} Cameron, Deborah. 1997. Theoretical debates in feminist linguistics: Questions of sex and gender. In \textit{Gender and discourse}, ed. Ruth Wodak, 99--119. London: Sage Publications.
%
% Monograph
\bibitem{humlinphil-mono} Cameron, Deborah. 1985. \textit{Feminism and linguistic theory.} New York: St. Martin's Press.
%
% Online Document
\bibitem{humlinphil-online} Dod, Jake. 1999. Effective substances. In: The dictionary of substances and their effects. Royal Society of Chemistry. Available via DIALOG. \\
http://www.rsc.org/dose/title of subordinate document. Cited 15 Jan 1999
%
% Journal article by DOI
\bibitem{humlinphil-DOI} Suleiman, Camelia, Daniel C. O'Connell, and Sabine Kowal. 2002. `If you and I, if we, in this later day, lose that sacred fire...': Perspective in political interviews. \textit{Journal of Psycholinguistic Research}. doi: 10.1023/A:1015592129296.
%
%
%
\bigskip
%
%
% Use the following syntax and markup for your references if 
% the subject of your book is from the field 
% "Computer Science, Economics, Engineering, Geosciences, Life Sciences"
%
%
% Contribution 
\bibitem{basic-contrib} Brown B, Aaron M (2001) The politics of nature. In: Smith J (ed) The rise of modern genomics, 3rd edn. Wiley, New York 
%
% Online Document
\bibitem{basic-online} Dod J (1999) Effective Substances. In: The dictionary of substances and their effects. Royal Society of Chemistry. Available via DIALOG. \\
\url{http://www.rsc.org/dose/title of subordinate document. Cited 15 Jan 1999}
%
% Journal article by DOI
\bibitem{basic-DOI} Slifka MK, Whitton JL (2000) Clinical implications of dysregulated cytokine production. J Mol Med, doi: 10.1007/s001090000086
%
% Journal article
\bibitem{basic-journal} Smith J, Jones M Jr, Houghton L et al (1999) Future of health insurance. N Engl J Med 965:325--329
%
% Monograph
\bibitem{basic-mono} South J, Blass B (2001) The future of modern genomics. Blackwell, London 
%
\end{thebibliography}
