\chapauthor{Самодумкин С.А.\\Шункевич Д.В.}
\chapter{Язык вопросов для ostis-систем}
\chapauthortoc{Самодумкин С.~А.\\Шункевич Д.~В.\\ Ивашенко В.~П.}
\label{chapter_requests}

\abstract{В главе уточнена формальная трактовка понятия вопроса, что позволило задать язык вопросов.}

В процессе эксплуатации интеллектуальной системы одним из ключевых моментов является возможность формулировать информационную потребность пользователями. Одним из способов выражения такой потребности является вопрос. В процессе диалогового общения всегда существует контекст, который определяет дополнительную информацию, способствующую правильному пониманию смысла сообщения. Особенность представления информации в базах знаний ostis-систем упрощает формирование информационной потребности пользователя, так как представленная инфрормация в базах знаний уже структурирована и известны отношения, заданные на определенном понятии, в отношении которого собственно и разрешается вопросно-проблемная ситуация. В работе Аверьянова показно, что вопросно-проблемная ситуация не может быть решена в рамках формальной логики и природа вопроса может быть понятна в системе субъектно-объектных отношений. В связи с тем, что при формировании баз знаний ostis-систем происходит формирование субъектно-объектных отношений в рамках заданной предметной области, тем самым упрощается выражение информационной потребности пользователем средствами SC-кода.   

Целью разработки языка вопросов и последующее его развитие является реализация возможности понимания действий, осуществляемых ostis-системой, при формировании ответа на поставленный вопрос. В процессе формирования вывода на поставленный вопрос возможны следующие варианты:
1) ответ на поставленный вопрос существует в базе знаний и происходит локализация фрагмента базы знаний в контексте выраженной срествами SC-кода информационной потребности пользователя;
2) ответ связан с разрешением некоторой задачной ситуации, которая содержится в контексте вопроса и формирование ответа возлагается на решатель задач.

\section{Синтаксис языка вопросов для ostis-систем}

Язык вопросов относится к семейству совместимых семантических языков – sc-языков и предназначен для формального описания поискового предписания ostis-систем с целью удовлетворения информационной потребности пользователей.

\section{Денотационная семантика языка вопросов для ostis-систем}

Объектами анализа языка вопросов являются типы вопросов и соответствующие классы ответов в соответствии с семантической классификацией (типологией) вопросов.
С целью семантической классификации вопросов будем исходить из того, что любая предметная область (ПО) содержательно состоит из конкретных понятий и отношений. В работе Сулейманов предложено множество конкретных понятий и отношений по определенным признакам разбить на конечное число типов понятий и типов отношений, которые названы семантическими единицами.
В работе Сулейманов выделено множество различных типов понятий, необходимых для формирования вопроса:
SS - \textit{множество главных понятий} - это понятия, относительно которых задан вопрос (в частном случае, может быть только одно главное понятие);
SO - понятие, состоящие в некотором определенном отношении с главным понятием;
Sоп - \textit{обобщенное понятие} (ОП), то есть такое понятие, находящееся по отношению к главному на более высоком уровне иерархии понятий предметной области
SA - \textit{понятие-аргумент};
SP - \textit{понятие-результат}.

Введём типы отношений, необходимых для формирования вопросов.

\begin{SCn}
	\scnheader{конкретное отношение}
	\scnidtf{определённое отношение между понятиями \textit{предметной области} в контексте вопроса}
\end{SCn}

\begin{SCn}
	\scnheader{типовое отношение}
	\scnidtf{\textit{обобщённое отношение}, объединяющее \textit{конкретное отношение} в семейства отношений, отражающих однотипный смысл и раскрывающих определённый признак понятий \textit{предметной области}}
	\begin{scnrelfromset}{декомпозиция}
		\scnitem{типовое отношение СОСТОЯНИЕ}
		\scnitem{типовое отношение ДЕЙСТВИЕ}
		\scnitem{типовое отношение СОСТАВ}
		\scnitem{типовое отношение ВКЛЮЧЕНИЕ}
		\scnitem{типовое отношение ВРЕМЕННОЕ ОТНОШЕНИЕ}
		\scnitem{типовое отношение ПРОСТРАНСТВЕННОЕ ОТНОШЕНИЕ}
		\scnitem{типовое отношение КОЛИЧЕСТВЕННОЕ ОТНОШЕНИЕ}
		\scnitem{типовое отношение КАЧЕСТВЕННОЕ ОТНОШЕНИЕ}
	\end{scnrelfromset}
\end{SCn}

Например, \textit{конкретные отношения} такие, как "играет", "спит", "плавает", объдиняются в семейство \textit{типовое отношение СОСТОЯНИЕ} по признаку выражать состояние понятия (раскрывает признак понятия предметной области - находиться в некотором состоянии).

\begin{SCn}
	\scnheader{составное отношение}
	\scnidtf{устойчивая комбинация двух типовых отшений ДЕЙСТВИЕ: действия, направленного на \textit{понятие-аргумент}, и действия, направленного на \textit{понятие-результат}}
	\scnsuperset{составное отношение ФУНКЦИЯ заданного понятия}
\end{SCn}

Например, составное отношение ФУНКЦИЯ понятия S1: "S1 переводит S2 в S3".

Смысловая типизация вопросов дает возможность противопоставить каждому типу вопроса ограниченный набор допустимых, то есть логически корректных конструкций, передающий правильный смысл вопроса в зависимости от типа вопроса. При этом семантическая типизация вопросов позваляет разбить множество вопросов на семантичекие классы, в каждом из которых требуется раскрытие некоторого однотипного смысла, определенного типом вопроса. 

\subsection{Семантическая классификация вопросов и ответов}
\label{chapter_questions_sec_sem_classification}

\begin{SCn}
	\scnheader{вопрос к ostis-системе}
	\begin{scnrelfromset}{декомпозиция}
		\scnitem{вопрос, требующий явного задания в ответе \textit{ключевого параметра}}
		\begin{scnindent}
			\begin{scnhaselementrolelist}{пример}
				\scnitem{"Назовите состав компилятора?"{}}
			\end{scnhaselementrolelist}
		\end{scnindent}
		\scnitem{вопрос, требующий раскрытия в ответе \textit{типового отношения} одного \textit{главного понятия}}
		\begin{scnindent}
			\begin{scnhaselementrolelist}{пример}
				\scnitem{"Что легче: железо или дерево?"{}}
			\end{scnhaselementrolelist}
		\end{scnindent}
		\scnitem{вопрос, требующий раскрытия в ответе \textit{специального отношения} одного \textit{главного понятия}}
		\begin{scnindent}
			\scntext{пояснение}{Такому классу вопросов соответствуют классы ответов, в которых\textit{ главное понятие} раскрывается через \textit{специальное отношение}.}
			\begin{scnhaselementrolelist}{пример}
				\scnitem{"Какую функцию выполняет компилятор?"{}}
			\end{scnhaselementrolelist}
		\end{scnindent}
		\scnitem{вопрос, требующий раскрытия в ответе произвольной комбинации \textit{типового отношения} и/или \textit{составного отношения} одного \textit{главного понятия}}
		\begin{scnindent}
			\scntext{пояснение}{Такому классу вопросов соответствуют классы ответов, в которых\textit{ главное понятие} раскрывается через \textit{специальное отношение}.}
			\begin{scnhaselementrolelist}{пример}
				\scnitem{"Какую функцию выполняет компилятор?"{}}
			\end{scnhaselementrolelist}
		\end{scnindent}
		\scnitem{вопрос, требующий раскрытия в ответе более чем одного \textit{главного понятия}}
		\begin{scnindent}
			\begin{scnhaselementrolelist}{пример}
				\scnitem{"Докажите теорему"{}}
			\end{scnhaselementrolelist}
		\end{scnindent}
	\end{scnrelfromset}
\end{SCn}

\begin{SCn}
	\scnheader{ответ на вопрос, требующий раскрытия в ответе \textit{типового отношения} одного \textit{главного понятия}}
	\begin{scnrelfromset}{декомпозиция}
		\scnitem{СОСТАВ}
		\scnidtf{класс ответов, в которых понятие S раскрывается через его типовое отношение СОСТАВ с составляющими понятиями P и Q}
		\begin{scnindent}
			\begin{scnhaselementrolelist}{пример}
				\scnitem{НАПИСАТЬ ПРИМЕР}
			\end{scnhaselementrolelist}
		\end{scnindent}
		\scnitem{ВКЛЮЧЕНИЕ}
		\scnidtf{класс ответов, в которых понятие S раскрывается через его типовое отношение ВКЛЮЧЕНИЕ к другому понятию P, содержащего S как часть}
		\begin{scnindent}
			\begin{scnhaselementrolelist}{пример}
				\scnitem{НАПИСАТЬ ПРИМЕР}
			\end{scnhaselementrolelist}
		\end{scnindent}
		\scnitem{СОСТОЯНИЕ}
		\scnidtf{класс ответов, в которых понятие S раскрывается через его типовое отношение СОСТОЯНИЕ}
		\begin{scnindent}
			\begin{scnhaselementrolelist}{пример}
				\scnitem{"S играет"{}}
			\end{scnhaselementrolelist}
		\end{scnindent}
		\scnitem{ДЕЙСТВИЕ}
		\scnidtf{класс ответов, в которых понятие S раскрывается через его типовое отношение ДЕЙСТВИЕ к другому понятию P}
		\begin{scnindent}
			\begin{scnhaselementrolelist}{пример}
				\scnitem{"S перемещает P"{}}
			\end{scnhaselementrolelist}
		\end{scnindent}
		\scnitem{ВРЕМЕННОЕ ОТНОШЕНИЕ}
		\scnidtf{класс ответов, в которых понятие S раскрывается через его типовое отношение ВРЕМЕННОЕ ОТНОШЕНИЕ к другому понятию P по некоторой временной шкале}
		\begin{scnindent}
			\begin{scnhaselementrolelist}{пример}
				\scnitem{"S выполняется раньше P"{}}
			\end{scnhaselementrolelist}
		\end{scnindent}
		\scnitem{ПРОСТРАНСТВЕННОЕ ОТНОШЕНИЕ}
		\scnidtf{класс ответов, в которых понятие S раскрывается через его типовое отношение ПРОСТРАНСТВЕННОЕ ОТНОШЕНИЕ, отражающее его положение в пространстве относительно другого понятия P}
		\begin{scnindent}
			\begin{scnhaselementrolelist}{пример}
				\scnitem{"S выполняется раньше P"{}}
			\end{scnhaselementrolelist}
		\end{scnindent}
		\scnitem{КОЛИЧЕСТВЕННОЕ ОТНОШЕНИЕ}
		\scnidtf{класс ответов, в которых раскрывается типовое отношение КОЛИЧЕСТВЕННОЕ ОТНОШЕНИЕ понятия S к другому понятию P}
		\begin{scnindent}
			\begin{scnhaselementrolelist}{пример}
				\scnitem{"S больше чем P"{}}
			\end{scnhaselementrolelist}
		\end{scnindent}
		\scnitem{КАЧЕСТВЕННОЕ ОТНОШЕНИЕ}
		\scnidtf{класс ответов, в которых раскрывается типовое отношение КАЧЕСТВЕННОЕ ОТНОШЕНИЕ понятия S к другому понятию P}
		\begin{scnindent}
			\begin{scnhaselementrolelist}{пример}
				\scnitem{"S легче чем P"{}}
			\end{scnhaselementrolelist}
		\end{scnindent}
	\end{scnrelfromset}
\end{SCn}

\begin{SCn}
	\scnheader{ответ на вопрос, требующий раскрытия в ответе произвольной комбинации \textit{типового отношения} и/или \textit{составного отношения} одного \textit{главного понятия}}
	\begin{scnrelfromset}{декомпозиция}
		\scnitem{ОПИСАНИЕ}
		\scnidtf{класс ответов, в которых раскрываются произвольные комбинации \textit{типового отношения} и/или \textit{составного отношения} одного \textit{главного понятия} S с другими понятиями P}
		\begin{scnindent}
			\begin{scnhaselementrolelist}{пример}
				\scnitem{"S состоит из P, Q, W. S переводит X и Y и выполняется раньше Z"{}}
			\end{scnhaselementrolelist}
		\end{scnindent}
		\scnitem{ОПРЕДЕЛЕНИЕ}
		\scnidtf{класс ответов, в которых понятие S раскрывается через \textit{обобщающее понятие} и \textit{класс ОПИСАНИЕ}}
		\begin{scnindent}
			\begin{scnhaselementrolelist}{пример}
				\scnitem{"Студент - это человек, который обучается в ВУЗе"{}}
				\scnitem{"Минск - это столица, которая находится в РБ"{}}
			\end{scnhaselementrolelist}
		\end{scnindent}
		\scnitem{ПРИЧИНА}
		\scnidtf{класс ответов, в которых раскрывается условие существования некоторых отношений S с другими понятими P}
		\begin{scnindent}
			\begin{scnhaselementrolelist}{пример}
				\scnitem{"Почему дерево не тонет в воде?"{}}
			\end{scnhaselementrolelist}
		\end{scnindent}
		\scnitem{СЛЕДСТВИЕ}
		\scnidtf{класс ответов, в которых раскрывается следствие от существования некоторых отношений S с другими понятими P}
		\begin{scnindent}
			\begin{scnhaselementrolelist}{пример}
				\scnitem{"Что следует из того, что удельный вес дерева меньше удельного веса воды?"{}}
			\end{scnhaselementrolelist}
		\end{scnindent}
	\end{scnrelfromset}
\end{SCn}

\begin{SCn}
	\scnheader{ответ на вопрос, требующий раскрытия в ответе более чем одного \textit{главного понятия}}
	\scnsuperset{ДЕТАЛИЗАЦИЯ}
	\begin{scnindent}
		\scnidtf{класс ответов, в которых происходит детализация понятий, стоящих в некоторых отношениях с \textit{главным понятием} P}
		\begin{scnhaselementrolelist}{пример}
			\scnitem{"Каким связь существует между институтом и заводом?"{}}
		\end{scnhaselementrolelist}
	\end{scnindent}
\end{SCn}


\section{Операционная семантика языка вопросов для ostis-систем}

Рассмотренные в \textit{\ref{chapter_questions_sec_sem_classification}~\nameref{chapter_questions_sec_sem_classification}} классы вопросов и ответов обуславливают необходимость определения операционной семантики Языка вопросов. Каждому классу вопросов должен соответствовать определённый коллектив sc-агентов, реализующих вывод из базы знаний ostis-системы соответствующих ответов. Следует отметить, что в зависимости от степени наполненности базы знаний ответы могут содержаться в базе знаний либо в текущей версии базы знаний отсутствовать. В случае наличия ответа в базе знаний информационная потребность пользователя реализуется информационно-поисковыми sc-агентами, в противном случае -- в зависимости от классов вопросов реализация вывода ответов осуществляется специализированные sc-агентами, которые в процессе работы дополнительно выполняют вычислительные задачи либо осуществляют синтез на основе логического вывода.

\begin{SCn}
	\scnheader{интерпретатор Языка вопросов}
	\begin{scnrelfromset}{декомпозиция}
		\scnitem{абстрактный sc-агент, решающий задачу поиска ответа на заданный вопрос}
		\begin{scnindent}
			\begin{scnrelfromset}{декомпозиция}
				\scnitem{абстрактный sc-агент поиска семантической окрестности заданного понятия}
				\scnitem{абстрактный sc-агент поиска ответа СОСТАВА для заданного понятия}
				\scnitem{абстрактный sc-агент поиска ответа ВКЛЮЧЕНИЕ заданного понятия как части другого понятия}
				\scnitem{абстрактный sc-агент поиска ответа СОСТОЯНИЕ заданного понятия}
				\scnitem{абстрактный sc-агент поиска ответа ДЕЙСТВИЕ заданного понятия к другому понятию}
				\scnitem{абстрактный sc-агент поиска ответа ВРЕМЕННОЕ ОТНОШЕНИЕ заданных понятий}
				\scnitem{абстрактный sc-агент поиска ответа ПРОСТРАНСТВЕННОЕ ОТНОШЕНИЕ заданных понятий}
				\scnitem{абстрактный sc-агент поиска ответа КОЛИЧЕСТВЕННОГО ОТНОШЕНИЕ заданных понятий}
				\scnitem{абстрактный sc-агент поиска ответа КАЧЕСТВЕННОЕ ОТНОШЕНИЕ заданных понятий}
				\scnitem{абстрактный sc-агент поиска ответа ОПИСАНИЕ заданного понятия с другими понятиями}
				\scnitem{абстрактный sc-агент поиска ответа ОПРЕДЕЛЕНИЕ заданного понятия с другими понятиями}
				\scnitem{абстрактный sc-агент поиска ответа ПРИЧИНА заданного понятия с другими понятиями}
				\scnitem{абстрактный sc-агент поиска ответа СЛЕДСТВИЕ заданного понятия с другими понятиями}
				\scnitem{абстрактный sc-агент поиска ответа ДЕТАЛИЗАЦИЯ понятий, состоящих в отношении с заданным понятием}
			\end{scnrelfromset}
		\end{scnindent}
		\scnitem{абстрактный sc-агент, решающий задачу синтеза ответа на заданный вопрос}
	\end{scnrelfromset}
\end{SCn}

%%%%%%%%%%%%%%%%%%%%%%%%% referenc.tex %%%%%%%%%%%%%%%%%%%%%%%%%%%%%%
% sample references
% %
% Use this file as a template for your own input.
%
%%%%%%%%%%%%%%%%%%%%%%%% Springer-Verlag %%%%%%%%%%%%%%%%%%%%%%%%%%
%
% BibTeX users please use
% \bibliographystyle{}
% \bibliography{}
%
\biblstarthook{In view of the parallel print and (chapter-wise) online publication of your book at \url{www.springerlink.com} it has been decided that -- as a genreral rule --  references should be sorted chapter-wise and placed at the end of the individual chapters. However, upon agreement with your contact at Springer you may list your references in a single seperate chapter at the end of your book. Deactivate the class option \texttt{sectrefs} and the \texttt{thebibliography} environment will be put out as a chapter of its own.\\\indent
References may be \textit{cited} in the text either by number (preferred) or by author/year.\footnote{Make sure that all references from the list are cited in the text. Those not cited should be moved to a separate \textit{Further Reading} section or chapter.} If the citatiion in the text is numbered, the reference list should be arranged in ascending order. If the citation in the text is author/year, the reference list should be \textit{sorted} alphabetically and if there are several works by the same author, the following order should be used:
\begin{enumerate}
\item all works by the author alone, ordered chronologically by year of publication
\item all works by the author with a coauthor, ordered alphabetically by coauthor
\item all works by the author with several coauthors, ordered chronologically by year of publication.
\end{enumerate}
The \textit{styling} of references\footnote{Always use the standard abbreviation of a journal's name according to the ISSN \textit{List of Title Word Abbreviations}, see \url{http://www.issn.org/en/node/344}} depends on the subject of your book:
\begin{itemize}
\item The \textit{two} recommended styles for references in books on \textit{mathematical, physical, statistical and computer sciences} are depicted in ~\cite{science-contrib, science-online, science-mono, science-journal, science-DOI} and ~\cite{phys-online, phys-mono, phys-journal, phys-DOI, phys-contrib}.
\item Examples of the most commonly used reference style in books on \textit{Psychology, Social Sciences} are~\cite{psysoc-mono, psysoc-online,psysoc-journal, psysoc-contrib, psysoc-DOI}.
\item Examples for references in books on \textit{Humanities, Linguistics, Philosophy} are~\cite{humlinphil-journal, humlinphil-contrib, humlinphil-mono, humlinphil-online, humlinphil-DOI}.
\item Examples of the basic Springer style used in publications on a wide range of subjects such as \textit{Computer Science, Economics, Engineering, Geosciences, Life Sciences, Medicine, Biomedicine} are ~\cite{basic-contrib, basic-online, basic-journal, basic-DOI, basic-mono}. 
\end{itemize}
}

\begin{thebibliography}{99.}%
% and use \bibitem to create references.
%
% Use the following syntax and markup for your references if 
% the subject of your book is from the field 
% "Mathematics, Physics, Statistics, Computer Science"
%
% Contribution 
\bibitem{science-contrib} Broy, M.: Software engineering --- from auxiliary to key technologies. In: Broy, M., Dener, E. (eds.) Software Pioneers, pp. 10-13. Springer, Heidelberg (2002)
%
% Online Document
\bibitem{science-online} Dod, J.: Effective substances. In: The Dictionary of Substances and Their Effects. Royal Society of Chemistry (1999) Available via DIALOG. \\
\url{http://www.rsc.org/dose/title of subordinate document. Cited 15 Jan 1999}
%
% Monograph
\bibitem{science-mono} Geddes, K.O., Czapor, S.R., Labahn, G.: Algorithms for Computer Algebra. Kluwer, Boston (1992) 
%
% Journal article
\bibitem{science-journal} Hamburger, C.: Quasimonotonicity, regularity and duality for nonlinear systems of partial differential equations. Ann. Mat. Pura. Appl. \textbf{169}, 321--354 (1995)
%
% Journal article by DOI
\bibitem{science-DOI} Slifka, M.K., Whitton, J.L.: Clinical implications of dysregulated cytokine production. J. Mol. Med. (2000) doi: 10.1007/s001090000086 
%
\bigskip

% Use the following (APS) syntax and markup for your references if 
% the subject of your book is from the field 
% "Mathematics, Physics, Statistics, Computer Science"
%
% Online Document
\bibitem{phys-online} J. Dod, in \textit{The Dictionary of Substances and Their Effects}, Royal Society of Chemistry. (Available via DIALOG, 1999), 
\url{http://www.rsc.org/dose/title of subordinate document. Cited 15 Jan 1999}
%
% Monograph
\bibitem{phys-mono} H. Ibach, H. L\"uth, \textit{Solid-State Physics}, 2nd edn. (Springer, New York, 1996), pp. 45-56 
%
% Journal article
\bibitem{phys-journal} S. Preuss, A. Demchuk Jr., M. Stuke, Appl. Phys. A \textbf{61}
%
% Journal article by DOI
\bibitem{phys-DOI} M.K. Slifka, J.L. Whitton, J. Mol. Med., doi: 10.1007/s001090000086
%
% Contribution 
\bibitem{phys-contrib} S.E. Smith, in \textit{Neuromuscular Junction}, ed. by E. Zaimis. Handbook of Experimental Pharmacology, vol 42 (Springer, Heidelberg, 1976), p. 593
%
\bigskip
%
% Use the following syntax and markup for your references if 
% the subject of your book is from the field 
% "Psychology, Social Sciences"
%
%
% Monograph
\bibitem{psysoc-mono} Calfee, R.~C., \& Valencia, R.~R. (1991). \textit{APA guide to preparing manuscripts for journal publication.} Washington, DC: American Psychological Association.
%
% Online Document
\bibitem{psysoc-online} Dod, J. (1999). Effective substances. In: The dictionary of substances and their effects. Royal Society of Chemistry. Available via DIALOG. \\
\url{http://www.rsc.org/dose/Effective substances.} Cited 15 Jan 1999.
%
% Journal article
\bibitem{psysoc-journal} Harris, M., Karper, E., Stacks, G., Hoffman, D., DeNiro, R., Cruz, P., et al. (2001). Writing labs and the Hollywood connection. \textit{J Film} Writing, 44(3), 213--245.
%
% Contribution 
\bibitem{psysoc-contrib} O'Neil, J.~M., \& Egan, J. (1992). Men's and women's gender role journeys: Metaphor for healing, transition, and transformation. In B.~R. Wainrig (Ed.), \textit{Gender issues across the life cycle} (pp. 107--123). New York: Springer.
%
% Journal article by DOI
\bibitem{psysoc-DOI}Kreger, M., Brindis, C.D., Manuel, D.M., Sassoubre, L. (2007). Lessons learned in systems change initiatives: benchmarks and indicators. \textit{American Journal of Community Psychology}, doi: 10.1007/s10464-007-9108-14.
%
%
% Use the following syntax and markup for your references if 
% the subject of your book is from the field 
% "Humanities, Linguistics, Philosophy"
%
\bigskip
%
% Journal article
\bibitem{humlinphil-journal} Alber John, Daniel C. O'Connell, and Sabine Kowal. 2002. Personal perspective in TV interviews. \textit{Pragmatics} 12:257--271
%
% Contribution 
\bibitem{humlinphil-contrib} Cameron, Deborah. 1997. Theoretical debates in feminist linguistics: Questions of sex and gender. In \textit{Gender and discourse}, ed. Ruth Wodak, 99--119. London: Sage Publications.
%
% Monograph
\bibitem{humlinphil-mono} Cameron, Deborah. 1985. \textit{Feminism and linguistic theory.} New York: St. Martin's Press.
%
% Online Document
\bibitem{humlinphil-online} Dod, Jake. 1999. Effective substances. In: The dictionary of substances and their effects. Royal Society of Chemistry. Available via DIALOG. \\
http://www.rsc.org/dose/title of subordinate document. Cited 15 Jan 1999
%
% Journal article by DOI
\bibitem{humlinphil-DOI} Suleiman, Camelia, Daniel C. O'Connell, and Sabine Kowal. 2002. `If you and I, if we, in this later day, lose that sacred fire...': Perspective in political interviews. \textit{Journal of Psycholinguistic Research}. doi: 10.1023/A:1015592129296.
%
%
%
\bigskip
%
%
% Use the following syntax and markup for your references if 
% the subject of your book is from the field 
% "Computer Science, Economics, Engineering, Geosciences, Life Sciences"
%
%
% Contribution 
\bibitem{basic-contrib} Brown B, Aaron M (2001) The politics of nature. In: Smith J (ed) The rise of modern genomics, 3rd edn. Wiley, New York 
%
% Online Document
\bibitem{basic-online} Dod J (1999) Effective Substances. In: The dictionary of substances and their effects. Royal Society of Chemistry. Available via DIALOG. \\
\url{http://www.rsc.org/dose/title of subordinate document. Cited 15 Jan 1999}
%
% Journal article by DOI
\bibitem{basic-DOI} Slifka MK, Whitton JL (2000) Clinical implications of dysregulated cytokine production. J Mol Med, doi: 10.1007/s001090000086
%
% Journal article
\bibitem{basic-journal} Smith J, Jones M Jr, Houghton L et al (1999) Future of health insurance. N Engl J Med 965:325--329
%
% Monograph
\bibitem{basic-mono} South J, Blass B (2001) The future of modern genomics. Blackwell, London 
%
\end{thebibliography}
