\chapter{Язык вопросов для ostis-систем}
\chapauthortoc{Самодумкин С.~А.\\Шункевич Д.~В.\\ Ивашенко В.~П.}
\label{chapter_requests}

\vspace{-7\baselineskip}

\begin{SCn}
\begin{scnrelfromlist}{авторы}
	\scnitem{Самодумкин С.~А.}
	\scnitem{Шункевич Д.~В.}
	\scnitem{Ивашенко В.~П.}
\end{scnrelfromlist}

\bigskip

\scntext{аннотация}{В главе уточнена формальная трактовка понятия вопроса, что позволило задать \textit{Язык вопросов}.}

\bigskip

\begin{scnrelfromlist}{подраздел}
	\scnitem{\nameref{sec_requests_syntax}~\ref{sec_requests_syntax}}
	\scnitem{\nameref{sec_requests_den_semantics}~\ref{sec_requests_den_semantics}}
	\scnitem{\nameref{sec_requests_op_semantics}~\ref{sec_requests_op_semantics}}
\end{scnrelfromlist}

\bigskip

\begin{scnrelfromlist}{ключевой знак}
	\scnitem{Язык вопросов}
\end{scnrelfromlist}

\bigskip

\begin{scnrelfromlist}{библиографическая ссылка}
	\scnitem{\scncite{Suleimanov2001}}
	\scnitem{\scncite{Suleimanov2014}}
	\scnitem{\scncite{Bukharev1990}}
	\scnitem{\scncite{Kwok2001}}
	\scnitem{\scncite{Emelyanov2007}}
	\scnitem{\scncite{Averyanov1993}}
	\scnitem{\scncite{Finn1976}}
	\scnitem{\scncite{Finn1981}}
	\scnitem{\scncite{Belnap1981}}
	\scnitem{\scncite{Sosnin2007}}
	\scnitem{\scncite{Zaharov2002}}
	\scnitem{\scncite{Hant1978}}
	\scnitem{\scncite{Lyubarsky1990}}
	\scnitem{\scncite{Samodumkin2009}}
	\scnitem{\scncite{Samodumkin2009a}}
\end{scnrelfromlist}
	
\end{SCn}

Одна из ключевых особенностей \textit{интеллектуальной системы} состоит в том, что \textit{пользователь} имеет возможность формулировать свою информационную потребность. Одним из способов выражения такой потребности является \textit{вопрос}. В процессе диалогового общения всегда существует контекст, который определяет дополнительную информацию, способствующую правильному пониманию \textit{смысла} сообщения. Особенность представления информации в \textit{базах знаний} \textit{ostis-систем} упрощает формирование информационной потребности пользователя, так как представленная информация в \textit{базах знаний} уже структурирована и известны отношения, заданные на определенном понятии, в отношении которого разрешается вопросно-проблемная ситуация. В работе \scncite{Averyanov1993} показно, что вопросно-проблемная ситуация не может быть решена в рамках формальной логики и природа вопроса может быть понятна в системе субъектно-объектных отношений. В связи с тем, что при формировании \textit{баз знаний} \textit{ostis-систем} происходит формирование субъектно-объектных отношений в рамках заданной \textit{предметной области}, тем самым упрощается выражение информационной потребности пользователем средствами \textit{SC-кода}.   

Целью разработки \textit{Языка вопросов} и последующее его развитие является реализация возможности понимания действий, осуществляемых \textit{ostis-системой}, при формировании ответа на поставленный \textit{вопрос}. В процессе формирования вывода на поставленный \textit{вопрос} возможны следующие варианты:
1) ответ на поставленный вопрос существует в \textit{базе знаний} и происходит локализация \textit{фрагмента базы знаний} в контексте выраженной средствами \textit{SC-кода} информационной потребности \textit{пользователя};
2) ответ связан с разрешением некоторой задачной ситуации, которая содержится в контексте \textit{вопроса} и формирование \textit{ответа на вопрос} возлагается на \textit{решатель задач}.

\section{Синтаксис языка вопросов для ostis-систем}
\label{sec_requests_syntax}

\textit{Язык вопросов} относится к семейству семантических совместимых языков – \textit{sc-языков} и предназначен для формального описания поискового предписания \textit{ostis-систем} с целью удовлетворения информационной потребности \textit{пользователей}.

\section{Денотационная семантика языка вопросов для ostis-систем}
\label{sec_requests_den_semantics}

\textit{Денотационная семантика Языка вопросов} включает классы \textit{вопросов} и соответствующие классы \textit{ответов}, необходимые для спецификации формулировок \textit{вопросов} и \textit{ответов} на них. В \textit{Семантическую классификацию вопросов} и \textit{Семантическую классификацию ответов} \textit{Языка вопросов} заложена идея, представленная в работе \scncite{Suleimanov2001}.

Базовыми понятиями в \textit{Языке вопросов} являются:
\begin{textitemize}
	\item \textit{основной знак в рамках заданного вопроса} - \textit{знак}, относительно которого задан вопрос;
	\item \textit{неосновной знак в рамках заданного вопроса} - \textit{знак}, стоящий в некотором отношении с \textit{основным знаком в рамках заданного вопроса};
    %\item \textit{первостепенное понятие в заданном вопросе} - \textit{понятие}, находящееся по отношению к \textit{основному понятию в заданном вопросе} на более высоком уровне иерархии понятий \textit{предметной области};
    \item \textit{параметр вопроса\scnrolesign};
	\item \textit{ответ на вопрос*}.
\end{textitemize}

Введём типы отношений, необходимых для формирования вопросов.

\begin{SCn}
\scnheader{отношение в рамках заданного вопроса\scnsupergroupsign}
\scnidtf{определённое отношение между знаками \textit{предметной области} в контексте вопроса}
\end{SCn}

\begin{SCn}
\scnheader{базовое отношение в рамках заданного вопроса}
\scnidtf{\textit{класс отношений}, объединяющий \textit{отношения в заданном вопросе}, отражающие однотипный \textit{смысл} и раскрывающие определённый признак \textit{знаков} \textit{предметной области}}
\begin{scnrelfromset}{декомпозиция}
	\scnitem{отношение состояния}
	\scnitem{отношение действия}
	\scnitem{отношение состава}
	\scnitem{теоретико-множественное отношение}
	\scnitem{темпоральное отношение}
	\scnitem{пространственное отношение}
	\scnitem{количественное отношение}
	\scnitem{качественное отношение}
\end{scnrelfromset}
\end{SCn}

Например, \textit{отношения в рамках заданного вопроса} такие, как \scnqqi{играет*}, \scnqqi{спит*}, \scnqqi{плавает*}, объединяются в класс \textit{отношений состояния} по признаку выражать состояние знака (раскрывает признак знака \textit{предметной области} --- \scnqqi{находиться в некотором состоянии}).

\begin{SCn}
\scnheader{составное отношение в рамках заданного вопроса}
\scnidtf{устойчивая комбинация двух \textit{отношений действия}: действия, направленного на \textit{параметр вопроса\scnrolesign}, и действия, направленного на \textit{ответ на вопрос*}}
\scnsuperset{составное отношение функции}
\end{SCn}

Например, \textit{составное отношение функции} знака \textit{S1} --- \scnqqi{S1 переводит S2 в S3}; \scnqqi{Нефтеперерабатывающий завод перерабатывает нефть в нефтепродукты}.

\subsection{Семантическая классификация вопросов и ответов}
\label{chapter_questions_sec_sem_classification}

Смысловая типизация \textit{вопросов} дает возможность противопоставить каждому типу вопроса ограниченный набор допустимых, то есть логически корректных конструкций, передающий правильный \textit{смысл} \textit{вопроса} в зависимости от класса \textit{вопроса}. При этом \textit{Семантическая классификация вопросов} позволяет разбить множество \textit{вопросов} на классы, в каждом из которых требуется раскрытие некоторого однотипного \textit{смысла}, заданного классом этого \textit{вопроса}. 

\begin{SCn}
\scnheader{вопрос}
\scnidtf{запрос}
\scnidtf{непроцедурная формулировка задачи на поиск (в текущем состоянии базы знаний) или на генерацию знания, удовлетворяющего заданным требованиям}
\begin{scnindent}
	\scnidtf{каким способом}
	\scnidtf{запрос метода (способа) решения заданного (указываемого) вида задач или класса задач либо плана решения конкретной указываемой задачи}
\end{scnindent}
\scnidtf{задача, направленная на удовлетворение информационной потребности некоторого субъекта-заказчика}
\begin{scnrelfromset}{декомпозиция}
	\scnitem{вопрос, требующий вывода семантической окрестности \textit{основного знака}}
	\begin{scnindent}
		\begin{scnhaselementrolelist}{пример}
			\scnitem{Вопрос. Какие отношения определены на понятии \textit{Город Минск}}
		\end{scnhaselementrolelist}
	\end{scnindent}
	\scnitem{вопрос, требующий раскрытия в ответе \textit{базового отношения} \textit{основного знака}}
%	\begin{scnindent}
%		\begin{scnhaselementrolelist}{пример}
%			\scnitem{Вопрос. Что легче: железо или дерево}
%		\end{scnhaselementrolelist}
%	\end{scnindent}
	\scnitem{вопрос, требующий раскрытия в ответе \textit{составного отношения} \textit{основного знака}}
	\begin{scnindent}
		\scntext{пояснение}{Такому классу \textit{вопросов} соответствуют классы \textit{ответов}, в которых \textit{главный знак} раскрывается через \textit{составное отношение}.}
%		\begin{scnhaselementrolelist}{пример}
%			\scnitem{Вопрос. Какую функцию выполняет компилятор}
%		\end{scnhaselementrolelist}
	\end{scnindent}
	\scnitem{вопрос, требующий раскрытия в ответе произвольной комбинации \textit{базового отношения} и/или \textit{составного отношения} \textit{основного знака}}
%	\begin{scnindent}
%		\begin{scnhaselementrolelist}{пример}
%			\scnitem{Вопрос. Дайте определение компилятора}
%		\end{scnhaselementrolelist}
%	\end{scnindent}
	\scnitem{вопрос, требующий раскрытия в ответе более чем одного \textit{основного знака}}
%	\begin{scnindent}
%		\begin{scnhaselementrolelist}{пример}
%			\scnitem{Вопрос. Докажите теорему Пифагора}
%		\end{scnhaselementrolelist}
%	\end{scnindent}
\end{scnrelfromset}
\end{SCn}

\begin{SCn}
\scnheader{ответ на вопрос, требующий раскрытия в ответе \textit{базового отношения} \textit{основного знака}}
\scnrelto{класс ответов}{вопрос, требующий раскрытия в ответе \textit{базового отношения} \textit{основного знака}}
\begin{scnrelfromset}{декомпозиция}
	\scnitem{ответ на вопрос, требующий раскрытия в ответе \textit{отношения состава} \textit{основного знака}}
	\begin{scnindent}
%		\scnidtf{класс ответов, в которых \textit{основной знак} \textit{S} раскрывается через его \textit{отношение состава} с составляющими знаками \textit{P} и \textit{Q}}
		\begin{scnhaselementrolelist}{пример}
%			\scnitem{\{S состоит из P и Q\}}
%			\begin{scnindent}
%				\scnrelto{ответ на вопрос}{Вопрос. Какие элементы принадлежат множеству P}
%			\end{scnindent}
			\scnitem{\{Железнодорожный район Города Витебск, Октябрьский район Города Витебск, Первомайский район Города Витебск\}}
			\begin{scnindent}
				\scneqimage[30em]{author/part3/figures/question\_about\_vitebsk\_regions\_answer.png}
			\end{scnindent}
			\begin{scnindent}
				\scnrelto{ответ на вопрос}{Вопрос. Какие административные районы входят в состав Города Витебск}
				\begin{scnindent}
					\scneqimage[30em]{author/part3/figures/question\_about\_vitebsk\_regions.png}
				\end{scnindent}
			\end{scnindent}
		\end{scnhaselementrolelist}
	\end{scnindent}
	\scnitem{ответ на вопрос, требующий раскрытия в ответе \textit{теоретико-множественного отношения} \textit{основного знака}}
	\begin{scnindent}
		\scnidtf{класс ответов, в которых \textit{основной знак} \textit{S} раскрывается через его \textit{теоретико-множественное отношение} к другому знаку \textit{P}, содержащего \textit{S} как часть}
		\begin{scnhaselementrolelist}{пример}
%			\scnitem{\{P включает S\}}
%			\begin{scnindent}
%				\scnrelto{ответ на вопрос}{Вопрос. В какое множество включается знак S}
%			\end{scnindent}
			\scnitem{\{Смолевичский район является частью Минской области\}}
			\begin{scnindent}
				\scnrelto{ответ на вопрос}{Вопрос. Частью какой области является Смолевичский район}
				\begin{scnindent}
					\scneqimage[30em]{author/part3/figures/question\_about\_smolevichi\_inclusion.png}
				\end{scnindent}
			\end{scnindent}
		\end{scnhaselementrolelist}
	\end{scnindent}
	\scnitem{ответ на вопрос, требующий раскрытия в ответе \textit{отношения состояния} \textit{основного знака}}
	\begin{scnindent}
		\scnidtf{класс ответов, в которых \textit{основной знак} \textit{S} раскрывается через его \textit{отношение состояния}}
		\begin{scnhaselementrolelist}{пример}
%			\scnitem{\{S играет\}}
%			\begin{scnindent}
%				\scnrelto{ответ на вопрос}{Вопрос. Какое состояние у S}
%			\end{scnindent}
			\scnitem{\{Брест, Волковыск, Гродно, Лида, Мозырь и другие имели Магдебургское право\}}
			\begin{scnindent}
				\scnrelto{ответ на вопрос}{Вопрос. Какие города современной территории Республики Беларусь имели Магдебургское право}
				\begin{scnindent}
					\scneqimage[30em]{author/part3/figures/question\_about\_minsk\_district\_town\_with\_mag\_act.png}
				\end{scnindent}
			\end{scnindent}
		\end{scnhaselementrolelist}
	\end{scnindent}
	\scnitem{ответ на вопрос, требующий раскрытия в ответе \textit{отношения действия} \textit{основного знака}}
	\begin{scnindent}
		\scnidtf{класс ответов, в которых \textit{основной знак} \textit{S} раскрывается через его \textit{отношение действия} по отношению к другому знаку \textit{P}}
		\begin{scnhaselementrolelist}{пример}
			\scnitem{\{S перемещает P\}}
			\begin{scnindent}
				\scnrelto{ответ на вопрос}{Вопрос. Какое действие совершает S}
			\end{scnindent}
		\end{scnhaselementrolelist}
	\end{scnindent}
	\scnitem{ответ на вопрос, требующий раскрытия в ответе \textit{темпорального отношения} \textit{основного знака}}
	\begin{scnindent}
		\scnidtf{класс ответов, в которых \textit{основной знак} \textit{S} раскрывается через его \textit{темпоральное отношение} к другому понятию \textit{P} по некоторой временной шкале}
		\begin{scnhaselementrolelist}{пример}
			\scnitem{\{S выполняется раньше P\}}
			\begin{scnindent}
				\scnrelto{ответ на вопрос}{Вопрос. Как связаны S и P по временной шкале}
			\end{scnindent}
			\scnitem{\{Первый раздел Речи Посполитой был раньше Бородинского сражения\}}
			\begin{scnindent}
				\scneqimage[30em]{author/part3/figures/question\_about\_event\_answer.png}
				\scnrelto{ответ на вопрос}{Вопрос. Какое событие произошло раньше: Первый раздел Речи Посполитой или Бородинское сражение}
				\begin{scnindent}
					\scneqimage[30em]{author/part3/figures/question\_about\_events.png}
				\end{scnindent}
			\end{scnindent}
		\end{scnhaselementrolelist}
	\end{scnindent}
	\scnitem{ответ на вопрос, требующий раскрытия в ответе \textit{пространственного отношения} \textit{основного знака}}
	\begin{scnindent}
		\scnidtf{класс ответов, в которых \textit{основной знак} \textit{S} раскрывается через \textit{пространственное отношение}, отражающее его положение в пространстве относительно другого знак \textit{P}}
		\begin{scnhaselementrolelist}{пример}
			\scnitem{\{S находится ближе чем P\}}
			\begin{scnindent}
				\scnrelto{ответ на вопрос}{Вопрос. Где находится S по отношению к P}
			\end{scnindent}
		\end{scnhaselementrolelist}
	\end{scnindent}
	\scnitem{ответ на вопрос, требующий раскрытия в ответе \textit{количественного отношения} \textit{основного знака}}
	\begin{scnindent}
		\scnidtf{класс ответов, в которых раскрывается \textit{количественное отношение} \textit{основного знака} }
		\begin{scnhaselementrolelist}{пример}
%			\scnitem{\{S больше чем P\}}
%			\begin{scnindent}
%				\scnrelto{ответ на вопрос}{Вопрос. Что больше: S или P}
%			\end{scnindent}
			\scnitem{\{Высота Горы Дзержинская - 345 м\}}
			\begin{scnindent}
				\scnrelto{ответ на вопрос}{Вопрос. Какова высота Горы Дзержинская}
				\begin{scnindent}
					\scneqimage[30em]{author/part3/figures/question\_about\_mountain\_length.png}
				\end{scnindent}
			\end{scnindent}
		\end{scnhaselementrolelist}
	\end{scnindent}
	\scnitem{ответ на вопрос, требующий раскрытия в ответе \textit{качественного отношения} \textit{основного знака}}
	\begin{scnindent}
		\scnidtf{класс ответов, в которых раскрывается \textit{качественное отношение} \textit{основного знака} \textit{S} к другому знаку \textit{P}}
		\begin{scnhaselementrolelist}{пример}
			\scnitem{\{S легче чем P\}}
			\begin{scnindent}
				\scnrelto{ответ на вопрос}{Вопрос. Что легче: S или P}
			\end{scnindent}
			\scnitem{\{Территория Минской области больше Брестской\}}
			\begin{scnindent}
				\scneqimage[30em]{author/part3/figures/question\_about\_district\_squares\_answer.png}
				\scnrelto{ответ на вопрос}{Вопрос. Территория какой административной области больше: Минской или Брестской}
				\begin{scnindent}
					\scneqimage[30em]{author/part3/figures/question\_about\_district\_squares.png}
				\end{scnindent}
			\end{scnindent}
		\end{scnhaselementrolelist}
	\end{scnindent}
\end{scnrelfromset}
\end{SCn}

\begin{SCn}
\scnheader{ответ на вопрос, требующий раскрытия в ответе произвольной комбинации \textit{базового отношения} и/или \textit{составного отношения} \textit{основного знака}}
\scnrelto{класс ответов}{вопрос, требующий раскрытия в ответе произвольной комбинации \textit{базового отношения} и/или \textit{составного отношения} \textit{основного знака}}
\begin{scnrelfromset}{декомпозиция}
	\scnitem{ответ на вопрос, требующий раскрытия в ответе произвольной комбинации \textit{составного отношения описания} \textit{основного знака}}
	\begin{scnindent}
		\scnidtf{класс ответов, в которых раскрываются произвольные комбинации \textit{базового отношения} и/или \textit{составного отношения} \textit{основного знака} \textit{S} с другими знаками}
		\begin{scnhaselementrolelist}{пример}
			\scnitem{\{S состоит из P, Q, W. S переводит X и Y и выполняется раньше Z\}}
			\begin{scnindent}
				\scnrelto{ответ на вопрос}{Вопрос. Что такое S}
			\end{scnindent}
		\end{scnhaselementrolelist}
	\end{scnindent}
	\scnitem{ответ на вопрос, требующий раскрытия в ответе произвольной комбинации \textit{составного отношения определения} \textit{основного знака}}
	\begin{scnindent}
		\scnidtf{класс ответов, в которых \textit{основной знак} \textit{S} раскрывается через \textit{первостепенное понятие} и его \textit{описание}}
		\begin{scnhaselementrolelist}{пример}
%			\scnitem{\{Студент - это человек, который обучается в ВУЗе\}}
%			\begin{scnindent}
%				\scnrelto{ответ на вопрос}{Вопрос. Как определяется понятие студента}
%			\end{scnindent}
			\scnitem{\{Минск - это столица, которая находится в РБ\}}
			\begin{scnindent}
				\scnrelto{ответ на вопросы}{Вопрос. Как определяется город Минск}
			\end{scnindent}
		\end{scnhaselementrolelist}
	\end{scnindent}
	\scnitem{ответ на вопрос, требующий раскрытия в ответе произвольной комбинации \textit{составного отношения причины} \textit{основного знака}}
	\begin{scnindent}
		\scnidtf{класс ответов, в которых раскрывается условие существования некоторых отношений \textit{основного знака} \textit{S} с другими знаками}
		\begin{scnhaselementrolelist}{пример}
			\scnitem{\{Время в пути от города Минска до города Борисова меньше чем время в пути от города Минска до города Орша, потому что расстояние от города Минска меньше до города Борисова, чем до города Орша\}}
			\begin{scnindent}
				\scnrelto{ответ на вопрос}{Вопрос. Почему время в пути от города Минска до города Борисова меньше чем время в пути от города Минска до города Орша}
			\end{scnindent}
		\end{scnhaselementrolelist}
	\end{scnindent}
	\scnitem{ответ на вопрос, требующий раскрытия в ответе произвольной комбинации \textit{составного отношения следствия} \textit{основного знака}}
	\scnidtf{класс ответов, в которых раскрывается следствие от существования некоторых отношений \textit{основного знака} \textit{S} с другими знаками}
	\begin{scnindent}
		\begin{scnhaselementrolelist}{пример}
			\scnitem{\{Расстояние от города Минска до города Борисова меньше расстояния от города Минска до города Орша, поэтому от города Минска до города Борисова время в пути меньше чем до города Орша\}}
			\begin{scnindent}
				\scnrelto{ответ на вопрос}{Вопрос. Что следует из того, что расстояние от города Минска до города Борисова меньше расстояния от города Минска до города Орша}
			\end{scnindent}
		\end{scnhaselementrolelist}
	\end{scnindent}
\end{scnrelfromset}
\end{SCn}

\begin{SCn}
\scnheader{ответ на вопрос, требующий раскрытия в ответе более чем одного \textit{основного знака}}
\scnrelto{класс ответов}{вопрос, требующий раскрытия в ответе более чем одного \textit{основного знака}}
\scnsuperset{ответ на вопрос, требующий раскрытия в ответе \textit{отношение детализации} знаков, стоящих в некоторых отношениях с \textit{основным знаком}}
\begin{scnindent}
	\scnidtf{класс ответов, в которых происходит детализация знаков, стоящих в некоторых отношениях с \textit{основным знаком} \textit{S}}
	\begin{scnhaselementrolelist}{пример}
		\scnitem{\{Город Минск расположен на реке Свислочь, которая впадает в реку Березина, протекающую через город Светлогорск\}}
		\begin{scnindent}
			\scnrelto{ответ на вопрос}{Вопрос. Какая связь водной сети существует  между городом Минск и городом Свтлогорск}
		\end{scnindent}
	\end{scnhaselementrolelist}
\end{scnindent}
\end{SCn}


\section{Операционная семантика языка вопросов для ostis-систем}
\label{sec_requests_op_semantics}

Рассмотренные в \textit{\ref{chapter_questions_sec_sem_classification}~\nameref{chapter_questions_sec_sem_classification}} классы вопросов и ответов обуславливают необходимость определения операционной семантики Языка вопросов. Каждому классу вопросов должен соответствовать определённый коллектив sc-агентов, реализующих вывод из базы знаний ostis-системы соответствующих ответов. Следует отметить, что в зависимости от степени наполненности базы знаний ответы могут содержаться в базе знаний либо в текущей версии базы знаний отсутствовать. В случае наличия ответа в базе знаний информационная потребность пользователя реализуется информационно-поисковыми sc-агентами, в противном случае -- в зависимости от классов вопросов реализация вывода ответов осуществляется специализированные sc-агентами, которые в процессе работы дополнительно выполняют вычислительные задачи либо осуществляют синтез на основе логического вывода.

\begin{SCn}
	\scnheader{интерпретатор Языка вопросов}
	\begin{scnrelfromset}{декомпозиция}
		\scnitem{абстрактный sc-агент, решающий задачу поиска ответа на заданный вопрос}
		\begin{scnindent}
			\begin{scnrelfromset}{декомпозиция}
				\scnitem{абстрактный sc-агент поиска семантической окрестности \textit{основного знака}}
				\scnitem{абстрактный sc-агент поиска ответа на вопрос, требующий раскрытия в ответе \textit{отношения состава} для \textit{основного знака}}
				\scnitem{абстрактный sc-агент поиска ответа на вопрос, требующий раскрытия в ответе \textit{теоретико-множественного отношения} для \textit{основного знака}}
				\scnitem{абстрактный sc-агент поиска ответа на вопрос, требующий раскрытия в ответе \textit{отношения состояния} для \textit{основного знака}}	
				\scnitem{абстрактный sc-агент поиска ответа на вопрос, требующий раскрытия в ответе \textit{отношения действия} для \textit{основного знака}}	
				\scnitem{абстрактный sc-агент поиска ответа на вопрос, требующий раскрытия в ответе \textit{темпорального отношения} для \textit{основного знака}}
				\scnitem{абстрактный sc-агент поиска ответа на вопрос, требующий раскрытия в ответе \textit{пространственного отношения} для \textit{основного знака}}
				\scnitem{абстрактный sc-агент поиска ответа на вопрос, требующий раскрытия в ответе \textit{количественного отношения} для \textit{основного знака}}
				\scnitem{абстрактный sc-агент поиска ответа на вопрос, требующий раскрытия в ответе \textit{качественного отношения} для \textit{основного знака}}
				\scnitem{абстрактный sc-агент поиска ответа на вопрос, требующий раскрытия в ответе \textit{отношения описания} для \textit{основного знака}}
				\scnitem{абстрактный sc-агент поиска ответа на вопрос, требующий раскрытия в ответе \textit{отношения определения} для \textit{основного знака}}
				\scnitem{абстрактный sc-агент поиска ответа на вопрос, требующий раскрытия в ответе \textit{отношения причины} для \textit{основного знака}}
				\scnitem{абстрактный sc-агент поиска ответа на вопрос, требующий раскрытия в ответе \textit{отношения следствия} для \textit{основного знака}}
				\scnitem{абстрактный sc-агент поиска ответа на вопрос, требующий раскрытия в ответе \textit{отношения детализации} для \textit{основного знака}}
			\end{scnrelfromset}
		\end{scnindent}
		\scnitem{абстрактный sc-агент, решающий задачу синтеза ответа на заданный вопрос}
	\end{scnrelfromset}
\end{SCn}

%%%%%%%%%%%%%%%%%%%%%%%%% referenc.tex %%%%%%%%%%%%%%%%%%%%%%%%%%%%%%
% sample references
% %
% Use this file as a template for your own input.
%
%%%%%%%%%%%%%%%%%%%%%%%% Springer-Verlag %%%%%%%%%%%%%%%%%%%%%%%%%%
%
% BibTeX users please use
% \bibliographystyle{}
% \bibliography{}
%
\biblstarthook{In view of the parallel print and (chapter-wise) online publication of your book at \url{www.springerlink.com} it has been decided that -- as a genreral rule --  references should be sorted chapter-wise and placed at the end of the individual chapters. However, upon agreement with your contact at Springer you may list your references in a single seperate chapter at the end of your book. Deactivate the class option \texttt{sectrefs} and the \texttt{thebibliography} environment will be put out as a chapter of its own.\\\indent
References may be \textit{cited} in the text either by number (preferred) or by author/year.\footnote{Make sure that all references from the list are cited in the text. Those not cited should be moved to a separate \textit{Further Reading} section or chapter.} If the citatiion in the text is numbered, the reference list should be arranged in ascending order. If the citation in the text is author/year, the reference list should be \textit{sorted} alphabetically and if there are several works by the same author, the following order should be used:
\begin{enumerate}
\item all works by the author alone, ordered chronologically by year of publication
\item all works by the author with a coauthor, ordered alphabetically by coauthor
\item all works by the author with several coauthors, ordered chronologically by year of publication.
\end{enumerate}
The \textit{styling} of references\footnote{Always use the standard abbreviation of a journal's name according to the ISSN \textit{List of Title Word Abbreviations}, see \url{http://www.issn.org/en/node/344}} depends on the subject of your book:
\begin{itemize}
\item The \textit{two} recommended styles for references in books on \textit{mathematical, physical, statistical and computer sciences} are depicted in ~\cite{science-contrib, science-online, science-mono, science-journal, science-DOI} and ~\cite{phys-online, phys-mono, phys-journal, phys-DOI, phys-contrib}.
\item Examples of the most commonly used reference style in books on \textit{Psychology, Social Sciences} are~\cite{psysoc-mono, psysoc-online,psysoc-journal, psysoc-contrib, psysoc-DOI}.
\item Examples for references in books on \textit{Humanities, Linguistics, Philosophy} are~\cite{humlinphil-journal, humlinphil-contrib, humlinphil-mono, humlinphil-online, humlinphil-DOI}.
\item Examples of the basic Springer style used in publications on a wide range of subjects such as \textit{Computer Science, Economics, Engineering, Geosciences, Life Sciences, Medicine, Biomedicine} are ~\cite{basic-contrib, basic-online, basic-journal, basic-DOI, basic-mono}. 
\end{itemize}
}

\begin{thebibliography}{99.}%
% and use \bibitem to create references.
%
% Use the following syntax and markup for your references if 
% the subject of your book is from the field 
% "Mathematics, Physics, Statistics, Computer Science"
%
% Contribution 
\bibitem{science-contrib} Broy, M.: Software engineering --- from auxiliary to key technologies. In: Broy, M., Dener, E. (eds.) Software Pioneers, pp. 10-13. Springer, Heidelberg (2002)
%
% Online Document
\bibitem{science-online} Dod, J.: Effective substances. In: The Dictionary of Substances and Their Effects. Royal Society of Chemistry (1999) Available via DIALOG. \\
\url{http://www.rsc.org/dose/title of subordinate document. Cited 15 Jan 1999}
%
% Monograph
\bibitem{science-mono} Geddes, K.O., Czapor, S.R., Labahn, G.: Algorithms for Computer Algebra. Kluwer, Boston (1992) 
%
% Journal article
\bibitem{science-journal} Hamburger, C.: Quasimonotonicity, regularity and duality for nonlinear systems of partial differential equations. Ann. Mat. Pura. Appl. \textbf{169}, 321--354 (1995)
%
% Journal article by DOI
\bibitem{science-DOI} Slifka, M.K., Whitton, J.L.: Clinical implications of dysregulated cytokine production. J. Mol. Med. (2000) doi: 10.1007/s001090000086 
%
\bigskip

% Use the following (APS) syntax and markup for your references if 
% the subject of your book is from the field 
% "Mathematics, Physics, Statistics, Computer Science"
%
% Online Document
\bibitem{phys-online} J. Dod, in \textit{The Dictionary of Substances and Their Effects}, Royal Society of Chemistry. (Available via DIALOG, 1999), 
\url{http://www.rsc.org/dose/title of subordinate document. Cited 15 Jan 1999}
%
% Monograph
\bibitem{phys-mono} H. Ibach, H. L\"uth, \textit{Solid-State Physics}, 2nd edn. (Springer, New York, 1996), pp. 45-56 
%
% Journal article
\bibitem{phys-journal} S. Preuss, A. Demchuk Jr., M. Stuke, Appl. Phys. A \textbf{61}
%
% Journal article by DOI
\bibitem{phys-DOI} M.K. Slifka, J.L. Whitton, J. Mol. Med., doi: 10.1007/s001090000086
%
% Contribution 
\bibitem{phys-contrib} S.E. Smith, in \textit{Neuromuscular Junction}, ed. by E. Zaimis. Handbook of Experimental Pharmacology, vol 42 (Springer, Heidelberg, 1976), p. 593
%
\bigskip
%
% Use the following syntax and markup for your references if 
% the subject of your book is from the field 
% "Psychology, Social Sciences"
%
%
% Monograph
\bibitem{psysoc-mono} Calfee, R.~C., \& Valencia, R.~R. (1991). \textit{APA guide to preparing manuscripts for journal publication.} Washington, DC: American Psychological Association.
%
% Online Document
\bibitem{psysoc-online} Dod, J. (1999). Effective substances. In: The dictionary of substances and their effects. Royal Society of Chemistry. Available via DIALOG. \\
\url{http://www.rsc.org/dose/Effective substances.} Cited 15 Jan 1999.
%
% Journal article
\bibitem{psysoc-journal} Harris, M., Karper, E., Stacks, G., Hoffman, D., DeNiro, R., Cruz, P., et al. (2001). Writing labs and the Hollywood connection. \textit{J Film} Writing, 44(3), 213--245.
%
% Contribution 
\bibitem{psysoc-contrib} O'Neil, J.~M., \& Egan, J. (1992). Men's and women's gender role journeys: Metaphor for healing, transition, and transformation. In B.~R. Wainrig (Ed.), \textit{Gender issues across the life cycle} (pp. 107--123). New York: Springer.
%
% Journal article by DOI
\bibitem{psysoc-DOI}Kreger, M., Brindis, C.D., Manuel, D.M., Sassoubre, L. (2007). Lessons learned in systems change initiatives: benchmarks and indicators. \textit{American Journal of Community Psychology}, doi: 10.1007/s10464-007-9108-14.
%
%
% Use the following syntax and markup for your references if 
% the subject of your book is from the field 
% "Humanities, Linguistics, Philosophy"
%
\bigskip
%
% Journal article
\bibitem{humlinphil-journal} Alber John, Daniel C. O'Connell, and Sabine Kowal. 2002. Personal perspective in TV interviews. \textit{Pragmatics} 12:257--271
%
% Contribution 
\bibitem{humlinphil-contrib} Cameron, Deborah. 1997. Theoretical debates in feminist linguistics: Questions of sex and gender. In \textit{Gender and discourse}, ed. Ruth Wodak, 99--119. London: Sage Publications.
%
% Monograph
\bibitem{humlinphil-mono} Cameron, Deborah. 1985. \textit{Feminism and linguistic theory.} New York: St. Martin's Press.
%
% Online Document
\bibitem{humlinphil-online} Dod, Jake. 1999. Effective substances. In: The dictionary of substances and their effects. Royal Society of Chemistry. Available via DIALOG. \\
http://www.rsc.org/dose/title of subordinate document. Cited 15 Jan 1999
%
% Journal article by DOI
\bibitem{humlinphil-DOI} Suleiman, Camelia, Daniel C. O'Connell, and Sabine Kowal. 2002. `If you and I, if we, in this later day, lose that sacred fire...': Perspective in political interviews. \textit{Journal of Psycholinguistic Research}. doi: 10.1023/A:1015592129296.
%
%
%
\bigskip
%
%
% Use the following syntax and markup for your references if 
% the subject of your book is from the field 
% "Computer Science, Economics, Engineering, Geosciences, Life Sciences"
%
%
% Contribution 
\bibitem{basic-contrib} Brown B, Aaron M (2001) The politics of nature. In: Smith J (ed) The rise of modern genomics, 3rd edn. Wiley, New York 
%
% Online Document
\bibitem{basic-online} Dod J (1999) Effective Substances. In: The dictionary of substances and their effects. Royal Society of Chemistry. Available via DIALOG. \\
\url{http://www.rsc.org/dose/title of subordinate document. Cited 15 Jan 1999}
%
% Journal article by DOI
\bibitem{basic-DOI} Slifka MK, Whitton JL (2000) Clinical implications of dysregulated cytokine production. J Mol Med, doi: 10.1007/s001090000086
%
% Journal article
\bibitem{basic-journal} Smith J, Jones M Jr, Houghton L et al (1999) Future of health insurance. N Engl J Med 965:325--329
%
% Monograph
\bibitem{basic-mono} South J, Blass B (2001) The future of modern genomics. Blackwell, London 
%
\end{thebibliography}
