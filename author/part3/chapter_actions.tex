\chapauthor{Шункевич Д.В.\\Ковалёв М.В.\\Никифоров С.А.}
\chapter{Формализация понятий действия, задачи, метода, средства, навыка и технологии}
\chapauthortoc{Шункевич Д.В.\\Ковалёв М.В.\\Никифоров С.А.}
\label{chapter_actions}

\abstract{Аннотация к главе.}

\section{Глобальная предметная область и онтология воздействий, действий, методов, средств и технологий}

\subsection{Понятие метода (программы)}
\label{sec_actions_method_concept}

Формально, \textit{метод} -- это спецификация решения задачи какого-то класса \cite{Standard2021}, \cite{Tuzov1986}. В состав спецификации каждого класса задач входит описание способа \scnqq{привязки} метода к исходным данным конкретной задачи, решаемой с помощью этого метода.

\begin{SCn}
	\scnheader{метод}
	\scnidtf{программа}
	\scnidtf{описание того, как может быть выполнено любое или почти любое действие, принадлежащее соответствующему классу действий}
	\scnidtf{метод решения соответствующего класса задач, обеспечивающий решение любой или большинства задач
		указанного класса}
	\scnidtf{обобщенная спецификация решения задач соответствующего класса}
	\scnidtf{программа решения задач соответствующего класса, которая может быть как процедурной, так и декларативной (непроцедурной)}
	\scnidtf{знание о том, как можно решать задачи соответствующего класса}
	\scnsubset{знание}
	\scniselement{вид знаний}
	\scnidtf{способ}
	\scnsuperset{модель решения задач}
\end{SCn}

\subsection{Понятие класса методов. Общая классификация методов}
\label{sec_actions_method_class_concept}

Иногда целесообразным может считаться выделение некоторого подмножества методов (например, множества методов, с помощью которых решается определённая задача), тогда в таком случае для этих методов можно описать требования, которые они должны выполнять. Такие множества методов являются \textit{классами методов} некоторого я.п.м., которым ставится в соответствие конкретная \textit{модель решения задач}. Методы могут быть как \textit{процедурные}, так и \textit{непроцедурные} \cite{Turner2007}.

\begin{SCn}
	\scnheader{класс методов}
	\scnrelto{семейство подклассов}{метод}
	\scnidtf{множество методов, для которых можно унифицировать представление (спецификацию) этих методов}
	\scnidtf{множество всевозможных методов решения задач, имеющих общий язык представления этих методов}
	\scnidtf{множество методов, для которых задан язык представления этих методов}
	\scnhaselement{процедурный метод решения задач}
	\begin{scnindent}
		\scnsuperset{алгоритмический метод решения задач}
	\end{scnindent}
	\scnhaselement{непроцедурный метод решения задач}
	\begin{scnindent}
		\scnsuperset{логический метод решения задач}
		\scnsuperset{продукционный метод решения задач}
		\scnsuperset{функциональный метод решения задач}
		\begin{scnindent}
			\scnsuperset{искусственная нейронная сеть}
			\scnsuperset{генетический \scnqq{алгоритм}}
		\end{scnindent}
	\end{scnindent}
	\scnidtf{множество методов, которому ставится в соответствие отдельная модель решения задач}
\end{SCn}

Поскольку каждому методу соответствует обобщенная формулировка задач, решаемых с помощью этого метода, то каждому классу методов должен соответствовать не только определенный я.п.м, принадлежащих указанному \textit{классу методов}, но и определенный язык представления обобщенных формулировок задач для различных классов задач, решаемых с помощью методов, принадлежащих указанному классу методов.

Для процедурных и непроцедурных методов хоть и можно задать \textit{входные} и \textit{выходные параметры}, но нельзя задать общую денотационную семантику их логических элементов: для процедурных методов -- это операторы, для непроцедурных -- математические объекты предметной области.

\subsection{Понятие языка представления методов (языка программирования)}
\label{sec_actions_method_representation_language_concept}

Каждому конкретному классу методов взаимно однозначно соответствует я.п.м., принадлежащих этому (специфицируемому) классу методов. Таким образом, спецификация каждого класса методов сводится к спецификации соответствующего я.п.м., т.е. к описанию его синтаксической, денотационной семантики и операционной семантики. Примерами я.п.м. являются все языки программирования, которые в основном относятся к подклассу я.п.м. Но сейчас все большую актуальность приобретает необходимость создания эффективных формальных я.п.м. выполнения действий во внешней среде кибернетических систем. Без этого комплексная автоматизация \cite{Pospelov2021}, в частности, в промышленной сфере невозможна.

Под \textit{языком представления методов} будем подразумевать формальный язык, (1) знаковыми конструкциями которого являются соответствующие методы, для которых существуют общие правила построения и (2) общие правила соотнесения с теми сущностями и связями между ними, которые описываются этими методами.

С помощью я.п.м. формируются \textit{сообщения} (методы) для компьютера. Эти сообщения должны быть понятны (семантически корректны и целостны) компьютеру.

\begin{SCn}
	\scnheader{язык представления методов}
	\scnidtf{язык программирования}
	\scnsubset{язык представления знаний}
	\begin{scnindent}
		\scnsubset{формальный язык}
	\end{scnindent}
	\scnidtf{компьютерный язык}
	\scnidtf{формальный язык, (1) знаковыми конструкциями которого являются соответствующие методы, для которых существуют общие правила построения и (2) общие правила соотнесения с теми сущностями и связями между ними, которые описываются этими методами}
	\scnidtf{средство общения между человеком (пользователем) и компьютером (исполнителем)}
	\scnidtf{инструмент для производства программных услуг}
\end{SCn}

Метод принадлежит языку представления методов, если он является синтаксически корректным, синтаксически целостным, семантически корректным и семантически целостным методом заданного я.п.м. (!).

\begin{SCn}
	\scnheader{отношение, заданное на множестве языков представления методов\scnsupergroupsign}
	\scnidtf{отношение, область определения которого включает в себя множество всевозможных языков представления методов}
	\scnhaselement{метод заданного языка представления методов*}
	\scnhaselement{синтаксически корректный метод для заданного языка представления методов*}
	\begin{scnindent}
		\scnidtf{метод, не содержащий синтаксических ошибок для заданного языка представления методов*}
		\scnsubset{синтаксически корректная знаковая конструкция для заданного языка*}
	\end{scnindent}
	\scnhaselement{синтаксически целостный метод для заданного языка представления методов*}
	\begin{scnindent}
		\scnsubset{синтаксически целостная знаковая конструкция для заданного языка*}
	\end{scnindent}
	\scnhaselement{семантически корректный метод для заданного языка представления методов*}
	\begin{scnindent}
		\scnidtf{метод, не содержащий семантических ошибок для заданного языка представления методов*}
		\scnsubset{семантически корректная знаковая конструкция для заданного языка*}
	\end{scnindent}
	\scnhaselement{семантически целостный метод для заданного языка представления методов*}
	\begin{scnindent}
		\scnsubset{семантически целостная знаковая конструкция для заданного языка*}
		\scnidtf{метод заданного языка представления методов, содержащий достаточную информацию для установления его
			истинности*}
	\end{scnindent}
\end{SCn}

\begin{SCn}
	\scnheader{метод заданного языка представления методов*}
	\scnidtf{метод, принадлежащий заданному языку программирования*}
	\scnsubset{текст заданного языка*}
	\scnrelfrom{второй домен}{метод}
	\begin{scnreltoset}{объединение}
		\scnitem{\scnnonamednode}
		\begin{scnindent}
			\begin{scnreltoset}{объединение}
				\scnitem{синтаксически корректный метод для заданного языка представления методов*}
				\scnitem{синтаксически целостный метод для заданного языка представления методов*}
			\end{scnreltoset}
		\end{scnindent}
		\scnitem{\scnnonamednode}
		\begin{scnindent}
			\begin{scnreltoset}{объединение}
				\scnitem{семантически корректный метод для заданного языка представления методов*}
				\scnitem{синтаксически целостный метод для заданного языка представления методов*}
			\end{scnreltoset}
		\end{scnindent}
	\end{scnreltoset}
\end{SCn}

\subsection{Общая классификация языков представления методов}
\label{sec_actions_method_representation_language_classification}

Языки представления методов в современном информационном обществе различают по их парадигмам: \textit{процедурные}, \textit{функциональные}, \textit{логические}, \textit{объектно-ориентированные} и т. д. Таким, например, в методах процедурного я.п.м. решение задачи компьютером формируется в виде последовательности операторов, в методах функционального я.п.м. — указанием других методов. В логическом я.п.м. применяются высказывания, а в объектно-ориентированном — объекты.

\begin{SCn}
	\scnheader{язык представления методов}
	\scnsuperset{язык представления методов общего назначения}
	\begin{scnindent}
		\scnidtf{язык программирования общего назначения}
	\end{scnindent}
	\scnsuperset{предметно-ориентированный язык представления методов}
	\begin{scnindent}
		\scnidtf{предметно-ориентированный язык программирования}
	\end{scnindent}
	\scnrelfrom{разбиение}{парадигма языка представления методов\scnsupergroupsign}
	\begin{scnindent}
		\begin{scneqtoset}
			\scnitem{процедурный язык представления методов}
			\scnitem{непроцедурный язык представления методов}
		\end{scneqtoset}
	\end{scnindent}
\end{SCn}

\textit{Процедурные языки представления методов} задают вычисления как последовательность операторов (команд).
Они ориентированы на компьютеры с архитектурой фон Неймана. Основные понятия процедурных я.п.м. тесно связаны с компонентами компьютера:
\begin{textitemize}
	\item переменными различных типов, которые моделируют ячейки памяти компьютера;
	\item операторами присваивания, которые моделируют пересылки данных между участками памяти;
	\item повторений действий в форме итерации, которые моделируют хранение информации в смежных ячейках памяти;
	\item и другое.
\end{textitemize}

\begin{SCn}
	\scnheader{процедурный язык представления методов}
	\scnidtf{императивный язык представления методов}
	\scnsuperset{структурный язык представления методов}
	\begin{scnindent}
		\begin{scnhaselementrolelist}{пример}
			\scnitem{Fortran}
			\scnitem{Си}
			\scnitem{Pascal}
		\end{scnhaselementrolelist}
	\end{scnindent}
	\scnsuperset{объектно-ориентированный язык представления методов}
	\begin{scnindent}
		\begin{scnhaselementrolelist}{пример}
			\scnitem{Java}
			\scnitem{Smalltalk}
			\scnitem{HTML}
		\end{scnhaselementrolelist}
		\scnsuperset{аспектно-ориентированный язык представления методов}
	\end{scnindent}
	\scnsuperset{скриптовый язык представления методов}
	\begin{scnindent}
		\scnidtf{склеивающий язык представления методов}
	\end{scnindent}
\end{SCn}

\textit{Непроцедурные языки представления методов}, в отличие от процедурных, задают вычисления как последовательность связанных между собой объектов. Основные понятия непроцедурных я.п.м. обычно не связаны с компонентами компьютера.

\begin{SCn}
	\scnheader{непроцедурный язык представления методов}
	\scnidtf{декларативный язык представления методов}
	\scnsuperset{логический язык представления методов}
	\begin{scnindent}
		\begin{scnhaselementrolelist}{пример}
			\scnitem{Prolog}
		\end{scnhaselementrolelist}
	\end{scnindent}
	\scnsuperset{продукционный язык представления методов}
	\scnsuperset{функциональный язык представления методов}
	\begin{scnindent}
		\scnidtf{аппликативный язык представления методов}
		\begin{scnhaselementrolelist}{пример}
			\scnitem{LISP}
		\end{scnhaselementrolelist}
	\end{scnindent}
\end{SCn}

\section{Локальные предметные области и онтологии действий}

%%%%%%%%%%%%%%%%%%%%%%%%% referenc.tex %%%%%%%%%%%%%%%%%%%%%%%%%%%%%%
% sample references
% %
% Use this file as a template for your own input.
%
%%%%%%%%%%%%%%%%%%%%%%%% Springer-Verlag %%%%%%%%%%%%%%%%%%%%%%%%%%
%
% BibTeX users please use
% \bibliographystyle{}
% \bibliography{}
%
\biblstarthook{In view of the parallel print and (chapter-wise) online publication of your book at \url{www.springerlink.com} it has been decided that -- as a genreral rule --  references should be sorted chapter-wise and placed at the end of the individual chapters. However, upon agreement with your contact at Springer you may list your references in a single seperate chapter at the end of your book. Deactivate the class option \texttt{sectrefs} and the \texttt{thebibliography} environment will be put out as a chapter of its own.\\\indent
References may be \textit{cited} in the text either by number (preferred) or by author/year.\footnote{Make sure that all references from the list are cited in the text. Those not cited should be moved to a separate \textit{Further Reading} section or chapter.} If the citatiion in the text is numbered, the reference list should be arranged in ascending order. If the citation in the text is author/year, the reference list should be \textit{sorted} alphabetically and if there are several works by the same author, the following order should be used:
\begin{enumerate}
\item all works by the author alone, ordered chronologically by year of publication
\item all works by the author with a coauthor, ordered alphabetically by coauthor
\item all works by the author with several coauthors, ordered chronologically by year of publication.
\end{enumerate}
The \textit{styling} of references\footnote{Always use the standard abbreviation of a journal's name according to the ISSN \textit{List of Title Word Abbreviations}, see \url{http://www.issn.org/en/node/344}} depends on the subject of your book:
\begin{itemize}
\item The \textit{two} recommended styles for references in books on \textit{mathematical, physical, statistical and computer sciences} are depicted in ~\cite{science-contrib, science-online, science-mono, science-journal, science-DOI} and ~\cite{phys-online, phys-mono, phys-journal, phys-DOI, phys-contrib}.
\item Examples of the most commonly used reference style in books on \textit{Psychology, Social Sciences} are~\cite{psysoc-mono, psysoc-online,psysoc-journal, psysoc-contrib, psysoc-DOI}.
\item Examples for references in books on \textit{Humanities, Linguistics, Philosophy} are~\cite{humlinphil-journal, humlinphil-contrib, humlinphil-mono, humlinphil-online, humlinphil-DOI}.
\item Examples of the basic Springer style used in publications on a wide range of subjects such as \textit{Computer Science, Economics, Engineering, Geosciences, Life Sciences, Medicine, Biomedicine} are ~\cite{basic-contrib, basic-online, basic-journal, basic-DOI, basic-mono}. 
\end{itemize}
}

\begin{thebibliography}{99.}%
% and use \bibitem to create references.
%
% Use the following syntax and markup for your references if 
% the subject of your book is from the field 
% "Mathematics, Physics, Statistics, Computer Science"
%
% Contribution 
\bibitem{science-contrib} Broy, M.: Software engineering --- from auxiliary to key technologies. In: Broy, M., Dener, E. (eds.) Software Pioneers, pp. 10-13. Springer, Heidelberg (2002)
%
% Online Document
\bibitem{science-online} Dod, J.: Effective substances. In: The Dictionary of Substances and Their Effects. Royal Society of Chemistry (1999) Available via DIALOG. \\
\url{http://www.rsc.org/dose/title of subordinate document. Cited 15 Jan 1999}
%
% Monograph
\bibitem{science-mono} Geddes, K.O., Czapor, S.R., Labahn, G.: Algorithms for Computer Algebra. Kluwer, Boston (1992) 
%
% Journal article
\bibitem{science-journal} Hamburger, C.: Quasimonotonicity, regularity and duality for nonlinear systems of partial differential equations. Ann. Mat. Pura. Appl. \textbf{169}, 321--354 (1995)
%
% Journal article by DOI
\bibitem{science-DOI} Slifka, M.K., Whitton, J.L.: Clinical implications of dysregulated cytokine production. J. Mol. Med. (2000) doi: 10.1007/s001090000086 
%
\bigskip

% Use the following (APS) syntax and markup for your references if 
% the subject of your book is from the field 
% "Mathematics, Physics, Statistics, Computer Science"
%
% Online Document
\bibitem{phys-online} J. Dod, in \textit{The Dictionary of Substances and Their Effects}, Royal Society of Chemistry. (Available via DIALOG, 1999), 
\url{http://www.rsc.org/dose/title of subordinate document. Cited 15 Jan 1999}
%
% Monograph
\bibitem{phys-mono} H. Ibach, H. L\"uth, \textit{Solid-State Physics}, 2nd edn. (Springer, New York, 1996), pp. 45-56 
%
% Journal article
\bibitem{phys-journal} S. Preuss, A. Demchuk Jr., M. Stuke, Appl. Phys. A \textbf{61}
%
% Journal article by DOI
\bibitem{phys-DOI} M.K. Slifka, J.L. Whitton, J. Mol. Med., doi: 10.1007/s001090000086
%
% Contribution 
\bibitem{phys-contrib} S.E. Smith, in \textit{Neuromuscular Junction}, ed. by E. Zaimis. Handbook of Experimental Pharmacology, vol 42 (Springer, Heidelberg, 1976), p. 593
%
\bigskip
%
% Use the following syntax and markup for your references if 
% the subject of your book is from the field 
% "Psychology, Social Sciences"
%
%
% Monograph
\bibitem{psysoc-mono} Calfee, R.~C., \& Valencia, R.~R. (1991). \textit{APA guide to preparing manuscripts for journal publication.} Washington, DC: American Psychological Association.
%
% Online Document
\bibitem{psysoc-online} Dod, J. (1999). Effective substances. In: The dictionary of substances and their effects. Royal Society of Chemistry. Available via DIALOG. \\
\url{http://www.rsc.org/dose/Effective substances.} Cited 15 Jan 1999.
%
% Journal article
\bibitem{psysoc-journal} Harris, M., Karper, E., Stacks, G., Hoffman, D., DeNiro, R., Cruz, P., et al. (2001). Writing labs and the Hollywood connection. \textit{J Film} Writing, 44(3), 213--245.
%
% Contribution 
\bibitem{psysoc-contrib} O'Neil, J.~M., \& Egan, J. (1992). Men's and women's gender role journeys: Metaphor for healing, transition, and transformation. In B.~R. Wainrig (Ed.), \textit{Gender issues across the life cycle} (pp. 107--123). New York: Springer.
%
% Journal article by DOI
\bibitem{psysoc-DOI}Kreger, M., Brindis, C.D., Manuel, D.M., Sassoubre, L. (2007). Lessons learned in systems change initiatives: benchmarks and indicators. \textit{American Journal of Community Psychology}, doi: 10.1007/s10464-007-9108-14.
%
%
% Use the following syntax and markup for your references if 
% the subject of your book is from the field 
% "Humanities, Linguistics, Philosophy"
%
\bigskip
%
% Journal article
\bibitem{humlinphil-journal} Alber John, Daniel C. O'Connell, and Sabine Kowal. 2002. Personal perspective in TV interviews. \textit{Pragmatics} 12:257--271
%
% Contribution 
\bibitem{humlinphil-contrib} Cameron, Deborah. 1997. Theoretical debates in feminist linguistics: Questions of sex and gender. In \textit{Gender and discourse}, ed. Ruth Wodak, 99--119. London: Sage Publications.
%
% Monograph
\bibitem{humlinphil-mono} Cameron, Deborah. 1985. \textit{Feminism and linguistic theory.} New York: St. Martin's Press.
%
% Online Document
\bibitem{humlinphil-online} Dod, Jake. 1999. Effective substances. In: The dictionary of substances and their effects. Royal Society of Chemistry. Available via DIALOG. \\
http://www.rsc.org/dose/title of subordinate document. Cited 15 Jan 1999
%
% Journal article by DOI
\bibitem{humlinphil-DOI} Suleiman, Camelia, Daniel C. O'Connell, and Sabine Kowal. 2002. `If you and I, if we, in this later day, lose that sacred fire...': Perspective in political interviews. \textit{Journal of Psycholinguistic Research}. doi: 10.1023/A:1015592129296.
%
%
%
\bigskip
%
%
% Use the following syntax and markup for your references if 
% the subject of your book is from the field 
% "Computer Science, Economics, Engineering, Geosciences, Life Sciences"
%
%
% Contribution 
\bibitem{basic-contrib} Brown B, Aaron M (2001) The politics of nature. In: Smith J (ed) The rise of modern genomics, 3rd edn. Wiley, New York 
%
% Online Document
\bibitem{basic-online} Dod J (1999) Effective Substances. In: The dictionary of substances and their effects. Royal Society of Chemistry. Available via DIALOG. \\
\url{http://www.rsc.org/dose/title of subordinate document. Cited 15 Jan 1999}
%
% Journal article by DOI
\bibitem{basic-DOI} Slifka MK, Whitton JL (2000) Clinical implications of dysregulated cytokine production. J Mol Med, doi: 10.1007/s001090000086
%
% Journal article
\bibitem{basic-journal} Smith J, Jones M Jr, Houghton L et al (1999) Future of health insurance. N Engl J Med 965:325--329
%
% Monograph
\bibitem{basic-mono} South J, Blass B (2001) The future of modern genomics. Blackwell, London 
%
\end{thebibliography}
