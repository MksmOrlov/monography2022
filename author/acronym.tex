%%%%%%%%%%%%%%%%%%%%%%acronym.tex%%%%%%%%%%%%%%%%%%%%%%%%%%%%%%%%%%%%%%%%%
% sample list of acronyms
%
% Use this file as a template for your own input.
%
%%%%%%%%%%%%%%%%%%%%%%%% Springer %%%%%%%%%%%%%%%%%%%%%%%%%%

\extrachap{Список сокращений}

\begin{description}
\item[ACL] англ. Agent Communication Language. Язык взаимодействия агентов, предложенный FIPA в качестве стандарта
\item[FIPA] англ. Foundation for Intelligent Physical Agents. Организация, осуществляющая разработку и продвижение стандартов в области многоагентных систем
\item[GPS] англ. General Problem Solver. Компьютерная программа, созданная в 1959 г. и предназначенная для работы в качестве универсальной машины для решения задач, сформулированных на языке хорновских дизъюнктов
\item[IACPaaS] англ. Intelligent Applications, Control and Platform as a Service. Исследовательская облачная платформа, объединяющая различные модели парадигмы облачных вычислений
\item[IMS] англ. Intelligent MetaSystem. Интеллектуальная метасистема поддержки проектирования интеллектуальных систем
\item[KIF] англ. Knowledge Interchange Language. Компьютерноориентированный язык для обмена знаниями между различными компьютерными программами
\item[KQML] англ. Knowledge Query and Manipulation Language. Язык взаимодействия между программными агентами и системами, основанными на знаниях
\item[OSTIS] англ. Open Semantic Technology for Intelligent Systems. Открытая семантическая технология проектирования интеллектуальных систем
\item[OWL] англ. Web Ontology Language. Язык описания онтологий для семантической паутины
\item[QA3] англ. Question Answer system ver. 3. Вопросно-ответная дедуктивная система, созданная в 1969 г. на языке LISP
\item[RDF] англ. Resource Description Framework. Разработанная Консорциумом Всемирной паутины модель для представления данных
\item[SCg-код] англ. Semantic Code graphic. Графический нелинейный вариант визуализации текстов SC-кода
\item[SCn-код] англ. Semantic Code natural. Гипертекстовый вариант визуализации текстов SC-кода
\item[SCP] англ. Semantic Code Programming. Графовый процедурный язык программирования, построенный на базе SC-кода
\item[SC-код] англ. Semantic Code. Универсальный базовый способ смыслового представления знаний в виде семантических сетей с базовой теоретикомножественной интерпретацией
\item[SPARQL] англ. SPARQL Protocol and RDF Query Language. Язык запросов к данным (является рекомендацией консорциума W3C и одной из технологий семантической паутины), представленным по модели RDF, а также протокол для передачи этих запросов и ответов на них
\item[SQL] англ. Structured Query Language --- ''язык структурированных запросов``. Универсальный язык запросов, применяемый для создания, модификации и управления данными в реляционных базах данных
\item[STRIPS] англ. Stanford Research Institute Problem Solver. Планирующая система, использующая декларативно-процедуральное представление знаний в сочетании с эвристическим поиском, создана в 1971 г.
\item[W3C] англ. World Wide Web Consortium, W3C. Консорциум Всемирной паутины, организация, разрабатывающая и внедряющая технологические стандарты для Всемирной паутины
\item[ГРЗ] гибридный решатель задач
\item[ИСС] интеллектуальная справочная система
\item[НИЦ ЭВТ] московский Научно-исследовательский центр электронной вычислительной техники
\item[ППР] Программа принятия решений, планирующая система для интеллектуального робота, созданная в 1977 г. под руководством В. П. Гладуна
\item[ПРИЗ] Пакет прикладных инженерных задач, система программирования, созданная под руководством Э. Х. Тыугу в 1970–1976 гг
\item[РЗ] решатель задач
\item[УСК] универсальный семантический код, разработанный В. В. Мартыновым
\end{description}
