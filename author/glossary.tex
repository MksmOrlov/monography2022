\begin{partbacktext}
\part*{Используемые сокращения и предметный указатель OSTIS}
\markboth{Используемые сокращения и предметный указатель OSTIS}{Используемые сокращения и предметный указатель OSTIS}
\label{part_glossary}
\addcontentsline{toc}{part}{Используемые сокращения и  предметный указатель OSTIS}

Предлагаемый вашему вниманию предметный указатель представляет собой алфавитный перечень всех основных терминов, используемых в данной монографии, которые \myuline{взаимно-однозначно} соответствуют элементам рафинированной семантической сети, представляющей собой базу знаний \textit{Метасистемы OSTIS}, основная часть которой семантически эквивалентна тексту данной монографии.

В рамках внутреннего языка \textit{ostis-систем} (в рамках \textit{SC-кода}) указанные основные термины называются \textit{основными sc-идентификаторами} (основными внешними идентификаторами sc-элементов --- элементов рафинированных семантических сетей).

В данный предметный указатель включаются как русскоязычные, так и англоязычные термины (если нет аналогичного русскоязычного), непереводимые интернациональные названия (например, названия различных программных систем, такие как \textit{Neo4j}, \textit{MySQL} и т.д.), а также различные используемые сокращения.

При этом для каждого \myuline{основного} русскоязычного термина указывается синонимичный ему \myuline{основной} англоязычный термин, например:

\begin{SCn}

\scnheader{sc-идентификатор}
\scnidtf{строка символов или пиктограмма, взаимно однозначно представляющая соответствующий sc-элемент, хранимый в sc-памяти}
\scnidtf{внешний идентификатор sc-элемента}

\begin{scnset}
\vspace{-\baselineskip}
\scnheader{sc-элемент}
\scnidtf{\textit{sc-element}}
\begin{scnindent}
	\scniselement{основной англоязычный sc-идентификатор}
	\begin{scnindent}
		\scnidtf{основной sc-идентификатор для англоязычного режима}
	\end{scnindent}
\end{scnindent}
\scnidtf{\textit{sc-элемент}}
\begin{scnindent}
	\scniselement{основной русскоязычный sc-идентификатор}
	\begin{scnindent}
		\scnidtf{основной sc-идентификатор для русскоязычного режима}
	\end{scnindent}
\end{scnindent}
\end{scnset}
\scnexplanation{Для каждого ключевого термина в предметном указателе указывается его основной русскоязычный идентификатор и основной англоязычный идентификатор. При этом по умолчанию считается, что русскоязычный термин является \myuline{основным} русскоязычным идентификатором, а соответствующий ему англоязычный --- \myuline{основным} англоязычным идентификатором.}
\end{SCn}

\bigskip

Для каждого неосновного, но часто используемого русскоязычного термина указывается синонимичный ему основной русскоязычный термин. Кроме того, для каждого основного русскоязычного термина указывается ссылка на соответствующую главу, параграф или пункт монографии, где сущность, обозначаемая этим термином, является ключевым знаком.
\end{partbacktext}

\begin{SCn}

\scnheader{Абстрактная scp-машина}
\scnidtf{\textit{Abstract scp-machine}}
\begin{scnreltolist}{ключевой знак}
	\scnitem{\ref{sec_ps_scp}~\nameref{sec_ps_scp}}
\end{scnreltolist}

\scnheader{абстрактный sc-агент}
\scnidtf{\textit{abstract sc-agent}}
\begin{scnreltolist}{ключевой знак}
	\scnitem{\ref{sec_ps_agents}~\nameref{sec_ps_agents}}
\end{scnreltolist}

\scnheader{абстрактный sc-агент, не реализуемый на Языке SCP}	
\scnidtf{\textit{abstract sc-agent, non-implementable in the SCP Language}}
\begin{scnreltolist}{ключевой знак}
	\scnitem{\ref{sec_ps_agents}~\nameref{sec_ps_agents}}
\end{scnreltolist}

\scnheader{абстрактный sc-агент, реализуемый на Языке SCP}
\scnidtf{\textit{abstract sc-agent, implementable in the SCP Language}}
\begin{scnreltolist}{ключевой знак}
	\scnitem{\ref{sec_ps_agents}~\nameref{sec_ps_agents}}
\end{scnreltolist}

\scnheader{адаптивный интерфейс}
\scnidtf{\textit{adaptive interface}}
\begin{scnreltolist}{ключевое понятие}
    \scnitem{Глава \ref{chapter_interfaces}~\nameref{chapter_interfaces}}
\end{scnreltolist}

\scnheader{ассоциативный семантический компьютер}
\scnidtf{\textit{associative semantic computer}}
\begin{scnreltolist}{ключевой знак}
	\scnitem{Глава \ref{chapter_computers}~\nameref{chapter_computers}}
	\scnitem{Глава \ref{chapter_interpreter}~\nameref{chapter_interpreter}}
\end{scnreltolist}

\scnheader{атомарный абстрактный sc-агент}
\scnidtf{\textit{atomic abstract sc-agent}}
\begin{scnreltolist}{ключевой знак}
	\scnitem{\ref{sec_ps_agents}~\nameref{sec_ps_agents}}
\end{scnreltolist}

\scnheader{базовая ostis-платформа}
\scnidtf{\textit{basic ostis-platform}}
\begin{scnreltolist}{ключевой знак}
	\scnitem{Глава \ref{chapter_interpreter}~\nameref{chapter_interpreter}}
\end{scnreltolist}

\scnheader{библиотека многократно используемых компонентов ostis-систем}
\scnidtftext{сокращение основного sc-идентификатора}{библиотека м.и.к. ostis-систем}
\scnidtf{\textit{library of ostis-systems reusable components}}
\begin{scnreltolist}{ключевой знак}
	\scnitem{Глава \ref{chapter_library}~\nameref{chapter_library}}
\end{scnreltolist}

\scnheader{Библиотека OSTIS}
\scnidtf{\textit{OSTIS library}}
\begin{scnreltolist}{ключевой знак}
	\scnitem{Глава \ref{chapter_library}~\nameref{chapter_library}}
\end{scnreltolist}

\scnheader{блокировка*}
\scnidtf{\textit{lock*}}
\begin{scnreltolist}{ключевой знак}
	\scnitem{\ref{sec_ps_sync}~\nameref{sec_ps_sync}}
\end{scnreltolist}

\scnheader{действие в sc-памяти}
\scnidtf{\textit{action in sc-memory}}
\begin{scnreltolist}{ключевой знак}
	\scnitem{\ref{sec_ps_actions}~\nameref{sec_ps_actions}}
\end{scnreltolist}

\scnheader{действие в sc-памяти, инициируемое вопросом}
\scnidtf{\textit{action in sc-memory, initiated by a question}}
\begin{scnreltolist}{ключевой знак}
		\scnitem{\ref{sec_ps_actions}~\nameref{sec_ps_actions}}
\end{scnreltolist}

\scnheader{действие редактирования базы знаний}
\scnidtf{\textit{action of knowledge base editing}}
\begin{scnreltolist}{ключевой знак}
	\scnitem{\ref{sec_ps_actions}~\nameref{sec_ps_actions}}
\end{scnreltolist}

\scnheader{денотационная семантика метода*}
\scnidtf{\textit{denotational semantics of method*}}
\begin{scnreltolist}{ключевое отношение}
	\scnitem{Глава \ref{chapter_programs}~\nameref{chapter_programs}}
\end{scnreltolist}

\scnheader{денотационная семантика языка представления методов*}
\scnidtf{\textit{denotational semantics of method representation language*}}
\begin{scnreltolist}{ключевое отношение}
	\scnitem{Глава \ref{chapter_programs}~\nameref{chapter_programs}}
\end{scnreltolist}

\scnheader{задача, решаемая в sc-памяти}
\scnidtf{\textit{problem, solved in sc-memory}}
\begin{scnreltolist}{ключевой знак}
	\scnitem{\ref{sec_ps_actions}~\nameref{sec_ps_actions}}
\end{scnreltolist}

\scnheader{интеллектуальная компьютерная система нового поколения}
\scnidtftext{сокращение основного sc-идентификатора}{и.к.с. нового поколения}
\scntext{определение}{интеллектуальная компьютерная система, обладающая:
\begin{scnitemize}
	\item высоким уровнем самообучаемости, обеспечивающим высокий уровень автоматизации собственной эволюции и, соответственно, высокие темпы этой эволюции;
	\item высоким уровнем интероперабельности.
\end{scnitemize}}
\scnsubset{самообучаемая интеллектуальная компьютерная система}
\scnsubset{интероперабельная интеллектуальная компьютерная система}
\begin{scnreltolist}{ключевое понятие}
	\scnitem{Глава~\ref{chapter_new_generation_systems}~\nameref{chapter_new_generation_systems}}
\end{scnreltolist}

\scnheader{инициируемое пользовательским интерфейсом действие*}
\scnidtf{\textit{action initiated by the user interface}}
\begin{scnreltolist}{ключевое отношение}
    \scnitem{Глава \ref{chapter_interfaces}~\nameref{chapter_interfaces}}
\end{scnreltolist}

\scnheader{интеллектуальный интерфейс}
\scnidtf{\textit{intelligent interface}}
\begin{scnreltolist}{ключевое понятие}
    \scnitem{Глава \ref{chapter_interfaces}~\nameref{chapter_interfaces}}
\end{scnreltolist}

\scnheader{интерпретатор пользовательских действий}
\scnidtf{\textit{interpreter of user actions}}
\begin{scnreltolist}{ключевое понятие}
    \scnitem{Глава \ref{chapter_interfaces}~\nameref{chapter_interfaces}}
\end{scnreltolist}

\scnheader{интерпретатор sc-моделей пользовательских интерфейсов}
\scnidtf{\textit{interpreter of the sc-models of the user interfaces}}
\begin{scnreltolist}{ключевое понятие}
    \scnitem{Глава \ref{chapter_interfaces}~\nameref{chapter_interfaces}}
\end{scnreltolist}

\scnheader{интерфейс}
\scnidtf{\textit{interface}}
\begin{scnreltolist}{ключевое понятие}
    \scnitem{Глава \ref{chapter_interfaces}~\nameref{chapter_interfaces}}
\end{scnreltolist}

\scnheader{интерфейсное действие пользователя}
\scnidtf{\textit{user interface action}}
\begin{scnreltolist}{ключевое понятие}
    \scnitem{Глава \ref{chapter_interfaces}~\nameref{chapter_interfaces}}
\end{scnreltolist}

\scnheader{интерфейс ostis-систем}
\scnidtf{\textit{интерфейс интеллектуальных компьютерных систем нового поколения}}
\scnidtf{\textit{ostis-system interface}}
\begin{scnreltolist}{ключевое понятие}
    \scnitem{Глава \ref{chapter_interfaces}~\nameref{chapter_interfaces}}
\end{scnreltolist}

\scnheader{класс логически атомарных действий}
\scnidtf{\textit{class of logically atomic actions}}
\begin{scnreltolist}{ключевой знак}
	\scnitem{\ref{sec_ps_actions}~\nameref{sec_ps_actions}}
\end{scnreltolist}

\scnheader{компонент пользовательского интерфейса}
\scnidtf{\textit{user interface component}}
\begin{scnreltolist}{ключевое понятие}
    \scnitem{Глава \ref{chapter_interfaces}~\nameref{chapter_interfaces}}
\end{scnreltolist}

\scnheader{компонентное проектирование}
\scnidtf{\textit{component design}}
\begin{scnreltolist}{ключевой знак}
	\scnitem{Глава \ref{chapter_library}~\nameref{chapter_library}}
\end{scnreltolist}

\scnheader{компонентное проектирование интеллектуальных систем}
\scnidtf{\textit{intelligent systems component design}}
\begin{scnreltolist}{ключевой знак}
	\scnitem{Глава \ref{chapter_library}~\nameref{chapter_library}}
\end{scnreltolist}

\scnheader{компьютерное зрение}	
\scnidtf{\textit{computer vision}}
\begin{scnreltolist}{ключевое понятие}
	\scnitem{\ref{sec_3d_models_computervision}~\nameref{sec_3d_models_computervision}}
\end{scnreltolist}

\scnheader{локальный признак изображения}	
\scnidtf{\textit{local image feature}}
\begin{scnreltolist}{ключевое понятие}
	\scnitem{\ref{sec_3d_models_computervision}~\nameref{sec_3d_models_computervision}}
\end{scnreltolist}

\scnheader{машина обработки знаний}
\scnidtf{\textit{knowledge processing machine}}
\begin{scnreltolist}{ключевой знак}
	\scnitem{Глава \ref{chapter_situation_management}~\nameref{chapter_situation_management}}
\end{scnreltolist}

\scnheader{метод}
\scnidtf{\textit{программа}}
\scnidtf{\textit{method}}
\begin{scnreltolist}{ключевое понятие}
	\scnitem{Глава \ref{chapter_programs}~\nameref{chapter_programs}}
\end{scnreltolist}

\scnheader{метаметод}
\scnidtf{\textit{метапрограмма}}
\scnidtf{\textit{meta-method}}
\begin{scnreltolist}{ключевое понятие}
	\scnitem{Глава \ref{chapter_programs}~\nameref{chapter_programs}}
\end{scnreltolist}

\scnheader{менеджер многократно используемых компонентов ostis-систем}
\scnidtftext{сокращение основного sc-идентификатора}{менеджер компонентов}
\scnidtf{\textit{ostis-systems reusable component manager}}
\scnidtf{\textit{sc-component-manager}}
\begin{scnreltolist}{ключевой знак}
	\scnitem{Глава \ref{chapter_library}~\nameref{chapter_library}}
\end{scnreltolist}

\scnheader{материнская ostis-система}
\scnidtf{\textit{maternal ostis-system}}
\begin{scnreltolist}{ключевой знак}
	\scnitem{\ref{ostis_library_section}~\nameref{ostis_library_section}}
\end{scnreltolist}

\scnheader{многократно используемый компонент ostis-систем}
\scnidtftext{сокращение основного sc-идентификатора}{м.и.к. ostis-систем}
\scnidtf{\textit{ostis-systems reusable component}}
\begin{scnreltolist}{ключевой знак}
	\scnitem{Глава \ref{chapter_library}~\nameref{chapter_library}}
\end{scnreltolist}

\scnheader{минимальная конфигурация ostis-системы}
\scnidtf{\textit{ostis-system minimal configuration}}
\begin{scnreltolist}{ключевой знак}
	\scnitem{Глава \ref{chapter_interpreter}~\nameref{chapter_interpreter}}
\end{scnreltolist}

\scnheader{мультимодальный интерфейс}
\scnidtf{\textit{multimodal interface}}
\begin{scnreltolist}{ключевое понятие}
    \scnitem{Глава \ref{chapter_interfaces}~\nameref{chapter_interfaces}}
\end{scnreltolist}

\scnheader{неатомарный абстрактный sc-агент}
\scnidtf{\textit{non-atomic abstract sc-agent}}
\begin{scnreltolist}{ключевой знак}
	\scnitem{\ref{sec_ps_agents}~\nameref{sec_ps_agents}}
\end{scnreltolist}

\scnheader{операционная семантика метода*}
\scnidtf{\textit{operational semantics of method*}}
\begin{scnreltolist}{ключевое отношение}
	\scnitem{Глава \ref{chapter_programs}~\nameref{chapter_programs}}
\end{scnreltolist}

\scnheader{операционная семантика языка представления методов*}
\scnidtf{\textit{operational semantics of method representation language*}}
\begin{scnreltolist}{ключевое отношение}
	\scnitem{Глава \ref{chapter_programs}~\nameref{chapter_programs}}
\end{scnreltolist}

\scnheader{оптическая система компьютерного зрения}	
\scnidtf{\textit{optical computer vision system}}
\begin{scnreltolist}{ключевое понятие}
	\scnitem{\ref{sec_3d_models_computervision}~\nameref{sec_3d_models_computervision}}
\end{scnreltolist}

\scnheader{пакетный менеджер}
\scnidtf{\textit{packet manager}}
\begin{scnreltolist}{ключевой знак}
	\scnitem{\ref{ostis_library_analysis}~\nameref{ostis_library_analysis}}
\end{scnreltolist}

\scnheader{параметр scp-программы\scnrolesign}
\scnidtf{\textit{scp-program parameter\scnrolesign}}
\begin{scnreltolist}{ключевой знак}
	\scnitem{\ref{sec_ps_scp}~\nameref{sec_ps_scp}}
\end{scnreltolist}

\scnheader{планируемая блокировка*}
\scnidtf{\textit{planned lock*}}
\begin{scnreltolist}{ключевой знак}
	\scnitem{\ref{sec_ps_sync}~\nameref{sec_ps_sync}}
\end{scnreltolist}

\scnheader{пользовательский интерфейс}
\scnidtf{\textit{user interface}}
\begin{scnreltolist}{ключевое понятие}
    \scnitem{Глава \ref{chapter_interfaces}~\nameref{chapter_interfaces}}
\end{scnreltolist}

\scnheader{приоритет блокировки*}
\scnidtf{\textit{lock priority*}}
\begin{scnreltolist}{ключевой знак}
	\scnitem{\ref{sec_ps_sync}~\nameref{sec_ps_sync}}
\end{scnreltolist}

\scnheader{программный вариант ostis-платформы}
\scnidtf{\textit{software version of ostis-platform}}
\begin{scnreltolist}{ключевой знак}
	\scnitem{Глава \ref{chapter_interpreter}~\nameref{chapter_interpreter}}
\end{scnreltolist}

\scnheader{программный интерфейс}
\scnidtf{\textit{application interface}}
\scnidtf{\textit{API}}
\begin{scnreltolist}{ключевое понятие}
    \scnitem{Глава \ref{chapter_interfaces}~\nameref{chapter_interfaces}}
\end{scnreltolist}

\scnheader{расширенная ostis-платформа}
\scnidtf{\textit{extended ostis-platform}}
\begin{scnreltolist}{ключевой знак}
	\scnitem{Глава \ref{chapter_interpreter}~\nameref{chapter_interpreter}}
\end{scnreltolist}

\scnheader{решатель задач пользовательского интерфейса ostis-систем}
\scnidtf{\textit{problem solver of the ostis-system user interface}}
\begin{scnreltolist}{ключевое понятие}
    \scnitem{Глава \ref{chapter_interfaces}~\nameref{chapter_interfaces}}
\end{scnreltolist}

\scnheader{решатель задач ostis-системы}
\scnidtf{\textit{problem solver of ostis-system}}
\begin{scnreltolist}{ключевой знак}
	\scnitem{Глава \ref{chapter_situation_management}~\nameref{chapter_situation_management}}
\end{scnreltolist}

\scnheader{Семантическая теория программ для ostis-систем}
\scnidtf{\textit{Semantic program theory for ostis-systems}}
\begin{scnreltolist}{ключевой знак}
	\scnitem{Глава \ref{chapter_programs}~\nameref{chapter_programs}}
\end{scnreltolist}

\scnheader{синтаксис метода*}
\scnidtf{\textit{method syntax*}}
\begin{scnreltolist}{ключевое отношение}
	\scnitem{Глава \ref{chapter_programs}~\nameref{chapter_programs}}
\end{scnreltolist}

\scnheader{система локального позиционирования}	
\scnidtf{\textit{real-time locating system}}
\begin{scnreltolist}{ключевое понятие}
	\scnitem{\ref{sec_3d_models_positioning}~\nameref{sec_3d_models_positioning}}
\end{scnreltolist}

\scnheader{сообщение}
\scnidtf{\textit{message}}
\begin{scnreltolist}{ключевое понятие}
    \scnitem{Глава \ref{chapter_interfaces}~\nameref{chapter_interfaces}}
\end{scnreltolist}

\scnheader{специализированная ostis-платформа}
\scnidtf{\textit{specialized ostis-platform}}
\begin{scnreltolist}{ключевой знак}
	\scnitem{Глава \ref{chapter_interpreter}~\nameref{chapter_interpreter}}
\end{scnreltolist}

\scnheader{спецификация метода*}
\scnidtf{\textit{specification of method*}}
\begin{scnreltolist}{ключевое отношение}
	\scnitem{Глава \ref{chapter_programs}~\nameref{chapter_programs}}
\end{scnreltolist}

\scnheader{спецификация языка представления методов*}
\scnidtf{\textit{specification of method representation language*}}
\begin{scnreltolist}{ключевое отношение}
	\scnitem{Глава \ref{chapter_programs}~\nameref{chapter_programs}}
\end{scnreltolist}

\scnheader{сцена в трехмерном представлении}	
\scnidtf{\textit{scene in 3D representation}}
\begin{scnreltolist}{ключевое понятие}
	\scnitem{\ref{sec_3d_models_semantics}~\nameref{sec_3d_models_semantics}}
\end{scnreltolist}

\scnheader{технология проектирования интеллектуальных систем}
\scnidtf{\textit{intelligent systems design technology}}
\begin{scnreltolist}{ключевой знак}
	\scnitem{Глава \ref{chapter_library}~\nameref{chapter_library}}
\end{scnreltolist}

\scnheader{тип блокировки}
\scnidtf{\textit{lock type}}
\begin{scnreltolist}{ключевой знак}
	\scnitem{\ref{sec_ps_sync}~\nameref{sec_ps_sync}}
\end{scnreltolist}

\scnheader{транзакция в sc-памяти}	
\scnidtf{\textit{transaction in sc-memory}}
\begin{scnreltolist}{ключевой знак}
	\scnitem{\ref{sec_ps_sync}~\nameref{sec_ps_sync}}
\end{scnreltolist}

\scnheader{трехмерная модель объекта}	
\scnidtf{\textit{object 3D model}}
\begin{scnreltolist}{ключевое понятие}
	\scnitem{\ref{sec_3d_models_representation}~\nameref{sec_3d_models_representation}}
\end{scnreltolist}

\scnheader{трехмерная реконструкция}	
\scnidtf{\textit{3D reconstruction}}
\begin{scnreltolist}{ключевое понятие}
	\scnitem{\ref{sec_3d_models_reconstruction}~\nameref{sec_3d_models_reconstruction}}
\end{scnreltolist}

\scnheader{трехмерное представление объекта}	
\scnidtf{\textit{object 3D representation}}
\begin{scnreltolist}{ключевое понятие}
	\scnitem{\ref{sec_3d_models_representation}~\nameref{sec_3d_models_representation}}
\end{scnreltolist}

\scnheader{удаляемые sc-элементы*}
\scnidtf{\textit{sc-elements to be deleted*}}
\begin{scnreltolist}{ключевой знак}
	\scnitem{\ref{sec_ps_sync}~\nameref{sec_ps_sync}}
\end{scnreltolist}

\scnheader{физический интерфейс}
\scnidtf{\textit{physical interface}}
\begin{scnreltolist}{ключевое понятие}
    \scnitem{Глава \ref{chapter_interfaces}~\nameref{chapter_interfaces}}
\end{scnreltolist}

\scnheader{эффективность метода}
\scnidtf{\textit{method efficiency}}
\begin{scnreltolist}{ключевое понятие}
	\scnitem{Глава \ref{chapter_programs}~\nameref{chapter_programs}}
\end{scnreltolist}

\scnheader{язык представления методов}
\scnidtf{язык программирования}
\scnidtf{\textit{method representation language}}
\begin{scnreltolist}{ключевое понятие}
	\scnitem{Глава \ref{chapter_programs}~\nameref{chapter_programs}}
\end{scnreltolist}

\scnheader{Язык SCP}
\scnidtf{\textit{SCP Language}}
\begin{scnreltolist}{ключевой знак}
	\scnitem{\ref{sec_ps_scp}~\nameref{sec_ps_scp}}
\end{scnreltolist}

\scnheader{ostis-платформа}
\scnidtf{\textit{ostis-platform}}
\begin{scnreltolist}{ключевой знак}
	\scnitem{Глава \ref{chapter_interpreter}~\nameref{chapter_interpreter}}
\end{scnreltolist}

\scnheader{sc-агент}
\scnidtf{\textit{sc-agent}}
\begin{scnreltolist}{ключевой знак}
	\scnitem{\ref{sec_ps_agents}~\nameref{sec_ps_agents}}
\end{scnreltolist}

\scnheader{sc-идентификатор}
\scnidtf{\textit{sc-identifier}}
\begin{scnreltolist}{ключевой знак}
	\scnitem{Глава \ref{chapter_ext_lang}~\nameref{chapter_ext_lang}}
\end{scnreltolist}

\scnheader{SCg-код}
\scnidtf{\textit{SCg-code}}
\begin{scnreltolist}{ключевой знак}
	\scnitem{\nameref{chapter_ext_lang}}
\end{scnreltolist}

\scnheader{SCn-код}
\scnidtf{\textit{SCn-code}}
\begin{scnreltolist}{ключевой знак}
	\scnitem{\nameref{chapter_ext_lang}}
\end{scnreltolist}

\scnheader{SCs-код}
\scnidtf{\textit{SCs-code}}
\begin{scnreltolist}{ключевой знак}
	\scnitem{\nameref{chapter_ext_lang}}
\end{scnreltolist}

\bigskip
\scnheader{\scnfilelong{\textbf{\textit{SC-код}}}}
\scnidtf{\textit{SC-code}}
\scnrelto{часто используемый sc-идентификатор}{sc-структура}
\begin{scnreltolist}{ключевой знак}
	\scnitem{Глава \ref{chapter_new_generation_systems}~\nameref{chapter_new_generation_systems}}
\end{scnreltolist}

\scnheader{sc-машина}
\scnidtf{\textit{sc-machine}}
\begin{scnreltolist}{ключевой знак}
	\scnitem{Глава \ref{chapter_interpreter}~\nameref{chapter_interpreter}}
\end{scnreltolist}

\scnheader{scp-операнд\scnrolesign}
\scnidtf{\textit{scp-operand\scnrolesign}}
\begin{scnreltolist}{ключевой знак}
	\scnitem{\ref{sec_ps_scp}~\nameref{sec_ps_scp}}
\end{scnreltolist}

\scnheader{scp-оператор}
\scnidtf{\textit{scp-operator}}
\begin{scnreltolist}{ключевой знак}
	\scnitem{\ref{sec_ps_scp}~\nameref{sec_ps_scp}}
\end{scnreltolist}

\scnheader{sc-элемент}
\scnidtf{\textit{sc-element}}
\begin{scnreltolist}{ключевой знак}
	\scnitem{Глава \ref{chapter_new_generation_systems}~\nameref{chapter_new_generation_systems}}
\end{scnreltolist}

\end{SCn}	