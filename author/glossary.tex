\begin{partbacktext}
\part*{Используемые сокращения и предметный указатель OSTIS}
\markboth{Используемые сокращения и предметный указатель OSTIS}{Используемые сокращения и предметный указатель OSTIS}
\label{part_glossary}
\addcontentsline{toc}{part}{Используемые сокращения и  предметный указатель OSTIS}

Предлагаемый вашему вниманию предметный указатель представляет собой алфавитный перечень основных терминов, используемых в данной монографии, которые \myuline{взаимно-однозначно} соответствуют элементам рафинированной семантической сети, представляющей собой базу знаний \textit{Метасистемы OSTIS}, основная часть которой семантически эквивалентна тексту данной монографии.

В рамках внутреннего языка \textit{ostis-систем} (в рамках \textit{SC-кода}) указанные основные термины называются \textit{основными sc-идентификаторами} (основными внешними идентификаторами sc-элементов --- элементов рафинированных семантических сетей).

%В данный предметный указатель включаются как русскоязычные, так и англоязычные термины (если нет аналогичного русскоязычного), непереводимые интернациональные названия (например, названия различных программных систем, такие как \textit{Neo4j}, \textit{MySQL} и так далее), а также различные используемые сокращения.

В данном предметном указателе в алфавитном порядке перечислены:
\begin{textitemize}
	\item все русскоязычные \myuline{основные} термины описываемых в монографии сущностей с указанием
	\begin{textitemize}
		\item их англоязычных эквивалентов;
		\item тех разделов монографии, в которых эти термины являются ключевыми знаками;
	\end{textitemize}
	\item все \myuline{интернациональные} основные термины (например, названия различных программных систем, такие как \textit{Neo4j}, \textit{MySQL} и так далее), используемые в монографии с указанием
	\begin{textitemize}
		\item соответствующих разделов монографии, где эти термины являются ключевыми знаками.
	\end{textitemize}
\end{textitemize}

При этом по умолчанию считается, что русскоязычный термин является \myuline{основным} русскоязычным идентификатором, а соответствующий ему англоязычный --- \myuline{основным} англоязычным идентификатором.

%TODO Перенести SCn вниз
\begin{comment}
\begin{SCn}
	\scnheader{sc-идентификатор}
	\scntext{часто используемый sc-идентификатор}{внешний идентификатор sc-элемента}
	\scnidtf{строка символов или пиктограмма, взаимно однозначно представляющая соответствующий sc-элемент, хранимый в sc-памяти}
	\begin{scnsubdividing}
		\scnitem{основной sc-идентификатор}
		\scnitem{неосновной sc-идентификатор}
		\begin{scnindent}
			\scnsuperset{часто используемый неосновной sc-идентификатор}
		\end{scnindent} 
	\end{scnsubdividing}

	\begin{scnsubdividing}
		\scnitem{русскоязычный sc-идентификатор}
		\scnitem{англоязычный sc-идентификатор}
		\scnitem{интернациональный sc-идентификатор}
	\end{scnsubdividing}
\end{SCn}

\end{comment}

Для каждого неосновного, но часто используемого русскоязычного термина указывается синонимичный ему основной русскоязычный термин. Кроме того, для каждого основного русскоязычного термина указывается ссылка на соответствующую главу, параграф или пункт монографии, где сущность, обозначаемая этим термином, является ключевым знаком.

\end{partbacktext}

\begin{SCn}

\scnheader{абстрактная сущность}
\scntext{трактовка}{трактовка этого термина имеет два аспекта:
	\begin{textitemize}
		\item вымышленная (придуманная, реально несуществующая) сущность в отличие оо реальной (материальной) сущности, например, множество, пространственная точка;
		\item сущность, имеющая неоднозначную спецификацию описывающую только те свойства, которые важны (существенны) только для некоторой точки зрения. Например, абстрактная (виртуальная) машина, абстрактный (виртуальный) пациент.
	\end{textitemize}
}
\begin{scnreltolist}{ключевое понятие}
	\scnitem{Глава \ref{chapter_sc_code}~\nameref{chapter_sc_code}}
\end{scnreltolist}

\scnheader{Абстрактная scp-машина}
\scnidtf{\textit{Abstract scp-machine}}
\begin{scnreltolist}{ключевой знак}
	\scnitem{\ref{sec_ps_scp}~\nameref{sec_ps_scp}}
\end{scnreltolist}

\scnheader{абстрактный sc-агент}
\scnidtf{\textit{abstract sc-agent}}
\begin{scnreltolist}{ключевое понятие}
	\scnitem{\ref{sec_ps_agents}~\nameref{sec_ps_agents}}
\end{scnreltolist}

\scnheader{абстрактный sc-агент, не реализуемый на Языке SCP}	
\scnidtf{\textit{abstract sc-agent not implemented in the SCP Language}}
\begin{scnreltolist}{ключевое понятие}
	\scnitem{\ref{sec_ps_agents}~\nameref{sec_ps_agents}}
\end{scnreltolist}

\scnheader{абстрактный sc-агент, реализуемый на Языке SCP}
\scnidtf{\textit{abstract sc-agent implemented in the SCP Language}}
\begin{scnreltolist}{ключевое понятие}
	\scnitem{\ref{sec_ps_agents}~\nameref{sec_ps_agents}}
\end{scnreltolist}

\scnheader{A/Б тестирование пользовательских интерфейсов}
\scnidtf{\textit{A/B testing of user interfaces}}
\begin{scnreltolist}{ключевое понятие}
	\scnitem{Глава \ref{chapter_ui_design}~\nameref{chapter_ui_design}}
\end{scnreltolist}

\scnheader{автоматизированная система управления технологическим процессом}
\scnidtf{\textit{automated technological process control system}}
\begin{scnreltolist}{ключевое понятие}
	\scnitem{\ref{sec_chapter_enterprise_characteristics}~\nameref{sec_chapter_enterprise_characteristics}}
\end{scnreltolist}

\scnheader{агент Экосистемы OSTIS}
\scnidtf{\textit{OSTIS Ecosystem agent}}
\begin{scnreltolist}{ключевое понятие}
	\scnitem{\ref{sec_ecosystem_structure_description}~\nameref{sec_ecosystem_structure_description}}
\end{scnreltolist}

\scnheader{адаптивное управление}
\scnidtf{\textit{adaptive control}}
\begin{scnreltolist}{ключевое понятие}
	\scnitem{\ref{sec_chapter_enterprise_characteristics}~\nameref{sec_chapter_enterprise_characteristics}}
\end{scnreltolist}

\scnheader{адаптивный интерфейс}
\scnidtf{\textit{adaptive interface}}
\begin{scnreltolist}{ключевое понятие}
    \scnitem{Глава \ref{chapter_interfaces}~\nameref{chapter_interfaces}}
\end{scnreltolist}

\scnheader{адъюнкт}
\scnidtf{\textit{adjunct}}
\begin{scnreltolist}{ключевое понятие}
	\scnitem{Глава \ref{chapter_lang}~\nameref{chapter_lang}}
\end{scnreltolist}

\scnheader{активный навык}
\scnidtf{\textit{active skill}}
\begin{scnreltolist}{ключевое понятие}
	\scnitem{\ref{sec_skill_concept}~\nameref{sec_skill_concept}}
\end{scnreltolist}

\scnheader{алфавит}
\scnidtf{\textit{alphabet}}
\begin{scnreltolist}{ключевое понятие}
	\scnitem{Глава \ref{chapter_inf_constr}~\nameref{chapter_inf_constr}}
\end{scnreltolist}

\scnheader{Алфавит SC-кода\scnsupergroupsign}
\scnidtf{\textit{SC-code Alphabet\scnsupergroupsign}}
\begin{scnreltolist}{ключевой параметр}
	\scnitem{\ref{sec_scg_syntax}~\nameref{sec_scg_syntax}}
\end{scnreltolist}

\scnheader{Алфавит SCg-кода\scnsupergroupsign}
\scnidtf{\textit{SCg-code Alphabet\scnsupergroupsign}}
\begin{scnreltolist}{ключевой параметр}
	\scnitem{\ref{sec_scg_syntax}~\nameref{sec_scg_syntax}}
\end{scnreltolist}

\scnheader{Алфавит SCn-кода\scnsupergroupsign}
\scnidtf{\textit{SCn-code Alphabet\scnsupergroupsign}}
\begin{scnreltolist}{ключевой параметр}
	\scnitem{\ref{sec_scn_syntax}~\nameref{sec_scn_syntax}}
\end{scnreltolist}

\scnheader{Алфавит SCs-кода\scnsupergroupsign}
\scnidtf{\textit{SCs-code Alphabet\scnsupergroupsign}}
\begin{scnreltolist}{ключевой параметр}
	\scnitem{\ref{sec_scs_syntax}~\nameref{sec_scs_syntax}}
\end{scnreltolist}

\scnheader{анализ}
\scnidtf{\textit{analysis}}
\begin{scnreltolist}{ключевое отношение}
	\scnitem{\ref{sec_activity_and_technology}~\nameref{sec_activity_and_technology}}
\end{scnreltolist}

\scnheader{арифметическая операция*}
\scnidtf{\textit{arithmetic operation*}}
\begin{scnreltolist}{ключевое отношение}
	\scnitem{Глава \ref{chapter_top_ontologies}~\nameref{chapter_top_ontologies}}
\end{scnreltolist}

\scnheader{арность}
\scnidtf{\textit{arity}}
\begin{scnreltolist}{ключевое понятие}
	\scnitem{Глава \ref{chapter_top_ontologies}~\nameref{chapter_top_ontologies}}
\end{scnreltolist}

\scnheader{архитектура вычислительной системы}
\scnidtf{\textit{computing system architecture}}
\begin{scnreltolist}{ключевое понятие}
	\scnitem{Глава \ref{chapter_computers}~\nameref{chapter_computers}}
\end{scnreltolist}

\scnheader{ассоциативный семантический компьютер}
\scnidtf{\textit{associative semantic computer}}
\begin{scnreltolist}{ключевой знак}
	\scnitem{Глава \ref{chapter_computers}~\nameref{chapter_computers}}
	\scnitem{Глава \ref{chapter_interpreter}~\nameref{chapter_interpreter}}
\end{scnreltolist}

\scnheader{атомарное действие}
\scnidtf{\textit{atomic action}}
\begin{scnreltolist}{ключевое отношение}
	\scnitem{\ref{sec_action_concept}~\nameref{sec_action_concept}}
\end{scnreltolist}

\scnheader{атомарное существование}
\scnidtf{\textit{atomic existence}}
\begin{scnreltolist}{ключевое понятие}
	\scnitem{Глава \ref{chapter_logic}~\nameref{chapter_logic}}
\end{scnreltolist}

\scnheader{атомарный абстрактный sc-агент}
\scnidtf{\textit{atomic abstract sc-agent}}
\begin{scnreltolist}{ключевое понятие}
	\scnitem{\ref{sec_ps_agents}~\nameref{sec_ps_agents}}	
\end{scnreltolist}

\scnheader{атрибут отношения*}
\scnidtf{\textit{relation attribute*}}
\begin{scnreltolist}{ключевое отношение}
	\scnitem{Глава \ref{chapter_top_ontologies}~\nameref{chapter_top_ontologies}}
\end{scnreltolist}

\scnheader{аудиоинтерфейс}
\scnidtf{\textit{audio interface}}
\begin{scnreltolist}{ключевое понятие}
	\scnitem{Глава \ref{chapter_audio_interfaces}~\nameref{chapter_audio_interfaces}}
\end{scnreltolist}

\scnheader{аудиосигнал}
\scnidtf{\textit{audio signal}}
\begin{scnreltolist}{ключевое понятие}
	\scnitem{Глава \ref{chapter_audio_interfaces}~\nameref{chapter_audio_interfaces}}
\end{scnreltolist}

\scnheader{база знаний}
\scnidtf{\textit{knowledge base}}
\begin{scnreltolist}{ключевое понятие}
	\scnitem{Глава \ref{chapter_top_ontologies}~\nameref{chapter_top_ontologies}}
	\scnitem{\ref{sec_kb}~\nameref{sec_kb}}
\end{scnreltolist}

\scnheader{база знаний ostis-системы}
\scnidtf{\textit{ostis-system knowledge base}}
\begin{scnreltolist}{ключевое понятие}
	\scnitem{\ref{sec_security_principles}~\nameref{sec_security_principles}}
\end{scnreltolist}

\scnheader{базовая ostis-платформа}
\scnidtf{\textit{basic ostis-platform}}
\begin{scnreltolist}{ключевое понятие}
	\scnitem{Глава \ref{chapter_interpreter}~\nameref{chapter_interpreter}}
\end{scnreltolist}

\scnheader{базовый класс описываемых сущностей}
\scnidtf{\textit{basic entity class}}
\begin{scnreltolist}{ключевое понятие}
	\scnitem{Глава \ref{chapter_top_ontologies}~\nameref{chapter_top_ontologies}}
\end{scnreltolist}

\scnheader{библиотека многократно используемых компонентов пользовательских интерфейсов ostis-систем}
\scnidtftext{сокращение основного sc-идентификатора}{библиотека м.и.к. п.и. ostis-систем}
\scnidtf{\textit{ostis-systems reusable user interface components library}}
\begin{scnreltolist}{ключевое понятие}
	\scnitem{Глава \ref{chapter_ui_design}~\nameref{chapter_ui_design}}
\end{scnreltolist}  

\scnheader{библиотека многократно используемых компонентов ostis-систем}
\scnidtftext{сокращение основного sc-идентификатора}{библиотека м.и.к. ostis-систем}
\scnidtf{\textit{ostis-systems reusable components library}}
\begin{scnreltolist}{ключевое понятие}
	\scnitem{Глава \ref{chapter_library}~\nameref{chapter_library}}
\end{scnreltolist}

\scnheader{Библиотека Метасистемы OSTIS}
\scnidtf{\textit{OSTIS Metasystem library}}
\begin{scnreltolist}{ключевой знак}
	\scnitem{Глава \ref{chapter_library}~\nameref{chapter_library}}
\end{scnreltolist}

\scnheader{Библиотека Экосистемы OSTIS}
\scnidtf{\textit{OSTIS Ecosystem library}}
\begin{scnreltolist}{ключевой знак}
	\scnitem{Глава \ref{chapter_library}~\nameref{chapter_library}}
\end{scnreltolist}

\scnheader{бинарное отношение}
\scnidtf{\textit{binary relation}}
\begin{scnreltolist}{ключевое понятие}
	\scnitem{Глава \ref{chapter_top_ontologies}~\nameref{chapter_top_ontologies}}
\end{scnreltolist}

\scnheader{блокировка*}
\scnidtf{\textit{lock*}}
\begin{scnreltolist}{ключевое отношение}
	\scnitem{\ref{sec_ps_sync}~\nameref{sec_ps_sync}}
\end{scnreltolist}

\scnheader{веб вещей}
\scnidtf{\textit{web of things}}
\begin{scnreltolist}{ключевое понятие}
	\scnitem{Глава \ref{chapter_smart_home}~\nameref{chapter_smart_home}}
\end{scnreltolist}

\scnheader{величина}
\scnidtf{\textit{value}}
\begin{scnreltolist}{ключевое понятие}
	\scnitem{Глава \ref{chapter_top_ontologies}~\nameref{chapter_top_ontologies}}
\end{scnreltolist}

\scnheader{верное высказывание*}
\scnidtf{\textit{correct statement*}}
\begin{scnreltolist}{ключевое отношение}
	\scnitem{\ref{sec_nonclass_logic}~\nameref{sec_nonclass_logic}}
\end{scnreltolist}

\scnheader{вероятностный технологический процесс}
\scnidtf{\textit{probabilistic technological process}}
\begin{scnreltolist}{ключевое понятие}
	\scnitem{\ref{sec_chapter_enterprise_characteristics}~\nameref{sec_chapter_enterprise_characteristics}}
\end{scnreltolist}

\scnheader{вершина}
\scnidtf{\textit{head}}
\begin{scnreltolist}{ключевое понятие}
	\scnitem{Глава \ref{chapter_lang}~\nameref{chapter_lang}}
\end{scnreltolist}

\scnheader{вид деятельности}
\scnidtf{\textit{activity type}}
\begin{scnreltolist}{ключевое понятие}
	\scnitem{Глава \ref{chapter_actions}~\nameref{chapter_actions}}
\end{scnreltolist}

\scnheader{вид знаний}
\scnidtf{\textit{knowledge type}}
\begin{scnreltolist}{ключевое понятие}
	\scnitem{\ref{sec_kb}~\nameref{sec_kb}}
\end{scnreltolist}

\scnheader{включение*}
\scnidtf{\textit{inclusion*}}
\begin{scnreltolist}{ключевое отношение}
	\scnitem{Глава \ref{chapter_top_ontologies}~\nameref{chapter_top_ontologies}}
\end{scnreltolist}

\scnheader{внешняя сущность}
\scnidtf{\textit{external entity}}
\begin{scnreltolist}{ключевое понятие}
	\scnitem{Глава \ref{chapter_sc_code}~\nameref{chapter_sc_code}}
\end{scnreltolist}

\scnheader{воздействие}
\scnidtf{\textit{manipulation}}
\begin{scnreltolist}{ключевое понятие}
	\scnitem{Глава \ref{chapter_actions}~\nameref{chapter_actions}}
\end{scnreltolist}

\scnheader{волновая микропрограмма}
\scnidtf{\textit{wave microprogram}}
\begin{scnreltolist}{ключевое понятие}
	\scnitem{Глава \ref{chapter_computers}~\nameref{chapter_computers}}
\end{scnreltolist}

\scnheader{волновой язык программирования}
\scnidtf{\textit{wave programming language}}
\begin{scnreltolist}{ключевое понятие}
	\scnitem{Глава \ref{chapter_computers}~\nameref{chapter_computers}}
\end{scnreltolist}

\scnheader{вопрос}
\scnidtf{\textit{question}}
\begin{scnreltolist}{ключевое понятие}
	\scnitem{\ref{sec_problem_concept}~\nameref{sec_problem_concept}}
	\scnitem{Глава \ref{chapter_requests}~\nameref{chapter_requests}}
\end{scnreltolist}

\scnheader{временная связь}
\scnidtf{\textit{temporary connection}}
\begin{scnreltolist}{ключевое понятие}
	\scnitem{Глава \ref{chapter_top_ontologies}~\nameref{chapter_top_ontologies}}
\end{scnreltolist}

\scnheader{временная сущность}
\scnidtf{\textit{temporary entity}}
\begin{scnreltolist}{ключевое понятие}
	\scnitem{Глава \ref{chapter_top_ontologies}~\nameref{chapter_top_ontologies}}
\end{scnreltolist}

\scnheader{встроенная ostis-система}
\scnidtf{\textit{built-in ostis-system}}
\begin{scnreltolist}{ключевое понятие}
	\scnitem{\ref{sec_ecosystem_structure}~\nameref{sec_ecosystem_structure}}
\end{scnreltolist}

\scnheader{выводимое множество}
\scnidtf{\textit{inferrable set}}
\begin{scnreltolist}{ключевое понятие}
	\scnitem{\ref{sec_nonclass_logic}~\nameref{sec_nonclass_logic}}
\end{scnreltolist}

\scnheader{высказывание}
\scnidtf{\textit{statement}}
\begin{scnreltolist}{ключевое понятие}
	\scnitem{Глава \ref{chapter_logic}~\nameref{chapter_logic}}
\end{scnreltolist}

\scnheader{высказывание*}
\scnidtf{\textit{statement*}}
\begin{scnreltolist}{ключевое отношение}
	\scnitem{Глава \ref{chapter_logic}~\nameref{chapter_logic}}
\end{scnreltolist}

\scnheader{геоинформационная система}
\scnidtf{\textit{geoinformation system}}
\begin{scnreltolist}{ключевое понятие}
	\scnitem{Глава \ref{chapter_gis}~\nameref{chapter_gis}}
\end{scnreltolist}

\scnheader{геоонтология}
\scnidtf{\textit{geoontology}}
\begin{scnreltolist}{ключевое понятие}
	\scnitem{Глава \ref{chapter_gis}~\nameref{chapter_gis}}
\end{scnreltolist}

\scnheader{геосемантическая характеристика объекта местности}
\scnidtf{\textit{geosemantic characteristic of terrain object}}
\begin{scnreltolist}{ключевое понятие}
	\scnitem{Глава \ref{chapter_gis}~\nameref{chapter_gis}}
\end{scnreltolist}

\scnheader{глобальный нетранслируемый sc-идентификатор}
\scnidtf{\textit{global non-translatable sc-identifier}}
\begin{scnreltolist}{ключевое понятие}
	\scnitem{\ref{sec_non_translation_identifier_concept}~\nameref{sec_non_translation_identifier_concept}}
\end{scnreltolist}

\scnheader{грамматическая категория}
\scnidtf{\textit{grammatical category}}
\begin{scnreltolist}{ключевое понятие}
	\scnitem{Глава \ref{chapter_lang}~\nameref{chapter_lang}}
\end{scnreltolist}

\scnheader{действие}
\scnidtf{\textit{action}}
\begin{scnreltolist}{ключевое понятие}
	\scnitem{Глава \ref{chapter_actions}~\nameref{chapter_actions}}
\end{scnreltolist}

\scnheader{действие в sc-памяти}
\scnidtf{\textit{action in sc-memory}}
\begin{scnreltolist}{ключевое понятие}
	\scnitem{Глава \ref{chapter_lang}~\nameref{chapter_lang}}
	\scnitem{\ref{sec_ps_actions}~\nameref{sec_ps_actions}}
\end{scnreltolist}

\scnheader{действие в sc-памяти, инициируемое вопросом}
\scnidtf{\textit{action in sc-memory initiated by a question}}
\begin{scnreltolist}{ключевое понятие}
		\scnitem{\ref{sec_ps_actions}~\nameref{sec_ps_actions}}
\end{scnreltolist}

\scnheader{действие по построению искусственных нейронных сетей}
\scnidtf{\textit{artificial neural networks building action}}
\begin{scnreltolist}{ключевое понятие}
	\scnitem{Глава \ref{chapter_ann}~\nameref{chapter_ann}}
\end{scnreltolist}

\scnheader{действие редактирования базы знаний}
\scnidtf{\textit{knowledge base editing action}}
\begin{scnreltolist}{ключевое понятие}
	\scnitem{\ref{sec_ps_actions}~\nameref{sec_ps_actions}}
\end{scnreltolist}

\scnheader{декартово произведение*}
\scnidtf{\textit{cartesian product*}}
\begin{scnreltolist}{ключевое отношение}
	\scnitem{Глава \ref{chapter_top_ontologies}~\nameref{chapter_top_ontologies}}
\end{scnreltolist}

\scnheader{декларативная формулировка задачи}
\scnidtf{\textit{declarative problem definition}}
\begin{scnreltolist}{ключевое понятие}
	\scnitem{\ref{sec_problem_concept}~\nameref{sec_problem_concept}}
\end{scnreltolist}

\scnheader{декларативная формулировка задачи*}
\scnidtf{\textit{declarative problem definition*}}
\begin{scnreltolist}{ключевое отношение}
	\scnitem{\ref{sec_problem_concept}~\nameref{sec_problem_concept}}
\end{scnreltolist}

\scnheader{декларативно-процедурная формулировка задачи}
\scnidtf{\textit{declarative-procedural problem definition}}
\begin{scnreltolist}{ключевое понятие}
	\scnitem{\ref{sec_problem_concept}~\nameref{sec_problem_concept}}
\end{scnreltolist}

\scnheader{денотационная семантика метода*}
\scnidtf{\textit{denotational semantics of method*}}
\begin{scnreltolist}{ключевое отношение}
	\scnitem{Глава \ref{chapter_programs}~\nameref{chapter_programs}}
	\scnitem{\ref{sec_skill_concept}~\nameref{sec_skill_concept}}
\end{scnreltolist}

\scnheader{денотационная семантика языка}
\scnidtf{\textit{denotational semantics of language}}
\begin{scnreltolist}{ключевое понятие}
	\scnitem{Глава \ref{chapter_inf_constr}~\nameref{chapter_inf_constr}}
\end{scnreltolist}

\scnheader{денотационная семантика языка представления методов*}
\scnidtf{\textit{denotational semantics of method representation language*}}
\begin{scnreltolist}{ключевое отношение}
	\scnitem{Глава \ref{chapter_programs}~\nameref{chapter_programs}}
\end{scnreltolist}

\scnheader{Денотационная семантика SCg-кода}
\scnidtf{\textit{Denotational semantics of SCg-code}}
\begin{scnreltolist}{ключевой знак}
	\scnitem{\ref{sec_scg_semantics}~\nameref{sec_scg_semantics}}
\end{scnreltolist}

\scnheader{Денотационная семантика SCn-кода}
\scnidtf{\textit{Denotational semantics of SCn-code}}
\begin{scnreltolist}{ключевой знак}
	\scnitem{\ref{sec_scn_semantics}~\nameref{sec_scn_semantics}}
\end{scnreltolist}

\scnheader{Денотационная семантика SCs-кода}
\scnidtf{\textit{Denotational semantics of SCs-code}}
\begin{scnreltolist}{ключевой знак}
	\scnitem{\ref{sec_scs_semantics}~\nameref{sec_scs_semantics}}
\end{scnreltolist}

\scnheader{деятельность}
\scnidtf{\textit{activity}}
\begin{scnreltolist}{ключевое понятие}
	\scnitem{Глава \ref{chapter_actions}~\nameref{chapter_actions}}
\end{scnreltolist}

\scnheader{дидактическая информация}
\scnidtf{\textit{didactic information}}
\begin{scnreltolist}{ключевое понятие}
	\scnitem{\ref{sec_didactic information}~\nameref{sec_didactic information}}
\end{scnreltolist}

\scnheader{дискретная информационная конструкция}
\scnidtf{\textit{discrete information construction}}
\begin{scnreltolist}{ключевое понятие}
	\scnitem{Глава \ref{chapter_inf_constr}~\nameref{chapter_inf_constr}}
\end{scnreltolist}

\scnheader{длительность\scnsupergroupsign}
\scnidtf{\textit{duration\scnsupergroupsign}}
\begin{scnreltolist}{ключевой параметр}
	\scnitem{Глава \ref{chapter_top_ontologies}~\nameref{chapter_top_ontologies}}
\end{scnreltolist}

\scnheader{домен*}
\scnidtf{\textit{domain*}}
\begin{scnreltolist}{ключевое отношение}
	\scnitem{Глава \ref{chapter_top_ontologies}~\nameref{chapter_top_ontologies}}
\end{scnreltolist}

\scnheader{древовидное тестирование пользовательских интерфейсов}
\scnidtf{\textit{tree testing of user interfaces}}
\begin{scnreltolist}{ключевое понятие}
	\scnitem{Глава \ref{chapter_ui_design}~\nameref{chapter_ui_design}}
\end{scnreltolist}

\scnheader{единица измерения*}
\scnidtf{\textit{measurement unit*}}
\begin{scnreltolist}{ключевое отношение}
	\scnitem{Глава \ref{chapter_top_ontologies}~\nameref{chapter_top_ontologies}}
\end{scnreltolist}

\scnheader{естественно-языковой интерфейс}
\scnidtf{\textit{natural language interface}}
\begin{scnreltolist}{ключевое понятие}
	\scnitem{Глава \ref{chapter_nl_interfaces}~\nameref{chapter_nl_interfaces}}
\end{scnreltolist}

\scnheader{жизненный цикл}
\scnidtf{\textit{life cycle}}
\begin{scnreltolist}{ключевое понятие}
	\scnitem{Часть \ref{part1}~\nameref{part1}}
\end{scnreltolist}

\scnheader{завершение\scnsupergroupsign}
\scnidtf{\textit{completion\scnsupergroupsign}}
\begin{scnreltolist}{ключевой параметр}
	\scnitem{Глава \ref{chapter_top_ontologies}~\nameref{chapter_top_ontologies}}
\end{scnreltolist}

\scnheader{задача}
\scnidtf{\textit{problem}}
\begin{scnreltolist}{ключевое понятие}
	\scnitem{Глава \ref{chapter_actions}~\nameref{chapter_actions}}
\end{scnreltolist}

\scnheader{задача геоинформационной системы}
\scnidtf{\textit{geoinformation system problem}}
\begin{scnreltolist}{ключевое понятие}
	\scnitem{Глава \ref{chapter_gis}~\nameref{chapter_gis}}
\end{scnreltolist}

\scnheader{задача, решаемая в sc-памяти}
\scnidtf{\textit{problem solved in sc-memory}}
\begin{scnreltolist}{ключевое понятие}
	\scnitem{\ref{sec_ps_actions}~\nameref{sec_ps_actions}}
\end{scnreltolist}

\scnheader{знак}
\scnidtf{\textit{sign}}
\begin{scnreltolist}{ключевое понятие}
	\scnitem{Глава \ref{chapter_inf_constr}~\nameref{chapter_inf_constr}}
\end{scnreltolist}

\scnheader{знаковая конструкция}
\scnidtf{\textit{sign construction}}
\begin{scnreltolist}{ключевое понятие}
	\scnitem{Глава \ref{chapter_inf_constr}~\nameref{chapter_inf_constr}}
\end{scnreltolist}

\scnheader{знание}
\scnidtf{\textit{knowledge}}
\begin{scnreltolist}{ключевое понятие}
	\scnitem{Глава \ref{chapter_top_ontologies}~\nameref{chapter_top_ontologies}}
	\scnitem{\ref{sec_kb}~\nameref{sec_kb}}
\end{scnreltolist}

\scnheader{идентификатор}
\scnidtf{\textit{identifier}}
\begin{scnreltolist}{ключевое понятие}
	\scnitem{Глава \ref{chapter_inf_constr}~\nameref{chapter_inf_constr}}
\end{scnreltolist}

\scnheader{измерение*}
\scnidtf{\textit{measurement*}}
\begin{scnreltolist}{ключевое отношение}
	\scnitem{Глава \ref{chapter_top_ontologies}~\nameref{chapter_top_ontologies}}
\end{scnreltolist}

\scnheader{измерение с фиксированной единицей измерения}
\scnidtf{\textit{measurement with fixed measurement unit}}
\begin{scnreltolist}{ключевое понятие}
	\scnitem{Глава \ref{chapter_top_ontologies}~\nameref{chapter_top_ontologies}}
\end{scnreltolist}

\scnheader{имя нарицательное}
\scnidtf{\textit{common name}}
\begin{scnreltolist}{ключевое понятие}
	\scnitem{\ref{sec_simple_identifier_concept}~\nameref{sec_simple_identifier_concept}}
\end{scnreltolist}

\scnheader{имя собственное}
\scnidtf{\textit{proper name}}
\begin{scnreltolist}{ключевое понятие}
	\scnitem{\ref{sec_simple_identifier_concept}~\nameref{sec_simple_identifier_concept}}
\end{scnreltolist}

\scnheader{индивидуальная кибернетическая система}
\scnidtf{\textit{individual cybernetic system}}
\begin{scnreltolist}{ключевое понятие}
	\scnitem{Глава \ref{chap_intro}~\nameref{chap_intro}}
\end{scnreltolist}

\scnheader{Индустрия 4.0}
\scnidtf{\textit{Industry 4.0}}
\begin{scnreltolist}{ключевой знак}
	\scnitem{\ref{sec_chapter_enterprise_model_creation_industry4}~\nameref{sec_chapter_enterprise_model_creation_industry4}}
\end{scnreltolist}

\scnheader{инициируемое пользовательским интерфейсом действие*}
\scnidtf{\textit{action initiated by the user interface}}
\begin{scnreltolist}{ключевое отношение}
	\scnitem{Глава \ref{chapter_interfaces}~\nameref{chapter_interfaces}}
\end{scnreltolist}

\scnheader{интеграция}
\scnidtf{\textit{integration}}
\begin{scnreltolist}{ключевое понятие}
	\scnitem{Глава \ref{chapter_integration}~\nameref{chapter_integration}}
\end{scnreltolist}

\scnheader{интеллект\scnsupergroupsign}
\scnidtf{\textit{intelligence\scnsupergroupsign}}
\scnidtf{уровень интеллекта\scnsupergroupsign}
\scniselement{параметр}
\scnrelfrom{область определения}{кибернетическая система}
\begin{scnreltolist}{ключевой параметр}
	\scnitem{Глава \ref{chap_intro}~\nameref{chap_intro}}
\end{scnreltolist}

\scnheader{интеллектуальная геоинформационная система}
\scnidtf{\textit{intelligent geoinformation system}}
\begin{scnreltolist}{ключевое понятие}
	\scnitem{Глава \ref{chapter_gis}~\nameref{chapter_gis}}
\end{scnreltolist}

\scnheader{интеллектуальная геоинформационная ostis-система}
\scnidtf{\textit{intelligent geoinformation ostis-system}}
\begin{scnreltolist}{ключевое понятие}
	\scnitem{Глава \ref{chapter_gis}~\nameref{chapter_gis}}
\end{scnreltolist}

\scnheader{интеллектуальная кибернетическая система}
\scnidtf{\textit{intelligent cybernetic system}}
\begin{scnreltolist}{ключевое понятие}
	\scnitem{Глава \ref{chap_intro}~\nameref{chap_intro}}
\end{scnreltolist}

\scnheader{интеллектуальная компьютерная система нового поколения}
\scnidtftext{сокращение основного sc-идентификатора}{и.к.с. нового поколения}
\scnidtf{\textit{next-generation intelligent computer system}}
\scntext{определение}{интеллектуальная компьютерная система, обладающая:
	\begin{scnitemize}
		\item высоким уровнем самообучаемости, обеспечивающим высокий уровень автоматизации собственной эволюции и, соответственно, высокие темпы этой эволюции;
		\item высоким уровнем интероперабельности.
\end{scnitemize}}
\scnsubset{самообучаемая интеллектуальная компьютерная система}
\scnsubset{интероперабельная интеллектуальная компьютерная система}
\begin{scnreltolist}{ключевое понятие}
	\scnitem{Глава~\ref{chapter_new_generation_systems}~\nameref{chapter_new_generation_systems}}
\end{scnreltolist}

\scnheader{интеллектуальная обучающая система}
\scnidtf{\textit{intelligent learning system}}
\begin{scnreltolist}{ключевое понятие}
	\scnitem{Глава \ref{chapter_learning_systems}~\nameref{chapter_learning_systems}}
\end{scnreltolist}

\scnheader{интеллектуальная справочная система}
\scnidtf{\textit{intelligent help system}}
\begin{scnreltolist}{ключевое понятие}
	\scnitem{Глава \ref{chapter_learning_systems}~\nameref{chapter_learning_systems}}
\end{scnreltolist}

\scnheader{интеллектуальный интерфейс}
\scnidtf{\textit{intelligent interface}}
\begin{scnreltolist}{ключевое понятие}
	\scnitem{Глава \ref{chapter_interfaces}~\nameref{chapter_interfaces}}
\end{scnreltolist}

\scnheader{интервальная величина}
\scnidtf{\textit{interval value}}
\begin{scnreltolist}{ключевое понятие}
	\scnitem{Глава \ref{chapter_top_ontologies}~\nameref{chapter_top_ontologies}}
\end{scnreltolist}

\scnheader{интернет вещей}
\scnidtf{\textit{internet of things}}
\begin{scnreltolist}{ключевое понятие}
	\scnitem{Глава \ref{chapter_smart_home}~\nameref{chapter_smart_home}}
\end{scnreltolist}

\scnheader{интерпретатор пользовательских действий}
\scnidtf{\textit{user action interpreter}}
\begin{scnreltolist}{ключевое понятие}
	\scnitem{Глава \ref{chapter_interfaces}~\nameref{chapter_interfaces}}
\end{scnreltolist}

\scnheader{интерпретатор sc-моделей пользовательских интерфейсов}
\scnidtf{\textit{user interface sc-model interpreter}}
\begin{scnreltolist}{ключевое понятие}
	\scnitem{Глава \ref{chapter_interfaces}~\nameref{chapter_interfaces}}
\end{scnreltolist}

\scnheader{интерфейс}
\scnidtf{\textit{interface}}
\begin{scnreltolist}{ключевое понятие}
	\scnitem{Глава \ref{chapter_interfaces}~\nameref{chapter_interfaces}}
\end{scnreltolist}

\scnheader{интерфейсное действие пользователя}
\scnidtf{\textit{interface action by user}}
\begin{scnreltolist}{ключевое понятие}
	\scnitem{Глава \ref{chapter_interfaces}~\nameref{chapter_interfaces}}
\end{scnreltolist}

\scnheader{интерфейс ostis-систем}
\scnidtf{\textit{интерфейс интеллектуальных компьютерных систем нового поколения}}
\scnidtf{\textit{ostis-system interface}}
\begin{scnreltolist}{ключевое понятие}
	\scnitem{Глава \ref{chapter_interfaces}~\nameref{chapter_interfaces}}
\end{scnreltolist}

\scnheader{информационное действие}
\scnidtf{\textit{information action}}
\begin{scnreltolist}{ключевое понятие}
	\scnitem{\ref{sec_action_concept}~\nameref{sec_action_concept}}
\end{scnreltolist}

\scnheader{информационная задача}
\scnidtf{\textit{information problem}}
\begin{scnreltolist}{ключевое понятие}
	\scnitem{\ref{sec_problem_concept}~\nameref{sec_problem_concept}}
\end{scnreltolist}

\scnheader{информационная конструкция}
\scnidtf{\textit{information construction}}
\begin{scnreltolist}{ключевое понятие}
	\scnitem{Глава \ref{chapter_inf_constr}~\nameref{chapter_inf_constr}}
\end{scnreltolist}

\scnheader{информационный ресурс}
\scnidtf{\textit{information resource}}
\begin{scnreltolist}{ключевое понятие}
	\scnitem{Глава \ref{chapter_integration}~\nameref{chapter_integration}}
\end{scnreltolist}

\scnheader{Искусственный интеллект}
\scnidtf{\textit{Artificial intelligence}}
\begin{scnreltolist}{ключевое понятие}
	\scnitem{Глава \ref{chap_intro}~\nameref{chap_intro}}
\end{scnreltolist}

\scnheader{картографический интерфейс}
\scnidtf{\textit{cartographic interface}}
\begin{scnreltolist}{ключевое понятие}
	\scnitem{Глава \ref{chapter_gis}~\nameref{chapter_gis}}
\end{scnreltolist}

\scnheader{качественный метод оценки пользовательских интерфейсов}
\scnidtf{\textit{qualitative method of user interface evaluation}}
\begin{scnreltolist}{ключевое понятие}
	\scnitem{Глава \ref{chapter_ui_design}~\nameref{chapter_ui_design}}
\end{scnreltolist}

\scnheader{квазибинарное отношение}
\scnidtf{\textit{quasybinary relation}}
\begin{scnreltolist}{ключевое понятие}
	\scnitem{Глава \ref{chapter_top_ontologies}~\nameref{chapter_top_ontologies}}
\end{scnreltolist}

\scnheader{квантор*}
\scnidtf{\textit{quantifier*}}
\begin{scnreltolist}{ключевое отношение}
	\scnitem{Глава \ref{chapter_logic}~\nameref{chapter_logic}}
\end{scnreltolist}

\scnheader{кибернетическая система}
\scnidtf{\textit{cybernetic system}}
\scnsuperset{компьютерная система}
\begin{scnindent}
	\scnidtf{искусственная кибернетическая система}
	\scnsuperset{интеллектуальная компьютерная система}
	\begin{scnindent}
		\scnsuperset{интеллектуальная компьютерная система нового поколения}
		\begin{scnindent}
			\scneq{\textup{(} самообучаемая компьютерная система $\cap$ интероперабельная компьютерная система \textup{)}}
			\scnsuperset{ostis-система}
			\begin{scnindent}
				\scnidtf{предлагаемое уточнение (вариант реализации) интеллектуальной компьютерной системы нового поколения}
			\end{scnindent}
		\end{scnindent} 
	\end{scnindent} 
\end{scnindent} 
\begin{scnreltolist}{ключевое понятие}
	\scnitem{Часть \ref{part1}~\nameref{part1}}
	\scnitem{Глава \ref{chap_intro}~\nameref{chap_intro}}
\end{scnreltolist}

\scnheader{класс}
\scnidtf{\textit{class}}
\begin{scnreltolist}{ключевое понятие}
	\scnitem{Глава \ref{chapter_top_ontologies}~\nameref{chapter_top_ontologies}}
\end{scnreltolist}

\scnheader{класс атомарных действий}
\scnidtf{\textit{atomic action class}}
\begin{scnreltolist}{ключевое понятие}
	\scnitem{\ref{sec_action_and_problem_classes}~\nameref{sec_action_and_problem_classes}}
\end{scnreltolist}

\scnheader{класс действий}
\scnidtf{\textit{action class}}
\begin{scnreltolist}{ключевое понятие}
	\scnitem{Глава \ref{chapter_actions}~\nameref{chapter_actions}}
\end{scnreltolist}

\scnheader{класс задач}
\scnidtf{\textit{problem class}}
\begin{scnreltolist}{ключевое понятие}
	\scnitem{Глава \ref{chapter_actions}~\nameref{chapter_actions}}
\end{scnreltolist}

\scnheader{класс легко выполнимых неатомарных действий}
\scnidtf{\textit{class of trivial non-atomic actions}}
\begin{scnreltolist}{ключевое понятие}
	\scnitem{\ref{sec_action_and_problem_classes}~\nameref{sec_action_and_problem_classes}}
\end{scnreltolist}

\scnheader{класс логически атомарных действий}
\scnidtf{\textit{class of logically atomic actions}}
\begin{scnreltolist}{ключевой знак}
	\scnitem{\ref{sec_ps_actions}~\nameref{sec_ps_actions}}
\end{scnreltolist}

\scnheader{класс методов}
\scnidtf{\textit{method class}}
\begin{scnreltolist}{ключевое понятие}
	\scnitem{\ref{sec_method_lang_concept}~\nameref{sec_method_lang_concept}}
\end{scnreltolist}

\scnheader{количественный метод оценки пользовательских интерфейсов}
\scnidtf{\textit{quantitative method of user interface evaluation}}
\begin{scnreltolist}{ключевое понятие}
	\scnitem{Глава \ref{chapter_ui_design}~\nameref{chapter_ui_design}}
\end{scnreltolist} 

\scnheader{коллектив ostis-систем}
\scnidtf{\textit{ostis-systems group}}
\begin{scnreltolist}{ключевое понятие}
    \scnitem{\ref{sec_ecosystem_structure}~\nameref{sec_ecosystem_structure}}
\end{scnreltolist}

\scnheader{команда}
\scnidtf{\textit{command}}
\begin{scnreltolist}{ключевое понятие}
	\scnitem{\ref{sec_problem_concept}~\nameref{sec_problem_concept}}
\end{scnreltolist}

\scnheader{комплемент}
\scnidtf{\textit{complement}}
\begin{scnreltolist}{ключевое понятие}
	\scnitem{Глава \ref{chapter_lang}~\nameref{chapter_lang}}
\end{scnreltolist}

\scnheader{компонент пользовательского интерфейса}
\scnidtf{\textit{user interface component}}
\begin{scnreltolist}{ключевое понятие}
    \scnitem{Глава \ref{chapter_interfaces}~\nameref{chapter_interfaces}}
\end{scnreltolist}

\scnheader{компонентное проектирование}
\scnidtf{\textit{component design}}
\begin{scnreltolist}{ключевое понятие}
	\scnitem{Глава \ref{chapter_library}~\nameref{chapter_library}}
\end{scnreltolist}

\scnheader{компонентное проектирование баз знаний интеллектуальных систем}
\scnidtf{\textit{intelligent system knowledge base component design}}
\begin{scnreltolist}{ключевое понятие}
	\scnitem{\ref{sc_kb_design_components}~\nameref{sc_kb_design_components}}
\end{scnreltolist}

\scnheader{компонентное проектирование интеллектуальных систем}
\scnidtf{\textit{intelligent system component design}}
\begin{scnreltolist}{ключевое понятие}
	\scnitem{Глава \ref{chapter_library}~\nameref{chapter_library}}
\end{scnreltolist}

\scnheader{компонент ostis-системы}
\scnidtf{\textit{ostis-system component}}
\begin{scnreltolist}{ключевое понятие}
	\scnitem{Глава \ref{reusable_component_section}~\nameref{reusable_component_section}}
\end{scnreltolist}

\scnheader{компьютерная алгебра}	
\scnidtf{\textit{computer algebra}}
\begin{scnreltolist}{ключевое понятие}
	\scnitem{\ref{sec_integration_algebra}~\nameref{sec_integration_algebra}}
\end{scnreltolist}

\scnheader{компьютерное зрение}	
\scnidtf{\textit{computer vision}}
\begin{scnreltolist}{ключевое понятие}
	\scnitem{\ref{sec_3d_models_computervision}~\nameref{sec_3d_models_computervision}}
\end{scnreltolist}

\scnheader{конструктивно истинное высказывание*}
\scnidtf{\textit{constructively true statement*}}
\begin{scnreltolist}{ключевое отношение}
	\scnitem{\ref{sec_nonclass_logic}~\nameref{sec_nonclass_logic}}
\end{scnreltolist}

\scnheader{контекст}
\scnidtf{\textit{context}}
\begin{scnreltolist}{ключевое понятие}
	\scnitem{Глава \ref{chapter_nl_interfaces}~\nameref{chapter_nl_interfaces}}
\end{scnreltolist}

\scnheader{контекст диалога}
\scnidtf{\textit{dialog context}}
\begin{scnreltolist}{ключевое понятие}
	\scnitem{Глава \ref{chapter_nl_interfaces}~\nameref{chapter_nl_interfaces}}
\end{scnreltolist}

\scnheader{контекстно-зависимая система}
\scnidtf{\textit{context-dependent system}}
\begin{scnreltolist}{ключевое понятие}
	\scnitem{Глава \ref{chapter_smart_home}~\nameref{chapter_smart_home}}
\end{scnreltolist}

\scnheader{контекст*}
\scnidtf{\textit{context*}}
\begin{scnreltolist}{ключевое отношение}
	\scnitem{\ref{sec_activity_and_technology}~\nameref{sec_activity_and_technology}}
\end{scnreltolist}

\scnheader{корпоративная система}
\scnidtf{\textit{corporate system}}
\begin{scnreltolist}{ключевое понятие}
	\scnitem{\ref{sec_corporate_ostis_system}~\nameref{sec_corporate_ostis_system}}
\end{scnreltolist}

\scnheader{Корпоративная система Экосистемы OSTIS}
\scnidtf{\textit{OSTIS Ecosystem corporate system}}
\begin{scnreltolist}{ключевой знак}
	\scnitem{\ref{sec_ecosystem_structure_description}~\nameref{sec_ecosystem_structure_description}}
\end{scnreltolist}

\scnheader{корпоративная ostis-система}
\scnidtf{\textit{corporate ostis-system}}
\begin{scnreltolist}{ключевое понятие}
	\scnitem{\ref{sec_corporate_ostis_system}~\nameref{sec_corporate_ostis_system}}
\end{scnreltolist}

\scnheader{лексема}
\scnidtf{\textit{lexeme}}
\begin{scnreltolist}{ключевое понятие}
	\scnitem{Глава \ref{chapter_lang}~\nameref{chapter_lang}}
\end{scnreltolist}

\scnheader{линия разметки sc.n-текста}
\scnidtf{\textit{sc.n-text markup line}}
\begin{scnreltolist}{ключевое понятие}
	\scnitem{\ref{sec_scn}~\nameref{sec_scn}}
\end{scnreltolist}

\scnheader{логическая формула}
\scnidtf{\textit{logical formula}}
\begin{scnreltolist}{ключевое понятие}
	\scnitem{Глава \ref{chapter_logic}~\nameref{chapter_logic}}
\end{scnreltolist}

\scnheader{логическая связка*}
\scnidtf{\textit{logical sheaf*}}
\begin{scnreltolist}{ключевое отношение}
	\scnitem{Глава \ref{chapter_logic}~\nameref{chapter_logic}}
\end{scnreltolist}

\scnheader{логический канал связи}
\scnidtf{\textit{logical link}}
\begin{scnreltolist}{ключевое понятие}
	\scnitem{Глава \ref{chapter_computers}~\nameref{chapter_computers}}
\end{scnreltolist}

\scnheader{локальный нетранслируемый sc-идентификатор}
\scnidtf{\textit{local non-translatable sc-identifier}}
\begin{scnreltolist}{ключевое понятие}
	\scnitem{\ref{sec_non_translation_identifier_concept}~\nameref{sec_non_translation_identifier_concept}}
\end{scnreltolist}

\scnheader{локальный признак изображения}	
\scnidtf{\textit{local image feature}}
\begin{scnreltolist}{ключевое понятие}
	\scnitem{\ref{sec_3d_models_computervision}~\nameref{sec_3d_models_computervision}}
\end{scnreltolist}

\scnheader{материнская ostis-система}
\scnidtf{\textit{maternal ostis-system}}
\scnidtf{\textit{parent ostis-system}}
\begin{scnreltolist}{ключевое понятие}
	\scnitem{\ref{ostis_library_section}~\nameref{ostis_library_section}}
\end{scnreltolist}

\scnheader{машина обработки знаний}
\scnidtf{\textit{knowledge processing machine}}
\begin{scnreltolist}{ключевое понятие}
	\scnitem{Глава \ref{chapter_situation_management}~\nameref{chapter_situation_management}}
\end{scnreltolist}

\scnheader{машина фон-Неймана}	
\scnidtf{\textit{von Neumann machine}}
\begin{scnreltolist}{ключевое понятие}
	\scnitem{Глава \ref{chapter_computers}~\nameref{chapter_computers}}
\end{scnreltolist}

\scnheader{менеджер многократно используемых компонентов ostis-систем}
\scnidtftext{сокращение основного sc-идентификатора}{менеджер компонентов}
\scnidtf{\textit{ostis-systems reusable component manager}}
\scnidtf{\textit{sc-component-manager}}
\begin{scnreltolist}{ключевое понятие}
	\scnitem{Глава \ref{chapter_library}~\nameref{chapter_library}}
\end{scnreltolist}

\scnheader{метаметод}
\scnidtf{\textit{метапрограмма}}
\scnidtf{\textit{meta-method}}
\begin{scnreltolist}{ключевое понятие}
	\scnitem{Глава \ref{chapter_programs}~\nameref{chapter_programs}}
\end{scnreltolist}

\scnheader{Метасистема OSTIS}
\scnidtf{\textit{OSTIS Metasystem}}
\scnidtftext{сокращение основного sc-идентификатора}{Intelligent MetaSystem}
\scnidtf{интеллектуальная метасистема поддержки проектирования интеллектуальных систем}
\begin{scnreltolist}{ключевой знак}
	\scnitem{Глава \ref{chapter_ims_standard}~\nameref{chapter_ims_standard}}
\end{scnreltolist}

\scnheader{метаструктура}
\scnidtf{\textit{meta-structure}}
\begin{scnreltolist}{ключевое понятие}
	\scnitem{\ref{sec_applied_logic}~\nameref{sec_applied_logic}}
\end{scnreltolist}

\scnheader{метод}
\scnidtf{\textit{программа}}
\scnidtf{\textit{method}}
\begin{scnreltolist}{ключевое понятие}
	\scnitem{Глава \ref{chapter_programs}~\nameref{chapter_programs}}
	\scnitem{Глава \ref{chapter_actions}~\nameref{chapter_actions}}
\end{scnreltolist}

\scnheader{метод заданного языка представления методов}
\scnidtf{\textit{specified method representation language method}}
\begin{scnreltolist}{ключевое понятие}
	\scnitem{\ref{sec_method_lang_concept}~\nameref{sec_method_lang_concept}}
\end{scnreltolist}

\scnheader{метод оценки пользовательских интерфейсов}
\scnidtf{\textit{user interface evaluation method}}
\begin{scnreltolist}{ключевое понятие}
	\scnitem{Глава \ref{chapter_ui_design}~\nameref{chapter_ui_design}}
\end{scnreltolist}

\scnheader{метрика}
\scnidtf{\textit{metric}}
\begin{scnreltolist}{ключевое понятие}
	\scnitem{\ref{sec_sr_semspace}~\nameref{sec_sr_semspace}}
\end{scnreltolist}

\scnheader{метрическое пространство}
\scnidtf{\textit{metric space}}
\begin{scnreltolist}{ключевое понятие}
	\scnitem{\ref{sec_sr_semspace}~\nameref{sec_sr_semspace}}
\end{scnreltolist}

\scnheader{метрическое конечное семантическое пространство}
\scnidtf{\textit{metric finite semantic space}}
\begin{scnreltolist}{ключевое понятие}
	\scnitem{\ref{sec_sr_semspace}~\nameref{sec_sr_semspace}}
\end{scnreltolist}

\scnheader{метрическое конечное синтаксическое пространство}
\scnidtf{\textit{metric finite syntactic space}}
\begin{scnreltolist}{ключевое понятие}
	\scnitem{\ref{sec_sr_semspace}~\nameref{sec_sr_semspace}}
\end{scnreltolist}

\scnheader{микротехнологическая операция}
\scnidtf{\textit{microtechnological operation}}
\begin{scnreltolist}{ключевое понятие}
	\scnitem{\ref{sec_chapter_enterprise_characteristics}~\nameref{sec_chapter_enterprise_characteristics}}
\end{scnreltolist}

\scnheader{минимальная конфигурация ostis-системы}
\scnidtf{\textit{ostis-system minimal configuration}}
\begin{scnreltolist}{ключевое понятие}
	\scnitem{Глава \ref{chapter_interpreter}~\nameref{chapter_interpreter}}
\end{scnreltolist}

\scnheader{многоагентная модель решения задач}
\scnidtf{\textit{multiagent problem solving model}}
\begin{scnreltolist}{ключевое понятие}
	\scnitem{Глава \ref{chapter_new_generation_systems}~\nameref{chapter_new_generation_systems}}
\end{scnreltolist}

\scnheader{многоагентная система}
\scnidtf{\textit{multiagent system}}
\scnidtf{кибернетическая система, представляющая собой коллектив взаимодействующих кибернетических систем, являющихся агентами (членами) этого коллектива}
\scnidtf{коллективная кибернетическая система}
\begin{scnreltolist}{ключевое понятие}
	\scnitem{Глава \ref{chap_intro}~\nameref{chap_intro}}
\end{scnreltolist}

\scnheader{многоагентная система обработки информации в общей памяти}
\scnidtf{\textit{multiagent system of information processing in common memory}}
\begin{scnreltolist}{ключевое понятие}
	\scnitem{Глава \ref{chapter_new_generation_systems}~\nameref{chapter_new_generation_systems}}
\end{scnreltolist}

\scnheader{многоагентный интерфейс интеллектуальной компьютерной системы нового поколения}
\scnidtf{\textit{next-generation intelligent computer system multiagent interface}}
\begin{scnreltolist}{ключевое понятие}
	\scnitem{Глава \ref{chapter_new_generation_systems}~\nameref{chapter_new_generation_systems}}
\end{scnreltolist}

\scnheader{многоагентный решатель задач}
\scnidtf{\textit{multiagent problem solver}}
\begin{scnreltolist}{ключевое понятие}
	\scnitem{Глава \ref{chapter_new_generation_systems}~\nameref{chapter_new_generation_systems}}
\end{scnreltolist}

\scnheader{многократно используемый компонент ostis-систем}
\scnidtftext{сокращение основного sc-идентификатора}{м.и.к. ostis-систем}
\scnidtf{\textit{ostis-systems reusable component}}
\begin{scnreltolist}{ключевое понятие}
	\scnitem{Глава \ref{chapter_library}~\nameref{chapter_library}}
\end{scnreltolist}

\scnheader{многократно используемый компонент решателей задач}
\scnidtf{\textit{reusable problem solver component}}
\begin{scnreltolist}{ключевое понятие}
	\scnitem{Глава \ref{chapter_ps_design}~\nameref{chapter_ps_design}}
\end{scnreltolist}

\scnheader{множество}
\scnidtf{\textit{set}}
\begin{scnreltolist}{ключевое понятие}
	\scnitem{Глава \ref{chapter_top_ontologies}~\nameref{chapter_top_ontologies}}
\end{scnreltolist}

\scnheader{модальный оператор}
\scnidtf{\textit{modal operator}}
\begin{scnreltolist}{ключевое понятие}
	\scnitem{\ref{sec_applied_logic}~\nameref{sec_applied_logic}}
\end{scnreltolist}

\scnheader{модальное правило вывода}
\scnidtf{\textit{modal inference rule}}
\begin{scnreltolist}{ключевое понятие}
	\scnitem{\ref{sec_applied_logic}~\nameref{sec_applied_logic}}
\end{scnreltolist}

\scnheader{модель решения задач}
\scnidtf{\textit{problem solving model}}
\begin{scnreltolist}{ключевое понятие}
	\scnitem{Глава \ref{chapter_actions}~\nameref{chapter_actions}}
\end{scnreltolist}

\scnheader{монотонное бинарное отношение*}
\scnidtf{\textit{monotonic binary relation*}}
\begin{scnreltolist}{ключевой отношение}
	\scnitem{\ref{sec_applied_logic}~\nameref{sec_applied_logic}}
\end{scnreltolist}

\scnheader{мощность множества}
\scnidtf{\textit{set power}}
\begin{scnreltolist}{ключевое понятие}
	\scnitem{Глава \ref{chapter_top_ontologies}~\nameref{chapter_top_ontologies}}
\end{scnreltolist}

\scnheader{мультимножество}
\scnidtf{\textit{multiset}}
\begin{scnreltolist}{ключевое понятие}
	\scnitem{Глава \ref{chapter_top_ontologies}~\nameref{chapter_top_ontologies}}
\end{scnreltolist}

\scnheader{мультимодальный интерфейс}
\scnidtf{\textit{multimodal interface}}
\begin{scnreltolist}{ключевое понятие}
    \scnitem{Глава \ref{chapter_interfaces}~\nameref{chapter_interfaces}}
\end{scnreltolist}

\scnheader{навык}
\scnidtf{\textit{skill}}
\begin{scnreltolist}{ключевое понятие}
	\scnitem{Глава \ref{chapter_actions}~\nameref{chapter_actions}}
\end{scnreltolist}

\scnheader{навык решения задач с помощью искусственных нейронных сетей}
\scnidtf{\textit{skill of problem solving using artificial neural networks}}
\begin{scnreltolist}{ключевое понятие}
	\scnitem{Глава \ref{chapter_ann}~\nameref{chapter_ann}}
\end{scnreltolist}

\scnheader{накопительный модуль}
\scnidtf{\textit{storage module}}
\begin{scnreltolist}{ключевое понятие}
	\scnitem{Глава \ref{chapter_computers}~\nameref{chapter_computers}}
\end{scnreltolist}

\scnheader{начало\scnsupergroupsign}
\scnidtf{\textit{begin\scnsupergroupsign}}
\begin{scnreltolist}{ключевой параметр}
	\scnitem{Глава \ref{chapter_top_ontologies}~\nameref{chapter_top_ontologies}}
\end{scnreltolist}

\scnheader{неатомарное действие}
\scnidtf{\textit{non-atomic action}}
\begin{scnreltolist}{ключевое отношение}
	\scnitem{\ref{sec_action_concept}~\nameref{sec_action_concept}}
\end{scnreltolist}

\scnheader{неатомарный абстрактный sc-агент}
\scnidtf{\textit{non-atomic abstract sc-agent}}
\begin{scnreltolist}{ключевое понятие}
	\scnitem{\ref{sec_ps_agents}~\nameref{sec_ps_agents}}
\end{scnreltolist}

\scnheader{небинарная связь}
\scnidtf{\textit{nonbinary sheaf}}
\begin{scnreltolist}{ключевое понятие}
	\scnitem{Глава \ref{chapter_top_ontologies}~\nameref{chapter_top_ontologies}}
\end{scnreltolist}

\scnheader{небинарное отношение}
\scnidtf{\textit{nonbinary relation}}
\begin{scnreltolist}{ключевое понятие}
	\scnitem{Глава \ref{chapter_top_ontologies}~\nameref{chapter_top_ontologies}}
\end{scnreltolist}

\scnheader{неискаженное высказывание*}
\scnidtf{\textit{undistorted statement*}}
\begin{scnreltolist}{ключевое отношение}
	\scnitem{\ref{sec_nonclass_logic}~\nameref{sec_nonclass_logic}}
\end{scnreltolist}

\scnheader{нейросетевая модель решения задач}
\scnidtf{\textit{neural network model of problem solving}}
\begin{scnreltolist}{ключевое понятие}
	\scnitem{Глава \ref{chapter_ann}~\nameref{chapter_ann}}
\end{scnreltolist}

\scnheader{нейросетевой метод решения задач}
\scnidtf{\textit{и.н.с.}}
\scnidtf{\textit{искусственная нейронная сеть}}
\scnidtf{\textit{artificial neural network}}
\scnidtf{\textit{artificial neural network method of problem solving}}
\begin{scnreltolist}{ключевое понятие}
	\scnitem{Глава \ref{chapter_ann}~\nameref{chapter_ann}}
\end{scnreltolist}

\scnheader{некорректность в scp-программе}
\scnidtf{\textit{inaccuracy in scp-program}}
\begin{scnreltolist}{ключевое понятие}
	\scnitem{\ref{chapter_ps_design}~\nameref{chapter_ps_design}}
\end{scnreltolist}

\scnheader{немонотонный вывод на конечном sc-множестве посылок}
\scnidtf{\textit{non-monotonic inference on a finite sc-set of premises}}
\begin{scnreltolist}{ключевое понятие}
	\scnitem{\ref{sec_nonclass_logic}~\nameref{sec_nonclass_logic}}
\end{scnreltolist}

\scnheader{неориентированное множество}
\scnidtf{\textit{non-oriented set}}
\begin{scnreltolist}{ключевое понятие}
	\scnitem{Глава \ref{chapter_top_ontologies}~\nameref{chapter_top_ontologies}}
\end{scnreltolist}

\scnheader{неосновной sc-идентификатор}
\scnidtf{\textit{auxiliary sc-identifier}}
\begin{scnreltolist}{ключевое понятие}
	\scnitem{\ref{sec_identifier_concept}~\nameref{sec_identifier_concept}}
\end{scnreltolist}

\scnheader{непроцедурный язык представления методов}
\scnidtf{\textit{non-procedural method representation language}}
\begin{scnreltolist}{ключевое понятие}
	\scnitem{\ref{sec_method_lang_classes}~\nameref{sec_method_lang_classes}}
\end{scnreltolist}

\scnheader{неролевое отношение}
\scnidtf{\textit{norole relation}}
\begin{scnreltolist}{ключевое понятие}
	\scnitem{Глава \ref{chapter_top_ontologies}~\nameref{chapter_top_ontologies}}
\end{scnreltolist}

\scnheader{неслотовое бинарное отношение}
\scnidtf{\textit{non-slot binary relation}}
\begin{scnreltolist}{ключевое понятие}
	\scnitem{\ref{sec_applied_logic}~\nameref{sec_applied_logic}}
\end{scnreltolist}

\scnheader{нестроковый sc-идентификатор}
\scnidtf{\textit{non-string sc-identifier}}
\begin{scnreltolist}{ключевое понятие}
	\scnitem{\ref{sec_main_and_system_identifiers_concept}~\nameref{sec_main_and_system_identifiers_concept}}
\end{scnreltolist}

\scnheader{нетранслируемый sc-идентификатор}
\scnidtf{\textit{non-translatable sc-identifier}}
\begin{scnreltolist}{ключевое понятие}
	\scnitem{\ref{sec_non_translation_identifier_concept}~\nameref{sec_non_translation_identifier_concept}}
\end{scnreltolist}

\scnheader{неточная величина}
\scnidtf{\textit{inexact value}}
\begin{scnreltolist}{ключевое понятие}
	\scnitem{Глава \ref{chapter_top_ontologies}~\nameref{chapter_top_ontologies}}
\end{scnreltolist}

\scnheader{нефактографическое высказывание}
\scnidtf{\textit{non-factual statement}}
\begin{scnreltolist}{ключевое понятие}
	\scnitem{Глава \ref{chapter_logic}~\nameref{chapter_logic}}
\end{scnreltolist}

\scnheader{нечеткая истинность*}
\scnidtf{\textit{fuzzy truth*}}
\begin{scnreltolist}{ключевое отношение}
	\scnitem{\ref{sec_nonclass_logic}~\nameref{sec_nonclass_logic}}
\end{scnreltolist}

\scnheader{область определения\scnrolesign}
\scnidtf{\textit{definitional domain\scnrolesign}}
\begin{scnreltolist}{ключевое отношение}
	\scnitem{Глава \ref{chapter_top_ontologies}~\nameref{chapter_top_ontologies}}
\end{scnreltolist}

\scnheader{область прибытия\scnrolesign}
\scnidtf{\textit{output set\scnrolesign}}
\begin{scnreltolist}{ключевое отношение}
	\scnitem{Глава \ref{chapter_top_ontologies}~\nameref{chapter_top_ontologies}}
\end{scnreltolist}

\scnheader{Обобщенный жизненный цикл ostis-систем}
\scnidtf{\textit{Generalized ostis-systems life cycle}}
\begin{scnreltolist}{ключевой знак}
	\scnitem{Глава \ref{chapter_ostis_tech}~\nameref{chapter_ostis_tech}}
\end{scnreltolist}

\scnheader{обозначение внешней сущности}
\scnidtf{\textit{external entity designation}}
\begin{scnreltolist}{ключевое понятие}
	\scnitem{Глава \ref{chapter_sc_code}~\nameref{chapter_sc_code}}
\end{scnreltolist}

\scnheader{обозначение sc-класса}
\scnidtf{\textit{sc-class designation}}
\begin{scnreltolist}{ключевое понятие}
	\scnitem{Глава \ref{chapter_sc_code}~\nameref{chapter_sc_code}}
\end{scnreltolist}

\scnheader{обозначение sc-множества}
\scnidtf{\textit{sc-set designation}}
\begin{scnreltolist}{ключевое понятие}
	\scnitem{Глава \ref{chapter_sc_code}~\nameref{chapter_sc_code}}
\end{scnreltolist}

\scnheader{обозначение sc-связки}
\scnidtf{\textit{sc-sheaf designation}}
\begin{scnreltolist}{ключевое понятие}
	\scnitem{Глава \ref{chapter_sc_code}~\nameref{chapter_sc_code}}
\end{scnreltolist}

\scnheader{обозначение sc-структуры}
\scnidtf{\textit{sc-structure designation}}
\begin{scnreltolist}{ключевое понятие}
	\scnitem{Глава \ref{chapter_sc_code}~\nameref{chapter_sc_code}}
\end{scnreltolist}

\scnheader{объединение*}
\scnidtf{\textit{union*}}
\begin{scnreltolist}{ключевое отношение}
	\scnitem{Глава \ref{chapter_top_ontologies}~\nameref{chapter_top_ontologies}}
\end{scnreltolist}

\scnheader{объект\scnrolesign}
\scnidtf{\textit{object\scnrolesign}}
\begin{scnreltolist}{ключевое отношение}
	\scnitem{\ref{sec_action_concept}~\nameref{sec_action_concept}}
\end{scnreltolist}

\scnheader{объект местности}
\scnidtf{\textit{terrain object}}
\begin{scnreltolist}{ключевое понятие}
	\scnitem{Глава \ref{chapter_gis}~\nameref{chapter_gis}}
\end{scnreltolist}

\scnheader{онтология}
\scnidtf{\textit{ontology}}
\begin{scnreltolist}{ключевое понятие}
%	\scnitem{Глава \ref{chapter_new_generation_systems}~\nameref{chapter_new_generation_systems}}
	\scnitem{Глава \ref{chapter_top_ontologies}~\nameref{chapter_top_ontologies}}
	\scnitem{\ref{sec_ontology}~\nameref{sec_ontology}}
\end{scnreltolist}

\scnheader{онтология верхнего уровня}
\scnidtf{\textit{top-level ontology}}
\begin{scnreltolist}{ключевое понятие}
	\scnitem{Глава \ref{chapter_top_ontologies}~\nameref{chapter_top_ontologies}}
	\scnitem{\ref{sec_top_level_ontologies}~\nameref{sec_top_level_ontologies}}
\end{scnreltolist}

\scnheader{операционная семантика метода*}
\scnidtf{\textit{operational semantics of method*}}
\begin{scnreltolist}{ключевое отношение}
	\scnitem{Глава \ref{chapter_programs}~\nameref{chapter_programs}}
	\scnitem{\ref{sec_skill_concept}~\nameref{sec_skill_concept}}
\end{scnreltolist}

\scnheader{операционная семантика языка представления методов*}
\scnidtf{\textit{operational semantics of method representation language*}}
\begin{scnreltolist}{ключевое отношение}
	\scnitem{Глава \ref{chapter_programs}~\nameref{chapter_programs}}
\end{scnreltolist}

\scnheader{оптическая система компьютерного зрения}	
\scnidtf{\textit{optical computer vision system}}
\begin{scnreltolist}{ключевое понятие}
	\scnitem{\ref{sec_3d_models_computervision}~\nameref{sec_3d_models_computervision}}
\end{scnreltolist}

\scnheader{ориентированное множество}
\scnidtf{\textit{oriented set}}
\begin{scnreltolist}{ключевое понятие}
	\scnitem{Глава \ref{chapter_top_ontologies}~\nameref{chapter_top_ontologies}}
\end{scnreltolist}

\scnheader{основной sc-идентификатор}
\scnidtf{\textit{main sc-identifier}}
\begin{scnreltolist}{ключевое понятие}
	\scnitem{\ref{sec_main_and_system_identifiers_concept}~\nameref{sec_main_and_system_identifiers_concept}}
\end{scnreltolist}

\scnheader{ответ на вопрос}
\scnidtf{\textit{question answer}}
\begin{scnreltolist}{ключевое понятие}
	\scnitem{Глава \ref{chapter_requests}~\nameref{chapter_requests}}
\end{scnreltolist}

\scnheader{отношение}
\scnidtf{\textit{relation}}
\begin{scnreltolist}{ключевое понятие}
	\scnitem{Глава \ref{chapter_top_ontologies}~\nameref{chapter_top_ontologies}}
\end{scnreltolist}

\scnheader{отношение порядка}
\scnidtf{\textit{order relation}}
\begin{scnreltolist}{ключевое понятие}
	\scnitem{Глава \ref{chapter_top_ontologies}~\nameref{chapter_top_ontologies}}
\end{scnreltolist}

\scnheader{отслеживание движения глаз}
\scnidtf{\textit{eye tracking}}
\begin{scnreltolist}{ключевое понятие}
	\scnitem{Глава \ref{chapter_ui_design}~\nameref{chapter_ui_design}}
\end{scnreltolist}

\scnheader{Отношение выводимости}
\scnidtf{\textit{Inferability relation}}
\begin{scnreltolist}{ключевой знак}
	\scnitem{\ref{sec_applied_logic}~\nameref{sec_applied_logic}}
\end{scnreltolist}

\scnheader{Отношение выводимости на конечных множествах}
\scnidtf{\textit{Inferability relation on finite sets}}
\begin{scnreltolist}{ключевой знак}
	\scnitem{\ref{sec_applied_logic}~\nameref{sec_applied_logic}}
\end{scnreltolist}

\scnheader{Отношение выводимости на конечных множествах полносвязно представленных множеств}
\scnidtf{\textit{Inferability relation on finite sets of fully connected sets}}
\begin{scnreltolist}{ключевой знак}
	\scnitem{\ref{sec_applied_logic}~\nameref{sec_applied_logic}}
\end{scnreltolist}

\scnheader{Отношение выводимости на секвенциях}
\scnidtf{\textit{Inferability relation on sequences}}
\begin{scnreltolist}{ключевой знак}
	\scnitem{\ref{sec_applied_logic}~\nameref{sec_applied_logic}}
\end{scnreltolist}

\scnheader{отношение, заданное на множестве знаний}
\scnidtf{\textit{relation defined on a set of knowledge}}
\begin{scnreltolist}{ключевое понятие}
	\scnitem{\ref{sec_kb}~\nameref{sec_kb}}
\end{scnreltolist}

\scnheader{отношение становления структур}
\scnidtf{\textit{relation of structure formation}}
\begin{scnreltolist}{ключевое понятие}
	\scnitem{\ref{sec_applied_logic}~\nameref{sec_applied_logic}}
\end{scnreltolist}

\scnheader{ошибка в scp-программе}	
\scnidtf{\textit{error in scp-program}}
\begin{scnreltolist}{ключевое понятие}
	\scnitem{Глава \ref{chapter_ps_design}~\nameref{chapter_ps_design}}
\end{scnreltolist}

\scnheader{пакетный менеджер}
\scnidtf{\textit{package manager}}
\begin{scnreltolist}{ключевое понятие}
	\scnitem{\ref{ostis_library_analysis}~\nameref{ostis_library_analysis}}
\end{scnreltolist}

\scnheader{параметр}
\scnidtf{\textit{parameter}}
\begin{scnreltolist}{ключевое понятие}
	\scnitem{Глава \ref{chap_intro}~\nameref{chap_intro}}
\end{scnreltolist}

\scnheader{параметр, заданный на множестве кибернетических систем}
%\scnidtf{\textit{parameter}}
\scnidtf{\textit{parameter defined on cybernetic systems set}}
\scnhaselement{качество физической оболочки кибернетической системы\scnsupergroupsign}
\scnhaselement{качество хранимой информации\scnsupergroupsign}
\begin{scnindent}
	\scnidtf{качество информации, хранимой в памяти кибернетической системы\scnsupergroupsign}
\end{scnindent} 
\scnhaselement{качество решателя задач\scnsupergroupsign}
\scnhaselement{качество интерфейса\scnsupergroupsign}
\scnhaselement{обучаемость\scnsupergroupsign}
\scnhaselement{гибкость кибернетической системы\scnsupergroupsign}
\scnhaselement{стратифицированность кибернетической системы\scnsupergroupsign}
\scnhaselement{рефлексивность кибернетической системы\scnsupergroupsign}
\scnhaselement{уровень эволюционных ограничений\scnsupergroupsign}
\begin{scnreltolist}{ключевое понятие}
	\scnitem{Глава \ref{chap_intro}~\nameref{chap_intro}}
\end{scnreltolist}

\scnheader{параметр scp-программы\scnrolesign}
\scnidtf{\textit{scp-program parameter\scnrolesign}}
\begin{scnreltolist}{ключевое отношение}
	\scnitem{\ref{sec_ps_scp}~\nameref{sec_ps_scp}}
\end{scnreltolist}

\scnheader{пассивный навык}
\scnidtf{\textit{passive skill}}
\begin{scnreltolist}{ключевое понятие}
	\scnitem{\ref{sec_skill_concept}~\nameref{sec_skill_concept}}
\end{scnreltolist}

\scnheader{пересечение*}
\scnidtf{\textit{intersection*}}
\begin{scnreltolist}{ключевое отношение}
	\scnitem{Глава \ref{chapter_top_ontologies}~\nameref{chapter_top_ontologies}}
\end{scnreltolist}

\scnheader{персональный ассистент}
\scnidtf{\textit{personal assistant}}
\begin{scnreltolist}{ключевое понятие}
	\scnitem{\ref{sec_ostis_assistant}~\nameref{sec_ostis_assistant}}
\end{scnreltolist}

\scnheader{персональный ostis-ассистент}
\scnidtf{\textit{personal ostis-assistant}}
\begin{scnreltolist}{ключевое понятие}
	\scnitem{\ref{sec_ostis_assistant}~\nameref{sec_ostis_assistant}}
\end{scnreltolist}

\scnheader{планируемая блокировка*}
\scnidtf{\textit{planned lock*}}
\begin{scnreltolist}{ключевое отношение}
	\scnitem{\ref{sec_ps_sync}~\nameref{sec_ps_sync}}
\end{scnreltolist}

\scnheader{поведенческая задача}
\scnidtf{\textit{behavioral problem}}
\begin{scnreltolist}{ключевое понятие}
	\scnitem{\ref{sec_problem_concept}~\nameref{sec_problem_concept}}
\end{scnreltolist}

\scnheader{поведенческое действие}
\scnidtf{\textit{behavioral action}}
\begin{scnreltolist}{ключевое понятие}
	\scnitem{\ref{sec_action_concept}~\nameref{sec_action_concept}}
\end{scnreltolist}

\scnheader{подметод*}
\scnidtf{\textit{submethod*}}
\begin{scnreltolist}{ключевое отношение}
	\scnitem{\ref{sec_method_concept}~\nameref{sec_method_concept}}
\end{scnreltolist}

\scnheader{подформула*}
\scnidtf{\textit{subformula*}}
\begin{scnreltolist}{ключевое отношение}
	\scnitem{Глава \ref{chapter_logic}~\nameref{chapter_logic}}
\end{scnreltolist}

\scnheader{полная семантическая окрестность элемента*}
\scnidtf{\textit{full semantic neighborhood of element*}}
\begin{scnreltolist}{ключевое отношение}
	\scnitem{\ref{sec_sr_semspace}~\nameref{sec_sr_semspace}}
\end{scnreltolist}

\scnheader{пользовательский интерфейс}
\scnidtf{\textit{user interface}}
\begin{scnreltolist}{ключевое понятие}
	\scnitem{Глава \ref{chapter_interfaces}~\nameref{chapter_interfaces}}
\end{scnreltolist}

\scnheader{пользователь Экосистемы OSTIS}
\scnidtf{\textit{OSTIS Ecosystem user}}
\begin{scnreltolist}{ключевое понятие}
	\scnitem{\ref{sec_ecosystem_structure_description}~\nameref{sec_ecosystem_structure_description}}
\end{scnreltolist}

\scnheader{понятие, переходящее из основного в неосновное}
\scnidtf{\textit{concept that transitions from main to auxiliary}}
\begin{scnreltolist}{ключевое понятие}
	\scnitem{Глава \ref{chapter_top_ontologies}~\nameref{chapter_top_ontologies}}
\end{scnreltolist}

\scnheader{понятие, переходящее из неосновного в основное}
\scnidtf{\textit{concept that transitions from auxiliary to main}}
\begin{scnreltolist}{ключевое понятие}
	\scnitem{Глава \ref{chapter_top_ontologies}~\nameref{chapter_top_ontologies}}
\end{scnreltolist}

\scnheader{последовательность мышления}
\scnidtf{\textit{thinking sequence}}
\begin{scnreltolist}{ключевое понятие}
	\scnitem{\ref{sec_applied_logic}~\nameref{sec_applied_logic}}
\end{scnreltolist}

\scnheader{портал знаний}
\scnidtf{\textit{knowledge portal}}
\begin{scnreltolist}{ключевое понятие}
    \scnitem{\ref{sec_ostis_scientific_portal}~\nameref{sec_ostis_scientific_portal}}
\end{scnreltolist}

\scnheader{Правила построения нетранслируемых sc-идентификаторов}
\scnidtf{\textit{Non-translatable sc-identifier construction rules}}
\begin{scnreltolist}{ключевое знание}
	\scnitem{Глава \ref{sec_non_translation_identifier_concept}~\nameref{sec_non_translation_identifier_concept}}
\end{scnreltolist}

\scnheader{Правила построения основных sc-идентификаторов}
\scnidtf{\textit{Main sc-identifier construction rules}}
\begin{scnreltolist}{ключевое знание}
	\scnitem{\ref{sec_main_and_system_identifiers_concept}~\nameref{sec_main_and_system_identifiers_concept}}
\end{scnreltolist}

\scnheader{Правила построения простых sc-идентификаторов}
\scnidtf{\textit{Simple sc-identifier construction rules}}
\begin{scnreltolist}{ключевое знание}
	\scnitem{\ref{sec_simple_identifier_concept}~\nameref{sec_simple_identifier_concept}}
\end{scnreltolist}

\scnheader{Правила построения системных sc-идентификаторов}
\scnidtf{\textit{System sc-identifier construction rules}}
\begin{scnreltolist}{ключевое знание}
	\scnitem{\ref{sec_main_and_system_identifiers_concept}~\nameref{sec_main_and_system_identifiers_concept}}
\end{scnreltolist}

\scnheader{Правила построения сложных sc-идентификаторов}
\scnidtf{\textit{Complex sc-identifier construction rules}}
\begin{scnreltolist}{ключевое знание}
	\scnitem{\ref{sec_complex_identifier_concept}~\nameref{sec_complex_identifier_concept}}
\end{scnreltolist}

\scnheader{Правила построения sc-идентификаторов}
\scnidtf{\textit{Sc-identifier construction rules}}
\begin{scnreltolist}{ключевое знание}
	\scnitem{\ref{sec_identifier_concept}~\nameref{sec_identifier_concept}}
\end{scnreltolist}

\scnheader{предметная область}
\scnidtf{\textit{subject domain}}
\begin{scnreltolist}{ключевое понятие}
	\scnitem{Глава \ref{chapter_kb}~\nameref{chapter_kb}}
	\scnitem{Глава \ref{chapter_top_ontologies}~\nameref{chapter_top_ontologies}}
	\scnitem{\ref{sec_sd}~\nameref{sec_sd}}
\end{scnreltolist}

\scnheader{приложение умного дома}
\scnidtf{\textit{smart home application}}
\begin{scnreltolist}{ключевое понятие}
	\scnitem{Глава \ref{chapter_smart_home}~\nameref{chapter_smart_home}}
\end{scnreltolist}

\scnheader{принадлежность*}
\scnidtf{\textit{belonging*}}
\begin{scnreltolist}{ключевое отношение}
	\scnitem{Глава \ref{chapter_top_ontologies}~\nameref{chapter_top_ontologies}}
\end{scnreltolist}

\scnheader{Принципы, лежащие в основе Технологии OSTIS}
\scnidtf{\textit{OSTIS Technology principles}}
\begin{scnreltolist}{ключевое знание}
	\scnitem{Глава \ref{chapter_ostis_tech}~\nameref{chapter_ostis_tech}}
\end{scnreltolist}

\scnheader{приоритет блокировки*}
\scnidtf{\textit{lock priority*}}
\begin{scnreltolist}{ключевое отношение}
	\scnitem{\ref{sec_ps_sync}~\nameref{sec_ps_sync}}
\end{scnreltolist}

\scnheader{Программный вариант реализации ostis-платформы}
\scnidtf{\textit{Программная платформа для ostis-систем}}
\scnidtf{\textit{Software version of ostis-platform implementation}}
\scnidtf{\textit{Software platform for ostis-systems}}
\begin{scnreltolist}{ключевое понятие}
	\scnitem{Глава \ref{chapter_interpreter}~\nameref{chapter_interpreter}}
\end{scnreltolist}

\scnheader{программный вариант ostis-платформы}
\scnidtf{\textit{software version of ostis-platform}}
\begin{scnreltolist}{ключевое понятие}
	\scnitem{Глава \ref{chapter_interpreter}~\nameref{chapter_interpreter}}
\end{scnreltolist}

\scnheader{программный интерфейс}
\scnidtf{\textit{application interface}}
\scnidtf{\textit{API}}
\begin{scnreltolist}{ключевое понятие}
    \scnitem{Глава \ref{chapter_interfaces}~\nameref{chapter_interfaces}}
\end{scnreltolist}

\scnheader{Программный интерфейс Реализации sc-памяти в ostis-платформе}
\scnidtf{\textit{Application interface of SC-memory implementation in ostis-platform}}
\begin{scnreltolist}{ключевое понятие}
	\scnitem{Глава \ref{chapter_soft_platform}~\nameref{chapter_soft_platform}}
\end{scnreltolist}

\scnheader{проектирование}
\scnidtf{\textit{design}}
\begin{scnreltolist}{ключевое понятие}
	\scnitem{\ref{sec_activity_and_technology}~\nameref{sec_activity_and_technology}}
\end{scnreltolist}

\scnheader{производство}
\scnidtf{\textit{production}}
\begin{scnreltolist}{ключевое понятие}
	\scnitem{\ref{sec_activity_and_technology}~\nameref{sec_activity_and_technology}}
\end{scnreltolist}

\scnheader{произведение*}
\scnidtf{\textit{multiplication*}}
\begin{scnreltolist}{ключевое отношение}
	\scnitem{Глава \ref{chapter_top_ontologies}~\nameref{chapter_top_ontologies}}
\end{scnreltolist}

\scnheader{простой sc-идентификатор}
\scnidtf{\textit{simple sc-identifier}}
\begin{scnreltolist}{ключевое понятие}
	\scnitem{\ref{sec_simple_identifier_concept}~\nameref{sec_simple_identifier_concept}}
\end{scnreltolist}

\scnheader{пространственное отношение}
\scnidtf{\textit{space relation}}
\begin{scnreltolist}{ключевое понятие}
	\scnitem{Глава \ref{chapter_gis}~\nameref{chapter_gis}}
\end{scnreltolist}

\scnheader{процедурная формулировка задачи}
\scnidtf{\textit{procedural problem definition}}
\begin{scnreltolist}{ключевое понятие}
	\scnitem{\ref{sec_problem_concept}~\nameref{sec_problem_concept}}
\end{scnreltolist}

\scnheader{процедурная формулировка задачи*}
\scnidtf{\textit{procedural problem definition*}}
\begin{scnreltolist}{ключевое отношение}
	\scnitem{\ref{sec_problem_concept}~\nameref{sec_problem_concept}}
\end{scnreltolist}

\scnheader{процедурный язык представления методов}
\scnidtf{\textit{procedural method representation language}}
\begin{scnreltolist}{ключевое понятие}
	\scnitem{\ref{sec_method_lang_classes}~\nameref{sec_method_lang_classes}}
\end{scnreltolist}

\scnheader{процесс}
\scnidtf{\textit{process}}
\begin{scnreltolist}{ключевое понятие}
	\scnitem{Глава \ref{chapter_top_ontologies}~\nameref{chapter_top_ontologies}}
\end{scnreltolist}

\scnheader{процессорный модуль}
\scnidtf{\textit{processor module}}
\begin{scnreltolist}{ключевое понятие}
	\scnitem{Глава \ref{chapter_computers}~\nameref{chapter_computers}}
\end{scnreltolist}

\scnheader{процессорный элемент}
\scnidtf{\textit{processor element}}
\begin{scnreltolist}{ключевое понятие}
	\scnitem{Глава \ref{chapter_computers}~\nameref{chapter_computers}}
\end{scnreltolist}

\scnheader{псевдометрика}
\scnidtf{\textit{pseudometric}}
\begin{scnreltolist}{ключевое понятие}
	\scnitem{\ref{sec_sr_semspace}~\nameref{sec_sr_semspace}}
\end{scnreltolist}

\scnheader{псевдометрическое пространство}
\scnidtf{\textit{pseudometric space}}
\begin{scnreltolist}{ключевое понятие}
	\scnitem{\ref{sec_sr_semspace}~\nameref{sec_sr_semspace}}
\end{scnreltolist}

\scnheader{псевдометрическое конечное семантическое пространство}
\scnidtf{\textit{pseudometric finite semantic space}}
\begin{scnreltolist}{ключевое понятие}
	\scnitem{\ref{sec_sr_semspace}~\nameref{sec_sr_semspace}}
\end{scnreltolist}

\scnheader{разбиение*}
\scnidtf{\textit{partition*}}
\scnidtf{\textit{subdividing*}}
\begin{scnreltolist}{ключевое отношение}
	\scnitem{Глава \ref{chapter_top_ontologies}~\nameref{chapter_top_ontologies}}
\end{scnreltolist}

\scnheader{разработка плана производства}
\scnidtf{\textit{production plan development}}
\begin{scnreltolist}{ключевое понятие}
	\scnitem{\ref{sec_activity_and_technology}~\nameref{sec_activity_and_technology}}
\end{scnreltolist}

\scnheader{распределенная система}
\scnidtf{\textit{distributed system}}
\begin{scnreltolist}{ключевое понятие}
	\scnitem{Глава \ref{sec_ecosystem_structure}~\nameref{sec_ecosystem_structure}}
\end{scnreltolist}

\scnheader{расширенная ostis-платформа}
\scnidtf{\textit{extended ostis-platform}}
\begin{scnreltolist}{ключевое понятие}
	\scnitem{Глава \ref{chapter_interpreter}~\nameref{chapter_interpreter}}
\end{scnreltolist}

\scnheader{Реализация интерпретатора sc-моделей пользовательских интерфейсов}
\scnidtf{\textit{Implementation of user interfaces sc-models}}
\begin{scnreltolist}{ключевое понятие}
	\scnitem{Глава \ref{chapter_soft_platform}~\nameref{chapter_soft_platform}}
\end{scnreltolist}

\scnheader{Реализация менеджера многократно используемых компонентов ostis-систем в ostis-платформе}
\scnidtf{\textit{Implementation of ostis-systems reusable component manager in ostis-platform}}
\begin{scnreltolist}{ключевое понятие}
	\scnitem{Глава \ref{chapter_library}~\nameref{chapter_library}}
	\scnitem{Глава \ref{chapter_soft_platform}~\nameref{chapter_soft_platform}}
\end{scnreltolist}

\scnheader{Реализация памяти в ostis-платформе}
\scnidtf{\textit{Memory implementation in ostis-platform}}
\begin{scnreltolist}{ключевое понятие}
	\scnitem{Глава \ref{chapter_soft_platform}~\nameref{chapter_soft_platform}}
\end{scnreltolist}

\scnheader{Реализация подсистемы взаимодействия ostis-платформы с внешней средой}
\scnidtf{\textit{Implementation of subsystem of interaction between ostis-platform and external environment}}
\begin{scnreltolist}{ключевое понятие}
	\scnitem{Глава \ref{chapter_soft_platform}~\nameref{chapter_soft_platform}}
\end{scnreltolist}

\scnheader{Реализация sc-памяти в ostis-платформе}
\scnidtf{\textit{SC-memory implementation in ostis-platform}}
\begin{scnreltolist}{ключевое понятие}
	\scnitem{Глава \ref{chapter_soft_platform}~\nameref{chapter_soft_platform}}
\end{scnreltolist}

\scnheader{Реализация файловой памяти в ostis-платформе}
\scnidtf{\textit{File memory implementation in ostis-platform}}
\begin{scnreltolist}{ключевое понятие}
	\scnitem{Глава \ref{chapter_soft_platform}~\nameref{chapter_soft_platform}}
\end{scnreltolist}

\scnheader{реинжиниринг}
\scnidtf{\textit{reengineering}}
\begin{scnreltolist}{ключевое понятие}
	\scnitem{\ref{sec_activity_and_technology}~\nameref{sec_activity_and_technology}}
\end{scnreltolist}

\scnheader{рецепторное действие}
\scnidtf{\textit{receptor action}}
\begin{scnreltolist}{ключевое понятие}
	\scnitem{\ref{sec_action_concept}~\nameref{sec_action_concept}}
\end{scnreltolist}

\scnheader{рецептурное производство}
\scnidtf{\textit{product formulation}}
\begin{scnreltolist}{ключевое понятие}
	\scnitem{\ref{sec_chapter_enterprise_smart_industry}~\nameref{sec_chapter_enterprise_smart_industry}}
\end{scnreltolist}

\scnheader{речевой интерфейс}
\scnidtf{\textit{speech interface}}
\begin{scnreltolist}{ключевое понятие}
	\scnitem{Глава \ref{chapter_nl_interfaces}~\nameref{chapter_nl_interfaces}}
	\scnitem{Глава \ref{chapter_audio_interfaces}~\nameref{chapter_audio_interfaces}}
\end{scnreltolist}

\scnheader{речевой сигнал}
\scnidtf{\textit{speech signal}}
\begin{scnreltolist}{ключевое понятие}
	\scnitem{Глава \ref{chapter_audio_interfaces}~\nameref{chapter_audio_interfaces}}
\end{scnreltolist}

\scnheader{решатель задач пользовательского интерфейса ostis-систем}
\scnidtf{\textit{ostis-system user interface problem solver}}
\begin{scnreltolist}{ключевое понятие}
    \scnitem{Глава \ref{chapter_interfaces}~\nameref{chapter_interfaces}}
\end{scnreltolist}

\scnheader{решатель задач ostis-системы}
\scnidtf{\textit{ostis-system problem solver}}
\begin{scnreltolist}{ключевое понятие}
	\scnitem{Глава \ref{chapter_situation_management}~\nameref{chapter_situation_management}}
	\scnitem{Глава \ref{chapter_ps_design}~\nameref{chapter_ps_design}}
\end{scnreltolist}

\scnheader{решатель задач}
\scnidtftext{сокращение основного sc-идентификатора}{р.з.}
\scnidtf{\textit{problem solver}}
\begin{scnreltolist}{ключевое понятие}
	\scnitem{Глава \ref{chapter_situation_management}~\nameref{chapter_situation_management}}
\end{scnreltolist}

\scnheader{ролевое отношение}
\scnidtf{\textit{role relation}}
\begin{scnreltolist}{ключевое понятие}
	\scnitem{Глава \ref{chapter_top_ontologies}~\nameref{chapter_top_ontologies}}
\end{scnreltolist}

\scnheader{самостоятельная ostis-система}
\scnidtf{\textit{independent ostis-system}}
\begin{scnreltolist}{ключевое понятие}
    \scnitem{Глава \ref{sec_ecosystem_structure}~\nameref{sec_ecosystem_structure}}
\end{scnreltolist}

\scnheader{связь}
\scnidtf{\textit{sheaf}}
\begin{scnreltolist}{ключевое понятие}
	\scnitem{Глава \ref{chapter_top_ontologies}~\nameref{chapter_top_ontologies}}
\end{scnreltolist}

\scnheader{секвенция}
\scnidtf{\textit{sequence}}
\begin{scnreltolist}{ключевое понятие}
	\scnitem{\ref{sec_applied_logic}~\nameref{sec_applied_logic}}
\end{scnreltolist}

\scnheader{семантическая метрика}
\scnidtf{\textit{semantic metric}}
\begin{scnreltolist}{ключевое понятие}
	\scnitem{\ref{sec_sr_semspace}~\nameref{sec_sr_semspace}}
\end{scnreltolist}

\scnheader{семантическая окрестность}
\scnidtf{\textit{semantic neighborhood}}
\begin{scnreltolist}{ключевое понятие}
	\scnitem{\ref{sec_sem_neighborhood}~\nameref{sec_sem_neighborhood}}
\end{scnreltolist}

\scnheader{семантическая сеть}
\scnidtf{\textit{semantic network}}
\scnsuperset{рафинированная семантическая сеть}
\scnsuperset{иерархическая семантическая сеть}
\begin{scnindent}
	\scnidtf{семантическая сеть, являющаяся метаграфовой}
\end{scnindent} 
\begin{scnreltolist}{ключевое понятие}
	\scnitem{Глава \ref{chapter_new_generation_systems}~\nameref{chapter_new_generation_systems}}
\end{scnreltolist}

\scnheader{Семантическая теория программ для ostis-систем}
\scnidtf{\textit{Semantic program theory for ostis-systems}}
\begin{scnreltolist}{ключевой знак}
	\scnitem{Глава \ref{chapter_programs}~\nameref{chapter_programs}}
\end{scnreltolist}

\scnheader{семантический электронный учебник}
\scnidtf{\textit{semantic electronic handbook}}
\begin{scnreltolist}{ключевое понятие}
	\scnitem{Глава \ref{chapter_learning_systems}~\nameref{chapter_learning_systems}}
\end{scnreltolist}

\scnheader{семейство множеств}
\scnidtf{\textit{set of sets}}
\begin{scnreltolist}{ключевое понятие}
	\scnitem{Глава \ref{chapter_top_ontologies}~\nameref{chapter_top_ontologies}}
\end{scnreltolist}

\scnheader{сервис}
\scnidtf{\textit{service}}
\begin{scnreltolist}{ключевое понятие}
    \scnitem{Глава \ref{chapter_integration}~\nameref{chapter_integration}}
\end{scnreltolist}

\scnheader{сигнал}
\scnidtf{\textit{signal}}
\begin{scnreltolist}{ключевое понятие}
	\scnitem{\ref{chapter_audio_interfaces}~\nameref{chapter_audio_interfaces}}
\end{scnreltolist}

\scnheader{Синтаксис SCg-кода}
\scnidtf{\textit{SCg-code syntax}}
\begin{scnreltolist}{ключевой знак}
	\scnitem{\ref{sec_scg_syntax}~\nameref{sec_scg_syntax}}
\end{scnreltolist}

\scnheader{Синтаксис SCn-кода}
\scnidtf{\textit{SCn-code syntax}}
\begin{scnreltolist}{ключевой знак}
	\scnitem{\ref{sec_scn_syntax}~\nameref{sec_scn_syntax}}
\end{scnreltolist}

\scnheader{Синтаксис SCs-кода}
\scnidtf{\textit{SCs-code syntax}}
\begin{scnreltolist}{ключевой знак}
	\scnitem{\ref{sec_scs_syntax}~\nameref{sec_scs_syntax}}
\end{scnreltolist}

\scnheader{синтаксис языка}
\scnidtf{\textit{language syntax}}
\begin{scnreltolist}{ключевое понятие}
	\scnitem{Глава \ref{chapter_inf_constr}~\nameref{chapter_inf_constr}}
\end{scnreltolist}

\scnheader{синтаксис языка представления методов*}
\scnidtf{\textit{syntax of method representation language*}}
\begin{scnreltolist}{ключевое отношение}
	\scnitem{Глава \ref{chapter_actions}~\nameref{chapter_actions}}
	\scnitem{Глава \ref{chapter_programs}~\nameref{chapter_programs}}
\end{scnreltolist}

\scnheader{синтаксическая группа}
\scnidtf{\textit{phrase}}
\begin{scnreltolist}{ключевое понятие}
	\scnitem{Глава \ref{chapter_lang}~\nameref{chapter_lang}}
\end{scnreltolist}

\scnheader{система компьютерной алгебры}
\scnidtf{\textit{c.к.а.}}
\scnidtf{\textit{computer algebra system}}
\begin{scnreltolist}{ключевое понятие}
	\scnitem{\ref{sec_integration_algebra}~\nameref{sec_integration_algebra}}
\end{scnreltolist}

\scnheader{система компьютерной математики}
\scnidtf{\textit{c.к.м.}}
\scnidtf{\textit{computer mathematics system}}
\begin{scnreltolist}{ключевое понятие}
	\scnitem{\ref{sec_integration_algebra}~\nameref{sec_integration_algebra}}
\end{scnreltolist}

\scnheader{система локального позиционирования}	
\scnidtf{\textit{real-time locating system}}
\begin{scnreltolist}{ключевое понятие}
	\scnitem{\ref{sec_3d_models_positioning}~\nameref{sec_3d_models_positioning}}
\end{scnreltolist}

\scnheader{системный sc-идентификатор}
\scnidtf{\textit{system sc-identifier}}
\begin{scnreltolist}{ключевое понятие}
	\scnitem{\ref{sec_main_and_system_identifiers_concept}~\nameref{sec_main_and_system_identifiers_concept}}
\end{scnreltolist}

\scnheader{ситуация}	
\scnidtf{\textit{situation}}
\begin{scnreltolist}{ключевое понятие}
	\scnitem{Глава \ref{chapter_top_ontologies}~\nameref{chapter_top_ontologies}}
\end{scnreltolist}

\scnheader{словоформа}
\scnidtf{\textit{word form}}
\begin{scnreltolist}{ключевое понятие}
	\scnitem{Глава \ref{chapter_lang}~\nameref{chapter_lang}}
\end{scnreltolist}

\scnheader{слотовое бинарное отношение}
\scnidtf{\textit{slot binary relation}}
\begin{scnreltolist}{ключевое понятие}
	\scnitem{\ref{sec_applied_logic}~\nameref{sec_applied_logic}}
\end{scnreltolist}

\scnheader{смысловое представление информации}
\scnidtf{\textit{semantic representation of information}}
\begin{scnreltolist}{ключевое понятие}
	\scnitem{Глава \ref{chapter_new_generation_systems}~\nameref{chapter_new_generation_systems}}
	\scnitem{Глава \ref{chapter_inf_constr}~\nameref{chapter_inf_constr}}
\end{scnreltolist}

\scnheader{событие в sc-памяти}
\scnidtf{\textit{event in sc-memory}}
\begin{scnreltolist}{ключевое понятие}
	\scnitem{Глава \ref{chapter_top_ontologies}~\nameref{chapter_top_ontologies}}
\end{scnreltolist}

\scnheader{сообщение}
\scnidtf{\textit{message}}
\begin{scnreltolist}{ключевое понятие}
    \scnitem{Глава \ref{chapter_interfaces}~\nameref{chapter_interfaces}}
\end{scnreltolist}

\scnheader{соответствие*}
\scnidtf{\textit{mapping*}}
\begin{scnreltolist}{ключевое отношение}
	\scnitem{Глава \ref{chapter_top_ontologies}~\nameref{chapter_top_ontologies}}
\end{scnreltolist}

\scnheader{составляющая}
\scnidtf{\textit{constituent}}
\begin{scnreltolist}{ключевое понятие}
	\scnitem{Глава \ref{chapter_lang}~\nameref{chapter_lang}}
\end{scnreltolist}

\scnheader{специализированная ostis-платформа}
\scnidtf{\textit{specialized ostis-platform}}
\begin{scnreltolist}{ключевое понятие}
	\scnitem{Глава \ref{chapter_interpreter}~\nameref{chapter_interpreter}}
\end{scnreltolist}

\scnheader{спецификатор}
\scnidtf{\textit{specifier}}
\begin{scnreltolist}{ключевое понятие}
	\scnitem{Глава \ref{chapter_lang}~\nameref{chapter_lang}}
\end{scnreltolist}

\scnheader{спецификация*}
\scnidtf{\textit{specification*}}
\begin{scnreltolist}{ключевое отношение}
	\scnitem{\ref{sec_problem_concept}~\nameref{sec_problem_concept}}
\end{scnreltolist}

\scnheader{спецификация метода*}
\scnidtf{\textit{method specification*}}
\begin{scnreltolist}{ключевое отношение}
	\scnitem{Глава \ref{chapter_programs}~\nameref{chapter_programs}}
\end{scnreltolist}

\scnheader{спецификация языка представления методов*}
\scnidtf{\textit{method representation language specification*}}
\begin{scnreltolist}{ключевое отношение}
	\scnitem{Глава \ref{chapter_programs}~\nameref{chapter_programs}}
\end{scnreltolist}

\scnheader{Стандарт Технологии OSTIS}
\scnidtf{\textit{Стандарт ostis-систем}}
\scnidtf{\textit{Стандарт OSTIS}}
\scnidtf{\textit{OSTIS Technology Standard}}
\scnidtf{\textit{Standard of ostis-systems}}
\begin{scnreltolist}{ключевой знак}
	\scnitem{Глава \ref{chapter_ostis_tech}~\nameref{chapter_ostis_tech}}
	\scnitem{\ref{sec_standard}~\nameref{sec_standard}}
\end{scnreltolist}

\scnheader{страница sc.n-текста}
\scnidtf{\textit{sc.n-text page}}
\begin{scnreltolist}{ключевое понятие}
	\scnitem{\ref{sec_scn}~\nameref{sec_scn}}
\end{scnreltolist}

\scnheader{стратегия решения задач}
\scnidtf{\textit{problem solving strategy}}
\begin{scnreltolist}{ключевое понятие}
	\scnitem{\ref{sec_method_concept}~\nameref{sec_method_concept}}
\end{scnreltolist}

\scnheader{стратегия решения информационных задач}
\scnidtf{\textit{information problem solving strategy}}
\begin{scnreltolist}{ключевое понятие}
	\scnitem{\ref{sec_method_concept}~\nameref{sec_method_concept}}
\end{scnreltolist}

\scnheader{строка sc.n-текста}
\scnidtf{\textit{sc.n-text string}}
\begin{scnreltolist}{ключевое понятие}
	\scnitem{\ref{sec_scn}~\nameref{sec_scn}}
\end{scnreltolist}

\scnheader{строковый sc-идентификатор}
\scnidtf{\textit{string sc-identifier}}
\begin{scnreltolist}{ключевое понятие}
	\scnitem{\ref{sec_main_and_system_identifiers_concept}~\nameref{sec_main_and_system_identifiers_concept}}
\end{scnreltolist}

\scnheader{структура}
\scnidtf{\textit{structure}}
\begin{scnreltolist}{ключевое понятие}
	\scnitem{\ref{sec_structure}~\nameref{sec_structure}}
\end{scnreltolist}

\scnheader{субъект}
\scnidtf{\textit{subject}}
\begin{scnreltolist}{ключевое понятие}
	\scnitem{\ref{sec_action_concept}~\nameref{sec_action_concept}}
\end{scnreltolist}

\scnheader{субъект\scnrolesign}
\scnidtf{\textit{subject\scnrolesign}}
\begin{scnreltolist}{ключевое отношение}
	\scnitem{\ref{sec_action_concept}~\nameref{sec_action_concept}}
\end{scnreltolist}

\scnheader{сумма*}
\scnidtf{\textit{sum*}}
\begin{scnreltolist}{ключевое отношение}
	\scnitem{Глава \ref{chapter_top_ontologies}~\nameref{chapter_top_ontologies}}
\end{scnreltolist}

\scnheader{сцена в трехмерном представлении}
\scnidtf{\textit{scene in 3D representation}}
\begin{scnreltolist}{ключевое понятие}
	\scnitem{\ref{sec_3d_models_semantics}~\nameref{sec_3d_models_semantics}}
\end{scnreltolist}

\scnheader{темпоральное отношение}
\scnidtf{\textit{temporal relation}}
\begin{scnreltolist}{ключевое понятие}
	\scnitem{Глава \ref{chapter_top_ontologies}~\nameref{chapter_top_ontologies}}
\end{scnreltolist}

\scnheader{терминальный модуль}
\scnidtf{\textit{terminal module}}
\begin{scnreltolist}{ключевое понятие}
	\scnitem{Глава \ref{chapter_computers}~\nameref{chapter_computers}}
\end{scnreltolist}

\scnheader{тестирование удобства использования пользовательских интерфейсов}
\scnidtf{\textit{usability testing of user interfaces}}
\begin{scnreltolist}{ключевое понятие}
	\scnitem{Глава \ref{chapter_ui_design}~\nameref{chapter_ui_design}}
\end{scnreltolist}

\scnheader{технология}
\scnidtf{\textit{technology}}
\scnidtf{комплекс моделей, методик, методов и средств, обеспечивающих выполнение соответствующего вида деятельности}
\scnsuperset{технология поддержки жизненного цикла}
\begin{scnindent}
	\scnsuperset{технология комплексной поддержки жизненного цикла}
	\begin{scnindent}
		\scnsuperset{технология комплексной поддержки жизненного цикла интеллектуальных компьютерных систем нового поколения}
		\begin{scnindent}
			\scnhaselement{Технология OSTIS}
			\begin{scnindent}
				\scnidtf{Технология комплексной поддержки жизненного цикла ostis-систем}
			\end{scnindent}
		\end{scnindent} 
	\end{scnindent} 
\end{scnindent} 
\begin{scnreltolist}{ключевое понятие}
	\scnitem{Часть \ref{part1}~\nameref{part1}}
	\scnitem{Глава \ref{chapter_actions}~\nameref{chapter_actions}}
\end{scnreltolist}

\scnheader{технология проектирования интеллектуальных систем}
\scnidtf{\textit{intelligent systems design technology}}
\begin{scnreltolist}{ключевое понятие}
	\scnitem{Глава \ref{chapter_library}~\nameref{chapter_library}}
\end{scnreltolist}

\scnheader{Технология OSTIS}
\scnidtf{\textit{OSTIS Technology}}
\begin{scnreltolist}{ключевой знак}
	\scnitem{Глава \ref{chapter_ostis_tech}~\nameref{chapter_ostis_tech}}
\end{scnreltolist}

\scnheader{технология*}
\scnidtf{\textit{technology*}}
\begin{scnreltolist}{ключевое отношение}
	\scnitem{\ref{sec_activity_and_technology}~\nameref{sec_activity_and_technology}}
\end{scnreltolist}

\scnheader{технологическая операция}	
\scnidtf{\textit{technological operation}}
\begin{scnreltolist}{ключевое понятие}
	\scnitem{\ref{sec_chapter_enterprise_characteristics}~\nameref{sec_chapter_enterprise_characteristics}}
\end{scnreltolist}

\scnheader{технологический процесс}	
\scnidtf{\textit{technological process}}
\begin{scnreltolist}{ключевое понятие}
	\scnitem{\ref{sec_chapter_enterprise_characteristics}~\nameref{sec_chapter_enterprise_characteristics}}
\end{scnreltolist}

\scnheader{тип блокировки}
\scnidtf{\textit{lock type}}
\begin{scnreltolist}{ключевое понятие}
	\scnitem{\ref{sec_ps_sync}~\nameref{sec_ps_sync}}
\end{scnreltolist}

\scnheader{топологическое пространство}
\scnidtf{\textit{topological space}}
\begin{scnreltolist}{ключевое понятие}
	\scnitem{\ref{sec_sr_semspace}~\nameref{sec_sr_semspace}}
\end{scnreltolist}

\scnheader{точка останова}
\scnidtf{\textit{breakpoint}}
\begin{scnreltolist}{ключевое понятие}
	\scnitem{\ref{chapter_ps_design}~\nameref{chapter_ps_design}}
\end{scnreltolist}

\scnheader{точная величина}
\scnidtf{\textit{exact value}}
\begin{scnreltolist}{ключевое понятие}
	\scnitem{Глава \ref{chapter_top_ontologies}~\nameref{chapter_top_ontologies}}
\end{scnreltolist}

\scnheader{точность*}
\scnidtf{\textit{accuracy*}}
\begin{scnreltolist}{ключевое отношение}
	\scnitem{Глава \ref{chapter_top_ontologies}~\nameref{chapter_top_ontologies}}
\end{scnreltolist}

\scnheader{транзакция в sc-памяти}	
\scnidtf{\textit{transaction in sc-memory}}
\begin{scnreltolist}{ключевое понятие}
	\scnitem{\ref{sec_ps_sync}~\nameref{sec_ps_sync}}
\end{scnreltolist}

\scnheader{трехмерная модель объекта}	
\scnidtf{\textit{object 3D model}}
\begin{scnreltolist}{ключевое понятие}
	\scnitem{\ref{sec_3d_models_representation}~\nameref{sec_3d_models_representation}}
\end{scnreltolist}

\scnheader{трехмерная реконструкция}	
\scnidtf{\textit{3D reconstruction}}
\begin{scnreltolist}{ключевое понятие}
	\scnitem{\ref{sec_3d_models_reconstruction}~\nameref{sec_3d_models_reconstruction}}
\end{scnreltolist}

\scnheader{трехмерное представление объекта}	
\scnidtf{\textit{object 3D representation}}
\begin{scnreltolist}{ключевое понятие}
	\scnitem{\ref{sec_3d_models_representation}~\nameref{sec_3d_models_representation}}
\end{scnreltolist}

\scnheader{удаляемые sc-элементы*}
\scnidtf{\textit{sc-elements to be deleted*}}
\begin{scnreltolist}{ключевое отношение}
	\scnitem{\ref{sec_ps_sync}~\nameref{sec_ps_sync}}
\end{scnreltolist}

\scnheader{умная больница}
\scnidtf{\textit{smart-больница}}
\scnidtf{\textit{smart-hospital}}
\begin{scnreltolist}{ключевое понятие}
	\scnitem{Глава \ref{chapter_ostis_tech}~\nameref{chapter_ostis_tech}}
\end{scnreltolist}

\scnheader{Умное общество}
\scnidtf{\textit{Smart-общество}}
\scnidtf{\textit{Общество 5.0}}
\scnidtf{\textit{Интеллектуальное общество}}
\scnidtf{\textit{Smart-society}}
\begin{scnreltolist}{ключевой знак}
	\scnitem{Глава \ref{chapter_ostis_tech}~\nameref{chapter_ostis_tech}}
\end{scnreltolist}

\scnheader{умное предприятие}
\scnidtf{\textit{smart-предприятие}}
\scnidtf{\textit{предприятие 5.0}}
\scnidtf{\textit{smart-enterprise}}
\begin{scnreltolist}{ключевое понятие}
	\scnitem{Глава \ref{chapter_ostis_tech}~\nameref{chapter_ostis_tech}}
	\scnitem{Глава \ref{chapter_enterprise}~\nameref{chapter_enterprise}}
\end{scnreltolist}

\scnheader{умный город}
\scnidtf{\textit{smart-город}}
\scnidtf{\textit{smart-city}}
\begin{scnreltolist}{ключевое понятие}
	\scnitem{Глава \ref{chapter_ostis_tech}~\nameref{chapter_ostis_tech}}
\end{scnreltolist}

\scnheader{умный дом}
\scnidtf{\textit{smart-дом}}
\scnidtf{\textit{smart-home}}
\begin{scnreltolist}{ключевое понятие}
	\scnitem{Глава \ref{chapter_smart_home}~\nameref{chapter_smart_home}}
\end{scnreltolist}

\scnheader{УСК}
\scnidtftext{сокращение основного sc-идентификатора}{\textit{Универсальный семантический код}}
\scnidtf{\textit{Универсальный семантический код, разработанный В.~В. Мартыновым}}
\scnidtf{\textit{Universal semantic code}}
\scnidtf{\textit{USC}}
\begin{scnreltolist}{ключевой знак}
	\scnitem{\ref{sec_ngics_sense_principles}~\nameref{sec_ngics_sense_principles}}
\end{scnreltolist}

\scnheader{файл}
\scnidtf{\textit{file}}
\begin{scnreltolist}{ключевое понятие}
	\scnitem{Глава \ref{chapter_inf_constr}~\nameref{chapter_inf_constr}}
\end{scnreltolist}

\scnheader{файл ostis-системы}
\scnidtf{\textit{ostis-system file}}
\scnidtf{\textit{file of ostis-system}}
\begin{scnreltolist}{ключевое понятие}
	\scnitem{Глава \ref{chapter_inf_constr}~\nameref{chapter_inf_constr}}
\end{scnreltolist}

\scnheader{фактографическое высказывание}
\scnidtf{\textit{factual statement}}
\begin{scnreltolist}{ключевое понятие}
	\scnitem{Глава \ref{chapter_logic}~\nameref{chapter_logic}}
\end{scnreltolist}

\scnheader{физический интерфейс}
\scnidtf{\textit{physical interface}}
\begin{scnreltolist}{ключевое понятие}
    \scnitem{Глава \ref{chapter_interfaces}~\nameref{chapter_interfaces}}
\end{scnreltolist}

\scnheader{физический канал связи}
\scnidtf{\textit{physical link}}
\begin{scnreltolist}{ключевое понятие}
	\scnitem{Глава \ref{chapter_computers}~\nameref{chapter_computers}}
\end{scnreltolist}

\scnheader{цель*}
\scnidtf{\textit{goal*}}
\begin{scnreltolist}{ключевое отношение}
	\scnitem{\ref{sec_action_concept}~\nameref{sec_action_concept}}
\end{scnreltolist}

\scnheader{цифра}
\scnidtf{\textit{digit}}
\begin{scnreltolist}{ключевое понятие}
	\scnitem{Глава \ref{chapter_top_ontologies}~\nameref{chapter_top_ontologies}}
\end{scnreltolist}

\scnheader{цифровая экосистема}
\scnidtf{\textit{digital ecosystem}}
\begin{scnreltolist}{ключевое понятие}
    \scnitem{\ref{sec_ecosystem_structure}~\nameref{sec_ecosystem_structure}}
\end{scnreltolist}

\scnheader{цифровой двойник}
\scnidtf{\textit{digital twin}}
\begin{scnreltolist}{ключевое понятие}
    \scnitem{Глава \ref{sec_chapter_enterprise_standard_examples_production}~\nameref{sec_chapter_enterprise_standard_examples_production}}
\end{scnreltolist}

\scnheader{часть речи}
\scnidtf{\textit{part of speech}}
\begin{scnreltolist}{ключевое понятие}
	\scnitem{Глава \ref{chapter_lang}~\nameref{chapter_lang}}
\end{scnreltolist}

\scnheader{число}
\scnidtf{\textit{number}}
\begin{scnreltolist}{ключевое понятие}
	\scnitem{Глава \ref{chapter_top_ontologies}~\nameref{chapter_top_ontologies}}
\end{scnreltolist}

\scnheader{эквивалентность задач*}
\scnidtf{\textit{problem equivalence*}}
\begin{scnreltolist}{ключевое отношение}
	\scnitem{\ref{sec_method_concept}~\nameref{sec_method_concept}}
\end{scnreltolist}

\scnheader{Экосистема OSTIS}
\scnidtf{\textit{OSTIS Ecosystem}}
\begin{scnreltolist}{ключевой знак}
	\scnitem{Глава \ref{chapter_ostis_tech}~\nameref{chapter_ostis_tech}}
	\scnitem{Глава \ref{chapter_ecosystem}~\nameref{chapter_ecosystem}}
	\scnitem{Глава \ref{chapter_integration}~\nameref{chapter_integration}}
\end{scnreltolist}

\scnheader{экспертная оценка пользовательских интерфейсов}
\scnidtf{\textit{expert evaluation of user interfaces}}
\begin{scnreltolist}{ключевое понятие}
	\scnitem{Глава \ref{chapter_ui_design}~\nameref{chapter_ui_design}}
\end{scnreltolist}

\scnheader{эффекторное действие}
\scnidtf{\textit{effector action}}
\begin{scnreltolist}{ключевое понятие}
	\scnitem{\ref{sec_action_concept}~\nameref{sec_action_concept}}
\end{scnreltolist}

\scnheader{эффективность метода}
\scnidtf{\textit{method efficiency}}
\begin{scnreltolist}{ключевое понятие}
	\scnitem{Глава \ref{chapter_programs}~\nameref{chapter_programs}}
\end{scnreltolist}

\scnheader{Ядро базы знаний ostis-системы}
\scnidtf{\textit{Knowledge base core of ostis-system}}
\begin{scnreltolist}{ключевой знак}
	\scnitem{Глава \ref{chapter_requests}~\nameref{chapter_requests}}
\end{scnreltolist}

\scnheader{язык}
\scnidtf{\textit{language}}
\begin{scnreltolist}{ключевое понятие}
	\scnitem{Глава \ref{chapter_inf_constr}~\nameref{chapter_inf_constr}}
	\scnitem{Глава \ref{chapter_lang}~\nameref{chapter_lang}}
\end{scnreltolist}

\scnheader{Язык вопросов для ostis-систем}
\scnidtf{\textit{Questions language for ostis-systems}}
\begin{scnreltolist}{ключевой знак}
	\scnitem{Глава \ref{chapter_requests}~\nameref{chapter_requests}}
\end{scnreltolist}

\scnheader{Язык карт для ostis-систем}
\scnidtf{\textit{Map language for ostis-systems}}
\begin{scnreltolist}{ключевое понятие}
	\scnitem{Глава \ref{chapter_gis}~\nameref{chapter_gis}}
\end{scnreltolist}

\scnheader{язык представления методов}
\scnidtf{\textit{я.п.м.}}
\scnidtf{\textit{язык программирования}}
\scnidtf{\textit{method representation language}}
\begin{scnreltolist}{ключевое понятие}
	\scnitem{Глава \ref{chapter_programs}~\nameref{chapter_programs}}
	\scnitem{Глава \ref{chapter_actions}~\nameref{chapter_actions}}
\end{scnreltolist}

\scnheader{Язык представления нейросетевых методов решения задач в базах знаний}
\scnidtf{\textit{Neural network methods problem solving representation language}}
\begin{scnreltolist}{ключевой знак}
	\scnitem{Глава \ref{chapter_ann}~\nameref{chapter_ann}}
\end{scnreltolist}

\scnheader{язык ostis-системы}
\scnidtf{\textit{ostis-system language}}
\begin{scnreltolist}{ключевое понятие}
	\scnitem{Глава \ref{chapter_inf_constr}~\nameref{chapter_inf_constr}}
\end{scnreltolist}

\scnheader{Язык SCL}
\scnidtf{\textit{SCL language}}
\scnidtftext{сокращение основного sc-идентификатора}{Semantic Code Logic}
\begin{scnreltolist}{ключевой знак}
	\scnitem{Глава \ref{chapter_logic}~\nameref{chapter_logic}}
	\scnitem{Глава \ref{chapter_logic_productions}~\nameref{chapter_logic_productions}}
\end{scnreltolist}

\scnheader{Язык SCP}
\scnidtf{\textit{SCP language}}
\scnidtftext{сокращение основного sc-идентификатора}{Semantic Code Programming}
\scnidtf{Графовый процедурный язык программирования, построенный на базе \textit{SC-кода}}
\begin{scnreltolist}{ключевой знак}
	\scnitem{\ref{sec_ps_scp}~\nameref{sec_ps_scp}}
\end{scnreltolist}

\scnheader{Язык SCPD}
\scnidtf{\textit{SCPD language}}
\scnidtftext{сокращение основного sc-идентификатора}{Semantic Code Distributed}
\begin{scnreltolist}{ключевой знак}
	\scnitem{\ref{sec_comp_curr_state}~\nameref{sec_comp_curr_state}}
\end{scnreltolist}

\scnheader{ambient assisted living}
\begin{scnreltolist}{ключевое понятие}
	\scnitem{Глава \ref{chapter_smart_home}~\nameref{chapter_smart_home}}
\end{scnreltolist}

\scnheader{ISA-88}
\scnidtf{\textit{ISA-88}}
\begin{scnreltolist}{ключевой знак}
	\scnitem{Глава \ref{sec_chapter_enterprise_standard_examples_production}~\nameref{sec_chapter_enterprise_standard_examples_production}}
\end{scnreltolist}

\scnheader{ISA-95}
\scnidtf{\textit{ISA-95}}
\begin{scnreltolist}{ключевой знак}
	\scnitem{Глава \ref{sec_chapter_enterprise_standard_examples_production}~\nameref{sec_chapter_enterprise_standard_examples_production}}
\end{scnreltolist}

\scnheader{Maple}
\begin{scnreltolist}{ключевой знак}
	\scnitem{\ref{sec_integration_algebra}~\nameref{sec_integration_algebra}}
\end{scnreltolist}

\scnheader{Mathematica}
\begin{scnreltolist}{ключевой знак}
	\scnitem{\ref{sec_integration_algebra}~\nameref{sec_integration_algebra}}
\end{scnreltolist}

\scnheader{MATLAB}
\begin{scnreltolist}{ключевой знак}
	\scnitem{\ref{sec_integration_algebra}~\nameref{sec_integration_algebra}}
\end{scnreltolist}

\scnheader{Maxima}
\begin{scnreltolist}{ключевой знак}
	\scnitem{\ref{sec_integration_algebra}~\nameref{sec_integration_algebra}}
\end{scnreltolist}

\scnheader{MQTT}
\begin{scnreltolist}{ключевой знак}
	\scnitem{Глава \ref{chapter_smart_home}~\nameref{chapter_smart_home}}
\end{scnreltolist}

\scnheader{Node-RED}
\begin{scnreltolist}{ключевой знак}
	\scnitem{Глава \ref{chapter_smart_home}~\nameref{chapter_smart_home}}
\end{scnreltolist}

\scnheader{OSTIS}
\scnidtftext{сокращение основного sc-идентификатора}{Open Semantic Technology for Intelligent Systems}
\scnidtftext{сокращение основного sc-идентификатора}{Открытая семантическая технология проектирования интеллектуальных систем}

\scnheader{ostis-}
\scnidtftext{сокращение основного sc-идентификатора}{for open semantic technology for intelligent systems}
\scntext{пример применения}{
	\begin{scnitemize}
		\item ostis-система
		\item ostis-платформа
		\item ostis-сообщество
	\end{scnitemize}
}

\scnheader{ostis-ассистент}
\scnidtf{\textit{ostis-assistant}}
\begin{scnreltolist}{ключевое понятие}
	\scnitem{Глава \ref{sec_ostis_assistant}~\nameref{sec_ostis_assistant}}
\end{scnreltolist}

\scnheader{ostis-платформа}
\scnidtf{\textit{ostis-platform}}
\begin{scnreltolist}{ключевое понятие}
	\scnitem{Глава \ref{chapter_interpreter}~\nameref{chapter_interpreter}}
\end{scnreltolist}

\scnheader{ostis-портал знаний}
\scnidtf{\textit{ostis-portal of knowledge}}
\begin{scnreltolist}{ключевое понятие}
    \scnitem{\ref{sec_ostis_scientific_portal}~\nameref{sec_ostis_scientific_portal}}
\end{scnreltolist}

\scnheader{ostis-система}
\scnidtf{\textit{ostis-system}}
\begin{scnreltolist}{ключевое понятие}
	\scnitem{Глава \ref{chapter_ostis_tech}~\nameref{chapter_ostis_tech}}
	\scnitem{\ref{sec_ecosystem_structure}~\nameref{sec_ecosystem_structure}}
\end{scnreltolist}

\scnheader{ostis-сообщество}
\scnidtf{\textit{ostis-community}}
\begin{scnreltolist}{ключевое понятие}
	\scnitem{\ref{sec_ecosystem_structure_description}~\nameref{sec_ecosystem_structure_description}}
\end{scnreltolist}

\scnheader{ostis-технология}
\scnidtf{\textit{частная технология, входящая в состав комплексной Технологии OSTIS}}
\scnidtf{\textit{ostis-technology}}
\begin{scnreltolist}{ключевой знак}
	\scnitem{Глава \ref{chapter_ostis_tech}~\nameref{chapter_ostis_tech}}
\end{scnreltolist}

\scnheader{OWL}
\scnidtftext{сокращение основного sc-идентификатора}{\textit{Web Ontology Language}}
\scnidtf{\textit{Язык описания онтологий для семантической паутины}}
\begin{scnreltolist}{ключевой знак}
	\scnitem{Глава \ref{chapter_kb}~\nameref{chapter_kb}}
\end{scnreltolist}

\scnheader{RDF}
\scnidtftext{сокращение основного sc-идентификатора}{\textit{Resource Description Framework}}
\scnidtf{\textit{Разработанная Консорциумом Всемирной паутины модель для представления данных}}
\begin{scnreltolist}{ключевой знак}
	\scnitem{Глава \ref{chapter_kb}~\nameref{chapter_kb}}
\end{scnreltolist}

\scnheader{REST API}
\begin{scnreltolist}{ключевой знак}
	\scnitem{Глава \ref{chapter_smart_home}~\nameref{chapter_smart_home}}
\end{scnreltolist}

\scnheader{SC-}
\scnidtftext{сокращение основного sc-идентификатора}{Semantic Code}
\scnidtftext{сокращение основного sc-идентификатора}{Semantic Computer}
\scntext{пример применения}{
	\begin{scnitemize}
		\item SC-код
		\item sc-элемент
		\item sc-множество
		\item sc-модель
	\end{scnitemize}
}

\scnheader{sc-агент}
\scnidtf{\textit{sc-agent}}
\begin{scnreltolist}{ключевое понятие}
	\scnitem{\ref{sec_ps_agents}~\nameref{sec_ps_agents}}
\end{scnreltolist}

\scnheader{sc-выражение}
\scnidtf{сложный sc-идентификатор}
\scnidtf{\textit{complex sc-identifier}}
\scnidtf{\textit{sc-expression}}
\begin{scnreltolist}{ключевое понятие}
	\scnitem{\ref{sec_complex_identifier_concept}~\nameref{sec_complex_identifier_concept}}
\end{scnreltolist}

\scnheader{sc-идентификатор}
\scnidtf{\textit{sc-identifier}}
\scntext{часто используемый sc-идентификатор}{внешний идентификатор sc-элемента}
\begin{scnreltolist}{ключевое понятие}
	\scnitem{\ref{sec_identifiers}~\nameref{sec_identifiers}}
\end{scnreltolist}

\scnheader{sc-класс}
\scnidtf{\textit{sc-class}}
\begin{scnreltolist}{ключевое понятие}
	\scnitem{Глава \ref{chapter_sc_code}~\nameref{chapter_sc_code}}
\end{scnreltolist}

\scnheader{\textit{SC-код}}
\scnidtf{\textit{SC-code}}
\scnidtftext{сокращение основного sc-идентификатора}{Semantic Code}
\scnidtf{Универсальный базовый способ смыслового представления знаний в виде семантических сетей с базовой теоретико-множественной интерпретацией}
\scnrelto{часто используемый sc-идентификатор}{sc-структура}
\scniselement{часто используемый sc-идентификатор}
\scniselement{имя собственное}
\scnidtf{sc-конструкция}
\begin{scnreltolist}{ключевой знак}
	\scnitem{Глава \ref{chapter_new_generation_systems}~\nameref{chapter_new_generation_systems}}
\end{scnreltolist}
\begin{scnreltolist}{ключевой термин}
	\scnitem{Параграф \ref{sec_external_information_constructs_external_lang}~\nameref{sec_external_information_constructs_external_lang}}
\end{scnreltolist}

\scnheader{sc-машина}
\scnidtf{\textit{sc-machine}}
\begin{scnreltolist}{ключевое понятие}
	\scnitem{Глава \ref{chapter_interpreter}~\nameref{chapter_interpreter}}
\end{scnreltolist}

\scnheader{sc-множество}
\scnidtf{\textit{sc-set}}
\begin{scnreltolist}{ключевое понятие}
	\scnitem{Глава \ref{chapter_sc_code}~\nameref{chapter_sc_code}}
\end{scnreltolist}

\scnheader{sc-память}
\scnidtf{\textit{sc-memory}}
\begin{scnreltolist}{ключевое понятие}
	\scnitem{Глава \ref{chapter_interpreter}~\nameref{chapter_interpreter}}
\end{scnreltolist}

\scnheader{sc-связка}
\scnidtf{\textit{sc-sheaf}}
\begin{scnreltolist}{ключевое понятие}
	\scnitem{Глава \ref{chapter_sc_code}~\nameref{chapter_sc_code}}
\end{scnreltolist}

\scnheader{sc-структура}
\scnidtf{\textit{sc-structure}}
\begin{scnreltolist}{ключевое понятие}
	\scnitem{Глава \ref{chapter_sc_code}~\nameref{chapter_sc_code}}
\end{scnreltolist}

\scnheader{sc-элемент}
\scnidtf{\textit{sc-element}}
\scnrelto{часто используемый sc-идентификатор}{сущность}
\begin{scnreltolist}{ключевое понятие}
	\scnitem{Глава \ref{chapter_new_generation_systems}~\nameref{chapter_new_generation_systems}}
	\scnitem{Глава \ref{chapter_top_ontologies}~\nameref{chapter_top_ontologies}}
	\scnitem{Глава \ref{chapter_sc_code}~\nameref{chapter_sc_code}}
	\scnitem{Глава \ref{chapter_situation_management}~\nameref{chapter_situation_management}}
\end{scnreltolist}

\scnheader{sc-язык}
\scnidtf{\textit{sc-language}}
\scnidtf{подъязык \textit{SC-кода}}
\begin{scnreltolist}{ключевое понятие}
	\scnitem{Глава \ref{chapter_kb}~\nameref{chapter_kb}}
\end{scnreltolist}

\scnheader{SCD-код}
\scnidtf{\textit{SCD-code}}
\scnidtftext{сокращение основного sc-идентификатора}{Semantic Code Distributed}
\begin{scnreltolist}{ключевой знак}
	\scnitem{\ref{sec_comp_curr_state}~\nameref{sec_comp_curr_state}}
\end{scnreltolist}

\scnheader{SCg-код}
\scnidtf{\textit{SCg-code}}
\scnidtftext{сокращение основного sc-идентификатора}{Semantic Code graphic}
\scnidtf{Графический нелинейный вариант визуализации текстов SC-кода}
\begin{scnreltolist}{ключевой знак}
	\scnitem{\ref{sec_scg}~\nameref{sec_scg}}
\end{scnreltolist}

\scnheader{SCfin-код}
\scnidtf{\textit{SCfin-code}}
\scnidtftext{сокращение основного sc-идентификатора}{Semantic Code file interior}
\begin{scnreltolist}{ключевой знак}
	\scnitem{Глава \ref{chapter_soft_platform}~\nameref{chapter_soft_platform}}
\end{scnreltolist}

\scnheader{SCin-код}
\scnidtf{\textit{SCin-code}}
\scnidtftext{сокращение основного sc-идентификатора}{Semantic Code interior}
\begin{scnreltolist}{ключевой знак}
	\scnitem{Глава \ref{chapter_soft_platform}~\nameref{chapter_soft_platform}}
\end{scnreltolist}

\scnheader{SC-JSON-код}
\begin{scnreltolist}{ключевой знак}
	\scnitem{Глава \ref{chapter_soft_platform}~\nameref{chapter_soft_platform}}
\end{scnreltolist}

\scnheader{sc.g-дуга}
\scnidtf{\textit{sc.g-arc}}
\begin{scnreltolist}{ключевое понятие}
	\scnitem{\ref{sec_scg}~\nameref{sec_scg}}
\end{scnreltolist}

\scnheader{sc.g-коннектор}
\scnidtf{\textit{sc.g-connector}}
\begin{scnreltolist}{ключевое понятие}
	\scnitem{\ref{sec_scg}~\nameref{sec_scg}}
\end{scnreltolist}

\scnheader{sc.g-рамка}
\scnidtf{\textit{sc.g-frame}}
\begin{scnreltolist}{ключевое понятие}
	\scnitem{\ref{sec_scg}~\nameref{sec_scg}}
\end{scnreltolist}

\scnheader{sc.g-ребро}
\scnidtf{\textit{sc.g-edge}}
\begin{scnreltolist}{ключевое понятие}
	\scnitem{\ref{sec_scg}~\nameref{sec_scg}}
\end{scnreltolist}

\scnheader{sc.g-элемент}
\scnidtf{\textit{sc.g-element}}
\begin{scnreltolist}{ключевое понятие}
	\scnitem{\ref{sec_scg}~\nameref{sec_scg}}
\end{scnreltolist}

\scnheader{sc.g-шина}
\scnidtf{\textit{sc.g-bus}}
\begin{scnreltolist}{ключевое понятие}
	\scnitem{\ref{sec_scg}~\nameref{sec_scg}}
\end{scnreltolist}

\scnheader{SCn-код}
\scnidtf{\textit{SCn-code}}
\scnidtftext{сокращение основного sc-идентификатора}{Semantic Code natural}
\scnidtf{Гипертекстовый вариант визуализации текстов \textit{SC-кода}}
\begin{scnreltolist}{ключевой знак}
	\scnitem{\ref{sec_scn}~\nameref{sec_scn}}
\end{scnreltolist}

\scnheader{sc.n-дуга}
\scnidtf{\textit{sc.n-arc}}
\begin{scnreltolist}{ключевое понятие}
	\scnitem{\ref{sec_scn}~\nameref{sec_scn}}
\end{scnreltolist}

\scnheader{sc.n-коннектор}
\scnidtf{\textit{sc.n-connector}}
\begin{scnreltolist}{ключевое понятие}
	\scnitem{\ref{sec_scn}~\nameref{sec_scn}}
\end{scnreltolist}

\scnheader{sc.n-контур}
\scnidtf{\textit{sc.n-contour}}
\begin{scnreltolist}{ключевое понятие}
	\scnitem{\ref{sec_scn}~\nameref{sec_scn}}
\end{scnreltolist}

\scnheader{sc.n-рамка}
\scnidtf{\textit{sc.n-frame}}
\begin{scnreltolist}{ключевое понятие}
	\scnitem{\ref{sec_scn}~\nameref{sec_scn}}
\end{scnreltolist}

\scnheader{sc.n-ребро}
\scnidtf{\textit{sc.n-edge}}
\begin{scnreltolist}{ключевое понятие}
	\scnitem{\ref{sec_scn}~\nameref{sec_scn}}
\end{scnreltolist}

\scnheader{sc.n-элемент}
\scnidtf{\textit{sc.n-element}}
\begin{scnreltolist}{ключевое понятие}
	\scnitem{\ref{sec_scn}~\nameref{sec_scn}}
\end{scnreltolist}

\scnheader{scp-компьютер}
\scnidtf{\textit{scp-computer}}
\begin{scnreltolist}{ключевое понятие}
	\scnitem{Глава \ref{chapter_computers}~\nameref{chapter_computers}}
\end{scnreltolist}

\scnheader{scp-операнд\scnrolesign}
\scnidtf{\textit{scp-operand\scnrolesign}}
\begin{scnreltolist}{ключевое отношение}
	\scnitem{\ref{sec_ps_scp}~\nameref{sec_ps_scp}}
\end{scnreltolist}

\scnheader{scp-оператор}
\scnidtf{\textit{scp-operator}}
\begin{scnreltolist}{ключевое понятие}
	\scnitem{\ref{sec_ps_scp}~\nameref{sec_ps_scp}}
\end{scnreltolist}

\scnheader{SCs-код}
\scnidtf{\textit{SCs-code}}
\scnidtftext{сокращение основного sc-идентификатора}{Semantic Code string}
\scnidtf{Линейный вариант визуализации текстов SC-кода}
\begin{scnreltolist}{ключевой знак}
	\scnitem{\ref{sec_scs}~\nameref{sec_scs}}
\end{scnreltolist}

\scnheader{sc.s-дуга}
\scnidtf{\textit{sc.s-arc}}
\begin{scnreltolist}{ключевое понятие}
	\scnitem{\ref{sec_scs}~\nameref{sec_scs}}
\end{scnreltolist}

\scnheader{sc.s-коннектор}
\scnidtf{\textit{sc.s-connector}}
\begin{scnreltolist}{ключевое понятие}
	\scnitem{\ref{sec_scs}~\nameref{sec_scs}}
\end{scnreltolist}

\scnheader{sc.s-модификатор}
\scnidtf{\textit{sc.s-modifier}}
\begin{scnreltolist}{ключевое понятие}
	\scnitem{\ref{sec_scs}~\nameref{sec_scs}}
\end{scnreltolist}

\scnheader{sc.s-ограничитель}
\scnidtf{\textit{sc.s-limiter}}
\begin{scnreltolist}{ключевое понятие}
	\scnitem{\ref{sec_scs}~\nameref{sec_scs}}
\end{scnreltolist}

\scnheader{sc.s-предложение}
\scnidtf{\textit{sc.s-sentence}}
\begin{scnreltolist}{ключевое понятие}
	\scnitem{\ref{sec_scs}~\nameref{sec_scs}}
\end{scnreltolist}

\scnheader{sc.s-разделитель}
\scnidtf{\textit{sc.s-separator}}
\begin{scnreltolist}{ключевое понятие}
	\scnitem{\ref{sec_scs}~\nameref{sec_scs}}
\end{scnreltolist}

\scnheader{sc.s-ребро}
\scnidtf{\textit{sc.s-edge}}
\begin{scnreltolist}{ключевое понятие}
	\scnitem{\ref{sec_scs}~\nameref{sec_scs}}
\end{scnreltolist}

\scnheader{sc.s-элемент}
\scnidtf{\textit{sc.s-element}}
\begin{scnreltolist}{ключевое понятие}
	\scnitem{\ref{sec_scs}~\nameref{sec_scs}}
\end{scnreltolist}

\scnheader{Yandex Cloud}
\begin{scnreltolist}{ключевой знак}
	\scnitem{Глава \ref{chapter_smart_home}~\nameref{chapter_smart_home}}
\end{scnreltolist}

\scnheader{Yandex IoT Core}
\begin{scnreltolist}{ключевой знак}
	\scnitem{Глава \ref{chapter_smart_home}~\nameref{chapter_smart_home}}
\end{scnreltolist}

\scnheader{WebSocket}
\begin{scnreltolist}{ключевой знак}
	\scnitem{Глава \ref{chapter_soft_platform}~\nameref{chapter_soft_platform}}
\end{scnreltolist}

\scnheader{Wolfram}
\begin{scnreltolist}{ключевой знак}
	\scnitem{\ref{sec_integration_algebra}~\nameref{sec_integration_algebra}}
\end{scnreltolist}

\scnheader{Wolfram Alpha}
\begin{scnreltolist}{ключевой знак}
	\scnitem{\ref{sec_integration_algebra}~\nameref{sec_integration_algebra}}
\end{scnreltolist}

\scnheader{Wolfram Mathematica}
\begin{scnreltolist}{ключевой знак}
	\scnitem{\ref{sec_integration_algebra}~\nameref{sec_integration_algebra}}
\end{scnreltolist}

\scnheader{Wolfram Language}
\begin{scnreltolist}{ключевой знак}
	\scnitem{\ref{sec_integration_algebra}~\nameref{sec_integration_algebra}}
\end{scnreltolist}

\end{SCn}
