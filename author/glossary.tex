
\chapter*{\LARGE Предметный указатель OSTIS}
\label{chap_glossary}
\addcontentsline{toc}{part}{Предметный указатель OSTIS}

Предлагаемый вашему вниманию предметный указатель представляет собой алфавитный перечень всех основных терминов, используемых в данной монографии, которые \underline{взаимно-однозначно} соответствуют элементам рафинированной семантической сети, представляющей собой базу знаний \textit{Метасистемы OSTIS}, основная часть которой семантически эквивалентна тексту данной монографии.

В рамках внутреннего языка \textit{ostis-систем} (в рамках \textit{SC-кода}) указанные основные термины называются \textit{основными sc-идентификаторами} (основными внешними идентификаторами sc-элементов -- элементов рафинированных семантических сетей).

В данный предметный указатель включаются как русскоязычные, так и англоязычные термины (если нет аналогичного русскоязычного), непереводимые интернациональные названия (например, названия различных программных систем, такие как \textit{Neo4j}, \textit{MySQL} и т.д.), а также различные используемые сокращения.

При этом для каждого \underline{основного} русскоязычного термина указывается синонимичный ему \underline{основной} англоязычный термин, например:

\bigskip

\begin{SCn}

\begin{scnset}
\vspace{-\baselineskip}
\scnheader{sc-элемент}
\scnidtf{\textit{sc-element}}
\begin{scnindent}
	\scniselement{основной англоязычный sc-идентификатор}
	\begin{scnindent}
		\scnidtf{основной sc-идентификатор для англоязычного режима}
	\end{scnindent}
\end{scnindent}
\scnidtf{\textit{sc-элемент}}
\begin{scnindent}
	\scniselement{основной русскоязычный sc-идентификатор}
	\begin{scnindent}
		\scnidtf{основной sc-идентификатор для русскоязычного режима}
	\end{scnindent}
\end{scnindent}
\end{scnset}
\scnexplanation{Для каждого ключевого термина в предметном указателе указывается его основной русскоязычный идентификатор и основной англоязычный идентификатор. При этом по умолчанию считается, что русскоязычный термин является \underline{основным} русскоязычным идентификатором, а соответствующий ему англоязычный -- \underline{основным} англоязычным идентификатором.}

\end{SCn}

\bigskip

Для каждого неосновного, но часто используемого русскоязычного термина указывается синонимичный ему основной русскоязычный термин. Кроме того, для каждого основного русскоязычного термина указывается ссылка на соответствующую главу, параграф или пункт монографии, где сущность, обозначаемая этим термином, является ключевым знаком.

\begin{SCn}

\scnheader{ассоциативный семантический компьютер}
\scnidtf{\textit{associative semantic computer}}
\begin{scnreltolist}{ключевой знак}
	\scnitem{Глава \ref{chapter_computers}~\nameref{chapter_computers}}
	\scnitem{Глава \ref{chapter_interpreter}~\nameref{chapter_interpreter}}
\end{scnreltolist}

\scnheader{базовая ostis-платформа}
\scnidtf{\textit{basic ostis-platform}}
\begin{scnreltolist}{ключевой знак}
	\scnitem{Глава \ref{chapter_interpreter}~\nameref{chapter_interpreter}}
\end{scnreltolist}

\scnheader{минимальная конфигурация ostis-системы}
\scnidtf{\textit{ostis-system minimal configuration}}
\begin{scnreltolist}{ключевой знак}
	\scnitem{Глава \ref{chapter_interpreter}~\nameref{chapter_interpreter}}
\end{scnreltolist}

\scnheader{программный вариант ostis-платформы}
\scnidtf{\textit{software version of ostis-platform}}
\begin{scnreltolist}{ключевой знак}
	\scnitem{Глава \ref{chapter_interpreter}~\nameref{chapter_interpreter}}
\end{scnreltolist}

\scnheader{расширенная ostis-платформа}
\scnidtf{\textit{extended ostis-platform}}
\begin{scnreltolist}{ключевой знак}
	\scnitem{Глава \ref{chapter_interpreter}~\nameref{chapter_interpreter}}
\end{scnreltolist}

\scnheader{специализированная ostis-платформа}
\scnidtf{\textit{specialized ostis-platform}}
\begin{scnreltolist}{ключевой знак}
	\scnitem{Глава \ref{chapter_interpreter}~\nameref{chapter_interpreter}}
\end{scnreltolist}

\scnheader{ostis-платформа}
\scnidtf{\textit{ostis-platform}}
\begin{scnreltolist}{ключевой знак}
	\scnitem{Глава \ref{chapter_interpreter}~\nameref{chapter_interpreter}}
\end{scnreltolist}

\scnheader{sc-идентификатор}
\scnidtf{\textit{sc-identifier}}
\begin{scnreltolist}{ключевой знак}
	\scnitem{Глава \ref{chapter_ext_lang}~\nameref{chapter_ext_lang}}
\end{scnreltolist}

\bigskip
\scnheader{\scnfilelong{\textbf{\textit{SC-код}}}}
\scnidtf{\textit{SC-code}}
\scnrelto{часто используемый sc-идентификатор}{sc-структура}
\begin{scnreltolist}{ключевой знак}
	\scnitem{Глава \ref{chapter_new_generation_systems}~\nameref{chapter_new_generation_systems}}
\end{scnreltolist}

\scnheader{sc-машина}
\scnidtf{\textit{sc-machine}}
\begin{scnreltolist}{ключевой знак}
	\scnitem{Глава \ref{chapter_interpreter}~\nameref{chapter_interpreter}}
\end{scnreltolist}

\scnheader{sc-элемент}
\scnidtf{\textit{sc-element}}
\begin{scnreltolist}{ключевой знак}
	\scnitem{Глава \ref{chapter_new_generation_systems}~\nameref{chapter_new_generation_systems}}
\end{scnreltolist}

\end{SCn}	