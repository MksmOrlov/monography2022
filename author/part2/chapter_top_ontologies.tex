\chapauthor{Бутрин С.В.\\Шункевич Д.В.}
\chapter{Представление формальных онтологий базовых классов сущностей в ostis-системах}
\chapauthortoc{Бутрин С.В.\\Шункевич Д.В.}
\label{chapter_top_ontologies}

\abstract{Аннотация к главе.}

Правки:
\begin{itemize}
	\item Ввести понятия для предметной области и онтологии
	\item Пересмотреть метериал по главе 2.4, где можно объеденить, убрать дублирование
	\item Добавить больше примеров и их описание
\end{itemize}

Вопросы:
\begin{itemize}
	\item Что должна включать аннотация?
	\item Где должно быть сравнение с аналогами?
	\item Где вводится понятие базы знаний? Должно ли оно быть?
\end{itemize}

Для обеспечения совместного использования различных видов знаний, входящих в состав базы знаний, необходимо обеспечить их совместимость с указанной базой знаний, которая включает семантическую совместимость, что подразумевает однозначную и единую для всех фрагментов базы знаний трактовку используемых понятий.

Среди многообразия средств представления знаний к наиболее эффективным относятся онтологии\cite{Davydenko2017}. Суть такого подхода при проектировании базы знаний состоит в рассмотрении базы знаний как иерархической системы выделенных предметных областей и соответствующих им онтологий. Однако онтологически можно по разному специфицировать знания. Чтобы решить эту проблему проектируются онтологии верхнего уровня.

Применение современных онтологий верхнего уровня при разработке баз знаний интеллектуальных компьютерных систем сопряжено с проблемами обеспечения их совместимости. Поскольку изначальной целью создания онтологий верхнего уровня являлось обеспечение  совместимости онтологий предметных областей и прикладных онтологий, а не самих интеллектуальных систем. 

Такими проблемами являются:
\begin{itemize}
    \item свобода трактовки понятий, вызванная отсутствием их четкого определения;
    \item отсутствие единой технологии проектирования баз знаний на основе онтологий верхнего уровня;
    \item отсутствие принадлежности онтологий верхнего уровня к какой-либо технологии, что не позволяет использовать их в качестве многократно используемых компонентов;
\end{itemize}

Поэтому возникает необходимость в разработке такой системы онтологии верхнего уровня, которая смогла бы обеспечить семантическую совместимость между большим количеством онтологий различных предметных областей.

Предлагаемый подход подразумевает разработку семейств Предметных областей и онтологий, которые бы содержали описания всех необходимых базовых классов сущностей для построения базы знаний интеллектуальной компьютерной системы.

К таким Предметным областям и онтологиям относятся:

\begin{itemize}
\item Предметная область и онтология множеств
\item Предметная область и онтология связок и отношений
\item Предметная область и онтология параметров, величин и шкал
\item Предметная область и онтология чисел и числовых структур
\item Предметная область и онтология структур
\item Предметная область и онтология темпоральных сущностей
\item Предметная область и онтология темпоральных сущностей баз знаний ostis-систем
\item Предметная область и онтология семантических окрестностей
\item Предметная область и онтология предметных областей
\item Предметная область и онтология онтологий
\item Предметная область и онтология логических формул, высказываний и формальных %теорий
\item Предметная область и онтология внешних информационных конструкций и файлов ostis-систем
\item Глобальная предметная область действий и задач и соответствующая ей онтология методов и технологий
\end{itemize}

Данные предметные области являются часть Ядра базы знаний, которое должно быть в каждой интеллектуальной системе. Это ядро гарантирует совместимость интеллектуальных компьютерных систем за счет общего понятийного аппарата. В зависимости от специфики конкретных систем могут выделяться различные Ядра базы знаний, но неизменным должна оставаться наличие базовая части, включающей в себя предметные области и онтологии указанные выше.

\section{Формальная онтология множеств}

%%  Введение для предметной области множеств


\scnheader{множество}

\begin{scnrelfromset}{разбиение}
\scnitem{конечное множество}
\scnitem{бесконечное множество}
\end{scnrelfromset}

\begin{scnrelfromset}{разбиение}
	\scnitem{множество без кратных элементов}
	\scnitem{мультимножество}
\end{scnrelfromset}


\begin{scnrelfromset}{разбиение}
	\scnitem{связка}
	\scnitem{класс}
	\begin{scnindent}
		\scnidtf{sc-элемент, обозначающий класс sc-элементов}
		\scnidtf{sc-знак множества sc-элементов, эквивалентных в том или ином смысле}
	\end{scnindent}
	\scnitem{структура}
	\begin{scnindent}
		\scnidtf{sc-знак множества sc-элементов, в состав которого входят sc-связки или структуры, связывающие эти sc-элементы}
	\end{scnindent}
\end{scnrelfromset}

\begin{scnrelfromset}{разбиение}
	\scnitem{четкое множество}
	\scnitem{нечеткое множество}
\end{scnrelfromset}

\begin{scnrelfromset}{разбиение}
	\scnitem{множество первичных сущностей}
	\scnitem{множество множеств}
	\scnitem{множество первичных сущностей и множеств}
\end{scnrelfromset}

\begin{scnrelfromset}{разбиение}
	\scnitem{рефлексивное множество}
	\scnitem{нерефлексивное множество}
\end{scnrelfromset}

\begin{scnrelfromset}{разбиение}
	\scnitem{сформированное множество}
	\scnitem{несформированное множество}
\end{scnrelfromset}

\begin{scnrelfromset}{разбиение}
	\scnitem{кортеж}
	\scnitem{неориентированное множество}
\end{scnrelfromset}

Под \textbf{\textit{множеством}} понимается соединение в некое целое M определённых хорошо различимых предметов m нашего созерцания или нашего мышления (которые будут называться «элементами» множества M). 
	
\textbf{\textit{множество}} – мысленная сущность, которая связывает одну или несколько сущностей в целое.
	
Более формально \textbf{\textit{множество}} – это абстрактный математический объект, состоящий из элементов. Связь множеств с их элементами задается бинарным ориентированным отношением \textbf{\textit{принадлежность*}}.

\textbf{\textit{Множество}} может быть полностью задано следующими тремя способами:

\begin{itemize}
		\item путем перечисления (явного указания) всех его элементов (очевидно, что таким способом можно задать только конечное множество)
		\item с помощью определяющего высказывания, содержащего описание общего характеристического свойства, которым обладают все те и только те объекты, которые являются элементами (т.е. принадлежат) задаваемого множества.
		\item с помощью теоретико-множественных операций, позволяющих однозначно задавать новые множества на основе уже заданных (это операции объединения, пересечения, разности множеств и др.)
\end{itemize}

Для любого семантически ненормализованного \textbf{\textit{множества}} существует единственное семантически нормализованное \textbf{\textit{множество}}, в котором все элементы, не являющиеся знаками множеств, заменены на знаки множеств.


\scnheader{принадлежность*}
\scnidtf{принадлежность элемента множеству*}
\scnidtf{отношение принадлежности элемента множеству*}
\scniselement{бинарное отношение}
\scniselement{ориентированное отношение}


\textbf{\textit{принадлежность*}} – это бинарное ориентированное отношение, каждая связка которого связывает множество с одним из его элементов. Элементами отношения \textbf{\textit{принадлежность*}} по умолчанию являются \textit{позитивные sc-дуги принадлежности}.

% \begin{SCn}
% \scnheader{мультимножество}
% \scnidtf{множество, имеющее кратные вхождения хотя бы одного элемента}
% \scnidtf{множество, по крайней мере один элемент которого входит в его состав многократно}
% \end{SCn}

% \textbf{\textit{мультимножество}} - это \textit{множество}, для которого существует хотя бы одна кратная пара принадлежности, выходящая из знака этого множества.

% \begin{SCn}
% \scnheader{кратность принадлежности}
% \scnidtf{кратность принадлежности элемента}
% \scnidtf{кратность вхождения элемента во множество}
% \scniselement{параметр}
% \end{SCn}

% \textbf{\textit{кратность принадлежности}} - \textit{параметр}, значением которого являются числовые величины, показывающие сколько раз входит тот или иной элемент в рассматриваемое множество. Элементами данного параметра являются классы \textit{позитивных sc-дуг принадлежности}, связывающих данное множество с элементом, кратность вхождения которого в данное множество мы хотим задать.
	
% 	Таким образом, кратное вхождение элемента в мультимножество может быть задано как явным указанием \textit{позитивных sc-дуг принадлежности} этого элемента данному \textit{множеству}, так и «склеиванием» этих дуг в одну и включением ее в некоторый класс \textbf{\textit{кратности принадлежности}}.
%%\scnrelfrom{описание примера}{
%%	\scnfilescg{figures/sd_sets/multiplicityOfMembership.png}
%%}

\scnheader{класс}
\scnidtf{класс sc-элементов}
\begin{scnrelfromset}{разбиение}
	\scnitem{класс первичных sc-элементов}
	\scnitem{класс множеств}
\end{scnrelfromset}


\textbf{\textit{класс}} – множество элементов, обладающих какими-либо явно указываемыми общими свойствами.


\scnheader{кортеж}
%\scnidtf{кортеж}
\scnidtf{вектор}


\textbf{\textit{кортеж}} – это множество, представляющее собой упорядоченный набор элементов, т.е. такое множество, порядок элементов в котором имеет значение. Пары принадлежности элементов \textbf{\textit{кортежу}} могут дополнительно принадлежать каким-либо \textit{ролевым отношениям}, при этом, в рамках каждого \textbf{\textit{кортежа}} должен существовать хотя бы один элемент, роль которого дополнительно уточнена \textit{ролевым отношением}.


\scnheader{включение*}
\scnidtf{включение множеств*}
\scnidtf{быть подмножеством*}
\scniselement{бинарное отношение}
\scniselement{ориентированное отношение}
\scniselement{транзитивное отношение}
\scnrelfrom{область определения}{множество}
\scnsuperset{строгое включение*}

\textbf{\textit{включение*}} – это бинарное ориентированное отношение, каждая связка которого связывает два множества. Будем говорить, что \textit{Множество Si} \textbf{\textit{включает*}} в себя \textit{Множество Sj} в том и только том случае, если каждый элемент \textit{Множества Sj} является также и элементом \textit{Множества Si}
%%\scnrelfrom{описание примера}{
%%	\scnfilescg{figures/sd_sets/inclusion.png}}
%%\scnaddlevel{1}
%%\scntext{пояснение}{Множество {Sj} включается во множество \textit{Si}.}
%%\scnaddlevel{-1}


\scnheader{объединение*}
\scnidtf{объединение множеств*}
\scniselement{квазибинарное отношение}
\scniselement{ориентированное отношение}

	
\textbf{\textit{объединение*}} – это \textit{квазибинарное ориентированное отношение}, областью определения которого является семейство всевозможных множеств. Будем говорить, что \textit{Множество Si} является объединением \textit{Множество Sj} и \textit{Множество Sk} тогда и только тогда, когда любой элемент \textit{Множество Si} является элементом или \textit{Множество Sj} или \textit{Множество Sk}.
%%\scnrelfrom{описание примера}
%%	\scnfilescg {figures/sd_sets/union.png}}
%%\scnaddlevel{1}
%%\scntext{пояснение}{Множество \textit{Si} является объединением %%множеств \textit{Sj}, \textit{Sk} и \textit{Sm}.}
%%\scnaddlevel{-1}
%%\scnrelfrom{изображение}{
%%	\scnfileimage{\includegraphics[width=0.6\linewidth]{figures/sd_sets/union2.png}}}


\scnheader{разбиение*}
\scnidtf{разбиение  множества*}
\scnidtf{объединение попарно непересекающихся множеств*}
\scnidtf{декомпозиция множества*}
\scniselement{квазибинарное отношение}
\scniselement{ориентированное отношение}
\scniselement{отношение декомпозиции}

	
\textbf{\textit{разбиение*}} – это \textit{квазибинарное ориентированное отношение}, областью определения которого является семейство всевозможных множеств. В результате разбиения множества получается множество попарно непересекающихся множеств, объединение которых есть исходное множество.\\
Семейство множеств \{\textit{S1…, Sn}\} является разбиением множества \textit{Si} в том и только том случае, если:
\begin{itemize}
		\item семейство \{\textit{S1…, Sn}\} является семейством \textit{попарно непересекающихся множеств};
		\item семейство \{\textit{S1…, Sn}\} является покрытием множества \textit{Si} (или другими словами, множество \textit{Si} является \textit{объединением} множеств, входящих в указанное выше семейство)
\end{itemize}
%%\scnrelfrom{описание примера}{
%%	\scnfilescg{figures/sd_sets/split.png}}
%\scnaddlevel{1}
%%\scnexplanation{Множество \textit{Si} разбивается на множества \textit{Sj}, \textit{Sk} и \textit{Sm}.}
%%\scnaddlevel{-1}
%%\scnrelfrom{изображение}{
%%	\scnfileimage{\includegraphics[width=0.5\linewidth]{figures/sd_sets/split2.png}}}


\scnheader{пересечение*}
\scnidtf{пересечение множеств*}
\scniselement{квазибинарное отношение}
\scniselement{ориентированное отношение}


\textbf{\textit{пересечение*}} – это операция над множествами, аргументами которой являются два или большее число множеств, а результатом является множество, элементами которого являются все те и только те сущности, которые одновременно принадлежат каждому множеству, которое входит в семейство аргументов этой операции.\\
	Будем говорить, что \textit{Множество Si} является пересечением \textit{Множество Sj} и \textit{Множество Sk} тогда и только тогда, когда любой элемент \textit{Множество Si} является элементом \textit{Множество Sj} и элементом \textit{Множество Sk}.
%%\scnrelfrom{описание примера}{
%%	\scnfilescg{figures/sd_sets/intersection.png}}
%%\scnaddlevel{1}
%%\scntext{пояснение}{Множество \textit{Si} является результатом пересечения множеств \textit{Sj}, \textit{Sk} и \textit{Sm}.}
%%\scnaddlevel{-1}
%%\scnrelfrom{изображение}{
%%	\scnfileimage{\includegraphics[width=0.5\linewidth]{figures/sd_sets/intersection2.png}}}


\scnheader{разность множеств*}
\scniselement{бинарное отношение}
\scniselement{ориентированное отношение}


\textbf{\textit{разность множеств*}} – это \textit{бинарное ориентированное отношение}, связывающее между собой \textit{ориентированную пару}, первым элементом которой является уменьшаемое множество, вторым - вычитаемое множество, и множество, являющееся результатом вычитания вычитаемого из уменьшаемого, в которое входят все элементы первого множества, не входящие во второе множество.
%%\scnrelfrom{описание примера}{
%%	\scnfilescg{figures/sd_sets/setDifference.png}}
%%\scnaddlevel{1}
%%\scnexplanation{Множество \textit{Si} является результатом разности множеств \textit{Sj} и \textit{Sk}.}
%%\scnaddlevel{-1}
%%\scnrelfrom{изображение}{\scnfileimage{\includegraphics[width=0.5\linewidth]{figures/sd_sets/setDifference2.png}}}


\scnheader{симметрическая разность множеств*}
\scniselement{бинарное отношение}
\scniselement{ориентированное отношение}


\textbf{\textit{симметрическая разность множеств*}} – это \textit{бинарное ориентированное отношение}, связывающее между собой \textit{пару} множеств и множество, являющееся результатом симметрической разности элементов указанной пары. Будем называть \textit{Множество Si} результатом симметрической разности \textit{Множества Sj} и \textit{Множества Sk} тогда и только тогда, когда любой элемент \textit{Множества Si} является или элементом \textit{Множества Sj} или \textit{Множества Sk}, но не является элементом обоих множеств.
%%\scnrelfrom{описание примера}{
%%	\scnfilescg{figures/sd_sets/symmetricDifferenceOfSets.png}
%%	\scnexplanation{Множество \textit{Si} является результатом симметрической разности множеств \textit{Sj} и \textit{Sk}.}}
%%\scnrelfrom{изображение}{
%%	\scnfileimage{\includegraphics[width=0.5\linewidth]{figures/sd_sets/symmetricDifferenceOfSets2.png}}}


\scnheader{декартово произведение*}
\scnidtf{декартово произведение множеств*}
\scnidtf{прямое произведение множеств*}
\scniselement{бинарное отношение}
\scniselement{ориентированное отношение}


\textbf{\textit{декартово произведение*}} – это \textit{бинарное ориентированное отношение} между \textit{ориентированной парой} множеств и \textit{множеством}, элементами которого являются всевозможные упорядоченные пары, первыми элементами которых являются элементы первого из указанных множеств, вторыми – элементы второго из указанных множеств.
%%\scnrelfrom{описание примера}{
%%	\scnfilescg{figures/sd_sets/cartesianMultiplication.png}}
%%\scnaddlevel{1}
%%\scnexplanation{Множество \textit{Si} является результатом декартова произведения множеств \textit{Sj} и \textit{Sk}.}
%%\scnaddlevel{-1}


\scnheader{булеан*}
\scnidtf{булеан множества*}
\scnidtf{семейство всевозможных подмножеств заданного множества*}
\scniselement{бинарное отношение}
\scniselement{ориентированное отношение}

	
\textbf{\textit{булеан*}} – это \textit{бинарное ориентированное отношение} между множеством и некоторым семейством множеств, каждое из которых является подмножеством первого множества.
%%\scnrelfrom{описание примера}{
%%	\scnfilescg{figures/sd_sets/boulean.png}
%%}


\scnheader{мощность множества}
\scnidtf{кардинальное число}
\scnidtf{общее число вхождений элементов в заданное множество}
\scnidtf{класс эквивалентности, элементами которого являются знаки всех тех и только тех множеств, которые имеют одинаковую мощность}
\scnidtf{класс эквивалентности, соответствующий отношению быть парой множеств, имеющих одинаковую мощность (равномощность множеств)}
\scnidtf{величина мощности множеств}
\scnidtf{трансфинитное число}
\scnidtf{мощность по Кантору}
\scniselement{параметр}


\textbf{\textit{мощность множества}} – это \textit{параметр}, элементами которых являются \textit{множества}, имеющие одинаковое количество элементов. Значением данного параметра является числовая величина, задающая количество элементов, входящих в данный класс множеств, т.е. по сути, количество \textit{позитивных sc-дуг принадлежности}, выходящих из данного \textit{множества}.
	
	Для двух множеств, имеющих одинаковую мощность, существует взаимно-однозначное соответствие между ними (между множествами вхождений элементов в эти множества – на случай мультимножеств).
%%\scnrelfrom{описание примера}{
%%	\scnfilescg{figures/sd_sets/power.png}
%%}

\section{Формальная онтология связок и отношений}
%% Введение в пердметную область
%%%%%%%%%%%%%%%%%%%%%%

	
\scnheader{Предметная область связок и отношений}
\scniselement{предметная область}
	% \scnsdmainclasssingle{связь}
	% \scnsdclass{бинарная связь;sc-коннектор;неатомарная бинарная связь;небинарная связь;неориентированная связь;ориентированная связь;отношение;класс равномощных связок;класс связок разной мощности;унарное отношение;бинарное отношение;квазибинарное отношение;тернарное отношение;небинарное отношение;ориентированное отношение;неориентированное отношение;рефлексивное отношение;антирефлексивное отношение;частично рефлексивное отношение;симметричное отношение;антисимметричное отношение;частично симметричное отношение;транзитивное отношение;антитранзитивное отношение;частично транзитивное отношение;связанное отношение;отношение порядка;отношение строгого порядка;отношение нестрогого порядка;отношение толерантности;отношение эквивалентности;ролевое отношение;числовой атрибут;неролевое отношение;неролевое бинарное отношение;арность;метаотношение;отношение декомпозиции;отношение интеграции}
	% \scnsdrelation{область определения*;атрибут отношения*;домен*;первый домен*;второй домен*;композиция отношений*;фактор-множество*;соответствие*;отношение соответствия*;область отправления\scnrolesign;область прибытия\scnrolesign;образ\scnrolesign;прообраз\scnrolesign;всюду определенное соответствие*;частично определенное соответствие*;сюръективное соответствие*;несюръективное соответствие*;однозначное соответствие*;обратное соответствие*;обратимое соответствие*;неоднозначное соответствие*;инъективное соответствие*;взаимно однозначное соответствие*;множество сочетаний*;множество размещений*;множество перестановок*}
\scnheader{связь}
\scnidtf{связка sc-элементов}
\scnidtf{sc-связка}


\textit{связь} -- множество, являющееся абстрактной моделью связи между описываемыми сущностями, которые или знаки которых являются элементами этого множества.
	
Напомним, что все элементы множества, представленного в SC-коде, являются знаками, но описываемыми сущностями могут быть не только сущности, обозначаемые sc-элементами, но и сами эти sc-элементы.
	


\begin{scnsubdividing}
	\scnitem{бинарная связь}
	\scnitem{небинарная связь}
\end{scnsubdividing}

\begin{scnsubdividing}
	\scnitem{неориентированная связь}
	\scnitem{ориентированная связь}
\end{scnsubdividing}
	
\scnheader{бинарная связь}
\begin{scnsubdividing}
	\scnitem{sc-коннекторь}
	\scnitem{неатомарная бинарная связь}
\end{scnsubdividing}


Данное разбиение осуществляется на основе синтаксического признака, а не семантического, поскольку каждый \textit{sc-коннектор} может быть записан в памяти при помощи семантически эквивалентной конструкции, содержащей знак самой связи и пары принадлежности, ведущие к ее элементам, уточненные, при необходимости ролевыми отношениями.
	
\scnheader{sc-коннектор}
\scnidtf{атомарная бинарная связь}

Каждый \textbf{\textit{sc-коннектор}} представлен в \textit{sc-памяти} одним \textit{sc-элементом} и семантически эквивалентен конструкции, содержащей знак некоторой \textit{бинарной связи} и пары принадлежности, ведущие к элементам этой связи, уточненные, при необходимости ролевыми отношениями.
	
Такая конструкция может быть обозначена \textbf{\textit{sc-коннектором}} только в случае, когда роли компонентов соответствующей бинарной связи указываются только при помощи \textit{числовых атрибутов 1\scnrolesign} и \textit{2\scnrolesign} или не уточняются вообще.
	
\scnheader{неатомарная бинарная связь}

\textbf{\textit{неатомарная бинарная связь}} -- \textit{бинарная связь}, роли компонентов которой не могут быть заданы только при помощи \textit{ролевых отношений 1\scnrolesign} и \textit{2\scnrolesign}, или не заданы совсем, а требуют дополнительного уточнения при помощи более частных ролевых отношений.
	
\scnheader{небинарная связь}
\textbf{\textit{небинарная связь}} -- связь, имеющая больше двух элементов.
	
	\scnheader{неориентированная связь}
	\scnsuperset{неориентированное множество}
	\scnexplanation{\textbf{\textit{неориентированная связь}} -- связь, все элементы которой имеют одинаковые роли (при этом соответствующее ролевое отношение, как правило, явно не указывается).}
	
	\scnheader{ориентированная связь}
	\scnsuperset{кортеж}
	\scnexplanation{\textbf{\textit{ориентированная связь}} -- связь, в которой с помощью ролевых отношений, указываются роли компонентов этой связи.}
	
	\scnheader{отношение}
	\scnidtf{класс связей}
	\scnidtf{класс sc-связок}
	\scnidtf{множество отношений}
	\scnidtf{Множество всевозможных отношений}
	\scntext{определение}{\textbf{\textit{отношение}}, \textit{заданное на множестве M} -- это подмножество \textit{декартового произведения} этого множества самого на себя некоторое количество раз}
	
В более широком смысле \textbf{\textit{отношение}} -- это математическая структура, которая формально определяет свойства различных объектов и их взаимосвязи.

\begin{scnsubdividing}
	\scnitem{класс равномощных связок}
	\scnitem{класс связок разной мощности}
\end{scnsubdividing}
\begin{scnsubdividing}
	\scnitem{бинарное отношение}
	\scnitem{небинарное отношение}
\end{scnsubdividing}
\begin{scnsubdividing}
	\scnitem{ориентированное отношение}
	\scnitem{неориентированное отношение}
\end{scnsubdividing}
\begin{scnsubdividing}
	\scnitem{ролевое отношение}
	\scnitem{неролевое отношение}
\end{scnsubdividing}

\scnheader{класс равномощных связок}
\scnidtf{класс связок фиксированной арности}
\scnidtf{отношение, обладающее свойством арности}
\scnsuperset{унарное отношение}
\scnsuperset{бинарное отношение}
\scnsuperset{тернарное отношение}
\scntext{определение}{\textbf{\textit{класс равномощных связок}} -- класс связок, имеющих одинаковую мощность.}
	
\scnheader{класс связок разной мощности}
\scnidtf{отношение нефиксированной арности}
\scnsubset{небинарное отношение}
\scntext{определение}{\textbf{\textit{класс связок разной мощности}} -- класс связок, имеющих разную мощность.}
	
\scnheader{унарное отношение}
\scnidtf{отношение арности один}
\scnidtf{одноместное отношение}
\scnidtf{множество синглетонов}
\scntext{определение}{\textbf{\textit{унарное отношение}} -- это множество таких отношений на множестве M, являющихся любым подмножеством множества M.}
	
\scnheader{бинарное отношение}
\scnidtf{отношение арности два}
\scnidtf{двухместное отношение}
\scnsuperset{квазибинарное отношение}
\scnsuperset{отношение порядка}
\scnsuperset{отношение толерантности}
\begin{scnsubdividing}
	\scnitem{рефлексивное отношение}
	\scnitem{антирефлексивное отношение}
	\scnitem{частично рефлексивное отношение}
\end{scnsubdividing}
\begin{scnsubdividing}
	\scnitem{симметричное отношение}
	\scnitem{антисимметричное отношение}
	\scnitem{частично симметричное отношение}
\end{scnsubdividing}
\begin{scnsubdividing}
	\scnitem{транзитивное отношение}
	\scnitem{антитранзитивное отношение}
	\scnitem{частично транзитивное отношение}
\end{scnsubdividing}
\begin{scnsubdividing}
	\scnitem{ролевое отношение}
	\scnitem{неролевое бинарное отношение}
\end{scnsubdividing}

\scntext{определение}{\textbf{\textit{бинарное отношение}} -- это множество таких отношений на множестве \textbf{\textit{M}}, являющихся подмножеством \textit{декартова произведения} множества \textbf{\textit{M}}.}

Если \textbf{\textit{бинарное отношение R}} задано на \textit{множестве} \textbf{\textit{М}} и два элемента этого множества \textbf{\textit{a}} и \textbf{\textit{b}} связаны данным отношением, то будем обозначать такую связь как \textbf{\textit{aRb}}.
	
\scnheader{квазибинарное отношение}
\scnexplanation{\textbf{\textit{квазибинарное отношение}} -- множество ориентированных пар, первые компоненты которых являются связками.}

Таким образом, \textit{sc-дуги}, принадлежащие \textbf{\textit{квазибинарным отношениям}}, всегда выходят из связок.

\scntext{sc-утверждение}{В область определения квазибинарного отношения будем включать:
\begin{scnitemize}
	\item вторые компоненты ориентированных пар, принадлежащих этому отношению;
	\item элементы первых компонентов ориентированных пар, принадлежащих этому отношению;
	\item других элементов область определения квазибинарного отношения не содержит.
\end{scnitemize}
}
	
\scnheader{небинарное отношение}
\scnexplanation{\textbf{\textit{небинарное отношение}} -- это множество отношений, хотя бы одна из связок каждого из которых имеет значение мощности больше двух.}
	
\scnheader{ориентированное отношение}
\scntext{определение}{\textbf{\textit{ориентированное отношение}} -- это множество таких отношений, каждая связка которых является кортежем.}
	
\scnheader{неориентированное отношение}
\scntext{определение}{\textbf{\textit{неориентированное отношение}} -- это множество таких отношений, каждая связка которых является неориентированным множеством.}
	
	% \scnheader{рефлексивное отношение}
	% \scntext{определение}{\textbf{\textit{рефлексивное отношение}} -- это \textit{бинарное отношение}, любая пара которого есть канторовское множество.}
	
	% \scnheader{антирефлексивное отношение}
	% \scntext{определение}{\textbf{\textit{антирефлексивное отношение R}} на \textit{множестве} \textbf{\textit{A}} -- это \textit{бинарное отношение}, в котором все элементы множества \textbf{\textit{A}} не находятся в отношении \textbf{\textit{R}} к самому себе.}
	
	% \scnheader{частично рефлексивное отношение}
	% \scntext{определение}{\textbf{\textit{частично рефлексивное отношение R}} на \textit{множестве} \textbf{\textit{A}} -- это \textit{бинарное отношение},  в котором хотя бы один (но не все) элемент множества \textbf{\textit{A}} находится в отношении \textbf{\textit{R}} к самому себе.}
	
	% \scnheader{симметричное отношение}
	% \scntext{определение}{\textbf{\textit{симметричное отношение R}} на \textit{множестве} \textbf{\textit{A}} -- это \textit{бинарное отношение}, в котором для каждой пары элементов \textbf{\textit{а}} и \textbf{\textit{b}} этого множества выполнение отношения \textbf{\textit{aRb}} влечёт выполнение \textbf{\textit{bRa}}.}
	
	% \scnheader{антисимметричное отношение}
	% \scntext{определение}{\textbf{\textit{антисимметричное отношение R}} на \textit{множестве} \textbf{\textit{A}} -- это \textit{бинарное отношение}, в котором для каждой пары элементов \textbf{\textit{а}} и \textbf{\textit{b}} этого множества выполнение отношений \textbf{\textit{aRb}} и \textbf{\textit{bRa}} влечёт равенство \textbf{\textit{a}} и \textbf{\textit{b}}.}
	
	% \scnheader{частично симметричное отношение}
	% \scntext{определение}{\textbf{\textit{частично симметричное отношение R}} на \textit{множестве} \textbf{\textit{A}} -- это \textit{бинарное отношение}, в котором для каждой пары элементов \textbf{\textit{а}} и \textbf{\textit{b}} (но не для всех таких пар) этого множества выполнение отношения \textbf{\textit{aRb}} влечёт выполнение \textbf{\textit{bRa}}.}
	
	% \scnheader{транзитивное отношение}
	% \scntext{определение}{\textbf{\textit{транзитивное отношение R}} на \textit{множестве} \textbf{\textit{A}} -- это \textit{бинарное отношение}, в котором для любых трёх элементов этого множества \textbf{\textit{a, b, c}} выполнение отношений \textbf{\textit{aRb}} и \textbf{\textit{bRc}} влечёт выполнение отношения \textbf{\textit{aRc}}.}
	
	% \scnheader{антитранзитивное отношение}
	% \scntext{определение}{\textbf{\textit{антитранзитивное отношение R}} на \textit{множестве} \textbf{\textit{A}} -- это \textit{бинарное отношение}, в котором для любых трёх элементов этого множества \textbf{\textit{a, b, c}} выполнение отношений \textbf{\textit{aRb}} и \textbf{\textit{bRc}} не влечёт выполнение отношения \textbf{\textit{aRc}}.}
	
	% \scnheader{частично транзитивное отношение}
	% \scntext{определение}{\textbf{\textit{частично транзитивное отношение R}} на \textit{множестве} \textbf{\textit{A}} -- это \textit{бинарное отношение}, в котором для каждых трёх элементов этого множества \textbf{\textit{a, b, c}} (но не для всех таких троек) выполнение отношений \textbf{\textit{aRb}} и \textbf{\textit{bRc}} влечёт выполнение отношения \textbf{\textit{aRc}}.}
	
	\scnheader{связанное отношение*}
	\scniselement{бинарное отношение}
	\scntext{определение}{\textbf{\textit{связанное отношение* R}} на \textit{множестве} \textbf{\textit{A}} -- это \textit{бинарное отношение}, в котором для каждой пары элементов \textbf{\textit{а}} и \textbf{\textit{b}} этого множества выполняется одно из двух отношений: \textbf{\textit{aRb}} или \textbf{\textit{bRa}}.}
	
	\scnheader{отношение порядка}
	\begin{scnsubdividing}
		\scnitem{отношение строгого порядка}
		\scnitem{отношение нестрогого порядка}
	\end{scnsubdividing}
	
	\scntext{определение}{\textbf{\textit{отношение порядка}} -- это \textit{бинарное отношение}, обладающее свойством транзитивности и антисимметричности.}
	
	\scnheader{отношение строгого порядка}
	\scntext{определение}{\textbf{\textit{отношение строгого порядка}} -- это \textit{отношение порядка}, обладающее свойством антирефлексивности.}
	
	\scnheader{отношение нестрогого порядка}
	\scntext{определение}{\textbf{\textit{отношение нестрогого порядка}} -- это \textit{отношение порядка}, обладающее свойством рефлексивности.}
	
	\scnheader{отношение толерантности}
	\scntext{определение}{\textbf{\textit{отношение толерантности}} -- это \textit{бинарное отношение}, принадлежащее классам \textit{симметричное отношение} и \textit{рефлексивное отношение}.}
	
	\scnheader{отношение эквивалентности}
	\scnidtf{максимальное семейство отношений эквивалентности}
	\scnsubset{отношение толерантности}
	\scntext{определение}{\textbf{\textit{отношение эквивалентности}} -- это \textit{отношение толерантности}, принадлежащее классу \textit{транзитивных отношений}}
	\scntext{примечание}{Каждое отношение эквивалентности уточняет то, что мы считаем эквивалентными сущностями, т.е. то, на какие сходства этих сущностей мы обращаем внимание и какие их отличия мы игнорируем (не учитываем).}
	
	\scnheader{ролевое отношение}
	\scnidtf{атрибут}
	\scnidtf{атрибутивное отношение}
	\scnidtf{отношение, которое задает роль элементов в рамках некоторого множества}
	\scnidtf{отношение, являющееся подмножеством отношения принадлежности}
	\scnrelto{семейство подмножеств}{принадлежность*}
	\scnsubset{бинарное отношение}
	\scnsuperset{числовой атрибут}
	\scnexplanation{\textbf{\textit{ролевое отношение}} -- это отношение, являющееся подмножеством отношения принадлежности.}
	\scntext{правило идентификации экземпляров}{В конце каждого \textit{идентификатора}, соответствующего экземплярам класса \textbf{\textit{ролевое отношение}}, не являющегося системным, ставится знак ``\scnrolesign''.
	
	Например:\\
	\textit{ключевой экземпляр\scnrolesign}
	
	Из-за ограничений в разрешенном алфавите символов, в системном идентификаторе не может быть использовать знак ``\scnrolesign'', поэтому в начале каждого \textit{системного идентификатора}, соответствующего экземплярам класса \textbf{\textit{ролевое отношение}} ставится префикс ``rrel\_''.
	
	Например:\\
	\textit{rrel\_key\_sc\_element}}
	
	\scnheader{числовой атрибут}
	\scnidtf{порядковый номер}
	\scnidtf{номер компонента ориентированной связки}
	\scnhaselement{\textbf{1\scnrolesign}; \textbf{2\scnrolesign}; \textbf{3\scnrolesign}; \textbf{4\scnrolesign}; \textbf{5\scnrolesign}; \textbf{6\scnrolesign}; \textbf{7\scnrolesign}; \textbf{8\scnrolesign}; \textbf{9\scnrolesign}; \textbf{10\scnrolesign}}
	\scnexplanation{\textbf{\textit{числовой атрибут}} -- \textit{ролевое отношение}, задающее порядковый номер элемента некоторой ориентированной связки, не уточняя при этом семантику такой принадлежности. Во многих случаях бывает достаточно использовать числовые атрибуты, чтобы различать компоненты связки, семантика каждого из которых дополнительно оговаривается, например, при определении отношения, которому данная связка принадлежит.}
	
	\scnheader{неролевое отношение}
	\begin{scnsubdividing}
		\scnitem{небинарное отношение}
		\scnitem{неролевое бинарное отношение}
	\end{scnsubdividing}
	\scnexplanation{\textbf{\textit{неролевое отношение}} -- отношение, не являющееся подмножеством отношения принадлежности.}
	\scntext{правило идентификации экземпляров}{В конце каждого \textit{идентификатора}, соответствующего экземплярам класса \textbf{\textit{неролевое отношение}}, не являющегося системным, ставится знак ``*''.
	
	Например:\\
	\textit{включение*}
	
	Из-за ограничений в разрешенном алфавите символов, в системном идентификаторе не может быть использовать знак ``*'', поэтому в начале каждого \textit{системного идентификатора}, соответствующего экземплярам класса \textbf{\textit{неролевое отношение}} ставится префикс ``nrel\_''.
	
	Например:\\
	\textit{nrel\_inclusion}}
	
	\scnheader{неролевое бинарное отношение}
	\scnexplanation{\textbf{\textit{неролевое бинарное отношение}} -- \textit{бинарное отношение}, не являющееся \textit{ролевым отношением}.}
	
	\scnheader{арность}
	\scnidtf{арность отношения}
	\scniselement{параметр}
	\scnexplanation{\textbf{\textit{арность}} -- это параметр, каждый элемент которого представляет собой класс \textit{отношений}, каждая связка которых имеет одинаковую \textit{мощность}. Значение данного \textit{параметра} совпадает со значением \textit{мощности} каждой из таких связок.}
	\scnrelfrom{описание примера}
	
	
	\scnheader{область определения*}
	\scnidtf{область определения отношения*}
	\scniselement{бинарное отношение}
	\scnexplanation{\textbf{\textit{область определения*}} -- это \textit{бинарное отношение}, связывающее отношение со множеством, являющимся его областью определения.
	
	Областью определения отношения будем называть результат теоретико-множественного объединения всех связок этого отношения, или, другими словами, результат теоретико-множественного объединения всех множеств, являющихся доменами данного отношения.}
%	\scnrelfrom{описание примера}{
	% \scnfilescg{figures/sd_relations/domain.png}}
	
	\scnheader{атрибут отношения*}
	\scnidtf{ролевой атрибут, используемый в связках заданного отношения*}
	\scniselement{бинарное отношение}
	\scnexplanation{\textbf{\textit{атрибут отношения*}} -- это \textit{бинарное отношение}, связывающее заданное отношение с \textit{ролевым отношением}, используемым в данном отношении для уточнения роли того или иного элемента связок данного отношения.}
%	\scnrelfrom{описание примера}{
	% \scnfilescg{figures/sd_relations/relationshipAttribute.png}}
	
	\newpage
	\scnheader{домен*}
	\scnidtf{домен отношения по заданному атрибуту*}
	\scniselement{бинарное отношение}
	\scnexplanation{\textbf{\textit{домен*}} -- это \textit{бинарное отношение}, связывающее связку отношения \textit{атрибут отношения*} со множеством, являющимся доменом заданного отношения по заданному атрибуту. Множество \textbf{\textit{di}} является доменом отношения \textbf{\textit{ri}} по атрибуту \textbf{\textit{ai}} в том и только том случае, если элементами этого множества являются все те и только те элементы связок отношения \textbf{\textit{ri}}, которые имеют в рамках этих связок атрибут \textbf{\textit{ai}}.}
%	\scnrelfrom{описание примера}{
	% \scnfilescg{figures/sd_relations/domen.png}}
	
	
	%% Зачем? Мы их используем?
	% \scnheader{первый домен*}
	% \scniselement{бинарное отношение}
	% \scntext{определение}{\textbf{\textit{первый домен*}} -- это \textit{бинарное отношение}, связывающее отношение с множеством, являющимся доменом по атрибуту \textbf{\textit {1\scnrolesign}} данного отношения.}
	% \scnrelfrom{описание примера}{
	% \scnfilescg{figures/sd_relations/firstDomen.png}}
	
	% \scnheader{второй домен*}
	% \scniselement{бинарное отношение}
	% \scntext{определение}{\textbf{\textit{второй домен*}} -- это \textit{бинарное отношение}, связывающее отношение с множеством, являющимся доменом по атрибуту \textbf{\textit{2\scnrolesign}} данного отношения.}
	% \scnrelfrom{описание примера}{
	% \scnfilescg{figures/sd_relations/secondDomen.png}}
	
	\scnheader{композиция отношений*}
	\scniselement{квазибинарное отношение}
	\scntext{определение}{\textbf{\textit{композиция отношений*}} -- это \textit{квазибинарное отношение}, связывающее два бинарных отношения с отношением, являющимся их композицией. Под композицией бинарных отношений \textbf{\textit{R}} и \textbf{\textit{S}} будем понимать множество $\{(x, y) | \exists z(xSz \wedge zRy)\}$}
%	\scnrelfrom{описание примера}{
%	\scnfilescg{figures/sd_relations/relationshipComposition.png}}
	
	% \scnheader{фактор-множество*}
	% \scnidtf{быть фактор-множеством*}
	% \scnidtf{множество всевозможных максимальных множеств из попарно эквивалентных элементов*}
	% \scnidtf{множество всевозможных классов эквивалентности для заданного отношения эквивалентности*}
	% \scniselement{бинарное отношение}
	% \scntext{определение}{\textbf{\textit{фактор множество*}} -- это бинарное ориентированное отношение, каждая связка которого связывает некоторое отношение эквивалентности со множеством всех соответствующих этому отношению классов эквивалентности. Каждый такой класс представляет собой максимальное множество сущностей, каждая пара которых принадлежит указанному выше отношению эквивалентности.}
	% \scnrelfrom{описание примера}{
	% \scnfilescg{figures/sd_relations/factor_set.png}}
	
	\scnheader{метаотношение}
	\scntext{определение}{метаотношение -- это \textit{отношение}, в каждой связке которого есть по крайней мере один компонент, являющийся знаком некоторого \textit{отношения}.}
	
	\scnheader{отношение декомпозиции}
	\scnhaselement{разбиение*}
	\scnhaselement{декомпозиция раздела*}
	\scnhaselement{декомпозиция абстрактного объекта*}
	
	% \scnheader{отношение интеграции}
	% \scnhaselement{объединение*}
	
	\scnheader{соответствие*}
	\scnidtf{наличие соответствия*}
	\scniselement{бинарное отношение}
	\begin{scnsubdividing}
		\scnitem{соответствие между непересекающимися множествами*}
		\scnitem{соответствие между строго пересекающимися множествами*}
		\scnitem{соответствие, область отправления и область прибытия которого совпадают*}
	\end{scnsubdividing}
\begin{scnsubdividing}
	\scnitem{всюду определенное соответствие*}
	\scnitem{частично определенное соответствие*}
\end{scnsubdividing}
\begin{scnsubdividing}
	\scnitem{сюръекция*}
	\scnitem{несюръективное соответствие*}
\end{scnsubdividing}
\begin{scnsubdividing}
	\scnitem{однозначное соответствие*}
	\scnitem{неоднозначное соответствие*}
\end{scnsubdividing}
	\scntext{определение}{\textbf{\textit{соответствие*}} -- \textit{бинарное ориентированное отношение}, каждая пара которого связывает два множества и указывает на наличие некоторого отношения, связывающего элементы этих двух множеств.}
%	\scnrelfrom{описание примера}{
%	\scnfilescg{figures/sd_relations/conformity.png}}
	
	\scnheader{отношение соответствия*}
	\scniselement{бинарное отношение}
	\scntext{определение}{\textbf{\textit{отношение соответствия*}} -- \textit{бинарное отношение}, связывающее ориентированную пару множеств, на которых задано \textit{соответствие*} и некоторое подмножество \textit{декартова произведения*} этих \textit{множеств}.}
%	\scnrelfrom{описание примера}{
%	\scnfilescg{figures/sd_relations/relationshipConformity.png}}
	
	\scnheader{область отправления\scnrolesign}
	\scnidtf{область отправления соответствия\scnrolesign}
	\scnidtf{область определения соответствия\scnrolesign}
	\scnidtf{первый компонент пары в отношении соответствия\scnrolesign}
	\scniselement{ролевое отношение}
	\scntext{определение}{\textbf{\textit{область отправления\scnrolesign}} -- \textit{ролевое отношение}, указывающее на первый компонент пары в рамках отношения \textit{соответствие*}.}
%	\scnrelfrom{описание примера}{
%	\scnfilescg{figures/sd_relations/departureArea.png}}
	
	\scnheader{область прибытия\scnrolesign}
	\scnidtf{область прибытия соответствия\scnrolesign}
	\scnidtf{область значений соответствия\scnrolesign}
	\scniselement{ролевое отношение}
	\scntext{определение}{\textbf{\textit{область прибытия\scnrolesign}} -- \textit{ролевое отношение}, указывающее на второй компонент пары в рамках отношения \textit{соответствие*}.}
%	\scnrelfrom{описание примера}{
%	\scnfilescg{figures/sd_relations/arrivalArea.png}}
	
	\scnheader{образ\scnrolesign}
	\scnidtf{образ соответствия\scnrolesign}
	\scniselement{ролевое отношение}
	\scntext{определение}{\textbf{\textit{образ\scnrolesign}} -- \textit{ролевое отношение}, указывающее на второй компонент каждой пары в рамках множества пар, которое является вторым компонентом \textit{отношения соответствия*}.}
%	\scnrelfrom{описание примера}{
%	\scnfilescg{figures/sd_relations/form.png}}
	
	\scnheader{прообраз\scnrolesign}
	\scnidtf{прообраз соответствия\scnrolesign}
	\scniselement{ролевое отношение}
	\scntext{определение}{\textbf{\textit{прообраз\scnrolesign}} -- \textit{ролевое отношение}, указывающее на первый компонент каждой пары в рамках множества пар, которое является первым компонентом \textit{отношения соответствия*}.}
%	\scnrelfrom{описание примера}{
%	\scnfilescg{figures/sd_relations/prototype.png}}
	
	\scnheader{всюду определенное соответствие*}
	\scnidtf{полное соответствие*}
	\scnidtf{наличие всюду определенного соответствия*}
	\scntext{определение}{\textbf{\textit{всюду определенное соответствие*}} -- это \textit{соответствие*}, при котором существует \textit{образ\scnrolesign} для каждого элемента \textit{области отправления\scnrolesign} данного \textit{соответствия*}.}
%	\scnrelfrom{описание примера}{
%	\scnfilescg{figures/sd_relations/surjection.png}}
%	\scnrelfrom{изображение}{
%	\scnfileimage{\includegraphics[width=0.5\linewidth]{figures/sd_relations/surjection2.png}}}
	
	
	\scnheader{частично определенное соответствие*}
	\scnidtf{наличие частично определенного соответствия*}
	\scntext{определение}{\textbf{\textit{частично определенное соответствие*}} -- это \textit{соответствие*}, при котором существует \textit{образ\scnrolesign} для некоторых, но не всех элементов \textit{области отправления\scnrolesign} данного \textit{соответствия*}.}
%	\scnrelfrom{описание примера}{
%	\scnfilescg{figures/sd_relations/partiallyDefinedConformity.png}}
%	\scnrelfrom{изображение}{
%	\scnfileimage{\includegraphics[width=0.5\linewidth]{figures/sd_relations/partiallySurjection.png}}}
	
	\scnheader{сюръективное соответствие*}
	\scnidtf{наличие сюръективного соответствия*}
	\scnidtf{сюръекция*}
	\scntext{определение}{\textbf{\textit{сюръективное соответствие*}} -- это \textit{соответствие*}, при котором существует \textit{прообраз\scnrolesign} для каждого элемента \textit{области прибытия\scnrolesign} данного \textit{соответствия*}.}
%	\scnrelfrom{описание примера}{
%	\scnfilescg{figures/sd_relations/surjectiveConformity.png}}
%	\scnrelfrom{изображение}{
%	\scnfileimage{\includegraphics[width=0.5\linewidth]{figures/sd_relations/surjectiveConformity2.png}}}
	
	\scnheader{несюръективное соответствие*}
	\scnidtf{наличие несюръективного соответствия*}
	\scntext{определение}{\textbf{\textit{несюръективное соответствие*}} -- это \textit{соответствие*}, при котором не для каждого элемента \textit{области прибытия\scnrolesign} данного \textit{соответствия*} существует \textit{прообраз\scnrolesign}.}
%	\scnrelfrom{описание примера}{
%	\scnfilescg{figures/sd_relations/nonSurjectiveConformity.png}}
%	\scnrelfrom{изображение}{
%	\scnfileimage{\includegraphics[width=0.5\linewidth]{figures/sd_relations/nonSurjectiveConformity2.png}}}
	
	\scnheader{однозначное соответствие*}
	\scnidtf{наличие однозначного соответствия*}
	\scnidtf{функциональное соответветствие*}
	\scnidtf{функция*}
	\scntext{определение}{\textbf{\textit{однозначное соответствие*}} -- это \textit{соответствие*}, при котором каждому элементу из \textit{области отправления\scnrolesign} соответствия ставится не более, чем один элемент из \textit{области прибытия\scnrolesign} соответствия.}
%	\scnrelfrom{описание примера}{
%	\scnfilescg{figures/sd_relations/singleConformity.png}}
%	\scnrelfrom{изображение}{
%	\scnfileimage{\includegraphics[width=0.5\linewidth]{figures/sd_relations/singleConformity2.png}}}
	

	%% Подумать что оставить
	% \scnheader{обратное соответствие*}
	% \scniselement{бинарное отношение}
	% \scnrelfrom{область определения}{соответствие*}
	% \scntext{определение}{\textbf{\textit{обратное соответствие*}} -- \textit{бинарное отношение}, связывающее два \textit{соответствия*}, при этом выполняются следующие условия:
	% \begin{scnitemize}
	% 	\item \textit{область отправления\scnrolesign} первого соответствия является \textit{областью прибытия\scnrolesign} второго;
	% 	\item \textit{область прибытия\scnrolesign} первого соответствия является \textit{областью отправления\scnrolesign} второго;
	% 	\item для каждой пары, входящей в состав отношения первого соответствия, существует пара, входящая в состав отношения второго соответствия, при этом \textit{образ\scnrolesign} и \textit{прообраз\scnrolesign} в рамках первой указанной пары являются соответственно \textit{прообразом\scnrolesign} и \textit{образом\scnrolesign} в рамках второй.
	% \end{scnitemize}
	% }
	
	% \scnheader{обратимое соответствие*}
	% \scnsubset{однозначное соответствие*}
	% \scntext{определение}{\textbf{\textit{обратимое соответствие*}} -- такое \textit{однозначное соответствие*}, для которого \textit{обратное соответствие*} также является \textit{однозначным соответствием*}.}
	
	% \newpage
	% \scnheader{неоднозначное соответствие*}
	% \scntext{определение}{\textbf{\textit{неоднозначное соответствие*}} -- это \textit{соответствие*}, при котором хотя бы одному элементу из \textit{области отправления\scnrolesign} соответствия ставится более, чем один элемент из \textit{области прибытия\scnrolesign} соответствия.}
	% \scnrelfrom{описание примера}{
	% \scnfilescg{figures/sd_relations/nonSingleConformity.png}}
	% \scnrelfrom{изображение}{
	% \scnfileimage{\includegraphics[width=0.5\linewidth]{figures/sd_relations/nonSingleConformity2.png}}}
	
	% \scnheader{инъективное соответствие*}
	% \scnidtf{инъекция*}
	% \scnsubset{однозначное соответствие*}
	% \scntext{определение}{\textbf{\textit{инъективное соответствие*}} -- это \textit{соответствие*}, при котором разным элементам из \textit{области отправления\scnrolesign} соответствия всегда соответствуют разные элементы из \textit{области прибытия\scnrolesign} соответствия и наоборот.}
	% \scnrelfrom{описание примера}{
	% \scnfilescg{figures/sd_relations/injectiveConformity.png}}
	% \scnrelfrom{изображение}{
	% \scnfileimage{\includegraphics[width=0.5\linewidth]{figures/sd_relations/injectiveConformity2.png}}}
	
	% \scnheader{взаимно однозначное соответствие*}
	% \scnidtf{биекция*}
	% \scnsubset{всюду определенное соответствие*}
	% \scnsubset{сюръективное соответствие*}
	% \scnsubset{инъективное соответствие*}
	% \scntext{определение}{\textbf{\textit{взаимно однозначное соответствие*}} -- это \textit{инъективное соответствие*}, являющееся всюду определенным и сюръективным.}
	% \scnrelfrom{описание примера}{
	% \scnfilescg{figures/sd_relations/bijectiveConformity.png}}
	% \scnrelfrom{изображение}{
	% \scnfileimage{\includegraphics[width=0.5\linewidth]{figures/sd_relations/bijectiveConformity2.png}}}
	
	
	\scnheader{множество сочетаний*}
	\scnidtf{множество всевозможных сочетаний*}
	\scnidtf{множество всевозможных сочетаний заданной арности из элементов заданного множества*}
	\scnidtf{множество всех неориентированных связок заданной арности*}
	\scnidtf{множество всех подмножеств заданной мощности*}
	\scnidtf{семейство всевозможных сочетаний*}
	\scntext{определение}{\textbf{\textit{множество сочетаний*}} -- \textit{отношение}, связывающее некоторое множество и семейство всевозможных множеств, имеющих значение мощности, меньше либо равное мощности исходного множества и состоящих из тех же элементов, что и это множество.}
	\scntext{утверждение}{Мощность \textbf{\textit{множества сочетаний*}} может быть вычислена как \textbf{n!/(k!(n-k)!)}, где \textbf{\textit{n}} -- мощность исходного множества, \textbf{\textit{k}} -- мощность элементов множества сочетаний.}
%	\scnrelfrom{описание примера}{
%	\scnfilescg{figures/sd_relations/setsOfCombinations.png}
%	\scnexplanation{Для Множества \textbf{\textit{Si}} представлено множество сочетаний по 2 элемента.}}
	
	\scnheader{множество размещений*}
	\scntext{определение}{\textbf{\textit{множество размещений*}} -- \textit{отношение}, связывающее некоторое множество и семейство всевозможных кортежей, имеющих значение мощности, меньше либо равное мощности исходного множества и состоящих из тех же элементов, что и это множество.}
	\scntext{утверждение}{Мощность \textbf{\textit{множества размещений*}} может быть вычислена как \textbf{n!/(n-k)!}, где \textbf{\textit{n}} -- мощность исходного множества, \textbf{\textit{k}} -- мощность элементов множества сочетаний.}
%	\scnrelfrom{описание примера}{
%	\scnfilescg{figures/sd_relations/setsOfPlacements.png}
%	\scnexplanation{Для Множества \textbf{\textit{Si}} представлено множество размещений по 2 элемента.}}
	
	\scnheader{множество перестановок*}
	\scnsubset{множество размещений*}
	\scntext{определение}{\textbf{\textit{множество перестановок*}} -- \textit{отношение}, связывающее некоторое множество и семейство всевозможных кортежей, равномощных исходному множеству и состоящих из тех же элементов, что и это множество.}
	\scntext{утверждение}{Мощность \textbf{\textit{множества перестановок*}} может быть вычислена как \textbf{n!}, где \textbf{\textit{n}} -- мощность исходного множества.}
%	\scnrelfrom{описание примера}{
%	\scnfilescg{figures/sd_relations/setsOfPermutations.png}
%	\scnexplanation{Для Множества \textbf{\textit{Si}} представлено его множество перестановок.}}
	

%%%%%%%%%%%%%%%%%

\section{Формальная онтология параметров, величин и шкал}
%% Введение в пердметную область

\section{Формальная онтология чисел и числовых структур}
%% Введение в пердметную область

\section{Формальная онтология темпоральных сущностей}
%% Введение в пердметную область
\section{Формальная онтология ситуаций и событий, описывающих динамику баз знаний ostis-систем}
%% Введение в пердметную область
\section{Формальная онтология пространственных сущностей}
%% Введение в пердметную область
\section{Формальная онтология материальных сущностей}
%% Введение в предметную область
\section{Иерархическая система онтологий верхнего уровня}
%% Введение в пердметную область

%%%%%%%%%%%%%%%%%%%%%%%%% referenc.tex %%%%%%%%%%%%%%%%%%%%%%%%%%%%%%
% sample references
% %
% Use this file as a template for your own input.
%
%%%%%%%%%%%%%%%%%%%%%%%% Springer-Verlag %%%%%%%%%%%%%%%%%%%%%%%%%%
%
% BibTeX users please use
% \bibliographystyle{}
% \bibliography{}
%
\biblstarthook{In view of the parallel print and (chapter-wise) online publication of your book at \url{www.springerlink.com} it has been decided that -- as a genreral rule --  references should be sorted chapter-wise and placed at the end of the individual chapters. However, upon agreement with your contact at Springer you may list your references in a single seperate chapter at the end of your book. Deactivate the class option \texttt{sectrefs} and the \texttt{thebibliography} environment will be put out as a chapter of its own.\\\indent
References may be \textit{cited} in the text either by number (preferred) or by author/year.\footnote{Make sure that all references from the list are cited in the text. Those not cited should be moved to a separate \textit{Further Reading} section or chapter.} If the citatiion in the text is numbered, the reference list should be arranged in ascending order. If the citation in the text is author/year, the reference list should be \textit{sorted} alphabetically and if there are several works by the same author, the following order should be used:
\begin{enumerate}
\item all works by the author alone, ordered chronologically by year of publication
\item all works by the author with a coauthor, ordered alphabetically by coauthor
\item all works by the author with several coauthors, ordered chronologically by year of publication.
\end{enumerate}
The \textit{styling} of references\footnote{Always use the standard abbreviation of a journal's name according to the ISSN \textit{List of Title Word Abbreviations}, see \url{http://www.issn.org/en/node/344}} depends on the subject of your book:
\begin{itemize}
\item The \textit{two} recommended styles for references in books on \textit{mathematical, physical, statistical and computer sciences} are depicted in ~\cite{science-contrib, science-online, science-mono, science-journal, science-DOI} and ~\cite{phys-online, phys-mono, phys-journal, phys-DOI, phys-contrib}.
\item Examples of the most commonly used reference style in books on \textit{Psychology, Social Sciences} are~\cite{psysoc-mono, psysoc-online,psysoc-journal, psysoc-contrib, psysoc-DOI}.
\item Examples for references in books on \textit{Humanities, Linguistics, Philosophy} are~\cite{humlinphil-journal, humlinphil-contrib, humlinphil-mono, humlinphil-online, humlinphil-DOI}.
\item Examples of the basic Springer style used in publications on a wide range of subjects such as \textit{Computer Science, Economics, Engineering, Geosciences, Life Sciences, Medicine, Biomedicine} are ~\cite{basic-contrib, basic-online, basic-journal, basic-DOI, basic-mono}. 
\end{itemize}
}

\begin{thebibliography}{99.}%
% and use \bibitem to create references.
%
% Use the following syntax and markup for your references if 
% the subject of your book is from the field 
% "Mathematics, Physics, Statistics, Computer Science"
%
% Contribution 
\bibitem{science-contrib} Broy, M.: Software engineering --- from auxiliary to key technologies. In: Broy, M., Dener, E. (eds.) Software Pioneers, pp. 10-13. Springer, Heidelberg (2002)
%
% Online Document
\bibitem{science-online} Dod, J.: Effective substances. In: The Dictionary of Substances and Their Effects. Royal Society of Chemistry (1999) Available via DIALOG. \\
\url{http://www.rsc.org/dose/title of subordinate document. Cited 15 Jan 1999}
%
% Monograph
\bibitem{science-mono} Geddes, K.O., Czapor, S.R., Labahn, G.: Algorithms for Computer Algebra. Kluwer, Boston (1992) 
%
% Journal article
\bibitem{science-journal} Hamburger, C.: Quasimonotonicity, regularity and duality for nonlinear systems of partial differential equations. Ann. Mat. Pura. Appl. \textbf{169}, 321--354 (1995)
%
% Journal article by DOI
\bibitem{science-DOI} Slifka, M.K., Whitton, J.L.: Clinical implications of dysregulated cytokine production. J. Mol. Med. (2000) doi: 10.1007/s001090000086 
%
\bigskip

% Use the following (APS) syntax and markup for your references if 
% the subject of your book is from the field 
% "Mathematics, Physics, Statistics, Computer Science"
%
% Online Document
\bibitem{phys-online} J. Dod, in \textit{The Dictionary of Substances and Their Effects}, Royal Society of Chemistry. (Available via DIALOG, 1999), 
\url{http://www.rsc.org/dose/title of subordinate document. Cited 15 Jan 1999}
%
% Monograph
\bibitem{phys-mono} H. Ibach, H. L\"uth, \textit{Solid-State Physics}, 2nd edn. (Springer, New York, 1996), pp. 45-56 
%
% Journal article
\bibitem{phys-journal} S. Preuss, A. Demchuk Jr., M. Stuke, Appl. Phys. A \textbf{61}
%
% Journal article by DOI
\bibitem{phys-DOI} M.K. Slifka, J.L. Whitton, J. Mol. Med., doi: 10.1007/s001090000086
%
% Contribution 
\bibitem{phys-contrib} S.E. Smith, in \textit{Neuromuscular Junction}, ed. by E. Zaimis. Handbook of Experimental Pharmacology, vol 42 (Springer, Heidelberg, 1976), p. 593
%
\bigskip
%
% Use the following syntax and markup for your references if 
% the subject of your book is from the field 
% "Psychology, Social Sciences"
%
%
% Monograph
\bibitem{psysoc-mono} Calfee, R.~C., \& Valencia, R.~R. (1991). \textit{APA guide to preparing manuscripts for journal publication.} Washington, DC: American Psychological Association.
%
% Online Document
\bibitem{psysoc-online} Dod, J. (1999). Effective substances. In: The dictionary of substances and their effects. Royal Society of Chemistry. Available via DIALOG. \\
\url{http://www.rsc.org/dose/Effective substances.} Cited 15 Jan 1999.
%
% Journal article
\bibitem{psysoc-journal} Harris, M., Karper, E., Stacks, G., Hoffman, D., DeNiro, R., Cruz, P., et al. (2001). Writing labs and the Hollywood connection. \textit{J Film} Writing, 44(3), 213--245.
%
% Contribution 
\bibitem{psysoc-contrib} O'Neil, J.~M., \& Egan, J. (1992). Men's and women's gender role journeys: Metaphor for healing, transition, and transformation. In B.~R. Wainrig (Ed.), \textit{Gender issues across the life cycle} (pp. 107--123). New York: Springer.
%
% Journal article by DOI
\bibitem{psysoc-DOI}Kreger, M., Brindis, C.D., Manuel, D.M., Sassoubre, L. (2007). Lessons learned in systems change initiatives: benchmarks and indicators. \textit{American Journal of Community Psychology}, doi: 10.1007/s10464-007-9108-14.
%
%
% Use the following syntax and markup for your references if 
% the subject of your book is from the field 
% "Humanities, Linguistics, Philosophy"
%
\bigskip
%
% Journal article
\bibitem{humlinphil-journal} Alber John, Daniel C. O'Connell, and Sabine Kowal. 2002. Personal perspective in TV interviews. \textit{Pragmatics} 12:257--271
%
% Contribution 
\bibitem{humlinphil-contrib} Cameron, Deborah. 1997. Theoretical debates in feminist linguistics: Questions of sex and gender. In \textit{Gender and discourse}, ed. Ruth Wodak, 99--119. London: Sage Publications.
%
% Monograph
\bibitem{humlinphil-mono} Cameron, Deborah. 1985. \textit{Feminism and linguistic theory.} New York: St. Martin's Press.
%
% Online Document
\bibitem{humlinphil-online} Dod, Jake. 1999. Effective substances. In: The dictionary of substances and their effects. Royal Society of Chemistry. Available via DIALOG. \\
http://www.rsc.org/dose/title of subordinate document. Cited 15 Jan 1999
%
% Journal article by DOI
\bibitem{humlinphil-DOI} Suleiman, Camelia, Daniel C. O'Connell, and Sabine Kowal. 2002. `If you and I, if we, in this later day, lose that sacred fire...': Perspective in political interviews. \textit{Journal of Psycholinguistic Research}. doi: 10.1023/A:1015592129296.
%
%
%
\bigskip
%
%
% Use the following syntax and markup for your references if 
% the subject of your book is from the field 
% "Computer Science, Economics, Engineering, Geosciences, Life Sciences"
%
%
% Contribution 
\bibitem{basic-contrib} Brown B, Aaron M (2001) The politics of nature. In: Smith J (ed) The rise of modern genomics, 3rd edn. Wiley, New York 
%
% Online Document
\bibitem{basic-online} Dod J (1999) Effective Substances. In: The dictionary of substances and their effects. Royal Society of Chemistry. Available via DIALOG. \\
\url{http://www.rsc.org/dose/title of subordinate document. Cited 15 Jan 1999}
%
% Journal article by DOI
\bibitem{basic-DOI} Slifka MK, Whitton JL (2000) Clinical implications of dysregulated cytokine production. J Mol Med, doi: 10.1007/s001090000086
%
% Journal article
\bibitem{basic-journal} Smith J, Jones M Jr, Houghton L et al (1999) Future of health insurance. N Engl J Med 965:325--329
%
% Monograph
\bibitem{basic-mono} South J, Blass B (2001) The future of modern genomics. Blackwell, London 
%
\end{thebibliography}

