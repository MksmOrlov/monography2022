\chapauthor{Никифоров С.А.\\Гойло А.А.}
\chapter{Информационные конструкции и языки}
\chapauthortoc{Никифоров С.А.\\Гойло А.А.}
\label{chapter_inf_constr}

\vspace{-7\baselineskip}

\begin{SCn}
    \begin{scnrelfromlist}{автор}
        \scnitem{Никифоров С.А.}
        \scnitem{Гойло А.А.}
    \end{scnrelfromlist}

    \bigskip

    \scntext{аннотация}{В главе уточняются базовые понятия, необходимые для формализации предметных областей информационных конструкций и языков. Определены такие понятия как знак, знаковая конструкция, информационная конструкция, синтаксис, денотационная семантика, язык и др. Кроме того, описываются параметры и отношения, заданные на множестве языков и информационных конструкций.}

    \bigskip

    \begin{scnrelfromlist}{подраздел}
        \scnitem{\ref{section_information_construction_formalization}~\nameref{section_information_construction_formalization}}
        \scnitem{\ref{section_external_information_constructions_and_languages}~\nameref{section_external_information_constructions_and_languages}}
    \end{scnrelfromlist}

    \bigskip

    \begin{scnrelfromlist}{ключевое понятие}
        \scnitem{знак}
        \scnitem{знаковая конструкция}
        \scnitem{информационная конструкция}
        \scnitem{дискретная информационная конструкция}
        \scnitem{язык}
        \scnitem{синтаксис языка}
        \scnitem{денотационная семантика языка}
        \scnitem{операционная семантика языка}
        \scnitem{смысл}
        \scnitem{идентификатор}

    \end{scnrelfromlist}

    \begin{scnrelfromlist}{ключевое знание}
        \scnitem{Отношения, заданные на множестве дискретных информационных конструкций}
        \scnitem{Соответствия, заданные на множестве дискретных информационных конструкций}
        \scnitem{Параметры, заданные на множестве дискретных информационных конструкций}
        \scnitem{Отношения, заданные на множестве языков}
        \scnitem{Параметры, заданные на множестве языков}
    \end{scnrelfromlist}

    \begin{scnrelfromlist}{ключевой знак}
        \scnitem{SC-код}
        \scnitem{SCg-код}
        \scnitem{SCn-код}
        \scnitem{SCs-код}
    \end{scnrelfromlist}

    \bigskip

    %todo: добить список, пока вставил пример
    \begin{scnrelfromlist}{библиографическая ссылка}
        \scnitem{\scncite{Standard2021}}
    \end{scnrelfromlist}

\end{SCn}

%todo: на последних созвонах говорили, что введени-заключение желательно, но не обяатльно.
% в оглавление оно не идет. так то как вариат - не делать его, если все оч грутсно.
%\section*{Введение в Главу \ref{chapter_inf_constr}}

%ы.

\section{Формализация понятия информационной конструкции}
\label{section_information_construction_formalization}

\begin{SCn}

    \scnheader{знак}
    \begin{scnrelfromset}{разбиение}
        \scnitem{знак, являющийся элементом дискретной информационной конструкции}
        \scnitem{знак, являющийся неатомарным фрагментом дискретной информационной конструкции}
    \end{scnrelfromset}

\end{SCn}

\textit{Знак} - фрагмент информационной конструкции, который условно представляет (изображает) некоторую описываемую сущность, которую называют денотатом знака.
При этом отсутствие знака, обозначающего некоторую сущность, не означает отсутствие самой этой сущности.
Это означает только то, что мы даже не догадываемся о её существовании и, следовательно, не приступили к её исследованию.

Поскольку все знаки являются дискретными информационными конструкциями, множество знаков является областью задания всех отношений, заданных на множестве дискретных информационных конструкций.
Тем не менее есть как минимум одно отношение, специфичное для множества знаков.

\begin{SCn}

    \scnheader{отношение, заданное на множестве знаков\scnsupergroupsign}
    \scnhaselement{синонимия знаков*}

\end{SCn}

Знаки являются синонимичными в том и только в том случае, если они обозначают одну и ту же сущность.
При этом синонимичные знаки могут быть синтаксически эквивалентными, а могут и не быть таковыми.

\begin{SCn}

    \scnheader{знаковая конструкция}
    \scnsubset{дискретная информационная конструкция}

\end{SCn}

\textit{Знаковая конструкция} - дискретная информационная конструкция, которая в общем случае представляет собой конфигурацию знаков и специальных фрагментов информационной конструкции, обеспечивающих структуризацию конфигурации знаков — различного вида разделителей и ограничителей.
Для некоторых знаковых конструкций и даже для некоторых языков необходимость в разделителях и ограничителях может отсутствовать.

\begin{SCn}

    \scnheader{отношение, заданное на множестве знаковых конструкций\scnsupergroupsign}
    \scnhaselement{знак*}
    \scnhaselement{разделитель знаковой конструкции*}
    \scnhaselement{разделители знаковой конструкции*}
    \scnhaselement{ограничитель знаковой конструкции*}
    \scnhaselement{ограничители знаковой конструкции*}
    \scnhaselement{семантическая смежность знаковых конструкций*}
    \scnhaselement{конкатенация знаковых конструкций*}
        \begin{scnindent}
        \scnidtf{декомпозиция заданной знаковой конструкции на последовательность знаковых конструкций*}
        \end{scnindent}

\end{SCn}

\textit{Знак*} - бинарное ориентированное отношение, связывающее знаковую конструкцию со множеством всех знаков, входящих в её состав.

\textit{Семантическая смежность знаковых конструкций*} - бинарное отношение, связывающее семантически смежные знаковые конструкции.
При этом знаковые конструкции $\bm{Ti}$ и $\bm{Tj}$ являются смежными в том и только в том случае, если существуют синонимичные знаки $\bm{Ti}$ и $\bm{Tj}$, один из которых входит в состав конструкции $\bm{Ti}$, а второй — в состав конструкции $\bm{Tj}$

\begin{SCn}

    \scnheader{класс знаковых конструкций\scnsupergroupsign}
    \scnhaselement{семантически элементарная знаковая конструкция}
    \scnhaselement{семантически связная знаковая конструкция}

\end{SCn}

\textit{Семантически элементарная знаковая конструкция} - знаковая конструкция, описывающая некоторую (одну) связь между некоторыми сущностями.

\textit{Семантически связная знаковая конструкция} - знаковая конструкция, которую можно представить в виде конкатенации семантически элементарных знаковых конструкций, каждая из которых семантически смежна предшествующей и последующей семантически элементарной знаковой конструкции

\begin{SCn}

    \scnheader{параметр, заданный на множестве знаковых конструкций\scnsupergroupsign}
    \scnhaselement{семантическая связность знаковых конструкций\scnsupergroupsign}
    \begin{scnindent}
        \scnhaselement{семантически связная знаковая конструкция}
        \scnhaselement{семантически несвязная знаковая конструкция}
    \end{scnindent}
    \scnhaselement{наличие разделителей и ограничителей\scnsupergroupsign}
    \begin{scnindent}
        \scnhaselement{знаковая конструкция, содержащая разделители и-или ограничители}
        \scnhaselement{знаковая конструкция без разделителей и ограничителей}
    \end{scnindent}

\end{SCn}

\textit{Информационная конструкция} - конструкция (структура), содержащая некоторые сведения о некоторых сущностях.
Форма представления ("изображения"{}, "материализации"{}), форма структуризации (синтаксическая структура), а также \textit{смысл*} (денотационная семантика) \textit{информационных конструкций} могут быть самыми различными.

\textit{Дискретная информационная конструкция} — это \textit{информационная конструкция}, смысл которой задается:
\begin{textitemize}
    \item множеством элементов (синтаксически атомарных фрагментов) этой информационной конструкции,
    \item алфавитом этих элементов — семейством классов синтаксически эквивалентных элементов информационной конструкции,
    \item принадлежностью каждого элемента информационной конструкции соответствующему классу синтаксически эквивалентных элементов информационной конструкции,
    \item конфигурацией связей инцидентности между элементами информационной конструкции.
\end{textitemize}

Следствием этого является то, что форма представления элементов дискретной информационной конструкции для анализа её смысла не требует уточнения.
Главным является:
\begin{textitemize}
    \item наличие простой процедуры выделения (сегментации) элементов дискретной информационной конструкции,
    \item наличие простой процедуры установления синтаксической эквивалентности разных элементов дискретной информационной конструкции,
    \item наличие простой процедуры установления принадлежности каждого элемента дискретной информационной конструкции соответствующему классу синтаксически эквивалентных элементов (т. е. соответствующему элементу алфавита).
\end{textitemize}

\textit{Элементом дискретной информационной конструкции} является синтаксически атомарный фрагмент (символ), входящий в состав дискретной информационной конструкции.
Поскольку дискретные информационные конструкции могут иметь общие элементы (атомарные фрагменты) и даже некоторые из них могут быть частями других информационных конструкций, элемент дискретной информационной конструкции может входить в состав сразу нескольких информационных конструкций.

Далее рассмотрим отношения, заданные на множестве элементов дискретных информационных конструкций.

\begin{SCn}

    \scnheader{отношение, заданное на множестве элементов дискретных информационных конструкций\scnsupergroupsign}
    \scnhaselement{элемент дискретной информационной конструкции*}
    \scnhaselement{синтаксическая эквивалентность элементов дискретных информационных конструкций*}
    \scnhaselement{инцидентность элементов дискретных информационных конструкций*}

\end{SCn}

\textit{Элемент дискретной информационной конструкции*} - бинарное ориентированное отношение, каждая пара которого связывает (1) знак некоторой дискретной информационной конструкции и (2) знак одного из элементов этой дискретной информационной конструкции*.

\textit{Cинтаксическая эквивалентность элементов дискретных информационных конструкций*} - отношение, связывающее синтаксически эквивалентные элементы (атомарные фрагменты) одной и той же или разных дискретных информационных конструкций, т. е. элементы, принадлежащими одному и тому же классу синтаксически эквивалентных \textit{элементов дискретных информационных конструкций*}.

\textit{Инцидентность элементов дискретных информационных конструкций*} для \textit{линейных информационных конструкций} -- это последовательность элементов (символов), входящих в состав этих конструкций.
Для дискретных информационных конструкций, конфигурация которых имеет нелинейный характер, отношение инцидентности их элементов может быть разбито на несколько частных отношений инцидентности, каждое из которых является \underline{подмножеством} объединенного отношения инцидентности.
Например, для двухмерных дискретных информационных конструкций это (1) инцидентность элементов информационных конструкций "по горизонтали"{} и (2) инцидентность элементов информационных конструкций "по вертикали"{}.

Далее рассмотрим отношения, заданные на множестве дискретных информационных конструкций.

\begin{SCn}

\scnheader{отношение, заданное на множестве элементов дискретных информационных конструкций\scnsupergroupsign}
\scnhaselement{неэлементарный фрагмент дискретной информационной конструкции*}
\scnhaselement{алфавит дискретной информационной конструкции*}
\scnhaselement{первичная синтаксическая структура дискретной информационной конструкции*}
\scnhaselement{синтаксическая эквивалентность дискретных информационных конструкций*}
\scnhaselement{копия дискретной информационной конструкции*}
\scnhaselement{семантическая эквивалентность дискретных информационных конструкций*}
\scnhaselement{семантическое расширение дискретной информационной конструкции*}
\scnhaselement{синтаксис информационной конструкции*}
\scnhaselement{смысл*}
\scnhaselement{операционная семантика информационной конструкции*}

\end{SCn}

\textit{Неэлементарный фрагмент дискретной информационной конструкции*} - бинарное ориентированное отношение, связывающее заданную дискретной информационной конструкцию с дискретной информационной конструкцией, которая является \underline{подструктурой} для нее, в состав которой входит (1) подмножество элементов заданной информационной конструкции и, соответственно, (2) подмножество пар инцидентности элементов заданной информационной конструкции.

\textit{Алфавит дискретной информационной конструкции*} - бинарное отношение, связывающее дискретную информационную конструкцию с семейством попарно непересекающихся \underline{классов} синтаксически эквивалентных элементов заданной дискретной информационной конструкции*

\textit{Первичная синтаксическая структура дискретной информационной конструкции*} - бинарное ориентированное отношение, связывающее дискретную информационную конструкцию с \textit{графовой структурой}, которая полностью описывает ее конфигурацию и которая включает в себя: (1) знаки всех тех классов синтаксически эквивалентных элементов, которым принадлежат элементы описываемой дискретной информационной конструкции, (2) знаки всех элементов (атомарных фрагментов) описываемой информационной конструкции, (3) пары, описывающие инцидентность элементов описываемой информационной конструкции, (4) пары, описывающие принадлежность элементов описываемой информационной конструкции соответствующим классам синтаксически эквивалентных элементов этой информационной конструкции.

\textit{Синтаксическая эквивалентность дискретных информационных конструкций*}: дискретные информационные конструкции $\bm{Ti}$ и $\bm{Tj}$ являются синтаксически эквивалентными в том и только в том случае, если между конструкцией $\bm{Ti}$ и конструкцией $\bm{Tj}$ существует \underline{изоморфизм}, в рамках которого каждому элементу конструкции $\bm{Ti}$ соответствует синтаксически эквивалентный элемент конструкции $\bm{Tj}$, т. е. элемент, принадлежащий тому же классу синтаксически эквивалентных элементов дискретных информационных конструкций.
И наоборот.

\textit{Копия дискретной информационной конструкции*} - бинарное ориентированное отношение, которое связывает дискретную информационную конструкцию с дискретной информационной конструкцией, которая не только синтаксически эквивалентна ей, но и содержит информацию о форме представления элементов данной копируемой информационной конструкции*

\begin{SCn}
    \scnheader{копия дискретной информационной конструкции*}
    \scnsubset{синтаксическая эквивалентность дискретных информационных конструкций*}
\end{SCn}

\textit{Семантическая эквивалентность дискретных информационных конструкций*}: информационная конструкция $\bm{Ti}$ и информационная конструкция $\bm{Tj}$ являются \underline{семантически эквивалентными} в том и только в том случае, если \underline{каждая} сущность (в том числе, и каждая связь между сущностями), описываемая в информационной конструкции $\bm{Ti}$ описывается также и в информационной конструкции $\bm{Tj}$.
И наоборот.

\textit{Семантическое расширение дискретной информационной конструкции*}: информационная конструкция $\bm{Tj}$ является семантическим расширением информационной конструкции $\bm{Ti}$ в том и только в том случае, если \underline{каждая} сущность, описываемая в $\bm{Ti}$, описывается также и в $\bm{Tj}$, но обратное неверно.

\textit{Cинтаксис информационной конструкции*} - описание того, из каких частей состоит заданная информационная конструкция и как эти части (фрагменты) связаны между собой.

%тут такое ощущение, что что-то пошло не так?
\textit{Смысл*} (\textit{денотационная семантика информационной конструкции*}) - каждая пара которого связывает некоторую информационную конструкцию с ее явным (формальным) представлением того, какие сущности описывает данная информационная конструкция и как эти сущности связаны между собой.

\textit{Операционная семантика информационной конструкции*} - бинарное ориентированное отношение, каждая пара которого связывает знак некоторой информационной конструкции со множеством правил ее трансформации - описанием того, на основании каких правил можно осуществлять действия по преобразования (обработке, трансформации) заданной информационной конструкции, оставляя ее в рамках класса синтаксически и семантически правильных информационных конструкций.

\begin{SCn}

    \scnheader{операционная семантика информационной конструкции*}
    \scnrelfrom{второй домен}{операционная семантика информационной конструкции}

\end{SCn}

Далее рассмотрим заданные на множестве дискретных информационных конструкций соответствия.

\begin{SCn}

    \scnheader{соответствие, заданное на множестве дискретных информационных конструкций}
    \scnhaselement{соответствие между синтаксической структурой информационной конструкции и смыслом этой конструкции*}
    \begin{scnindent}
        \scnsubset{соответствие*}
    \end{scnindent}

\end{SCn}

\textit{Соответствие, заданное на множестве дискретных информационных конструкций} - множество ориентированных пар, первым компонентом которых является ориентированная пара, состоящая из (1) знака синтаксической структуры некоторой информационной конструкции и (2) знака смысловой структуры этой конструкции, а вторым компонентом которых является множество ориентированных пар, связывающих фрагменты синтаксической структуры заданной информационной конструкции (которые описывают либо структуру фрагментов заданной конструкции, либо связи между фрагментами этой конструкции) с теми фрагментами смысловой структуры заданной информационной конструкции, которые семантически эквивалентны либо синтаксически представленным фрагментам заданной информационной конструкции, либо синтаксически представленным связям между такими фрагментами.

\begin{SCn}

    \scnheader{параметр, заданный на множестве дискретных информационных конструкций\scnsupergroupsign}
    \scnhaselement{размерность дискретных информационных конструкций\scnsupergroupsign}
    \begin{scnindent}
        \scnidtf{типология дискретных информационных конструкций, определяемая их размерностью}
        \scnhaselement{линейная информационная конструкция}
        \scnhaselement{двухмерная информационная конструкция}
        \scnhaselement{трехмерная информационная конструкция}
        \scnhaselement{четырехмерная информационная конструкция}
        \scnhaselement{графовая информационная конструкция}
    \end{scnindent}

\end{SCn}

\textit{Линейная информационная конструкция} - дискретная информационная конструкция, каждый элемент которой может иметь не более двух инцидентных ему элементов (один слева, другой справа).

\textit{Двухмерная информационная конструкция} - дискретная информационная конструкция, каждый элемент которой может иметь не более четырех инцидентных ему элементов (слева-справа, сверху-снизу).

\textit{Трехмерная информационная конструкция} - дискретная информационная конструкция, каждый элемент которой может иметь не более шести инцидентных ему элементов (слева-справа, сверху-снизу, сзади-спереди).

\textit{Четырехмерная информационная конструкция} - дискретная информационная конструкция, каждый элемент которой может иметь не более восьми инцидентных ему элементов (например, слева-справа, сверху-снизу, сзади-спереди, раньше-позже).

\textit{Графовая информационная конструкция} - дискретная информационная конструкция, множество элементов которой разбивается на два подмножества — связки и узлы.
При этом узлы могут иметь \underline{неограниченное} количество инцидентных им связок.
В некоторых графовых информационных конструкциях и связки могут иметь неограниченное количество инцидентных им других связок.

\begin{SCn}

    \scnheader{параметр, заданный на множестве дискретных информационных конструкций\scnsupergroupsign}
    \scnhaselement{типология дискретных информационных конструкций, определяемая их носителем\scnsupergroupsign}
    \begin{scnindent}
        \scnhaselement{некомпьютерная форма представления дискретных информационных конструкций}
        \begin{scnindent}
            \scnsuperset{аудио-сообщение}
            \scnsuperset{информационная конструкция, представленная на языке жестов}
            \scnsuperset{информационная конструкция, представленная в письменной форме}
        \end{scnindent}
        \scnhaselement{файл}
    \end{scnindent}

\end{SCn}

Представление информационных конструкций в виде \textit{файлов} ориентировано на представление \underline{дискретных} (!) информационных конструкций.
Поэтому "файловое"{} представление недискретных информационных конструкций (например, различного рода сигналов) предполагает "дискретизацию"{} таких конструкций, т. е. преобразование их в дискретные.
Так преобразуются аудио-сигналы (в частности, речевые сообщения), изображения, видео-сигналы и др.

\begin{SCn}

    \scnheader{параметр, заданный на множестве дискретных информационных конструкций\scnsupergroupsign}
    \scnhaselement{уровень унификации представления синтаксически эквивалентных элементов дискретных информационных конструкций\scnsupergroupsign}
    \begin{scnindent}
        \scnhaselement{дискретная информационная конструкция с низким уровнем унификации представления элементов}
        \begin{scnindent}
            \scnsuperset{аудио-сообщение}
            \scnsuperset{информационная конструкция, представленная на языке жестов}
            \scnsuperset{рукопись или её копия}
        \end{scnindent}
        \scnhaselement{дискретная информационная конструкция с высоким уровнем унификации представления элементов}
        \begin{scnindent}
            \scnsuperset{печатный текст}
            \scnsuperset{файл}
        \end{scnindent}
    \end{scnindent}

\end{SCn}

\textit{Уровень унификации представления синтаксически эквивалентных элементов дискретных информационных конструкций\scnsupergroupsign} - уровень "членораздельности"{} дискретных информационных конструкций.
Чем выше уровень унификации представления элементов дискретных информационных конструкций, тем проще реализуется (1) процедура выделения (сегментации) элементов дискретной информационной конструкции, (2) процедура установления синтаксической эквивалентности этих элементов и (3) процедура их распознавания, т. е. процедура установления их принадлежности соответствующим классам синтаксически эквивалентных элементов.

Чем выше уровень унификации представления элементов дискретных информационных конструкций, тем проще реализуется:
\begin{textitemize}
    \item процедура выделения (сегментации) элементов дискретной информационной конструкции,
    \item процедура установления синтаксической эквивалентности этих элементов,
    \item процедура их распознавания, т. е. процедура установления их принадлежности соответствующим классам синтаксически эквивалентных элементов.
\end{textitemize}

Уточнив понятия знака, знаковой конструкции, информационной конструкции, дискретной информационной конструкции и рассмотрев соответствующие отношения, можно перейти к формализации понятия язык.

\textit{Язык} - класс знаковых конструкций, для которого существуют (1) общие правила их построения и (2) общие правила их соотнесения с теми сущностями и конфигурациями сущностей, которые описываются (отражаются) указанными знаковыми конструкциями.

\begin{SCn}

    \scnheader{язык}
    \begin{scnrelfromset}{разбиение}
        \scnitem{язык, у которого все знаки, входящие в состав его знаковых конструкций, являются элементарными фрагментами этих конструкций}
        \scnitem{язык, у которого знаки, входящие в состав его знаковых конструкций, в общем случае не являются элементарными фрагментами этих конструкций}
        \begin{scnindent}
	        \begin{scnrelfromset}{разбиение}
	            \scnitem{язык, знаковые конструкции которого содержат разделители и ограничители}
	            \scnitem{язык, знаковые конструкции которого не содержат разделителей и ограничителей}
    	    \end{scnrelfromset}
	    \end{scnindent}
    \end{scnrelfromset}

\end{SCn}

Для \textit{языков, у которого все знаки, входящие в состав его знаковых конструкций, являются элементарными фрагментами этих конструкций} существенно упрощаются методы обработки их текстов.

\textit{язык, знаковые конструкции которого не содержат разделителей и ограничителей} - язык, знаковые конструкции такого языка состоят \underline{только} из знаков.

\textit{Отношение, заданное на множестве языков\scnsupergroupsign} - отношение, область определения которого включает в себя множество всевозможных языков.

\textit{Текст заданного языка*} - бинарное отношение, связывающее язык и синтаксически правильную (правильно построенную) знаковую конструкцию данного языка.
\textit{Синтаксически корректная знаковая конструкция для заданного языка*} - бинарное отношение, связывающее язык и знаковую конструкцию, не содержащая синтаксических ошибок для данного языка.

\begin{SCn}
    \scnheader{отношение, заданное на множестве языков\scnsupergroupsign}
    \scnidtf{отношение, область определения которого включает в себя множество всевозможных языков}
    \scnhaselement{текст заданного языка*}
    \begin{scnindent}
        \scneq{{\normalfont(}синтаксически корректная знаковая конструкция для заданного языка* $\cap$ синтаксически целостная знаковая конструкция для заданного языка*{\normalfont)}}
    \end{scnindent}
    \scnhaselement{синтаксически корректная знаковая конструкция для заданного языка*}
    \scnhaselement{синтаксически целостная знаковая конструкция для заданного языка*}
    \scnhaselement{синтаксически неправильная знаковая конструкция для заданного языка*}
    \begin{scnindent}
        \scneq{{\normalfont(}синтаксически некорректная знаковая конструкция для заданного языка* $\cup$ синтаксически нецелостная знаковая конструкция для заданного языка*{\normalfont)}}
        \scnsuperset{синтаксически некорректная знаковая конструкция для заданного языка*}
        \scnsuperset{синтаксически нецелостная знаковая конструкция для заданного языка*}
    \end{scnindent}
    \scnhaselement{знание, представленное на заданном языке*}
    \begin{scnindent}
        \scnidtf{семантически правильный текст заданного языка*}
        \scneq{(семантически корректный текст заданного языка* $\cap$ семантически целостный текст заданного языка*)}
        \scnidtf{истинный текст заданного языка*}
        \scnidtf{истинное высказывание, представленное на заданном языке*}
    \end{scnindent}
    \scnhaselement{семантически корректный текст заданного языка*}
    \begin{scnindent}
        \scnidtf{текст заданного языка, не содержащий семантических ошибок, противоречащих признанным закономерностям и фактам*}
    \end{scnindent}
    \scnhaselement{семантически целостный текст заданного языка*}
    \begin{scnindent}
        \scnidtf{текст заданного языка, содержащий достаточную информацию для установления его истинности*}
    \end{scnindent}
    \scnhaselement{семантически неправильный текст для заданного языка*}
    \begin{scnindent}
        \scneq{(семантически некорректный текст для заданного языка* $\cup$ семантически нецелостный текст для заданного языка*)}
        \scnsuperset{семантически некорректный текст для заданного языка*}
        \scnsuperset{семантически нецелостный текст для заданного языка*}
    \end{scnindent}
    \scnhaselement{алфавит*}
    \begin{scnindent}
        \scnidtf{алфавит заданной информационной конструкции или заданного языка*}
        \scnidtf{семейство классов синтаксически эквивалентных элементов (элементарных фрагментов) заданной информационной конструкции или информационных конструкций заданного языка*}
    \end{scnindent}
    \scnhaselement{семейство классов синтаксически эквивалентных разделителей*}
    \begin{scnindent}
        \scnidtf{семейство классов синтаксически эквивалентных разделителей, входящих в состав заданной информационной конструкции или в состав информационных конструкций заданного языка*}
    \end{scnindent}
    \scnhaselement{семейство классов синтаксически эквивалентных ограничителей*}
    \begin{scnindent}
        \scnidtf{семейство классов синтаксически эквивалентных ограничителей, входящих в состав заданной информационной конструкции или в состав информационных конструкций заданного языка*}
    \end{scnindent}
    \scnhaselement{синтаксис языка*}
    \begin{scnindent}
        \scnidtf{быть теорией правильно построенных информационных конструкций, принадлежащих заданному языку*}
        \scnidtf{определение понятия правильно построенной информационной конструкции заданного языка*}
        \scnidtf{синтаксические правила заданного языка*}
        \scnidtf{быть синтаксическими правилами заданного языка*}
        \scnidtf{бинарное ориентированное отношение, каждая пара которого связывает знак некоторого языка с описанием синтаксически выделяемых классов фрагментов конструкций заданного языка, с описанием отношений, заданных на этих классах и с конъюнкцией кванторных высказываний, являющихся синтаксическими правилами заданного языка, то есть правилами, которым должны удовлетворять все синтаксические правильные (правильно построенные) конструкции (тексты) указанного языка*}
        \scntext{примечание}{При представлении синтаксиса (синтаксических правил) заданного языка используются все те понятия, которые вводятся для представления синтаксических структур информационных конструкций, принадлежащих указанному языку. Это и синтаксически выделяемые классы фрагментов указанных информационных конструкций, и отношения, заданные на множестве таких фрагментов.}
        \scnrelfrom{второй домен}{синтаксис языка}
    \end{scnindent}
    \scnhaselement{описание синтаксических понятий языка*}
    \begin{scnindent}
        \scnidtf{описание синтаксически выделяемых классов фрагментов конструкций заданного языка*}
        \scnrelfrom{второй домен}{описание синтаксических понятий языка}
        \begin{scnindent}
            \scnrelto{обобщенное включение}{синтаксис языка}
        \end{scnindent}
    \end{scnindent}
    \scnhaselement{синтаксические правила языка*}
    \begin{scnindent}
        \scnidtf{синтаксические правила заданного языка*}
        \scnrelfrom{второй домен}{синтаксические правила языка}
    \end{scnindent}
    \scnhaselement{денотационная семантика языка*}
    \begin{scnindent}
        \scnidtf{быть теорией морфизмов, связывающих правильно построенные информационные конструкции заданного языка с описываемыми конфигурациями описываемых сущностей*}
    \end{scnindent}

    \scnhaselement{денотационная семантика языка*}
    \begin{scnindent}
        \scnidtf{семантические правила заданного языка*}
        \scnidtf{быть семантическими правилами заданного языка*}
        \scnidtf{бинарное ориентированное отношение, каждая пара которого связывает знак некоторого языка с описанием основных семантических понятий заданного языка и конъюнкцией кванторных высказываний, являющихся семантическими правилами заданного языка, то есть правилами, которым должны удовлетворять семантически правильные \underline{смысловые} информационные конструкции, соответствующие (семантические эквивалентные) синтаксически правильным конструкциям (текстам) заданного языка*}
        \scntext{примечание}{При формулировке семантических правил заданного языка используются понятия, введенные в рамках базовых онтологий (онтологий самого высокого уровня, в которых рассматриваются самые общие классы описываемых сущностей, самые общие отношения и параметры).}
        \scnrelfrom{второй домен}{денотационная семантика языка}
    \end{scnindent}
    \scnhaselement{описание семантических понятий языка*}
    \begin{scnindent}
        \scnrelfrom{второй домен}{описание семантических понятий языка}
    \end{scnindent}
    \scnhaselement{семантические правила языка*}
    \begin{scnindent}
        \scnrelfrom{второй домен}{семантические правила языка}
    \end{scnindent}
    \bigskip
    \scnhaselement{семантическая эквивалентность языков*}
    \begin{scnindent}
        \scnidtf{быть семантически эквивалентными языками*}
        \scntext{определение}{Язык $\bm{Li}$ и язык $\bm{Lj}$ будем считать \textit{семантически эквивалентными языками*} в том и только в том случае, если для каждого текста, принадлежащего языку $\bm{Li}$, существует \textit{семантически эквивалентный текст*}, принадлежащий языку $\bm{Lj}$, и наоборот.}
    \end{scnindent}
    \scnhaselement{семантическое расширение языка*}
    \begin{scnindent}
        \scnrelboth{обратное отношение}{семантическое сужение языка*}
        \scntext{определение}{Язык $\bm{Lj}$ будем считать \textit{семантическим расширением*} языка $\bm{Li}$ в том и только в том случае, есть ли для каждого текста, принадлежащего языку $\bm{Li}$, существует \textit{семантически эквивалентный текст*}, принадлежащий языку $\bm{Lj}$, но обратное неверно.}
    \end{scnindent}
    \scnhaselement{синтаксическое расширение языка*}
    \begin{scnindent}
        \scnidtf{быть семантически эквивалентным надмножеством заданного языка*}
        \scntext{определение}{Язык $\bm{Lj}$ будем считать \textit{синтаксическим расширением*} языка $\bm{Li}$ в том и только в том случае, если:
            \begin{scnitemize}
                \item $\bm{L_j} \supset \bm{Li}$ (то есть все тексты языка $\bm{Li}$ являются также и текстами языка $\bm{Lj}$, но обратное неверно);
                \item Язык $\bm{Lj}$ и язык $\bm{Li}$ являются \textit{семантически эквивалентными языками*}.
            \end{scnitemize}
        }
    \end{scnindent}
    \scnhaselement{синтаксическое ядро языка*}
    \begin{scnindent}
        \scnidtf{быть синтаксическим ядром заданного языка*}
        \scnidtf{быть семантически эквивалентным подмножеством заданного языка, имеющим минимальную синтаксическую сложность*}
    \end{scnindent}
    \scnhaselement{направление синтаксического расширения ядра заданного языка*}
    \begin{scnindent}
        \scnidtf{быть правилом трансформации информационных конструкций, принадлежащих заданному языку, которое описывает одно из направлений перехода от множества конструкций ядра этого языка ко множеству всех принадлежащих ему информационных конструкций*}
    \end{scnindent}
    \scnhaselement{операционная семантика языка*}
    \begin{scnindent}
        \scnidtf{бинарное ориентированное отношение, каждая пара которого связывает знак некоторого языка со множеством правил трансформации текстов этого языка*}
        \scnrelfrom{второй домен}{операционная семантика языка}
    \end{scnindent}
    \scnhaselement{внутренний язык*}
    \begin{scnindent}
        \scnidtf{быть внутренним языком для заданной системы, основанной на обработке информации, или заданного множества таких систем*}
        \scnidtf{быть языком внутреннего представления информации в памяти заданной системы, основанной на обработке информации или заданного класса таких систем*}
    \end{scnindent}
    \scnhaselement{внешний язык*}
    \begin{scnindent}
        \scnidtf{быть внешним языком для заданной системы, основанной на обработке информации, или заданного множества таких систем*}
        \scnidtf{быть языком, используемым для обмена информацией заданной системы, основанной на обработке информации, или заданного множества таких систем с другими системами, основанными на обработке информации, (в том числе, и с себе подобными)*}
    \end{scnindent}
    \scnhaselement{используемый язык*}
    \begin{scnindent}
        \scneq{{\normalfont(}внутренний язык* $\cup$ внешний язык*{\normalfont)}}
        \scnidtf{язык, используемый заданной системой, основанной на обработке информации или заданного множества таких систем*}
        \scnidtf{язык, носителем которого является (которым владеет) данная система, основанная на обработке информации}
    \end{scnindent}
    \scnhaselement{используемые языки*}

\end{SCn}

\textit{Параметр, заданный на множестве языков\scnsupergroupsign} - семейство классов эквивалентности языков, трактуемой в контексте того или иного свойства (характеристики), присущего языкам.

\begin{SCn}

    \scnheader{параметр, заданный на множестве языков\scnsupergroupsign}
    \scnhaselement{семантическая мощность языка\scnsupergroupsign}
    \begin{scnindent}
        \scnidtf{класс языков, семантически эквивалентных друг другу}
        \scnhaselement{универсальный язык}
        \begin{scnindent}
            \scnidtf{класс всевозможных универсальных языков}
            \scntext{примечание}{Очевидно, что все универсальные языки (если они действительно таковыми являются, а не только претендуют на это) семантически эквивалентны друг другу, т. е. имеют одинаковую семантическую мощность.}
        \end{scnindent}
    \end{scnindent}
    \scnhaselement{уровень синтаксической сложности представления знаков в текстах языка\scnsupergroupsign}
    \begin{scnindent}
        \scnhaselement{язык, в текстах которого все знаки представлены синтаксически элементарными фрагментами}
        \scnhaselement{язык, в текстах которого знаки в общем случае представлены синтаксически неэлементарными фрагментами}
    \end{scnindent}
    \scnhaselement{использование разделителей и ограничителей в текстах языка\scnsupergroupsign}
    \begin{scnindent}
        \scnhaselement{язык, в текстах которого не используются разделители и ограничители}
        \scnhaselement{язык, в текстах которого используются разделители и ограничители}
    \end{scnindent}
    \scnhaselement{уровень сложности процедуры установления синонимии знаков в текстах языка\scnsupergroupsign}
    \begin{scnindent}
        \scnhaselement{язык, в рамках каждого текста которого синонимичные знаки отсутствуют}
        \begin{scnindent}
            \scntext{пояснение}{В текстах такого языка знак каждой описываемой сущности входит \underline{однократно}.}
        \end{scnindent}
        \scnhaselement{язык, в рамках которого синонимичные знаки представлены синтаксически эквивалентными фрагментами текстов}
        \scnhaselement{флективный язык}
        \begin{scnindent}
            \scnidtf{язык, в рамках которого синонимичные знаки могут быть представлены синтаксически неэквивалентными фрагментами, но фрагментами, являющимися модификациями некоторого "ядра"{} этих фрагментов (при склонении и спряжении этих знаков).}
        \end{scnindent}
        \scnhaselement{язык, в рамках которого синонимичные знаки в общем случае могут быть представлены синтаксически неэквивалентными текстовыми фрагментами, структура которых носит непредсказуемый характер}
    \end{scnindent}
    \scnhaselement{наличие омонимии в текстах языка\scnsupergroupsign}
    \begin{scnindent}
        \scnhaselement{язык, в текстах которого присутствует омонимия знаков}
        \begin{scnindent}
            \scnidtf{язык, в текстах которого присутствуют синтаксически эквивалентные, не синонимичные знаки}
        \end{scnindent}
        \scnhaselement{язык, в текстах которого омонимия знаков отсутствует}
    \end{scnindent}

    \scnheader{семантически выделяемый класс дискретных информационных конструкций}
    \scnhaselement{синтаксическая структура информационной конструкции}
    \begin{scnindent}
        \scnrelto{второй домен}{синтаксис информационной конструкции*}
        \scnsuperset{первичная синтаксическая структура информационной конструкции}
        \scnsuperset{вторичная синтаксическая структура информационной конструкции}
    \end{scnindent}
    \scnhaselement{синтаксис языка}
    \scnhaselement{описание синтаксических понятий языка}
    \scnhaselement{синтаксические правила языка}
    \scnhaselement{денотационная семантика языка}
    \scnhaselement{описание семантических понятий языка}
    \scnhaselement{семантические правила языка}
    \scnhaselement{операционная семантика языка}
    \scnhaselement{смысл}
    \begin{scnindent}
        \scnrelto{второй домен}{смысл*}
    \end{scnindent}

\end{SCn}

\textit{Смысл} - явное (формальное) представление описываемых сущностей и связей между ними.
Для явного представления описываемых сущностей и связей между ними требуется существенное упрощение синтаксической структуры информационных конструкций.

\textit{Язык ostis-системы} - язык представления информационных конструкций в ostis-системах.

\begin{SCn}

    \scnheader{язык ostis-системы}
    \scnsubset{формальный язык}
    \scnsubset{универсальный язык}
    \scnrelto{используемые языки}{ostis-система}

    \scnhaselement{SC-код}
    \begin{scnindent}
        \scnidtf{Semantic Computer Code}
        \scnrelto{внутренний язык}{ostis-система}
        \scniselement{универсальный язык}
    \end{scnindent}

\end{SCn}

Для формального описания различного рода языков, включая рассматриваемые нами языки (SCg-код, SCs-код, SCn-код) используется целый ряд метаязыковых понятий.

Перечислим некоторые из них: \textit{идентификатор}, \textit{класс синтаксически эквивалентных идентификаторов}, \textit{имя}, \textit{простое имя}, \textit{выражение}, \textit{внешний идентификатор*}, \textit{алфавит*}, \textit{разделители*}, \textit{ограничители*}, \textit{предложения*}

Синтаксис \textit{языков представления знаний в ostis-системах} может быть формально описан различными способами.
Так, например, можно использовать метаязык Бэкуса-Наура для описания синтаксиса SCs-кода или его расширение для описания синтаксиса SCn-кода.
Однако значительно более логично и целесообразно описывать синтаксис всех форм внешнего отображения sc-текстов с помощью самого SC-кода.
Такой подход позволит ostis-системам самостоятельно понимать, анализировать и генерировать тексты указанных языков на основе принципов, общих для любых форм внешнего представления информации, в том числе нелинейных.

\textit{Алфавит*} - бинарное отношение, связывающее множество текстов с  семейством максимальных множеств синтаксически однотипных элементарных (атомарных) фрагментов текстов, принадлежащих заданному множеству текстов.

\scnheader{ограничители*}
\scnidtf{Отношение, связывающее заданный класс информационных конструкций с соответствующим классом их ограничителей}
\scnidtf{быть ограничителями, используемыми в заданном множестве информационных конструкций*}

\scnheader{разделители*}
\scnidtf{быть разделителями, используемыми в заданном множестве информационных конструкций*}
\scnrelfrom{второй домен}{разделитель}

\textit{Идентификатор} -- структурированный знак соответствующей (обозначаемой) сущности, который чаще всего представляет собой строку (цепочку символов), которую будем называть именем соответствующей сущности.
В формальных текстах (в том числе текстах SC-кода, SCg-кода, SCs-кода, SCn-кода) основные используемые идентификаторы не должны быть омонимичными, то есть должны \underline{однозначно} соответствовать идентифицируемым сущностям.
Следовательно, каждая пара идентификаторов, имеющих \underline{одинаковую} структуру, должны обозначать одну и ту же сущность.

\begin{SCn}

    \scnheader{имя}
    \scnsubset{идентификатор}
    \scnidtf{строковый идентификатор}
    \scnidtf{идентификатор, представляющий собой строку (цепочку) символов}
    \begin{scnrelfromset}{декомпозиция действия}
        \scnitem{простое имя}
        \begin{scnindent}
            \scnidtf{атомарное имя}
            \scnidtf{имя, в состав которого другие имена не входят}
        \end{scnindent}
        \scnitem{выражение}
        \begin{scnindent}
            \scnidtf{неатомарное имя}
        \end{scnindent}
    \end{scnrelfromset}

\end{SCn}

Внешний идентификатор* - бинарное ориентированное отношение, каждая связка (sc-дуга) которого связывает некоторый элемент с файлом, содержимым которого является внешний идентификатор (чаще всего, имя), соответствующий указанному элементу.
Понятие внешнего идентификатора является понятием относительным и важным для ostis-систем, поскольку внутреннее для ostis-систем представление информации (в виде текстов SC-кода) оперирует не идентификаторами описываемых сущностей, а знаками, структура которых никакого значения не имеет.

\scnheader{предложения*}
\scnidtf{быть множеством всех предложений заданного текста, не являющихся встроенными предложениями, то есть предложениями, входящими в состав других предложений*}
\scnrelfrom{второй домен}{предложение}

\scnheader{предложение}
\scnexplanation{минимальный семантически целостный фрагмент текста, представляющий собой конфигурацию знаков, входящих в этот фрагмент и связываемых между собой отношениями инцидентности (в частности, отношением непосредственной последовательности в строке), а также различного вида разделителями и ограничителями}

\section{Внешние информационные конструкции и внешние языки ostis-систем}
\label{sec_external_information_constructs_external_lang}

Существует несколько языков внешнего представления sc-текстов \scncite{Standart2021}:

\begin{itemize}
    \item SCn-код;
    \item SCs-код;
    \item SCg-код.
\end{itemize}

\textit{SCg-код} - \textit{внешний язык*} ostis-систем, тексты которого представляют собой графовые структуры общего вида с точно заданной \textit{денотационной семантикой*}.

\begin{SCn}

    \scnheader{SCg-код}
    \scniselement{язык ostis-системы}
    \begin{scnindent}
        \scnidtf{Semantic Code graphical}
        \scnrelto{внешний язык}{ostis-система}
        \scniselement{универсальный язык}
    \end{scnindent}

\end{SCn}

\textit{SCs-код} - \textit{внешний язык*} ostis-систем, тексты которого представляют собой строки (цепочки) символов.

\begin{SCn}

    \scnheader{SCs-код}
    \scniselement{язык ostis-системы}
    \begin{scnindent}
        \scnidtf{Semantic Code string}
        \scnrelto{внешний язык}{ostis-система}
        \scniselement{универсальный язык}
    \end{scnindent}

\end{SCn}

\textit{SCn-код} - \textit{внешний язык*} ostis-систем, тексты которого представляют собой двухмерные матрицы символов, являющиеся результатом форматирования, двухмерной структуризации текстов SCs-кода.

\begin{SCn}

    \scnheader{SCn-код}
    \scniselement{язык ostis-системы}
    \begin{scnindent}
        \scnidtf{Semantic Code natural}
        \scnrelto{внешний язык}{ostis-система}
        \scniselement{универсальный язык}
    \end{scnindent}

\end{SCn}

Для представления \textit{баз знаний ostis-систем} используется целый ряд как \textit{универсальных языков}, так и \textit{специализированных языков}, как \textit{формальных языков}, так и \textit{естественных языков}, как \textit{внутренних языков}, обеспечивающих представление информации в памяти \textit{ostis-систем}, так и \textit{внешних языков}, обеспечивающих представление информации, вводимой в память \textit{ostis-систем}, либо выводимой из этой памяти. \textit{Естественные языки} используются исключительно для представления \textit{файлов}, хранимых в памяти \textit{ostis-системы} и формально специфицируемых в рамках \textit{базы знаний} этой \textit{ostis-системы}.

Для эксплуатации \textit{интеллектуальных компьютерных систем}, построенных на основе \textit{SC-кода}, кроме способа абстрактного внутреннего представления баз знаний (\textit{SC-кода}) потребуются несколько способов внешнего изображения абстрактных \textit{sc-текстов}, удобных для пользователей и используемых при оформлении исходных текстов \textit{баз знаний} указанных интеллектуальных компьютерных систем и исходных текстов фрагментов этих \textit{баз знаний}, а также используемых для отображения пользователям различных фрагментов \textit{баз знаний} по пользовательским запросам.
В качестве таких способов внешнего отображения \textit{sc-текстов} и предлагаются указанные выше внешние языки ostis-систем (\textit{SCg-код}, \textit{SCs-код} и  \textit{SCn-код}).

%Для описания перечисленных \textit{языков}, используемых \textit{ostis-системами}, в каждом из них мы выделим \textit{ядро языка*}, которое является \textit{семантически эквивалентным языком*} для соответствующего языка и имеет минимальную синтаксическую сложность. Описание каждого из указанных языков строится как описание нескольких направлений синтаксического расширения выделенного \textit{языка-ядра}.

Все основные внешние формальные языки, используемые ostis-системами (\textit{SCg-код}, \textit{SCs-код}, \textit{SCn-код}) являются различными вариантами внешнего представления текстов внутреннего языка ostis-систем -- SC-кода.
Указанные языки являются универсальными и, следовательно, \textit{семантически эквивалентными языками*}.

При этом, каждая ostis-система может приобрести способность использовать любой внешний язык (как универсальный, так и специализированный, как естественный, так и искусственный), если синтаксис и денотационная семантика этого языка будут описаны в памяти ostis-системы на ее внутреннем языке (SC-коде).

Файл - информационная конструкция, представленная в "цифровой" форме, хранимой в какой-либо компьютерной памяти, но не являющийся sc-конструкцией, хранимой в памяти ostis-системы.

\scnheader{файл}
\begin{scnrelfromset}{разбиение}
    \scnitem{файл ostis-системы}
        \begin{scnindent}
        \scnidtf{файл, хранимый в памяти ostis-системы}
            \begin{scnrelfromset}{разбиение}
                \scnitem{внутренний файл ostis-системы}
                \scnitem{файл другой ostis-системы}
            \end{scnrelfromset}
        \end{scnindent}
    \scnitem{файл компьютерной системы, не явлющейся ostis-системой}
\end{scnrelfromset}

\scnheader{внутренний файл ostis-системы}
\begin{scnrelfromset}{разбиение}
    \scnitem{сформированный внутренний файл ostis-системы}
    \scnitem{несформированный внутренний файл ostis-системы}
\end{scnrelfromset}

Файл ostis-системы - инородная для sc-кода информационная конструкция, которая может как храниться в памяти ostis-системы, так и вне ее.

\begin{SCn}

    \scnheader{файл ostis-систем}
    \begin{scnrelfromset}{разбиение}
        \scnitem{хранимый файл ostis-систем}
        \scnitem{пустой файл ostis-систем}
    \end{scnrelfromset}
    \begin{scnrelfromset}{разбиение}
        \scnitem{файл-экземпляр}
            \begin{scnindent}
            \scnidtf{обозначение одного из вхождений (однорго из экземпляров) информационной конструкции]}
            \end{scnindent}
        \scnitem{файл-класс}
            \begin{scnindent}
            \scnidtf{обозначение всевозможнных информационных конструкций, каждая из которых эквивалентна той, что представлена содержимым данного sc-узла}
            \end{scnindent}
    \end{scnrelfromset}
    \begin{scnindent}
        \scnrelto{включение}{текстовый файл ostis-систем}
        \begin{scnindent}
            \scnrelto{включение}{естественно-языковой файл ostis-систем}
        \end{scnindent}
    \end{scnindent}

    \textit{Файлом ostis-системы} может стать любой \textit{файл} современной компьютерной системы.
    Некоторые \textit{текстовые файлы} (в первую очередь \textit{естественно-языковые файлы} - тексты \textit{естественных} языков) могут быть некоторым образом размечены путем указания связей размеченного файла с другими файлами ostis-системы, а также с различными sc-элементами.
    Так, например: (1) в любом месте размеченного текста можно сделать ссылку на другой файл, трактуемый как примечание к данному месту размеченного текста, (2) в размеченном тексте можно выделить используемые в нем основные идентификаторы (жирным курсивом) и неосновные идентификаторы (курсивом), (3) для выделенных неосновных sc-идентификаторов можно выделить соответствующие им основные.

    Eсли в естественно-языковом тексте (например в цитате) используется неосновной sc-идентификатор, то при разметке этого текста после указанного sc-идентификатора приводится соответствующий (синонимичный) основной sc-идентификатор.
    Eсли в естественно-языковом тексте приводится подряд несколько основных sc-идентификаторов, выделенных курсивом, то при разметке этого текста после указанных идентификаторов приводится перечень (через точку с запятой) соответствующих sc-идентификаторов.

    В состав текстового файла ostis-систем могут входить выделенные курсивом \textit{основные sc-идентификаторы}, являющиеся внешними идентификаторами соответствующих sc-элементов.
    В частности, это могут быть \textit{sc-идентификаторы}, обозначающие (1) другие текстовые файлы, являющиеся примечаниями и(или) пояснениями к соответствующим местам заданного файла, (2) библиографические ссылки.
    Такого рода \textit{sc-идентификаторы} в исходном тексте файла ограничителями ссылок на информационные конструкции (квадратными скобками).

    \scnheader{отношение, заданное на естественно-языковых файлах\scnsupergroupsign}
    \scnhaselement{ссылка*}
        \begin{scnindent}
        \scnidtf{бинарное ориентированное отношение, любая пара которого связывает \textit{естественно-языковой файл} с другим \textit{файлом}, на который указанный файл ссылается. При этом (1) второй файл может быть не только естественно-языковым файлом (2) в тексте первого файла может быть явно указано место с которого ссылка осуществляется, (3) частным видом ссылки может быть примечание.}
        \end{scnindent}
    \scnhaselement{ключевой знак*}
    \scnhaselement{семантическая смежность*}
    \scnhaselement{перевод*}
    \scnhaselement{авторская ссылка*}
    \scnhaselement{авторское примечание*}

\end{SCn}

%%%%%%%%%%%%%%%%%%%%%%%%% referenc.tex %%%%%%%%%%%%%%%%%%%%%%%%%%%%%%
% sample references
% %
% Use this file as a template for your own input.
%
%%%%%%%%%%%%%%%%%%%%%%%% Springer-Verlag %%%%%%%%%%%%%%%%%%%%%%%%%%
%
% BibTeX users please use
% \bibliographystyle{}
% \bibliography{}
%
\biblstarthook{In view of the parallel print and (chapter-wise) online publication of your book at \url{www.springerlink.com} it has been decided that -- as a genreral rule --  references should be sorted chapter-wise and placed at the end of the individual chapters. However, upon agreement with your contact at Springer you may list your references in a single seperate chapter at the end of your book. Deactivate the class option \texttt{sectrefs} and the \texttt{thebibliography} environment will be put out as a chapter of its own.\\\indent
References may be \textit{cited} in the text either by number (preferred) or by author/year.\footnote{Make sure that all references from the list are cited in the text. Those not cited should be moved to a separate \textit{Further Reading} section or chapter.} If the citatiion in the text is numbered, the reference list should be arranged in ascending order. If the citation in the text is author/year, the reference list should be \textit{sorted} alphabetically and if there are several works by the same author, the following order should be used:
\begin{enumerate}
\item all works by the author alone, ordered chronologically by year of publication
\item all works by the author with a coauthor, ordered alphabetically by coauthor
\item all works by the author with several coauthors, ordered chronologically by year of publication.
\end{enumerate}
The \textit{styling} of references\footnote{Always use the standard abbreviation of a journal's name according to the ISSN \textit{List of Title Word Abbreviations}, see \url{http://www.issn.org/en/node/344}} depends on the subject of your book:
\begin{itemize}
\item The \textit{two} recommended styles for references in books on \textit{mathematical, physical, statistical and computer sciences} are depicted in ~\cite{science-contrib, science-online, science-mono, science-journal, science-DOI} and ~\cite{phys-online, phys-mono, phys-journal, phys-DOI, phys-contrib}.
\item Examples of the most commonly used reference style in books on \textit{Psychology, Social Sciences} are~\cite{psysoc-mono, psysoc-online,psysoc-journal, psysoc-contrib, psysoc-DOI}.
\item Examples for references in books on \textit{Humanities, Linguistics, Philosophy} are~\cite{humlinphil-journal, humlinphil-contrib, humlinphil-mono, humlinphil-online, humlinphil-DOI}.
\item Examples of the basic Springer style used in publications on a wide range of subjects such as \textit{Computer Science, Economics, Engineering, Geosciences, Life Sciences, Medicine, Biomedicine} are ~\cite{basic-contrib, basic-online, basic-journal, basic-DOI, basic-mono}. 
\end{itemize}
}

\begin{thebibliography}{99.}%
% and use \bibitem to create references.
%
% Use the following syntax and markup for your references if 
% the subject of your book is from the field 
% "Mathematics, Physics, Statistics, Computer Science"
%
% Contribution 
\bibitem{science-contrib} Broy, M.: Software engineering --- from auxiliary to key technologies. In: Broy, M., Dener, E. (eds.) Software Pioneers, pp. 10-13. Springer, Heidelberg (2002)
%
% Online Document
\bibitem{science-online} Dod, J.: Effective substances. In: The Dictionary of Substances and Their Effects. Royal Society of Chemistry (1999) Available via DIALOG. \\
\url{http://www.rsc.org/dose/title of subordinate document. Cited 15 Jan 1999}
%
% Monograph
\bibitem{science-mono} Geddes, K.O., Czapor, S.R., Labahn, G.: Algorithms for Computer Algebra. Kluwer, Boston (1992) 
%
% Journal article
\bibitem{science-journal} Hamburger, C.: Quasimonotonicity, regularity and duality for nonlinear systems of partial differential equations. Ann. Mat. Pura. Appl. \textbf{169}, 321--354 (1995)
%
% Journal article by DOI
\bibitem{science-DOI} Slifka, M.K., Whitton, J.L.: Clinical implications of dysregulated cytokine production. J. Mol. Med. (2000) doi: 10.1007/s001090000086 
%
\bigskip

% Use the following (APS) syntax and markup for your references if 
% the subject of your book is from the field 
% "Mathematics, Physics, Statistics, Computer Science"
%
% Online Document
\bibitem{phys-online} J. Dod, in \textit{The Dictionary of Substances and Their Effects}, Royal Society of Chemistry. (Available via DIALOG, 1999), 
\url{http://www.rsc.org/dose/title of subordinate document. Cited 15 Jan 1999}
%
% Monograph
\bibitem{phys-mono} H. Ibach, H. L\"uth, \textit{Solid-State Physics}, 2nd edn. (Springer, New York, 1996), pp. 45-56 
%
% Journal article
\bibitem{phys-journal} S. Preuss, A. Demchuk Jr., M. Stuke, Appl. Phys. A \textbf{61}
%
% Journal article by DOI
\bibitem{phys-DOI} M.K. Slifka, J.L. Whitton, J. Mol. Med., doi: 10.1007/s001090000086
%
% Contribution 
\bibitem{phys-contrib} S.E. Smith, in \textit{Neuromuscular Junction}, ed. by E. Zaimis. Handbook of Experimental Pharmacology, vol 42 (Springer, Heidelberg, 1976), p. 593
%
\bigskip
%
% Use the following syntax and markup for your references if 
% the subject of your book is from the field 
% "Psychology, Social Sciences"
%
%
% Monograph
\bibitem{psysoc-mono} Calfee, R.~C., \& Valencia, R.~R. (1991). \textit{APA guide to preparing manuscripts for journal publication.} Washington, DC: American Psychological Association.
%
% Online Document
\bibitem{psysoc-online} Dod, J. (1999). Effective substances. In: The dictionary of substances and their effects. Royal Society of Chemistry. Available via DIALOG. \\
\url{http://www.rsc.org/dose/Effective substances.} Cited 15 Jan 1999.
%
% Journal article
\bibitem{psysoc-journal} Harris, M., Karper, E., Stacks, G., Hoffman, D., DeNiro, R., Cruz, P., et al. (2001). Writing labs and the Hollywood connection. \textit{J Film} Writing, 44(3), 213--245.
%
% Contribution 
\bibitem{psysoc-contrib} O'Neil, J.~M., \& Egan, J. (1992). Men's and women's gender role journeys: Metaphor for healing, transition, and transformation. In B.~R. Wainrig (Ed.), \textit{Gender issues across the life cycle} (pp. 107--123). New York: Springer.
%
% Journal article by DOI
\bibitem{psysoc-DOI}Kreger, M., Brindis, C.D., Manuel, D.M., Sassoubre, L. (2007). Lessons learned in systems change initiatives: benchmarks and indicators. \textit{American Journal of Community Psychology}, doi: 10.1007/s10464-007-9108-14.
%
%
% Use the following syntax and markup for your references if 
% the subject of your book is from the field 
% "Humanities, Linguistics, Philosophy"
%
\bigskip
%
% Journal article
\bibitem{humlinphil-journal} Alber John, Daniel C. O'Connell, and Sabine Kowal. 2002. Personal perspective in TV interviews. \textit{Pragmatics} 12:257--271
%
% Contribution 
\bibitem{humlinphil-contrib} Cameron, Deborah. 1997. Theoretical debates in feminist linguistics: Questions of sex and gender. In \textit{Gender and discourse}, ed. Ruth Wodak, 99--119. London: Sage Publications.
%
% Monograph
\bibitem{humlinphil-mono} Cameron, Deborah. 1985. \textit{Feminism and linguistic theory.} New York: St. Martin's Press.
%
% Online Document
\bibitem{humlinphil-online} Dod, Jake. 1999. Effective substances. In: The dictionary of substances and their effects. Royal Society of Chemistry. Available via DIALOG. \\
http://www.rsc.org/dose/title of subordinate document. Cited 15 Jan 1999
%
% Journal article by DOI
\bibitem{humlinphil-DOI} Suleiman, Camelia, Daniel C. O'Connell, and Sabine Kowal. 2002. `If you and I, if we, in this later day, lose that sacred fire...': Perspective in political interviews. \textit{Journal of Psycholinguistic Research}. doi: 10.1023/A:1015592129296.
%
%
%
\bigskip
%
%
% Use the following syntax and markup for your references if 
% the subject of your book is from the field 
% "Computer Science, Economics, Engineering, Geosciences, Life Sciences"
%
%
% Contribution 
\bibitem{basic-contrib} Brown B, Aaron M (2001) The politics of nature. In: Smith J (ed) The rise of modern genomics, 3rd edn. Wiley, New York 
%
% Online Document
\bibitem{basic-online} Dod J (1999) Effective Substances. In: The dictionary of substances and their effects. Royal Society of Chemistry. Available via DIALOG. \\
\url{http://www.rsc.org/dose/title of subordinate document. Cited 15 Jan 1999}
%
% Journal article by DOI
\bibitem{basic-DOI} Slifka MK, Whitton JL (2000) Clinical implications of dysregulated cytokine production. J Mol Med, doi: 10.1007/s001090000086
%
% Journal article
\bibitem{basic-journal} Smith J, Jones M Jr, Houghton L et al (1999) Future of health insurance. N Engl J Med 965:325--329
%
% Monograph
\bibitem{basic-mono} South J, Blass B (2001) The future of modern genomics. Blackwell, London 
%
\end{thebibliography}
